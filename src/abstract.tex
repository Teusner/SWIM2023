% LaTeX template for submitting an abstract to  
%
%           SWIM 2023
%           July 11-12-13, 2023
%           Angers, FRANCE
%
%% Please use the least number of macros and packages as possible
%% which will help us parsing your abstracts. Thank you! Additionally, 
%% please use UTF8 for font encoding!!! The last but not least, please 
%% do not change the font size and the paper format. 
%%
%% Thank you! We are looking forward for your contributions.
%%
%% Many thanks to all previous SWIM organizers for allowing us to use
%% their template as basis for our style file.

\documentclass[14pt, a4paper]{article}

%% Please use only the following packages. Thank you!
\usepackage{extsizes}
\usepackage{amsmath}
\usepackage{amsthm}
\usepackage{amssymb}
\usepackage{url}
\usepackage{graphicx}
% please use UTF8 for the encoding!!!
\usepackage[utf8]{inputenc}

\pagenumbering{arabic}
\pagestyle{myheadings}
\markright{LARIS/Polytech Angers \hfill SWIM 2023\hfill }
\clearpage

% DEFINITIONS
\newtheorem{prop}{Proposition}

% USER PACKAGES
%\usepackage{solvehaltingproblem}
\usepackage{pgf}
\usepackage{subcaption}

% AUTHORS - if all affiliations are the same, upper indices can be excluded
\newcommand\authors{Quentin Brateau$^{1}$, Fabrice Le Bars $^{1}$ and Luc Jaulin$^{1}$}

% TITLE - required 
\newcommand\papertitle{Union of adjacent contractors}

\begin{document}

	\begin{center}

	% BEGIN DO NOT MODIFY 
	\section*{\papertitle}
	% #REPLACE FOR MENU#
	\vspace*{0.8cm}
	{\large \authors}
	% END DO NOT MODIFY

	\bigskip

	% AFFILIATION - required with address, email at least for corresponding author
	{
		\small $^{1}$ENSTA Bretagne, UMR 6285, Lab-STICC, IAO, ROBEX\\
		2 rue François Verny, 29806 Brest CEDEX 09, \textsc{France} \\
		\medskip
		\texttt{quentin.brateau@ensta-bretagne.org}\\
		\texttt{fabrice.le\_bars@ensta-bretagne.org}\\
		\texttt{lucjaulin@gmail.com}\\
	}

	\end{center}

	\bigskip

	% KEYWORDS - required
	{\noindent\bf Keywords:} Set union, Contractor programming, Geometric contractors, Localization

	\subsection*{Introduction}
		Set theory provides a fundamental structure for interval analysis, which must conform to its formalism~[1,2]. Trivial operations are defined on sets, such as union, intersection, deprivation, cartesian product, and projection. They should be applicable to intervals and therefore to contractors.

		In the case of union of adjacent contractors, typical paving algorithm bisects unnecessarily the boxes and then reveals the common boundary between the two sets as shown in Figure~\ref{fig:sepvisible}. This behavior noticed on contractors union is not consistent with the set union as defined in set theory.

	\subsection*{Geometric contractors}
		Geometric contractors are a class of contractors based on geometric constraints, particularly used for localization in robotics~[2]. They can be used to characterize all possible robot states based on measurements. Geometric contractors are often defined for segments, which are one of the most simple geometric shapes. Then, by using set operators, more complex contractors can be built, such as contractors based on polygons~[3].

		By defining more complex contractors in this way, adjacent boundary-overlapping sets appear at each vertex. Figure~\ref{fig:sepvisible} shows the paving of a visibility separator from a point, implemented by Rémy Guyonneau~[3]. The visibility separator works well on two individual segments but fails to characterize inner subpaving when dealing with polygons.

		\begin{figure}[!htb]
			\centering
			\begin{subfigure}[t]{.31\textwidth}
				%% Creator: Matplotlib, PGF backend
%%
%% To include the figure in your LaTeX document, write
%%   \input{<filename>.pgf}
%%
%% Make sure the required packages are loaded in your preamble
%%   \usepackage{pgf}
%%
%% Also ensure that all the required font packages are loaded; for instance,
%% the lmodern package is sometimes necessary when using math font.
%%   \usepackage{lmodern}
%%
%% Figures using additional raster images can only be included by \input if
%% they are in the same directory as the main LaTeX file. For loading figures
%% from other directories you can use the `import` package
%%   \usepackage{import}
%%
%% and then include the figures with
%%   \import{<path to file>}{<filename>.pgf}
%%
%% Matplotlib used the following preamble
%%
\begingroup%
\makeatletter%
\begin{pgfpicture}%
\pgfpathrectangle{\pgfpointorigin}{\pgfqpoint{2.100000in}{2.100000in}}%
\pgfusepath{use as bounding box, clip}%
\begin{pgfscope}%
\pgfsetbuttcap%
\pgfsetmiterjoin%
\definecolor{currentfill}{rgb}{1.000000,1.000000,1.000000}%
\pgfsetfillcolor{currentfill}%
\pgfsetlinewidth{0.000000pt}%
\definecolor{currentstroke}{rgb}{1.000000,1.000000,1.000000}%
\pgfsetstrokecolor{currentstroke}%
\pgfsetdash{}{0pt}%
\pgfpathmoveto{\pgfqpoint{0.000000in}{0.000000in}}%
\pgfpathlineto{\pgfqpoint{2.100000in}{0.000000in}}%
\pgfpathlineto{\pgfqpoint{2.100000in}{2.100000in}}%
\pgfpathlineto{\pgfqpoint{0.000000in}{2.100000in}}%
\pgfpathlineto{\pgfqpoint{0.000000in}{0.000000in}}%
\pgfpathclose%
\pgfusepath{fill}%
\end{pgfscope}%
\begin{pgfscope}%
\pgfpathrectangle{\pgfqpoint{0.150000in}{0.150000in}}{\pgfqpoint{1.800000in}{1.800000in}}%
\pgfusepath{clip}%
\pgfsetbuttcap%
\pgfsetroundjoin%
\definecolor{currentfill}{rgb}{0.933333,0.600000,0.666667}%
\pgfsetfillcolor{currentfill}%
\pgfsetlinewidth{1.003750pt}%
\definecolor{currentstroke}{rgb}{0.600000,0.266667,0.333333}%
\pgfsetstrokecolor{currentstroke}%
\pgfsetdash{}{0pt}%
\pgfpathmoveto{\pgfqpoint{0.510000in}{1.162665in}}%
\pgfpathlineto{\pgfqpoint{0.518295in}{1.162665in}}%
\pgfpathlineto{\pgfqpoint{0.518295in}{1.205116in}}%
\pgfpathlineto{\pgfqpoint{0.510000in}{1.205116in}}%
\pgfpathlineto{\pgfqpoint{0.510000in}{1.162665in}}%
\pgfpathclose%
\pgfusepath{stroke,fill}%
\end{pgfscope}%
\begin{pgfscope}%
\pgfpathrectangle{\pgfqpoint{0.150000in}{0.150000in}}{\pgfqpoint{1.800000in}{1.800000in}}%
\pgfusepath{clip}%
\pgfsetbuttcap%
\pgfsetroundjoin%
\definecolor{currentfill}{rgb}{0.933333,0.600000,0.666667}%
\pgfsetfillcolor{currentfill}%
\pgfsetlinewidth{1.003750pt}%
\definecolor{currentstroke}{rgb}{0.600000,0.266667,0.333333}%
\pgfsetstrokecolor{currentstroke}%
\pgfsetdash{}{0pt}%
\pgfpathmoveto{\pgfqpoint{0.651728in}{0.690000in}}%
\pgfpathlineto{\pgfqpoint{0.672777in}{0.690000in}}%
\pgfpathlineto{\pgfqpoint{0.672777in}{0.741668in}}%
\pgfpathlineto{\pgfqpoint{0.651728in}{0.741668in}}%
\pgfpathlineto{\pgfqpoint{0.651728in}{0.690000in}}%
\pgfpathclose%
\pgfusepath{stroke,fill}%
\end{pgfscope}%
\begin{pgfscope}%
\pgfpathrectangle{\pgfqpoint{0.150000in}{0.150000in}}{\pgfqpoint{1.800000in}{1.800000in}}%
\pgfusepath{clip}%
\pgfsetbuttcap%
\pgfsetroundjoin%
\definecolor{currentfill}{rgb}{0.933333,0.600000,0.666667}%
\pgfsetfillcolor{currentfill}%
\pgfsetlinewidth{1.003750pt}%
\definecolor{currentstroke}{rgb}{0.600000,0.266667,0.333333}%
\pgfsetstrokecolor{currentstroke}%
\pgfsetdash{}{0pt}%
\pgfpathmoveto{\pgfqpoint{0.630940in}{1.886586in}}%
\pgfpathlineto{\pgfqpoint{0.641317in}{1.886586in}}%
\pgfpathlineto{\pgfqpoint{0.641317in}{1.950000in}}%
\pgfpathlineto{\pgfqpoint{0.630940in}{1.950000in}}%
\pgfpathlineto{\pgfqpoint{0.630940in}{1.886586in}}%
\pgfpathclose%
\pgfusepath{stroke,fill}%
\end{pgfscope}%
\begin{pgfscope}%
\pgfpathrectangle{\pgfqpoint{0.150000in}{0.150000in}}{\pgfqpoint{1.800000in}{1.800000in}}%
\pgfusepath{clip}%
\pgfsetbuttcap%
\pgfsetroundjoin%
\definecolor{currentfill}{rgb}{0.933333,0.600000,0.666667}%
\pgfsetfillcolor{currentfill}%
\pgfsetlinewidth{1.003750pt}%
\definecolor{currentstroke}{rgb}{0.600000,0.266667,0.333333}%
\pgfsetstrokecolor{currentstroke}%
\pgfsetdash{}{0pt}%
\pgfpathmoveto{\pgfqpoint{0.612074in}{1.782818in}}%
\pgfpathlineto{\pgfqpoint{0.620564in}{1.782818in}}%
\pgfpathlineto{\pgfqpoint{0.620564in}{1.834702in}}%
\pgfpathlineto{\pgfqpoint{0.612074in}{1.834702in}}%
\pgfpathlineto{\pgfqpoint{0.612074in}{1.782818in}}%
\pgfpathclose%
\pgfusepath{stroke,fill}%
\end{pgfscope}%
\begin{pgfscope}%
\pgfpathrectangle{\pgfqpoint{0.150000in}{0.150000in}}{\pgfqpoint{1.800000in}{1.800000in}}%
\pgfusepath{clip}%
\pgfsetbuttcap%
\pgfsetroundjoin%
\definecolor{currentfill}{rgb}{0.933333,0.600000,0.666667}%
\pgfsetfillcolor{currentfill}%
\pgfsetlinewidth{1.003750pt}%
\definecolor{currentstroke}{rgb}{0.600000,0.266667,0.333333}%
\pgfsetstrokecolor{currentstroke}%
\pgfsetdash{}{0pt}%
\pgfpathmoveto{\pgfqpoint{0.593207in}{1.688483in}}%
\pgfpathlineto{\pgfqpoint{0.601697in}{1.688483in}}%
\pgfpathlineto{\pgfqpoint{0.601697in}{1.740368in}}%
\pgfpathlineto{\pgfqpoint{0.593207in}{1.740368in}}%
\pgfpathlineto{\pgfqpoint{0.593207in}{1.688483in}}%
\pgfpathclose%
\pgfusepath{stroke,fill}%
\end{pgfscope}%
\begin{pgfscope}%
\pgfpathrectangle{\pgfqpoint{0.150000in}{0.150000in}}{\pgfqpoint{1.800000in}{1.800000in}}%
\pgfusepath{clip}%
\pgfsetbuttcap%
\pgfsetroundjoin%
\definecolor{currentfill}{rgb}{0.933333,0.600000,0.666667}%
\pgfsetfillcolor{currentfill}%
\pgfsetlinewidth{1.003750pt}%
\definecolor{currentstroke}{rgb}{0.600000,0.266667,0.333333}%
\pgfsetstrokecolor{currentstroke}%
\pgfsetdash{}{0pt}%
\pgfpathmoveto{\pgfqpoint{0.558903in}{1.516966in}}%
\pgfpathlineto{\pgfqpoint{0.567393in}{1.516966in}}%
\pgfpathlineto{\pgfqpoint{0.567393in}{1.568850in}}%
\pgfpathlineto{\pgfqpoint{0.558903in}{1.568850in}}%
\pgfpathlineto{\pgfqpoint{0.558903in}{1.516966in}}%
\pgfpathclose%
\pgfusepath{stroke,fill}%
\end{pgfscope}%
\begin{pgfscope}%
\pgfpathrectangle{\pgfqpoint{0.150000in}{0.150000in}}{\pgfqpoint{1.800000in}{1.800000in}}%
\pgfusepath{clip}%
\pgfsetbuttcap%
\pgfsetroundjoin%
\definecolor{currentfill}{rgb}{0.933333,0.600000,0.666667}%
\pgfsetfillcolor{currentfill}%
\pgfsetlinewidth{1.003750pt}%
\definecolor{currentstroke}{rgb}{0.600000,0.266667,0.333333}%
\pgfsetstrokecolor{currentstroke}%
\pgfsetdash{}{0pt}%
\pgfpathmoveto{\pgfqpoint{1.098704in}{1.050189in}}%
\pgfpathlineto{\pgfqpoint{1.158000in}{1.050189in}}%
\pgfpathlineto{\pgfqpoint{1.158000in}{1.098704in}}%
\pgfpathlineto{\pgfqpoint{1.098704in}{1.098704in}}%
\pgfpathlineto{\pgfqpoint{1.098704in}{1.050189in}}%
\pgfpathclose%
\pgfusepath{stroke,fill}%
\end{pgfscope}%
\begin{pgfscope}%
\pgfpathrectangle{\pgfqpoint{0.150000in}{0.150000in}}{\pgfqpoint{1.800000in}{1.800000in}}%
\pgfusepath{clip}%
\pgfsetbuttcap%
\pgfsetroundjoin%
\definecolor{currentfill}{rgb}{0.933333,0.600000,0.666667}%
\pgfsetfillcolor{currentfill}%
\pgfsetlinewidth{1.003750pt}%
\definecolor{currentstroke}{rgb}{0.600000,0.266667,0.333333}%
\pgfsetstrokecolor{currentstroke}%
\pgfsetdash{}{0pt}%
\pgfpathmoveto{\pgfqpoint{0.510000in}{1.085483in}}%
\pgfpathlineto{\pgfqpoint{0.532445in}{1.085483in}}%
\pgfpathlineto{\pgfqpoint{0.532445in}{1.162665in}}%
\pgfpathlineto{\pgfqpoint{0.510000in}{1.162665in}}%
\pgfpathlineto{\pgfqpoint{0.510000in}{1.085483in}}%
\pgfpathclose%
\pgfusepath{stroke,fill}%
\end{pgfscope}%
\begin{pgfscope}%
\pgfpathrectangle{\pgfqpoint{0.150000in}{0.150000in}}{\pgfqpoint{1.800000in}{1.800000in}}%
\pgfusepath{clip}%
\pgfsetbuttcap%
\pgfsetroundjoin%
\definecolor{currentfill}{rgb}{0.933333,0.600000,0.666667}%
\pgfsetfillcolor{currentfill}%
\pgfsetlinewidth{1.003750pt}%
\definecolor{currentstroke}{rgb}{0.600000,0.266667,0.333333}%
\pgfsetstrokecolor{currentstroke}%
\pgfsetdash{}{0pt}%
\pgfpathmoveto{\pgfqpoint{0.558173in}{0.945150in}}%
\pgfpathlineto{\pgfqpoint{0.583900in}{0.945150in}}%
\pgfpathlineto{\pgfqpoint{0.583900in}{1.008300in}}%
\pgfpathlineto{\pgfqpoint{0.558173in}{1.008300in}}%
\pgfpathlineto{\pgfqpoint{0.558173in}{0.945150in}}%
\pgfpathclose%
\pgfusepath{stroke,fill}%
\end{pgfscope}%
\begin{pgfscope}%
\pgfpathrectangle{\pgfqpoint{0.150000in}{0.150000in}}{\pgfqpoint{1.800000in}{1.800000in}}%
\pgfusepath{clip}%
\pgfsetbuttcap%
\pgfsetroundjoin%
\definecolor{currentfill}{rgb}{0.933333,0.600000,0.666667}%
\pgfsetfillcolor{currentfill}%
\pgfsetlinewidth{1.003750pt}%
\definecolor{currentstroke}{rgb}{0.600000,0.266667,0.333333}%
\pgfsetstrokecolor{currentstroke}%
\pgfsetdash{}{0pt}%
\pgfpathmoveto{\pgfqpoint{0.604950in}{0.804818in}}%
\pgfpathlineto{\pgfqpoint{0.630678in}{0.804818in}}%
\pgfpathlineto{\pgfqpoint{0.630678in}{0.867967in}}%
\pgfpathlineto{\pgfqpoint{0.604950in}{0.867967in}}%
\pgfpathlineto{\pgfqpoint{0.604950in}{0.804818in}}%
\pgfpathclose%
\pgfusepath{stroke,fill}%
\end{pgfscope}%
\begin{pgfscope}%
\pgfpathrectangle{\pgfqpoint{0.150000in}{0.150000in}}{\pgfqpoint{1.800000in}{1.800000in}}%
\pgfusepath{clip}%
\pgfsetbuttcap%
\pgfsetroundjoin%
\definecolor{currentfill}{rgb}{0.933333,0.600000,0.666667}%
\pgfsetfillcolor{currentfill}%
\pgfsetlinewidth{1.003750pt}%
\definecolor{currentstroke}{rgb}{0.600000,0.266667,0.333333}%
\pgfsetstrokecolor{currentstroke}%
\pgfsetdash{}{0pt}%
\pgfpathmoveto{\pgfqpoint{0.719170in}{0.690000in}}%
\pgfpathlineto{\pgfqpoint{0.801600in}{0.690000in}}%
\pgfpathlineto{\pgfqpoint{0.801600in}{0.719170in}}%
\pgfpathlineto{\pgfqpoint{0.719170in}{0.719170in}}%
\pgfpathlineto{\pgfqpoint{0.719170in}{0.690000in}}%
\pgfpathclose%
\pgfusepath{stroke,fill}%
\end{pgfscope}%
\begin{pgfscope}%
\pgfpathrectangle{\pgfqpoint{0.150000in}{0.150000in}}{\pgfqpoint{1.800000in}{1.800000in}}%
\pgfusepath{clip}%
\pgfsetbuttcap%
\pgfsetroundjoin%
\definecolor{currentfill}{rgb}{0.933333,0.600000,0.666667}%
\pgfsetfillcolor{currentfill}%
\pgfsetlinewidth{1.003750pt}%
\definecolor{currentstroke}{rgb}{0.600000,0.266667,0.333333}%
\pgfsetstrokecolor{currentstroke}%
\pgfsetdash{}{0pt}%
\pgfpathmoveto{\pgfqpoint{1.877527in}{1.818231in}}%
\pgfpathlineto{\pgfqpoint{1.950000in}{1.818231in}}%
\pgfpathlineto{\pgfqpoint{1.950000in}{1.877527in}}%
\pgfpathlineto{\pgfqpoint{1.877527in}{1.877527in}}%
\pgfpathlineto{\pgfqpoint{1.877527in}{1.818231in}}%
\pgfpathclose%
\pgfusepath{stroke,fill}%
\end{pgfscope}%
\begin{pgfscope}%
\pgfpathrectangle{\pgfqpoint{0.150000in}{0.150000in}}{\pgfqpoint{1.800000in}{1.800000in}}%
\pgfusepath{clip}%
\pgfsetbuttcap%
\pgfsetroundjoin%
\definecolor{currentfill}{rgb}{0.933333,0.600000,0.666667}%
\pgfsetfillcolor{currentfill}%
\pgfsetlinewidth{1.003750pt}%
\definecolor{currentstroke}{rgb}{0.600000,0.266667,0.333333}%
\pgfsetstrokecolor{currentstroke}%
\pgfsetdash{}{0pt}%
\pgfpathmoveto{\pgfqpoint{1.758935in}{1.710420in}}%
\pgfpathlineto{\pgfqpoint{1.818231in}{1.710420in}}%
\pgfpathlineto{\pgfqpoint{1.818231in}{1.758935in}}%
\pgfpathlineto{\pgfqpoint{1.758935in}{1.758935in}}%
\pgfpathlineto{\pgfqpoint{1.758935in}{1.710420in}}%
\pgfpathclose%
\pgfusepath{stroke,fill}%
\end{pgfscope}%
\begin{pgfscope}%
\pgfpathrectangle{\pgfqpoint{0.150000in}{0.150000in}}{\pgfqpoint{1.800000in}{1.800000in}}%
\pgfusepath{clip}%
\pgfsetbuttcap%
\pgfsetroundjoin%
\definecolor{currentfill}{rgb}{0.933333,0.600000,0.666667}%
\pgfsetfillcolor{currentfill}%
\pgfsetlinewidth{1.003750pt}%
\definecolor{currentstroke}{rgb}{0.600000,0.266667,0.333333}%
\pgfsetstrokecolor{currentstroke}%
\pgfsetdash{}{0pt}%
\pgfpathmoveto{\pgfqpoint{1.651124in}{1.602609in}}%
\pgfpathlineto{\pgfqpoint{1.710420in}{1.602609in}}%
\pgfpathlineto{\pgfqpoint{1.710420in}{1.651124in}}%
\pgfpathlineto{\pgfqpoint{1.651124in}{1.651124in}}%
\pgfpathlineto{\pgfqpoint{1.651124in}{1.602609in}}%
\pgfpathclose%
\pgfusepath{stroke,fill}%
\end{pgfscope}%
\begin{pgfscope}%
\pgfpathrectangle{\pgfqpoint{0.150000in}{0.150000in}}{\pgfqpoint{1.800000in}{1.800000in}}%
\pgfusepath{clip}%
\pgfsetbuttcap%
\pgfsetroundjoin%
\definecolor{currentfill}{rgb}{0.933333,0.600000,0.666667}%
\pgfsetfillcolor{currentfill}%
\pgfsetlinewidth{1.003750pt}%
\definecolor{currentstroke}{rgb}{0.600000,0.266667,0.333333}%
\pgfsetstrokecolor{currentstroke}%
\pgfsetdash{}{0pt}%
\pgfpathmoveto{\pgfqpoint{1.455104in}{1.406589in}}%
\pgfpathlineto{\pgfqpoint{1.514400in}{1.406589in}}%
\pgfpathlineto{\pgfqpoint{1.514400in}{1.455104in}}%
\pgfpathlineto{\pgfqpoint{1.455104in}{1.455104in}}%
\pgfpathlineto{\pgfqpoint{1.455104in}{1.406589in}}%
\pgfpathclose%
\pgfusepath{stroke,fill}%
\end{pgfscope}%
\begin{pgfscope}%
\pgfpathrectangle{\pgfqpoint{0.150000in}{0.150000in}}{\pgfqpoint{1.800000in}{1.800000in}}%
\pgfusepath{clip}%
\pgfsetbuttcap%
\pgfsetroundjoin%
\definecolor{currentfill}{rgb}{0.933333,0.600000,0.666667}%
\pgfsetfillcolor{currentfill}%
\pgfsetlinewidth{1.003750pt}%
\definecolor{currentstroke}{rgb}{0.600000,0.266667,0.333333}%
\pgfsetstrokecolor{currentstroke}%
\pgfsetdash{}{0pt}%
\pgfpathmoveto{\pgfqpoint{0.612074in}{1.834702in}}%
\pgfpathlineto{\pgfqpoint{0.630940in}{1.834702in}}%
\pgfpathlineto{\pgfqpoint{0.630940in}{1.950000in}}%
\pgfpathlineto{\pgfqpoint{0.612074in}{1.950000in}}%
\pgfpathlineto{\pgfqpoint{0.612074in}{1.834702in}}%
\pgfpathclose%
\pgfusepath{stroke,fill}%
\end{pgfscope}%
\begin{pgfscope}%
\pgfpathrectangle{\pgfqpoint{0.150000in}{0.150000in}}{\pgfqpoint{1.800000in}{1.800000in}}%
\pgfusepath{clip}%
\pgfsetbuttcap%
\pgfsetroundjoin%
\definecolor{currentfill}{rgb}{0.933333,0.600000,0.666667}%
\pgfsetfillcolor{currentfill}%
\pgfsetlinewidth{1.003750pt}%
\definecolor{currentstroke}{rgb}{0.600000,0.266667,0.333333}%
\pgfsetstrokecolor{currentstroke}%
\pgfsetdash{}{0pt}%
\pgfpathmoveto{\pgfqpoint{0.577770in}{1.646033in}}%
\pgfpathlineto{\pgfqpoint{0.593207in}{1.646033in}}%
\pgfpathlineto{\pgfqpoint{0.593207in}{1.740368in}}%
\pgfpathlineto{\pgfqpoint{0.577770in}{1.740368in}}%
\pgfpathlineto{\pgfqpoint{0.577770in}{1.646033in}}%
\pgfpathclose%
\pgfusepath{stroke,fill}%
\end{pgfscope}%
\begin{pgfscope}%
\pgfpathrectangle{\pgfqpoint{0.150000in}{0.150000in}}{\pgfqpoint{1.800000in}{1.800000in}}%
\pgfusepath{clip}%
\pgfsetbuttcap%
\pgfsetroundjoin%
\definecolor{currentfill}{rgb}{0.933333,0.600000,0.666667}%
\pgfsetfillcolor{currentfill}%
\pgfsetlinewidth{1.003750pt}%
\definecolor{currentstroke}{rgb}{0.600000,0.266667,0.333333}%
\pgfsetstrokecolor{currentstroke}%
\pgfsetdash{}{0pt}%
\pgfpathmoveto{\pgfqpoint{0.543467in}{1.474515in}}%
\pgfpathlineto{\pgfqpoint{0.558903in}{1.474515in}}%
\pgfpathlineto{\pgfqpoint{0.558903in}{1.568850in}}%
\pgfpathlineto{\pgfqpoint{0.543467in}{1.568850in}}%
\pgfpathlineto{\pgfqpoint{0.543467in}{1.474515in}}%
\pgfpathclose%
\pgfusepath{stroke,fill}%
\end{pgfscope}%
\begin{pgfscope}%
\pgfpathrectangle{\pgfqpoint{0.150000in}{0.150000in}}{\pgfqpoint{1.800000in}{1.800000in}}%
\pgfusepath{clip}%
\pgfsetbuttcap%
\pgfsetroundjoin%
\definecolor{currentfill}{rgb}{0.933333,0.600000,0.666667}%
\pgfsetfillcolor{currentfill}%
\pgfsetlinewidth{1.003750pt}%
\definecolor{currentstroke}{rgb}{0.600000,0.266667,0.333333}%
\pgfsetstrokecolor{currentstroke}%
\pgfsetdash{}{0pt}%
\pgfpathmoveto{\pgfqpoint{0.515400in}{1.320150in}}%
\pgfpathlineto{\pgfqpoint{0.528030in}{1.320150in}}%
\pgfpathlineto{\pgfqpoint{0.528030in}{1.397333in}}%
\pgfpathlineto{\pgfqpoint{0.515400in}{1.397333in}}%
\pgfpathlineto{\pgfqpoint{0.515400in}{1.320150in}}%
\pgfpathclose%
\pgfusepath{stroke,fill}%
\end{pgfscope}%
\begin{pgfscope}%
\pgfpathrectangle{\pgfqpoint{0.150000in}{0.150000in}}{\pgfqpoint{1.800000in}{1.800000in}}%
\pgfusepath{clip}%
\pgfsetbuttcap%
\pgfsetroundjoin%
\definecolor{currentfill}{rgb}{0.933333,0.600000,0.666667}%
\pgfsetfillcolor{currentfill}%
\pgfsetlinewidth{1.003750pt}%
\definecolor{currentstroke}{rgb}{0.600000,0.266667,0.333333}%
\pgfsetstrokecolor{currentstroke}%
\pgfsetdash{}{0pt}%
\pgfpathmoveto{\pgfqpoint{1.050189in}{0.961980in}}%
\pgfpathlineto{\pgfqpoint{1.158000in}{0.961980in}}%
\pgfpathlineto{\pgfqpoint{1.158000in}{1.050189in}}%
\pgfpathlineto{\pgfqpoint{1.050189in}{1.050189in}}%
\pgfpathlineto{\pgfqpoint{1.050189in}{0.961980in}}%
\pgfpathclose%
\pgfusepath{stroke,fill}%
\end{pgfscope}%
\begin{pgfscope}%
\pgfpathrectangle{\pgfqpoint{0.150000in}{0.150000in}}{\pgfqpoint{1.800000in}{1.800000in}}%
\pgfusepath{clip}%
\pgfsetbuttcap%
\pgfsetroundjoin%
\definecolor{currentfill}{rgb}{0.933333,0.600000,0.666667}%
\pgfsetfillcolor{currentfill}%
\pgfsetlinewidth{1.003750pt}%
\definecolor{currentstroke}{rgb}{0.600000,0.266667,0.333333}%
\pgfsetstrokecolor{currentstroke}%
\pgfsetdash{}{0pt}%
\pgfpathmoveto{\pgfqpoint{0.873771in}{0.801600in}}%
\pgfpathlineto{\pgfqpoint{0.961980in}{0.801600in}}%
\pgfpathlineto{\pgfqpoint{0.961980in}{0.873771in}}%
\pgfpathlineto{\pgfqpoint{0.873771in}{0.873771in}}%
\pgfpathlineto{\pgfqpoint{0.873771in}{0.801600in}}%
\pgfpathclose%
\pgfusepath{stroke,fill}%
\end{pgfscope}%
\begin{pgfscope}%
\pgfpathrectangle{\pgfqpoint{0.150000in}{0.150000in}}{\pgfqpoint{1.800000in}{1.800000in}}%
\pgfusepath{clip}%
\pgfsetbuttcap%
\pgfsetroundjoin%
\definecolor{currentfill}{rgb}{0.933333,0.600000,0.666667}%
\pgfsetfillcolor{currentfill}%
\pgfsetlinewidth{1.003750pt}%
\definecolor{currentstroke}{rgb}{0.600000,0.266667,0.333333}%
\pgfsetstrokecolor{currentstroke}%
\pgfsetdash{}{0pt}%
\pgfpathmoveto{\pgfqpoint{0.510000in}{0.945150in}}%
\pgfpathlineto{\pgfqpoint{0.558173in}{0.945150in}}%
\pgfpathlineto{\pgfqpoint{0.558173in}{1.085483in}}%
\pgfpathlineto{\pgfqpoint{0.510000in}{1.085483in}}%
\pgfpathlineto{\pgfqpoint{0.510000in}{0.945150in}}%
\pgfpathclose%
\pgfusepath{stroke,fill}%
\end{pgfscope}%
\begin{pgfscope}%
\pgfpathrectangle{\pgfqpoint{0.150000in}{0.150000in}}{\pgfqpoint{1.800000in}{1.800000in}}%
\pgfusepath{clip}%
\pgfsetbuttcap%
\pgfsetroundjoin%
\definecolor{currentfill}{rgb}{0.933333,0.600000,0.666667}%
\pgfsetfillcolor{currentfill}%
\pgfsetlinewidth{1.003750pt}%
\definecolor{currentstroke}{rgb}{0.600000,0.266667,0.333333}%
\pgfsetstrokecolor{currentstroke}%
\pgfsetdash{}{0pt}%
\pgfpathmoveto{\pgfqpoint{0.604950in}{0.690000in}}%
\pgfpathlineto{\pgfqpoint{0.651728in}{0.690000in}}%
\pgfpathlineto{\pgfqpoint{0.651728in}{0.804818in}}%
\pgfpathlineto{\pgfqpoint{0.604950in}{0.804818in}}%
\pgfpathlineto{\pgfqpoint{0.604950in}{0.690000in}}%
\pgfpathclose%
\pgfusepath{stroke,fill}%
\end{pgfscope}%
\begin{pgfscope}%
\pgfpathrectangle{\pgfqpoint{0.150000in}{0.150000in}}{\pgfqpoint{1.800000in}{1.800000in}}%
\pgfusepath{clip}%
\pgfsetbuttcap%
\pgfsetroundjoin%
\definecolor{currentfill}{rgb}{0.933333,0.600000,0.666667}%
\pgfsetfillcolor{currentfill}%
\pgfsetlinewidth{1.003750pt}%
\definecolor{currentstroke}{rgb}{0.600000,0.266667,0.333333}%
\pgfsetstrokecolor{currentstroke}%
\pgfsetdash{}{0pt}%
\pgfpathmoveto{\pgfqpoint{1.818231in}{1.710420in}}%
\pgfpathlineto{\pgfqpoint{1.950000in}{1.710420in}}%
\pgfpathlineto{\pgfqpoint{1.950000in}{1.818231in}}%
\pgfpathlineto{\pgfqpoint{1.818231in}{1.818231in}}%
\pgfpathlineto{\pgfqpoint{1.818231in}{1.710420in}}%
\pgfpathclose%
\pgfusepath{stroke,fill}%
\end{pgfscope}%
\begin{pgfscope}%
\pgfpathrectangle{\pgfqpoint{0.150000in}{0.150000in}}{\pgfqpoint{1.800000in}{1.800000in}}%
\pgfusepath{clip}%
\pgfsetbuttcap%
\pgfsetroundjoin%
\definecolor{currentfill}{rgb}{0.933333,0.600000,0.666667}%
\pgfsetfillcolor{currentfill}%
\pgfsetlinewidth{1.003750pt}%
\definecolor{currentstroke}{rgb}{0.600000,0.266667,0.333333}%
\pgfsetstrokecolor{currentstroke}%
\pgfsetdash{}{0pt}%
\pgfpathmoveto{\pgfqpoint{1.602609in}{1.514400in}}%
\pgfpathlineto{\pgfqpoint{1.710420in}{1.514400in}}%
\pgfpathlineto{\pgfqpoint{1.710420in}{1.602609in}}%
\pgfpathlineto{\pgfqpoint{1.602609in}{1.602609in}}%
\pgfpathlineto{\pgfqpoint{1.602609in}{1.514400in}}%
\pgfpathclose%
\pgfusepath{stroke,fill}%
\end{pgfscope}%
\begin{pgfscope}%
\pgfpathrectangle{\pgfqpoint{0.150000in}{0.150000in}}{\pgfqpoint{1.800000in}{1.800000in}}%
\pgfusepath{clip}%
\pgfsetbuttcap%
\pgfsetroundjoin%
\definecolor{currentfill}{rgb}{0.933333,0.600000,0.666667}%
\pgfsetfillcolor{currentfill}%
\pgfsetlinewidth{1.003750pt}%
\definecolor{currentstroke}{rgb}{0.600000,0.266667,0.333333}%
\pgfsetstrokecolor{currentstroke}%
\pgfsetdash{}{0pt}%
\pgfpathmoveto{\pgfqpoint{1.406589in}{1.318380in}}%
\pgfpathlineto{\pgfqpoint{1.514400in}{1.318380in}}%
\pgfpathlineto{\pgfqpoint{1.514400in}{1.406589in}}%
\pgfpathlineto{\pgfqpoint{1.406589in}{1.406589in}}%
\pgfpathlineto{\pgfqpoint{1.406589in}{1.318380in}}%
\pgfpathclose%
\pgfusepath{stroke,fill}%
\end{pgfscope}%
\begin{pgfscope}%
\pgfpathrectangle{\pgfqpoint{0.150000in}{0.150000in}}{\pgfqpoint{1.800000in}{1.800000in}}%
\pgfusepath{clip}%
\pgfsetbuttcap%
\pgfsetroundjoin%
\definecolor{currentfill}{rgb}{0.933333,0.600000,0.666667}%
\pgfsetfillcolor{currentfill}%
\pgfsetlinewidth{1.003750pt}%
\definecolor{currentstroke}{rgb}{0.600000,0.266667,0.333333}%
\pgfsetstrokecolor{currentstroke}%
\pgfsetdash{}{0pt}%
\pgfpathmoveto{\pgfqpoint{1.230171in}{1.158000in}}%
\pgfpathlineto{\pgfqpoint{1.318380in}{1.158000in}}%
\pgfpathlineto{\pgfqpoint{1.318380in}{1.230171in}}%
\pgfpathlineto{\pgfqpoint{1.230171in}{1.230171in}}%
\pgfpathlineto{\pgfqpoint{1.230171in}{1.158000in}}%
\pgfpathclose%
\pgfusepath{stroke,fill}%
\end{pgfscope}%
\begin{pgfscope}%
\pgfpathrectangle{\pgfqpoint{0.150000in}{0.150000in}}{\pgfqpoint{1.800000in}{1.800000in}}%
\pgfusepath{clip}%
\pgfsetbuttcap%
\pgfsetroundjoin%
\definecolor{currentfill}{rgb}{0.933333,0.600000,0.666667}%
\pgfsetfillcolor{currentfill}%
\pgfsetlinewidth{1.003750pt}%
\definecolor{currentstroke}{rgb}{0.600000,0.266667,0.333333}%
\pgfsetstrokecolor{currentstroke}%
\pgfsetdash{}{0pt}%
\pgfpathmoveto{\pgfqpoint{0.577770in}{1.740368in}}%
\pgfpathlineto{\pgfqpoint{0.612074in}{1.740368in}}%
\pgfpathlineto{\pgfqpoint{0.612074in}{1.950000in}}%
\pgfpathlineto{\pgfqpoint{0.577770in}{1.950000in}}%
\pgfpathlineto{\pgfqpoint{0.577770in}{1.740368in}}%
\pgfpathclose%
\pgfusepath{stroke,fill}%
\end{pgfscope}%
\begin{pgfscope}%
\pgfpathrectangle{\pgfqpoint{0.150000in}{0.150000in}}{\pgfqpoint{1.800000in}{1.800000in}}%
\pgfusepath{clip}%
\pgfsetbuttcap%
\pgfsetroundjoin%
\definecolor{currentfill}{rgb}{0.933333,0.600000,0.666667}%
\pgfsetfillcolor{currentfill}%
\pgfsetlinewidth{1.003750pt}%
\definecolor{currentstroke}{rgb}{0.600000,0.266667,0.333333}%
\pgfsetstrokecolor{currentstroke}%
\pgfsetdash{}{0pt}%
\pgfpathmoveto{\pgfqpoint{0.515400in}{1.397333in}}%
\pgfpathlineto{\pgfqpoint{0.543467in}{1.397333in}}%
\pgfpathlineto{\pgfqpoint{0.543467in}{1.568850in}}%
\pgfpathlineto{\pgfqpoint{0.515400in}{1.568850in}}%
\pgfpathlineto{\pgfqpoint{0.515400in}{1.397333in}}%
\pgfpathclose%
\pgfusepath{stroke,fill}%
\end{pgfscope}%
\begin{pgfscope}%
\pgfpathrectangle{\pgfqpoint{0.150000in}{0.150000in}}{\pgfqpoint{1.800000in}{1.800000in}}%
\pgfusepath{clip}%
\pgfsetbuttcap%
\pgfsetroundjoin%
\definecolor{currentfill}{rgb}{0.933333,0.600000,0.666667}%
\pgfsetfillcolor{currentfill}%
\pgfsetlinewidth{1.003750pt}%
\definecolor{currentstroke}{rgb}{0.600000,0.266667,0.333333}%
\pgfsetstrokecolor{currentstroke}%
\pgfsetdash{}{0pt}%
\pgfpathmoveto{\pgfqpoint{0.961980in}{0.801600in}}%
\pgfpathlineto{\pgfqpoint{1.158000in}{0.801600in}}%
\pgfpathlineto{\pgfqpoint{1.158000in}{0.961980in}}%
\pgfpathlineto{\pgfqpoint{0.961980in}{0.961980in}}%
\pgfpathlineto{\pgfqpoint{0.961980in}{0.801600in}}%
\pgfpathclose%
\pgfusepath{stroke,fill}%
\end{pgfscope}%
\begin{pgfscope}%
\pgfpathrectangle{\pgfqpoint{0.150000in}{0.150000in}}{\pgfqpoint{1.800000in}{1.800000in}}%
\pgfusepath{clip}%
\pgfsetbuttcap%
\pgfsetroundjoin%
\definecolor{currentfill}{rgb}{0.933333,0.600000,0.666667}%
\pgfsetfillcolor{currentfill}%
\pgfsetlinewidth{1.003750pt}%
\definecolor{currentstroke}{rgb}{0.600000,0.266667,0.333333}%
\pgfsetstrokecolor{currentstroke}%
\pgfsetdash{}{0pt}%
\pgfpathmoveto{\pgfqpoint{0.510000in}{0.690000in}}%
\pgfpathlineto{\pgfqpoint{0.604950in}{0.690000in}}%
\pgfpathlineto{\pgfqpoint{0.604950in}{0.945150in}}%
\pgfpathlineto{\pgfqpoint{0.510000in}{0.945150in}}%
\pgfpathlineto{\pgfqpoint{0.510000in}{0.690000in}}%
\pgfpathclose%
\pgfusepath{stroke,fill}%
\end{pgfscope}%
\begin{pgfscope}%
\pgfpathrectangle{\pgfqpoint{0.150000in}{0.150000in}}{\pgfqpoint{1.800000in}{1.800000in}}%
\pgfusepath{clip}%
\pgfsetbuttcap%
\pgfsetroundjoin%
\definecolor{currentfill}{rgb}{0.933333,0.600000,0.666667}%
\pgfsetfillcolor{currentfill}%
\pgfsetlinewidth{1.003750pt}%
\definecolor{currentstroke}{rgb}{0.600000,0.266667,0.333333}%
\pgfsetstrokecolor{currentstroke}%
\pgfsetdash{}{0pt}%
\pgfpathmoveto{\pgfqpoint{1.710420in}{1.514400in}}%
\pgfpathlineto{\pgfqpoint{1.950000in}{1.514400in}}%
\pgfpathlineto{\pgfqpoint{1.950000in}{1.710420in}}%
\pgfpathlineto{\pgfqpoint{1.710420in}{1.710420in}}%
\pgfpathlineto{\pgfqpoint{1.710420in}{1.514400in}}%
\pgfpathclose%
\pgfusepath{stroke,fill}%
\end{pgfscope}%
\begin{pgfscope}%
\pgfpathrectangle{\pgfqpoint{0.150000in}{0.150000in}}{\pgfqpoint{1.800000in}{1.800000in}}%
\pgfusepath{clip}%
\pgfsetbuttcap%
\pgfsetroundjoin%
\definecolor{currentfill}{rgb}{0.933333,0.600000,0.666667}%
\pgfsetfillcolor{currentfill}%
\pgfsetlinewidth{1.003750pt}%
\definecolor{currentstroke}{rgb}{0.600000,0.266667,0.333333}%
\pgfsetstrokecolor{currentstroke}%
\pgfsetdash{}{0pt}%
\pgfpathmoveto{\pgfqpoint{1.318380in}{1.158000in}}%
\pgfpathlineto{\pgfqpoint{1.514400in}{1.158000in}}%
\pgfpathlineto{\pgfqpoint{1.514400in}{1.318380in}}%
\pgfpathlineto{\pgfqpoint{1.318380in}{1.318380in}}%
\pgfpathlineto{\pgfqpoint{1.318380in}{1.158000in}}%
\pgfpathclose%
\pgfusepath{stroke,fill}%
\end{pgfscope}%
\begin{pgfscope}%
\pgfpathrectangle{\pgfqpoint{0.150000in}{0.150000in}}{\pgfqpoint{1.800000in}{1.800000in}}%
\pgfusepath{clip}%
\pgfsetbuttcap%
\pgfsetroundjoin%
\definecolor{currentfill}{rgb}{0.933333,0.600000,0.666667}%
\pgfsetfillcolor{currentfill}%
\pgfsetlinewidth{1.003750pt}%
\definecolor{currentstroke}{rgb}{0.600000,0.266667,0.333333}%
\pgfsetstrokecolor{currentstroke}%
\pgfsetdash{}{0pt}%
\pgfpathmoveto{\pgfqpoint{0.515400in}{1.568850in}}%
\pgfpathlineto{\pgfqpoint{0.577770in}{1.568850in}}%
\pgfpathlineto{\pgfqpoint{0.577770in}{1.950000in}}%
\pgfpathlineto{\pgfqpoint{0.515400in}{1.950000in}}%
\pgfpathlineto{\pgfqpoint{0.515400in}{1.568850in}}%
\pgfpathclose%
\pgfusepath{stroke,fill}%
\end{pgfscope}%
\begin{pgfscope}%
\pgfpathrectangle{\pgfqpoint{0.150000in}{0.150000in}}{\pgfqpoint{1.800000in}{1.800000in}}%
\pgfusepath{clip}%
\pgfsetbuttcap%
\pgfsetroundjoin%
\definecolor{currentfill}{rgb}{0.933333,0.600000,0.666667}%
\pgfsetfillcolor{currentfill}%
\pgfsetlinewidth{1.003750pt}%
\definecolor{currentstroke}{rgb}{0.600000,0.266667,0.333333}%
\pgfsetstrokecolor{currentstroke}%
\pgfsetdash{}{0pt}%
\pgfpathmoveto{\pgfqpoint{0.801600in}{0.690000in}}%
\pgfpathlineto{\pgfqpoint{1.158000in}{0.690000in}}%
\pgfpathlineto{\pgfqpoint{1.158000in}{0.801600in}}%
\pgfpathlineto{\pgfqpoint{0.801600in}{0.801600in}}%
\pgfpathlineto{\pgfqpoint{0.801600in}{0.690000in}}%
\pgfpathclose%
\pgfusepath{stroke,fill}%
\end{pgfscope}%
\begin{pgfscope}%
\pgfpathrectangle{\pgfqpoint{0.150000in}{0.150000in}}{\pgfqpoint{1.800000in}{1.800000in}}%
\pgfusepath{clip}%
\pgfsetbuttcap%
\pgfsetroundjoin%
\definecolor{currentfill}{rgb}{0.933333,0.600000,0.666667}%
\pgfsetfillcolor{currentfill}%
\pgfsetlinewidth{1.003750pt}%
\definecolor{currentstroke}{rgb}{0.600000,0.266667,0.333333}%
\pgfsetstrokecolor{currentstroke}%
\pgfsetdash{}{0pt}%
\pgfpathmoveto{\pgfqpoint{1.514400in}{1.158000in}}%
\pgfpathlineto{\pgfqpoint{1.950000in}{1.158000in}}%
\pgfpathlineto{\pgfqpoint{1.950000in}{1.514400in}}%
\pgfpathlineto{\pgfqpoint{1.514400in}{1.514400in}}%
\pgfpathlineto{\pgfqpoint{1.514400in}{1.158000in}}%
\pgfpathclose%
\pgfusepath{stroke,fill}%
\end{pgfscope}%
\begin{pgfscope}%
\pgfpathrectangle{\pgfqpoint{0.150000in}{0.150000in}}{\pgfqpoint{1.800000in}{1.800000in}}%
\pgfusepath{clip}%
\pgfsetbuttcap%
\pgfsetroundjoin%
\definecolor{currentfill}{rgb}{0.933333,0.600000,0.666667}%
\pgfsetfillcolor{currentfill}%
\pgfsetlinewidth{1.003750pt}%
\definecolor{currentstroke}{rgb}{0.600000,0.266667,0.333333}%
\pgfsetstrokecolor{currentstroke}%
\pgfsetdash{}{0pt}%
\pgfpathmoveto{\pgfqpoint{0.510000in}{1.257000in}}%
\pgfpathlineto{\pgfqpoint{0.515400in}{1.257000in}}%
\pgfpathlineto{\pgfqpoint{0.515400in}{1.950000in}}%
\pgfpathlineto{\pgfqpoint{0.510000in}{1.950000in}}%
\pgfpathlineto{\pgfqpoint{0.510000in}{1.257000in}}%
\pgfpathclose%
\pgfusepath{stroke,fill}%
\end{pgfscope}%
\begin{pgfscope}%
\pgfpathrectangle{\pgfqpoint{0.150000in}{0.150000in}}{\pgfqpoint{1.800000in}{1.800000in}}%
\pgfusepath{clip}%
\pgfsetbuttcap%
\pgfsetroundjoin%
\definecolor{currentfill}{rgb}{0.933333,0.600000,0.666667}%
\pgfsetfillcolor{currentfill}%
\pgfsetlinewidth{1.003750pt}%
\definecolor{currentstroke}{rgb}{0.600000,0.266667,0.333333}%
\pgfsetstrokecolor{currentstroke}%
\pgfsetdash{}{0pt}%
\pgfpathmoveto{\pgfqpoint{1.158000in}{0.690000in}}%
\pgfpathlineto{\pgfqpoint{1.950000in}{0.690000in}}%
\pgfpathlineto{\pgfqpoint{1.950000in}{1.158000in}}%
\pgfpathlineto{\pgfqpoint{1.158000in}{1.158000in}}%
\pgfpathlineto{\pgfqpoint{1.158000in}{0.690000in}}%
\pgfpathclose%
\pgfusepath{stroke,fill}%
\end{pgfscope}%
\begin{pgfscope}%
\pgfpathrectangle{\pgfqpoint{0.150000in}{0.150000in}}{\pgfqpoint{1.800000in}{1.800000in}}%
\pgfusepath{clip}%
\pgfsetbuttcap%
\pgfsetroundjoin%
\definecolor{currentfill}{rgb}{0.933333,0.600000,0.666667}%
\pgfsetfillcolor{currentfill}%
\pgfsetlinewidth{1.003750pt}%
\definecolor{currentstroke}{rgb}{0.600000,0.266667,0.333333}%
\pgfsetstrokecolor{currentstroke}%
\pgfsetdash{}{0pt}%
\pgfpathmoveto{\pgfqpoint{0.510000in}{0.150000in}}%
\pgfpathlineto{\pgfqpoint{1.950000in}{0.150000in}}%
\pgfpathlineto{\pgfqpoint{1.950000in}{0.690000in}}%
\pgfpathlineto{\pgfqpoint{0.510000in}{0.690000in}}%
\pgfpathlineto{\pgfqpoint{0.510000in}{0.150000in}}%
\pgfpathclose%
\pgfusepath{stroke,fill}%
\end{pgfscope}%
\begin{pgfscope}%
\pgfpathrectangle{\pgfqpoint{0.150000in}{0.150000in}}{\pgfqpoint{1.800000in}{1.800000in}}%
\pgfusepath{clip}%
\pgfsetbuttcap%
\pgfsetroundjoin%
\definecolor{currentfill}{rgb}{0.933333,0.600000,0.666667}%
\pgfsetfillcolor{currentfill}%
\pgfsetlinewidth{1.003750pt}%
\definecolor{currentstroke}{rgb}{0.600000,0.266667,0.333333}%
\pgfsetstrokecolor{currentstroke}%
\pgfsetdash{}{0pt}%
\pgfpathmoveto{\pgfqpoint{0.150000in}{0.150000in}}%
\pgfpathlineto{\pgfqpoint{0.510000in}{0.150000in}}%
\pgfpathlineto{\pgfqpoint{0.510000in}{1.950000in}}%
\pgfpathlineto{\pgfqpoint{0.150000in}{1.950000in}}%
\pgfpathlineto{\pgfqpoint{0.150000in}{0.150000in}}%
\pgfpathclose%
\pgfusepath{stroke,fill}%
\end{pgfscope}%
\begin{pgfscope}%
\pgfpathrectangle{\pgfqpoint{0.150000in}{0.150000in}}{\pgfqpoint{1.800000in}{1.800000in}}%
\pgfusepath{clip}%
\pgfsetbuttcap%
\pgfsetroundjoin%
\definecolor{currentfill}{rgb}{0.400000,0.600000,0.800000}%
\pgfsetfillcolor{currentfill}%
\pgfsetlinewidth{1.003750pt}%
\definecolor{currentstroke}{rgb}{0.000000,0.266667,0.533333}%
\pgfsetstrokecolor{currentstroke}%
\pgfsetdash{}{0pt}%
\pgfpathmoveto{\pgfqpoint{0.518295in}{1.205116in}}%
\pgfpathlineto{\pgfqpoint{0.532445in}{1.205116in}}%
\pgfpathlineto{\pgfqpoint{0.532445in}{1.257000in}}%
\pgfpathlineto{\pgfqpoint{0.518295in}{1.257000in}}%
\pgfpathlineto{\pgfqpoint{0.518295in}{1.205116in}}%
\pgfpathclose%
\pgfusepath{stroke,fill}%
\end{pgfscope}%
\begin{pgfscope}%
\pgfpathrectangle{\pgfqpoint{0.150000in}{0.150000in}}{\pgfqpoint{1.800000in}{1.800000in}}%
\pgfusepath{clip}%
\pgfsetbuttcap%
\pgfsetroundjoin%
\definecolor{currentfill}{rgb}{0.400000,0.600000,0.800000}%
\pgfsetfillcolor{currentfill}%
\pgfsetlinewidth{1.003750pt}%
\definecolor{currentstroke}{rgb}{0.000000,0.266667,0.533333}%
\pgfsetstrokecolor{currentstroke}%
\pgfsetdash{}{0pt}%
\pgfpathmoveto{\pgfqpoint{0.672777in}{0.741668in}}%
\pgfpathlineto{\pgfqpoint{0.719170in}{0.741668in}}%
\pgfpathlineto{\pgfqpoint{0.719170in}{0.804818in}}%
\pgfpathlineto{\pgfqpoint{0.672777in}{0.804818in}}%
\pgfpathlineto{\pgfqpoint{0.672777in}{0.741668in}}%
\pgfpathclose%
\pgfusepath{stroke,fill}%
\end{pgfscope}%
\begin{pgfscope}%
\pgfpathrectangle{\pgfqpoint{0.150000in}{0.150000in}}{\pgfqpoint{1.800000in}{1.800000in}}%
\pgfusepath{clip}%
\pgfsetbuttcap%
\pgfsetroundjoin%
\definecolor{currentfill}{rgb}{0.400000,0.600000,0.800000}%
\pgfsetfillcolor{currentfill}%
\pgfsetlinewidth{1.003750pt}%
\definecolor{currentstroke}{rgb}{0.000000,0.266667,0.533333}%
\pgfsetstrokecolor{currentstroke}%
\pgfsetdash{}{0pt}%
\pgfpathmoveto{\pgfqpoint{0.641317in}{1.834702in}}%
\pgfpathlineto{\pgfqpoint{0.654000in}{1.834702in}}%
\pgfpathlineto{\pgfqpoint{0.654000in}{1.886586in}}%
\pgfpathlineto{\pgfqpoint{0.641317in}{1.886586in}}%
\pgfpathlineto{\pgfqpoint{0.641317in}{1.834702in}}%
\pgfpathclose%
\pgfusepath{stroke,fill}%
\end{pgfscope}%
\begin{pgfscope}%
\pgfpathrectangle{\pgfqpoint{0.150000in}{0.150000in}}{\pgfqpoint{1.800000in}{1.800000in}}%
\pgfusepath{clip}%
\pgfsetbuttcap%
\pgfsetroundjoin%
\definecolor{currentfill}{rgb}{0.400000,0.600000,0.800000}%
\pgfsetfillcolor{currentfill}%
\pgfsetlinewidth{1.003750pt}%
\definecolor{currentstroke}{rgb}{0.000000,0.266667,0.533333}%
\pgfsetstrokecolor{currentstroke}%
\pgfsetdash{}{0pt}%
\pgfpathmoveto{\pgfqpoint{0.620564in}{1.740368in}}%
\pgfpathlineto{\pgfqpoint{0.630940in}{1.740368in}}%
\pgfpathlineto{\pgfqpoint{0.630940in}{1.782818in}}%
\pgfpathlineto{\pgfqpoint{0.620564in}{1.782818in}}%
\pgfpathlineto{\pgfqpoint{0.620564in}{1.740368in}}%
\pgfpathclose%
\pgfusepath{stroke,fill}%
\end{pgfscope}%
\begin{pgfscope}%
\pgfpathrectangle{\pgfqpoint{0.150000in}{0.150000in}}{\pgfqpoint{1.800000in}{1.800000in}}%
\pgfusepath{clip}%
\pgfsetbuttcap%
\pgfsetroundjoin%
\definecolor{currentfill}{rgb}{0.400000,0.600000,0.800000}%
\pgfsetfillcolor{currentfill}%
\pgfsetlinewidth{1.003750pt}%
\definecolor{currentstroke}{rgb}{0.000000,0.266667,0.533333}%
\pgfsetstrokecolor{currentstroke}%
\pgfsetdash{}{0pt}%
\pgfpathmoveto{\pgfqpoint{0.601697in}{1.646033in}}%
\pgfpathlineto{\pgfqpoint{0.612074in}{1.646033in}}%
\pgfpathlineto{\pgfqpoint{0.612074in}{1.688483in}}%
\pgfpathlineto{\pgfqpoint{0.601697in}{1.688483in}}%
\pgfpathlineto{\pgfqpoint{0.601697in}{1.646033in}}%
\pgfpathclose%
\pgfusepath{stroke,fill}%
\end{pgfscope}%
\begin{pgfscope}%
\pgfpathrectangle{\pgfqpoint{0.150000in}{0.150000in}}{\pgfqpoint{1.800000in}{1.800000in}}%
\pgfusepath{clip}%
\pgfsetbuttcap%
\pgfsetroundjoin%
\definecolor{currentfill}{rgb}{0.400000,0.600000,0.800000}%
\pgfsetfillcolor{currentfill}%
\pgfsetlinewidth{1.003750pt}%
\definecolor{currentstroke}{rgb}{0.000000,0.266667,0.533333}%
\pgfsetstrokecolor{currentstroke}%
\pgfsetdash{}{0pt}%
\pgfpathmoveto{\pgfqpoint{0.567393in}{1.474515in}}%
\pgfpathlineto{\pgfqpoint{0.577770in}{1.474515in}}%
\pgfpathlineto{\pgfqpoint{0.577770in}{1.516966in}}%
\pgfpathlineto{\pgfqpoint{0.567393in}{1.516966in}}%
\pgfpathlineto{\pgfqpoint{0.567393in}{1.474515in}}%
\pgfpathclose%
\pgfusepath{stroke,fill}%
\end{pgfscope}%
\begin{pgfscope}%
\pgfpathrectangle{\pgfqpoint{0.150000in}{0.150000in}}{\pgfqpoint{1.800000in}{1.800000in}}%
\pgfusepath{clip}%
\pgfsetbuttcap%
\pgfsetroundjoin%
\definecolor{currentfill}{rgb}{0.400000,0.600000,0.800000}%
\pgfsetfillcolor{currentfill}%
\pgfsetlinewidth{1.003750pt}%
\definecolor{currentstroke}{rgb}{0.000000,0.266667,0.533333}%
\pgfsetstrokecolor{currentstroke}%
\pgfsetdash{}{0pt}%
\pgfpathmoveto{\pgfqpoint{1.050189in}{1.098704in}}%
\pgfpathlineto{\pgfqpoint{1.098704in}{1.098704in}}%
\pgfpathlineto{\pgfqpoint{1.098704in}{1.158000in}}%
\pgfpathlineto{\pgfqpoint{1.050189in}{1.158000in}}%
\pgfpathlineto{\pgfqpoint{1.050189in}{1.098704in}}%
\pgfpathclose%
\pgfusepath{stroke,fill}%
\end{pgfscope}%
\begin{pgfscope}%
\pgfpathrectangle{\pgfqpoint{0.150000in}{0.150000in}}{\pgfqpoint{1.800000in}{1.800000in}}%
\pgfusepath{clip}%
\pgfsetbuttcap%
\pgfsetroundjoin%
\definecolor{currentfill}{rgb}{0.400000,0.600000,0.800000}%
\pgfsetfillcolor{currentfill}%
\pgfsetlinewidth{1.003750pt}%
\definecolor{currentstroke}{rgb}{0.000000,0.266667,0.533333}%
\pgfsetstrokecolor{currentstroke}%
\pgfsetdash{}{0pt}%
\pgfpathmoveto{\pgfqpoint{0.532445in}{1.162665in}}%
\pgfpathlineto{\pgfqpoint{0.558173in}{1.162665in}}%
\pgfpathlineto{\pgfqpoint{0.558173in}{1.257000in}}%
\pgfpathlineto{\pgfqpoint{0.532445in}{1.257000in}}%
\pgfpathlineto{\pgfqpoint{0.532445in}{1.162665in}}%
\pgfpathclose%
\pgfusepath{stroke,fill}%
\end{pgfscope}%
\begin{pgfscope}%
\pgfpathrectangle{\pgfqpoint{0.150000in}{0.150000in}}{\pgfqpoint{1.800000in}{1.800000in}}%
\pgfusepath{clip}%
\pgfsetbuttcap%
\pgfsetroundjoin%
\definecolor{currentfill}{rgb}{0.400000,0.600000,0.800000}%
\pgfsetfillcolor{currentfill}%
\pgfsetlinewidth{1.003750pt}%
\definecolor{currentstroke}{rgb}{0.000000,0.266667,0.533333}%
\pgfsetstrokecolor{currentstroke}%
\pgfsetdash{}{0pt}%
\pgfpathmoveto{\pgfqpoint{0.583900in}{1.008300in}}%
\pgfpathlineto{\pgfqpoint{0.604950in}{1.008300in}}%
\pgfpathlineto{\pgfqpoint{0.604950in}{1.085483in}}%
\pgfpathlineto{\pgfqpoint{0.583900in}{1.085483in}}%
\pgfpathlineto{\pgfqpoint{0.583900in}{1.008300in}}%
\pgfpathclose%
\pgfusepath{stroke,fill}%
\end{pgfscope}%
\begin{pgfscope}%
\pgfpathrectangle{\pgfqpoint{0.150000in}{0.150000in}}{\pgfqpoint{1.800000in}{1.800000in}}%
\pgfusepath{clip}%
\pgfsetbuttcap%
\pgfsetroundjoin%
\definecolor{currentfill}{rgb}{0.400000,0.600000,0.800000}%
\pgfsetfillcolor{currentfill}%
\pgfsetlinewidth{1.003750pt}%
\definecolor{currentstroke}{rgb}{0.000000,0.266667,0.533333}%
\pgfsetstrokecolor{currentstroke}%
\pgfsetdash{}{0pt}%
\pgfpathmoveto{\pgfqpoint{0.630678in}{0.867967in}}%
\pgfpathlineto{\pgfqpoint{0.651728in}{0.867967in}}%
\pgfpathlineto{\pgfqpoint{0.651728in}{0.945150in}}%
\pgfpathlineto{\pgfqpoint{0.630678in}{0.945150in}}%
\pgfpathlineto{\pgfqpoint{0.630678in}{0.867967in}}%
\pgfpathclose%
\pgfusepath{stroke,fill}%
\end{pgfscope}%
\begin{pgfscope}%
\pgfpathrectangle{\pgfqpoint{0.150000in}{0.150000in}}{\pgfqpoint{1.800000in}{1.800000in}}%
\pgfusepath{clip}%
\pgfsetbuttcap%
\pgfsetroundjoin%
\definecolor{currentfill}{rgb}{0.400000,0.600000,0.800000}%
\pgfsetfillcolor{currentfill}%
\pgfsetlinewidth{1.003750pt}%
\definecolor{currentstroke}{rgb}{0.000000,0.266667,0.533333}%
\pgfsetstrokecolor{currentstroke}%
\pgfsetdash{}{0pt}%
\pgfpathmoveto{\pgfqpoint{0.719170in}{0.801600in}}%
\pgfpathlineto{\pgfqpoint{0.801600in}{0.801600in}}%
\pgfpathlineto{\pgfqpoint{0.801600in}{0.804818in}}%
\pgfpathlineto{\pgfqpoint{0.719170in}{0.804818in}}%
\pgfpathlineto{\pgfqpoint{0.719170in}{0.801600in}}%
\pgfpathclose%
\pgfusepath{stroke,fill}%
\end{pgfscope}%
\begin{pgfscope}%
\pgfpathrectangle{\pgfqpoint{0.150000in}{0.150000in}}{\pgfqpoint{1.800000in}{1.800000in}}%
\pgfusepath{clip}%
\pgfsetbuttcap%
\pgfsetroundjoin%
\definecolor{currentfill}{rgb}{0.400000,0.600000,0.800000}%
\pgfsetfillcolor{currentfill}%
\pgfsetlinewidth{1.003750pt}%
\definecolor{currentstroke}{rgb}{0.000000,0.266667,0.533333}%
\pgfsetstrokecolor{currentstroke}%
\pgfsetdash{}{0pt}%
\pgfpathmoveto{\pgfqpoint{1.818231in}{1.877527in}}%
\pgfpathlineto{\pgfqpoint{1.877527in}{1.877527in}}%
\pgfpathlineto{\pgfqpoint{1.877527in}{1.950000in}}%
\pgfpathlineto{\pgfqpoint{1.818231in}{1.950000in}}%
\pgfpathlineto{\pgfqpoint{1.818231in}{1.877527in}}%
\pgfpathclose%
\pgfusepath{stroke,fill}%
\end{pgfscope}%
\begin{pgfscope}%
\pgfpathrectangle{\pgfqpoint{0.150000in}{0.150000in}}{\pgfqpoint{1.800000in}{1.800000in}}%
\pgfusepath{clip}%
\pgfsetbuttcap%
\pgfsetroundjoin%
\definecolor{currentfill}{rgb}{0.400000,0.600000,0.800000}%
\pgfsetfillcolor{currentfill}%
\pgfsetlinewidth{1.003750pt}%
\definecolor{currentstroke}{rgb}{0.000000,0.266667,0.533333}%
\pgfsetstrokecolor{currentstroke}%
\pgfsetdash{}{0pt}%
\pgfpathmoveto{\pgfqpoint{1.710420in}{1.758935in}}%
\pgfpathlineto{\pgfqpoint{1.758935in}{1.758935in}}%
\pgfpathlineto{\pgfqpoint{1.758935in}{1.818231in}}%
\pgfpathlineto{\pgfqpoint{1.710420in}{1.818231in}}%
\pgfpathlineto{\pgfqpoint{1.710420in}{1.758935in}}%
\pgfpathclose%
\pgfusepath{stroke,fill}%
\end{pgfscope}%
\begin{pgfscope}%
\pgfpathrectangle{\pgfqpoint{0.150000in}{0.150000in}}{\pgfqpoint{1.800000in}{1.800000in}}%
\pgfusepath{clip}%
\pgfsetbuttcap%
\pgfsetroundjoin%
\definecolor{currentfill}{rgb}{0.400000,0.600000,0.800000}%
\pgfsetfillcolor{currentfill}%
\pgfsetlinewidth{1.003750pt}%
\definecolor{currentstroke}{rgb}{0.000000,0.266667,0.533333}%
\pgfsetstrokecolor{currentstroke}%
\pgfsetdash{}{0pt}%
\pgfpathmoveto{\pgfqpoint{1.602609in}{1.651124in}}%
\pgfpathlineto{\pgfqpoint{1.651124in}{1.651124in}}%
\pgfpathlineto{\pgfqpoint{1.651124in}{1.710420in}}%
\pgfpathlineto{\pgfqpoint{1.602609in}{1.710420in}}%
\pgfpathlineto{\pgfqpoint{1.602609in}{1.651124in}}%
\pgfpathclose%
\pgfusepath{stroke,fill}%
\end{pgfscope}%
\begin{pgfscope}%
\pgfpathrectangle{\pgfqpoint{0.150000in}{0.150000in}}{\pgfqpoint{1.800000in}{1.800000in}}%
\pgfusepath{clip}%
\pgfsetbuttcap%
\pgfsetroundjoin%
\definecolor{currentfill}{rgb}{0.400000,0.600000,0.800000}%
\pgfsetfillcolor{currentfill}%
\pgfsetlinewidth{1.003750pt}%
\definecolor{currentstroke}{rgb}{0.000000,0.266667,0.533333}%
\pgfsetstrokecolor{currentstroke}%
\pgfsetdash{}{0pt}%
\pgfpathmoveto{\pgfqpoint{1.406589in}{1.455104in}}%
\pgfpathlineto{\pgfqpoint{1.455104in}{1.455104in}}%
\pgfpathlineto{\pgfqpoint{1.455104in}{1.514400in}}%
\pgfpathlineto{\pgfqpoint{1.406589in}{1.514400in}}%
\pgfpathlineto{\pgfqpoint{1.406589in}{1.455104in}}%
\pgfpathclose%
\pgfusepath{stroke,fill}%
\end{pgfscope}%
\begin{pgfscope}%
\pgfpathrectangle{\pgfqpoint{0.150000in}{0.150000in}}{\pgfqpoint{1.800000in}{1.800000in}}%
\pgfusepath{clip}%
\pgfsetbuttcap%
\pgfsetroundjoin%
\definecolor{currentfill}{rgb}{0.400000,0.600000,0.800000}%
\pgfsetfillcolor{currentfill}%
\pgfsetlinewidth{1.003750pt}%
\definecolor{currentstroke}{rgb}{0.000000,0.266667,0.533333}%
\pgfsetstrokecolor{currentstroke}%
\pgfsetdash{}{0pt}%
\pgfpathmoveto{\pgfqpoint{0.630940in}{1.740368in}}%
\pgfpathlineto{\pgfqpoint{0.654000in}{1.740368in}}%
\pgfpathlineto{\pgfqpoint{0.654000in}{1.834702in}}%
\pgfpathlineto{\pgfqpoint{0.630940in}{1.834702in}}%
\pgfpathlineto{\pgfqpoint{0.630940in}{1.740368in}}%
\pgfpathclose%
\pgfusepath{stroke,fill}%
\end{pgfscope}%
\begin{pgfscope}%
\pgfpathrectangle{\pgfqpoint{0.150000in}{0.150000in}}{\pgfqpoint{1.800000in}{1.800000in}}%
\pgfusepath{clip}%
\pgfsetbuttcap%
\pgfsetroundjoin%
\definecolor{currentfill}{rgb}{0.400000,0.600000,0.800000}%
\pgfsetfillcolor{currentfill}%
\pgfsetlinewidth{1.003750pt}%
\definecolor{currentstroke}{rgb}{0.000000,0.266667,0.533333}%
\pgfsetstrokecolor{currentstroke}%
\pgfsetdash{}{0pt}%
\pgfpathmoveto{\pgfqpoint{0.593207in}{1.568850in}}%
\pgfpathlineto{\pgfqpoint{0.612074in}{1.568850in}}%
\pgfpathlineto{\pgfqpoint{0.612074in}{1.646033in}}%
\pgfpathlineto{\pgfqpoint{0.593207in}{1.646033in}}%
\pgfpathlineto{\pgfqpoint{0.593207in}{1.568850in}}%
\pgfpathclose%
\pgfusepath{stroke,fill}%
\end{pgfscope}%
\begin{pgfscope}%
\pgfpathrectangle{\pgfqpoint{0.150000in}{0.150000in}}{\pgfqpoint{1.800000in}{1.800000in}}%
\pgfusepath{clip}%
\pgfsetbuttcap%
\pgfsetroundjoin%
\definecolor{currentfill}{rgb}{0.400000,0.600000,0.800000}%
\pgfsetfillcolor{currentfill}%
\pgfsetlinewidth{1.003750pt}%
\definecolor{currentstroke}{rgb}{0.000000,0.266667,0.533333}%
\pgfsetstrokecolor{currentstroke}%
\pgfsetdash{}{0pt}%
\pgfpathmoveto{\pgfqpoint{0.558903in}{1.397333in}}%
\pgfpathlineto{\pgfqpoint{0.577770in}{1.397333in}}%
\pgfpathlineto{\pgfqpoint{0.577770in}{1.474515in}}%
\pgfpathlineto{\pgfqpoint{0.558903in}{1.474515in}}%
\pgfpathlineto{\pgfqpoint{0.558903in}{1.397333in}}%
\pgfpathclose%
\pgfusepath{stroke,fill}%
\end{pgfscope}%
\begin{pgfscope}%
\pgfpathrectangle{\pgfqpoint{0.150000in}{0.150000in}}{\pgfqpoint{1.800000in}{1.800000in}}%
\pgfusepath{clip}%
\pgfsetbuttcap%
\pgfsetroundjoin%
\definecolor{currentfill}{rgb}{0.400000,0.600000,0.800000}%
\pgfsetfillcolor{currentfill}%
\pgfsetlinewidth{1.003750pt}%
\definecolor{currentstroke}{rgb}{0.000000,0.266667,0.533333}%
\pgfsetstrokecolor{currentstroke}%
\pgfsetdash{}{0pt}%
\pgfpathmoveto{\pgfqpoint{0.528030in}{1.257000in}}%
\pgfpathlineto{\pgfqpoint{0.543467in}{1.257000in}}%
\pgfpathlineto{\pgfqpoint{0.543467in}{1.320150in}}%
\pgfpathlineto{\pgfqpoint{0.528030in}{1.320150in}}%
\pgfpathlineto{\pgfqpoint{0.528030in}{1.257000in}}%
\pgfpathclose%
\pgfusepath{stroke,fill}%
\end{pgfscope}%
\begin{pgfscope}%
\pgfpathrectangle{\pgfqpoint{0.150000in}{0.150000in}}{\pgfqpoint{1.800000in}{1.800000in}}%
\pgfusepath{clip}%
\pgfsetbuttcap%
\pgfsetroundjoin%
\definecolor{currentfill}{rgb}{0.400000,0.600000,0.800000}%
\pgfsetfillcolor{currentfill}%
\pgfsetlinewidth{1.003750pt}%
\definecolor{currentstroke}{rgb}{0.000000,0.266667,0.533333}%
\pgfsetstrokecolor{currentstroke}%
\pgfsetdash{}{0pt}%
\pgfpathmoveto{\pgfqpoint{0.961980in}{1.050189in}}%
\pgfpathlineto{\pgfqpoint{1.050189in}{1.050189in}}%
\pgfpathlineto{\pgfqpoint{1.050189in}{1.158000in}}%
\pgfpathlineto{\pgfqpoint{0.961980in}{1.158000in}}%
\pgfpathlineto{\pgfqpoint{0.961980in}{1.050189in}}%
\pgfpathclose%
\pgfusepath{stroke,fill}%
\end{pgfscope}%
\begin{pgfscope}%
\pgfpathrectangle{\pgfqpoint{0.150000in}{0.150000in}}{\pgfqpoint{1.800000in}{1.800000in}}%
\pgfusepath{clip}%
\pgfsetbuttcap%
\pgfsetroundjoin%
\definecolor{currentfill}{rgb}{0.400000,0.600000,0.800000}%
\pgfsetfillcolor{currentfill}%
\pgfsetlinewidth{1.003750pt}%
\definecolor{currentstroke}{rgb}{0.000000,0.266667,0.533333}%
\pgfsetstrokecolor{currentstroke}%
\pgfsetdash{}{0pt}%
\pgfpathmoveto{\pgfqpoint{0.801600in}{0.873771in}}%
\pgfpathlineto{\pgfqpoint{0.873771in}{0.873771in}}%
\pgfpathlineto{\pgfqpoint{0.873771in}{0.961980in}}%
\pgfpathlineto{\pgfqpoint{0.801600in}{0.961980in}}%
\pgfpathlineto{\pgfqpoint{0.801600in}{0.873771in}}%
\pgfpathclose%
\pgfusepath{stroke,fill}%
\end{pgfscope}%
\begin{pgfscope}%
\pgfpathrectangle{\pgfqpoint{0.150000in}{0.150000in}}{\pgfqpoint{1.800000in}{1.800000in}}%
\pgfusepath{clip}%
\pgfsetbuttcap%
\pgfsetroundjoin%
\definecolor{currentfill}{rgb}{0.400000,0.600000,0.800000}%
\pgfsetfillcolor{currentfill}%
\pgfsetlinewidth{1.003750pt}%
\definecolor{currentstroke}{rgb}{0.000000,0.266667,0.533333}%
\pgfsetstrokecolor{currentstroke}%
\pgfsetdash{}{0pt}%
\pgfpathmoveto{\pgfqpoint{0.558173in}{1.085483in}}%
\pgfpathlineto{\pgfqpoint{0.604950in}{1.085483in}}%
\pgfpathlineto{\pgfqpoint{0.604950in}{1.257000in}}%
\pgfpathlineto{\pgfqpoint{0.558173in}{1.257000in}}%
\pgfpathlineto{\pgfqpoint{0.558173in}{1.085483in}}%
\pgfpathclose%
\pgfusepath{stroke,fill}%
\end{pgfscope}%
\begin{pgfscope}%
\pgfpathrectangle{\pgfqpoint{0.150000in}{0.150000in}}{\pgfqpoint{1.800000in}{1.800000in}}%
\pgfusepath{clip}%
\pgfsetbuttcap%
\pgfsetroundjoin%
\definecolor{currentfill}{rgb}{0.400000,0.600000,0.800000}%
\pgfsetfillcolor{currentfill}%
\pgfsetlinewidth{1.003750pt}%
\definecolor{currentstroke}{rgb}{0.000000,0.266667,0.533333}%
\pgfsetstrokecolor{currentstroke}%
\pgfsetdash{}{0pt}%
\pgfpathmoveto{\pgfqpoint{0.651728in}{0.804818in}}%
\pgfpathlineto{\pgfqpoint{0.801600in}{0.804818in}}%
\pgfpathlineto{\pgfqpoint{0.801600in}{0.945150in}}%
\pgfpathlineto{\pgfqpoint{0.651728in}{0.945150in}}%
\pgfpathlineto{\pgfqpoint{0.651728in}{0.804818in}}%
\pgfpathclose%
\pgfusepath{stroke,fill}%
\end{pgfscope}%
\begin{pgfscope}%
\pgfpathrectangle{\pgfqpoint{0.150000in}{0.150000in}}{\pgfqpoint{1.800000in}{1.800000in}}%
\pgfusepath{clip}%
\pgfsetbuttcap%
\pgfsetroundjoin%
\definecolor{currentfill}{rgb}{0.400000,0.600000,0.800000}%
\pgfsetfillcolor{currentfill}%
\pgfsetlinewidth{1.003750pt}%
\definecolor{currentstroke}{rgb}{0.000000,0.266667,0.533333}%
\pgfsetstrokecolor{currentstroke}%
\pgfsetdash{}{0pt}%
\pgfpathmoveto{\pgfqpoint{1.710420in}{1.818231in}}%
\pgfpathlineto{\pgfqpoint{1.818231in}{1.818231in}}%
\pgfpathlineto{\pgfqpoint{1.818231in}{1.950000in}}%
\pgfpathlineto{\pgfqpoint{1.710420in}{1.950000in}}%
\pgfpathlineto{\pgfqpoint{1.710420in}{1.818231in}}%
\pgfpathclose%
\pgfusepath{stroke,fill}%
\end{pgfscope}%
\begin{pgfscope}%
\pgfpathrectangle{\pgfqpoint{0.150000in}{0.150000in}}{\pgfqpoint{1.800000in}{1.800000in}}%
\pgfusepath{clip}%
\pgfsetbuttcap%
\pgfsetroundjoin%
\definecolor{currentfill}{rgb}{0.400000,0.600000,0.800000}%
\pgfsetfillcolor{currentfill}%
\pgfsetlinewidth{1.003750pt}%
\definecolor{currentstroke}{rgb}{0.000000,0.266667,0.533333}%
\pgfsetstrokecolor{currentstroke}%
\pgfsetdash{}{0pt}%
\pgfpathmoveto{\pgfqpoint{1.514400in}{1.602609in}}%
\pgfpathlineto{\pgfqpoint{1.602609in}{1.602609in}}%
\pgfpathlineto{\pgfqpoint{1.602609in}{1.710420in}}%
\pgfpathlineto{\pgfqpoint{1.514400in}{1.710420in}}%
\pgfpathlineto{\pgfqpoint{1.514400in}{1.602609in}}%
\pgfpathclose%
\pgfusepath{stroke,fill}%
\end{pgfscope}%
\begin{pgfscope}%
\pgfpathrectangle{\pgfqpoint{0.150000in}{0.150000in}}{\pgfqpoint{1.800000in}{1.800000in}}%
\pgfusepath{clip}%
\pgfsetbuttcap%
\pgfsetroundjoin%
\definecolor{currentfill}{rgb}{0.400000,0.600000,0.800000}%
\pgfsetfillcolor{currentfill}%
\pgfsetlinewidth{1.003750pt}%
\definecolor{currentstroke}{rgb}{0.000000,0.266667,0.533333}%
\pgfsetstrokecolor{currentstroke}%
\pgfsetdash{}{0pt}%
\pgfpathmoveto{\pgfqpoint{1.318380in}{1.406589in}}%
\pgfpathlineto{\pgfqpoint{1.406589in}{1.406589in}}%
\pgfpathlineto{\pgfqpoint{1.406589in}{1.514400in}}%
\pgfpathlineto{\pgfqpoint{1.318380in}{1.514400in}}%
\pgfpathlineto{\pgfqpoint{1.318380in}{1.406589in}}%
\pgfpathclose%
\pgfusepath{stroke,fill}%
\end{pgfscope}%
\begin{pgfscope}%
\pgfpathrectangle{\pgfqpoint{0.150000in}{0.150000in}}{\pgfqpoint{1.800000in}{1.800000in}}%
\pgfusepath{clip}%
\pgfsetbuttcap%
\pgfsetroundjoin%
\definecolor{currentfill}{rgb}{0.400000,0.600000,0.800000}%
\pgfsetfillcolor{currentfill}%
\pgfsetlinewidth{1.003750pt}%
\definecolor{currentstroke}{rgb}{0.000000,0.266667,0.533333}%
\pgfsetstrokecolor{currentstroke}%
\pgfsetdash{}{0pt}%
\pgfpathmoveto{\pgfqpoint{1.158000in}{1.230171in}}%
\pgfpathlineto{\pgfqpoint{1.230171in}{1.230171in}}%
\pgfpathlineto{\pgfqpoint{1.230171in}{1.318380in}}%
\pgfpathlineto{\pgfqpoint{1.158000in}{1.318380in}}%
\pgfpathlineto{\pgfqpoint{1.158000in}{1.230171in}}%
\pgfpathclose%
\pgfusepath{stroke,fill}%
\end{pgfscope}%
\begin{pgfscope}%
\pgfpathrectangle{\pgfqpoint{0.150000in}{0.150000in}}{\pgfqpoint{1.800000in}{1.800000in}}%
\pgfusepath{clip}%
\pgfsetbuttcap%
\pgfsetroundjoin%
\definecolor{currentfill}{rgb}{0.400000,0.600000,0.800000}%
\pgfsetfillcolor{currentfill}%
\pgfsetlinewidth{1.003750pt}%
\definecolor{currentstroke}{rgb}{0.000000,0.266667,0.533333}%
\pgfsetstrokecolor{currentstroke}%
\pgfsetdash{}{0pt}%
\pgfpathmoveto{\pgfqpoint{0.612074in}{1.568850in}}%
\pgfpathlineto{\pgfqpoint{0.654000in}{1.568850in}}%
\pgfpathlineto{\pgfqpoint{0.654000in}{1.740368in}}%
\pgfpathlineto{\pgfqpoint{0.612074in}{1.740368in}}%
\pgfpathlineto{\pgfqpoint{0.612074in}{1.568850in}}%
\pgfpathclose%
\pgfusepath{stroke,fill}%
\end{pgfscope}%
\begin{pgfscope}%
\pgfpathrectangle{\pgfqpoint{0.150000in}{0.150000in}}{\pgfqpoint{1.800000in}{1.800000in}}%
\pgfusepath{clip}%
\pgfsetbuttcap%
\pgfsetroundjoin%
\definecolor{currentfill}{rgb}{0.400000,0.600000,0.800000}%
\pgfsetfillcolor{currentfill}%
\pgfsetlinewidth{1.003750pt}%
\definecolor{currentstroke}{rgb}{0.000000,0.266667,0.533333}%
\pgfsetstrokecolor{currentstroke}%
\pgfsetdash{}{0pt}%
\pgfpathmoveto{\pgfqpoint{0.543467in}{1.257000in}}%
\pgfpathlineto{\pgfqpoint{0.577770in}{1.257000in}}%
\pgfpathlineto{\pgfqpoint{0.577770in}{1.397333in}}%
\pgfpathlineto{\pgfqpoint{0.543467in}{1.397333in}}%
\pgfpathlineto{\pgfqpoint{0.543467in}{1.257000in}}%
\pgfpathclose%
\pgfusepath{stroke,fill}%
\end{pgfscope}%
\begin{pgfscope}%
\pgfpathrectangle{\pgfqpoint{0.150000in}{0.150000in}}{\pgfqpoint{1.800000in}{1.800000in}}%
\pgfusepath{clip}%
\pgfsetbuttcap%
\pgfsetroundjoin%
\definecolor{currentfill}{rgb}{0.400000,0.600000,0.800000}%
\pgfsetfillcolor{currentfill}%
\pgfsetlinewidth{1.003750pt}%
\definecolor{currentstroke}{rgb}{0.000000,0.266667,0.533333}%
\pgfsetstrokecolor{currentstroke}%
\pgfsetdash{}{0pt}%
\pgfpathmoveto{\pgfqpoint{0.801600in}{0.961980in}}%
\pgfpathlineto{\pgfqpoint{0.961980in}{0.961980in}}%
\pgfpathlineto{\pgfqpoint{0.961980in}{1.158000in}}%
\pgfpathlineto{\pgfqpoint{0.801600in}{1.158000in}}%
\pgfpathlineto{\pgfqpoint{0.801600in}{0.961980in}}%
\pgfpathclose%
\pgfusepath{stroke,fill}%
\end{pgfscope}%
\begin{pgfscope}%
\pgfpathrectangle{\pgfqpoint{0.150000in}{0.150000in}}{\pgfqpoint{1.800000in}{1.800000in}}%
\pgfusepath{clip}%
\pgfsetbuttcap%
\pgfsetroundjoin%
\definecolor{currentfill}{rgb}{0.400000,0.600000,0.800000}%
\pgfsetfillcolor{currentfill}%
\pgfsetlinewidth{1.003750pt}%
\definecolor{currentstroke}{rgb}{0.000000,0.266667,0.533333}%
\pgfsetstrokecolor{currentstroke}%
\pgfsetdash{}{0pt}%
\pgfpathmoveto{\pgfqpoint{0.604950in}{0.945150in}}%
\pgfpathlineto{\pgfqpoint{0.801600in}{0.945150in}}%
\pgfpathlineto{\pgfqpoint{0.801600in}{1.257000in}}%
\pgfpathlineto{\pgfqpoint{0.604950in}{1.257000in}}%
\pgfpathlineto{\pgfqpoint{0.604950in}{0.945150in}}%
\pgfpathclose%
\pgfusepath{stroke,fill}%
\end{pgfscope}%
\begin{pgfscope}%
\pgfpathrectangle{\pgfqpoint{0.150000in}{0.150000in}}{\pgfqpoint{1.800000in}{1.800000in}}%
\pgfusepath{clip}%
\pgfsetbuttcap%
\pgfsetroundjoin%
\definecolor{currentfill}{rgb}{0.400000,0.600000,0.800000}%
\pgfsetfillcolor{currentfill}%
\pgfsetlinewidth{1.003750pt}%
\definecolor{currentstroke}{rgb}{0.000000,0.266667,0.533333}%
\pgfsetstrokecolor{currentstroke}%
\pgfsetdash{}{0pt}%
\pgfpathmoveto{\pgfqpoint{1.514400in}{1.710420in}}%
\pgfpathlineto{\pgfqpoint{1.710420in}{1.710420in}}%
\pgfpathlineto{\pgfqpoint{1.710420in}{1.950000in}}%
\pgfpathlineto{\pgfqpoint{1.514400in}{1.950000in}}%
\pgfpathlineto{\pgfqpoint{1.514400in}{1.710420in}}%
\pgfpathclose%
\pgfusepath{stroke,fill}%
\end{pgfscope}%
\begin{pgfscope}%
\pgfpathrectangle{\pgfqpoint{0.150000in}{0.150000in}}{\pgfqpoint{1.800000in}{1.800000in}}%
\pgfusepath{clip}%
\pgfsetbuttcap%
\pgfsetroundjoin%
\definecolor{currentfill}{rgb}{0.400000,0.600000,0.800000}%
\pgfsetfillcolor{currentfill}%
\pgfsetlinewidth{1.003750pt}%
\definecolor{currentstroke}{rgb}{0.000000,0.266667,0.533333}%
\pgfsetstrokecolor{currentstroke}%
\pgfsetdash{}{0pt}%
\pgfpathmoveto{\pgfqpoint{1.158000in}{1.318380in}}%
\pgfpathlineto{\pgfqpoint{1.318380in}{1.318380in}}%
\pgfpathlineto{\pgfqpoint{1.318380in}{1.514400in}}%
\pgfpathlineto{\pgfqpoint{1.158000in}{1.514400in}}%
\pgfpathlineto{\pgfqpoint{1.158000in}{1.318380in}}%
\pgfpathclose%
\pgfusepath{stroke,fill}%
\end{pgfscope}%
\begin{pgfscope}%
\pgfpathrectangle{\pgfqpoint{0.150000in}{0.150000in}}{\pgfqpoint{1.800000in}{1.800000in}}%
\pgfusepath{clip}%
\pgfsetbuttcap%
\pgfsetroundjoin%
\definecolor{currentfill}{rgb}{0.400000,0.600000,0.800000}%
\pgfsetfillcolor{currentfill}%
\pgfsetlinewidth{1.003750pt}%
\definecolor{currentstroke}{rgb}{0.000000,0.266667,0.533333}%
\pgfsetstrokecolor{currentstroke}%
\pgfsetdash{}{0pt}%
\pgfpathmoveto{\pgfqpoint{0.577770in}{1.257000in}}%
\pgfpathlineto{\pgfqpoint{0.654000in}{1.257000in}}%
\pgfpathlineto{\pgfqpoint{0.654000in}{1.568850in}}%
\pgfpathlineto{\pgfqpoint{0.577770in}{1.568850in}}%
\pgfpathlineto{\pgfqpoint{0.577770in}{1.257000in}}%
\pgfpathclose%
\pgfusepath{stroke,fill}%
\end{pgfscope}%
\begin{pgfscope}%
\pgfpathrectangle{\pgfqpoint{0.150000in}{0.150000in}}{\pgfqpoint{1.800000in}{1.800000in}}%
\pgfusepath{clip}%
\pgfsetbuttcap%
\pgfsetroundjoin%
\definecolor{currentfill}{rgb}{0.400000,0.600000,0.800000}%
\pgfsetfillcolor{currentfill}%
\pgfsetlinewidth{1.003750pt}%
\definecolor{currentstroke}{rgb}{0.000000,0.266667,0.533333}%
\pgfsetstrokecolor{currentstroke}%
\pgfsetdash{}{0pt}%
\pgfpathmoveto{\pgfqpoint{0.801600in}{1.158000in}}%
\pgfpathlineto{\pgfqpoint{1.158000in}{1.158000in}}%
\pgfpathlineto{\pgfqpoint{1.158000in}{1.257000in}}%
\pgfpathlineto{\pgfqpoint{0.801600in}{1.257000in}}%
\pgfpathlineto{\pgfqpoint{0.801600in}{1.158000in}}%
\pgfpathclose%
\pgfusepath{stroke,fill}%
\end{pgfscope}%
\begin{pgfscope}%
\pgfpathrectangle{\pgfqpoint{0.150000in}{0.150000in}}{\pgfqpoint{1.800000in}{1.800000in}}%
\pgfusepath{clip}%
\pgfsetbuttcap%
\pgfsetroundjoin%
\definecolor{currentfill}{rgb}{0.400000,0.600000,0.800000}%
\pgfsetfillcolor{currentfill}%
\pgfsetlinewidth{1.003750pt}%
\definecolor{currentstroke}{rgb}{0.000000,0.266667,0.533333}%
\pgfsetstrokecolor{currentstroke}%
\pgfsetdash{}{0pt}%
\pgfpathmoveto{\pgfqpoint{1.158000in}{1.514400in}}%
\pgfpathlineto{\pgfqpoint{1.514400in}{1.514400in}}%
\pgfpathlineto{\pgfqpoint{1.514400in}{1.950000in}}%
\pgfpathlineto{\pgfqpoint{1.158000in}{1.950000in}}%
\pgfpathlineto{\pgfqpoint{1.158000in}{1.514400in}}%
\pgfpathclose%
\pgfusepath{stroke,fill}%
\end{pgfscope}%
\begin{pgfscope}%
\pgfpathrectangle{\pgfqpoint{0.150000in}{0.150000in}}{\pgfqpoint{1.800000in}{1.800000in}}%
\pgfusepath{clip}%
\pgfsetbuttcap%
\pgfsetroundjoin%
\definecolor{currentfill}{rgb}{0.400000,0.600000,0.800000}%
\pgfsetfillcolor{currentfill}%
\pgfsetlinewidth{1.003750pt}%
\definecolor{currentstroke}{rgb}{0.000000,0.266667,0.533333}%
\pgfsetstrokecolor{currentstroke}%
\pgfsetdash{}{0pt}%
\pgfpathmoveto{\pgfqpoint{0.654000in}{1.257000in}}%
\pgfpathlineto{\pgfqpoint{1.158000in}{1.257000in}}%
\pgfpathlineto{\pgfqpoint{1.158000in}{1.950000in}}%
\pgfpathlineto{\pgfqpoint{0.654000in}{1.950000in}}%
\pgfpathlineto{\pgfqpoint{0.654000in}{1.257000in}}%
\pgfpathclose%
\pgfusepath{stroke,fill}%
\end{pgfscope}%
\begin{pgfscope}%
\pgfpathrectangle{\pgfqpoint{0.150000in}{0.150000in}}{\pgfqpoint{1.800000in}{1.800000in}}%
\pgfusepath{clip}%
\pgfsetbuttcap%
\pgfsetroundjoin%
\definecolor{currentfill}{rgb}{0.933333,0.800000,0.400000}%
\pgfsetfillcolor{currentfill}%
\pgfsetlinewidth{1.003750pt}%
\definecolor{currentstroke}{rgb}{0.600000,0.466667,0.000000}%
\pgfsetstrokecolor{currentstroke}%
\pgfsetdash{}{0pt}%
\pgfpathmoveto{\pgfqpoint{0.510000in}{1.205116in}}%
\pgfpathlineto{\pgfqpoint{0.518295in}{1.205116in}}%
\pgfpathlineto{\pgfqpoint{0.518295in}{1.257000in}}%
\pgfpathlineto{\pgfqpoint{0.510000in}{1.257000in}}%
\pgfpathlineto{\pgfqpoint{0.510000in}{1.205116in}}%
\pgfpathclose%
\pgfusepath{stroke,fill}%
\end{pgfscope}%
\begin{pgfscope}%
\pgfpathrectangle{\pgfqpoint{0.150000in}{0.150000in}}{\pgfqpoint{1.800000in}{1.800000in}}%
\pgfusepath{clip}%
\pgfsetbuttcap%
\pgfsetroundjoin%
\definecolor{currentfill}{rgb}{0.933333,0.800000,0.400000}%
\pgfsetfillcolor{currentfill}%
\pgfsetlinewidth{1.003750pt}%
\definecolor{currentstroke}{rgb}{0.600000,0.466667,0.000000}%
\pgfsetstrokecolor{currentstroke}%
\pgfsetdash{}{0pt}%
\pgfpathmoveto{\pgfqpoint{0.518295in}{1.162665in}}%
\pgfpathlineto{\pgfqpoint{0.532445in}{1.162665in}}%
\pgfpathlineto{\pgfqpoint{0.532445in}{1.205116in}}%
\pgfpathlineto{\pgfqpoint{0.518295in}{1.205116in}}%
\pgfpathlineto{\pgfqpoint{0.518295in}{1.162665in}}%
\pgfpathclose%
\pgfusepath{stroke,fill}%
\end{pgfscope}%
\begin{pgfscope}%
\pgfpathrectangle{\pgfqpoint{0.150000in}{0.150000in}}{\pgfqpoint{1.800000in}{1.800000in}}%
\pgfusepath{clip}%
\pgfsetbuttcap%
\pgfsetroundjoin%
\definecolor{currentfill}{rgb}{0.933333,0.800000,0.400000}%
\pgfsetfillcolor{currentfill}%
\pgfsetlinewidth{1.003750pt}%
\definecolor{currentstroke}{rgb}{0.600000,0.466667,0.000000}%
\pgfsetstrokecolor{currentstroke}%
\pgfsetdash{}{0pt}%
\pgfpathmoveto{\pgfqpoint{0.651728in}{0.741668in}}%
\pgfpathlineto{\pgfqpoint{0.672777in}{0.741668in}}%
\pgfpathlineto{\pgfqpoint{0.672777in}{0.804818in}}%
\pgfpathlineto{\pgfqpoint{0.651728in}{0.804818in}}%
\pgfpathlineto{\pgfqpoint{0.651728in}{0.741668in}}%
\pgfpathclose%
\pgfusepath{stroke,fill}%
\end{pgfscope}%
\begin{pgfscope}%
\pgfpathrectangle{\pgfqpoint{0.150000in}{0.150000in}}{\pgfqpoint{1.800000in}{1.800000in}}%
\pgfusepath{clip}%
\pgfsetbuttcap%
\pgfsetroundjoin%
\definecolor{currentfill}{rgb}{0.933333,0.800000,0.400000}%
\pgfsetfillcolor{currentfill}%
\pgfsetlinewidth{1.003750pt}%
\definecolor{currentstroke}{rgb}{0.600000,0.466667,0.000000}%
\pgfsetstrokecolor{currentstroke}%
\pgfsetdash{}{0pt}%
\pgfpathmoveto{\pgfqpoint{0.672777in}{0.690000in}}%
\pgfpathlineto{\pgfqpoint{0.719170in}{0.690000in}}%
\pgfpathlineto{\pgfqpoint{0.719170in}{0.741668in}}%
\pgfpathlineto{\pgfqpoint{0.672777in}{0.741668in}}%
\pgfpathlineto{\pgfqpoint{0.672777in}{0.690000in}}%
\pgfpathclose%
\pgfusepath{stroke,fill}%
\end{pgfscope}%
\begin{pgfscope}%
\pgfpathrectangle{\pgfqpoint{0.150000in}{0.150000in}}{\pgfqpoint{1.800000in}{1.800000in}}%
\pgfusepath{clip}%
\pgfsetbuttcap%
\pgfsetroundjoin%
\definecolor{currentfill}{rgb}{0.933333,0.800000,0.400000}%
\pgfsetfillcolor{currentfill}%
\pgfsetlinewidth{1.003750pt}%
\definecolor{currentstroke}{rgb}{0.600000,0.466667,0.000000}%
\pgfsetstrokecolor{currentstroke}%
\pgfsetdash{}{0pt}%
\pgfpathmoveto{\pgfqpoint{0.641317in}{1.886586in}}%
\pgfpathlineto{\pgfqpoint{0.654000in}{1.886586in}}%
\pgfpathlineto{\pgfqpoint{0.654000in}{1.950000in}}%
\pgfpathlineto{\pgfqpoint{0.641317in}{1.950000in}}%
\pgfpathlineto{\pgfqpoint{0.641317in}{1.886586in}}%
\pgfpathclose%
\pgfusepath{stroke,fill}%
\end{pgfscope}%
\begin{pgfscope}%
\pgfpathrectangle{\pgfqpoint{0.150000in}{0.150000in}}{\pgfqpoint{1.800000in}{1.800000in}}%
\pgfusepath{clip}%
\pgfsetbuttcap%
\pgfsetroundjoin%
\definecolor{currentfill}{rgb}{0.933333,0.800000,0.400000}%
\pgfsetfillcolor{currentfill}%
\pgfsetlinewidth{1.003750pt}%
\definecolor{currentstroke}{rgb}{0.600000,0.466667,0.000000}%
\pgfsetstrokecolor{currentstroke}%
\pgfsetdash{}{0pt}%
\pgfpathmoveto{\pgfqpoint{0.630940in}{1.834702in}}%
\pgfpathlineto{\pgfqpoint{0.641317in}{1.834702in}}%
\pgfpathlineto{\pgfqpoint{0.641317in}{1.886586in}}%
\pgfpathlineto{\pgfqpoint{0.630940in}{1.886586in}}%
\pgfpathlineto{\pgfqpoint{0.630940in}{1.834702in}}%
\pgfpathclose%
\pgfusepath{stroke,fill}%
\end{pgfscope}%
\begin{pgfscope}%
\pgfpathrectangle{\pgfqpoint{0.150000in}{0.150000in}}{\pgfqpoint{1.800000in}{1.800000in}}%
\pgfusepath{clip}%
\pgfsetbuttcap%
\pgfsetroundjoin%
\definecolor{currentfill}{rgb}{0.933333,0.800000,0.400000}%
\pgfsetfillcolor{currentfill}%
\pgfsetlinewidth{1.003750pt}%
\definecolor{currentstroke}{rgb}{0.600000,0.466667,0.000000}%
\pgfsetstrokecolor{currentstroke}%
\pgfsetdash{}{0pt}%
\pgfpathmoveto{\pgfqpoint{0.620564in}{1.782818in}}%
\pgfpathlineto{\pgfqpoint{0.630940in}{1.782818in}}%
\pgfpathlineto{\pgfqpoint{0.630940in}{1.834702in}}%
\pgfpathlineto{\pgfqpoint{0.620564in}{1.834702in}}%
\pgfpathlineto{\pgfqpoint{0.620564in}{1.782818in}}%
\pgfpathclose%
\pgfusepath{stroke,fill}%
\end{pgfscope}%
\begin{pgfscope}%
\pgfpathrectangle{\pgfqpoint{0.150000in}{0.150000in}}{\pgfqpoint{1.800000in}{1.800000in}}%
\pgfusepath{clip}%
\pgfsetbuttcap%
\pgfsetroundjoin%
\definecolor{currentfill}{rgb}{0.933333,0.800000,0.400000}%
\pgfsetfillcolor{currentfill}%
\pgfsetlinewidth{1.003750pt}%
\definecolor{currentstroke}{rgb}{0.600000,0.466667,0.000000}%
\pgfsetstrokecolor{currentstroke}%
\pgfsetdash{}{0pt}%
\pgfpathmoveto{\pgfqpoint{0.612074in}{1.740368in}}%
\pgfpathlineto{\pgfqpoint{0.620564in}{1.740368in}}%
\pgfpathlineto{\pgfqpoint{0.620564in}{1.782818in}}%
\pgfpathlineto{\pgfqpoint{0.612074in}{1.782818in}}%
\pgfpathlineto{\pgfqpoint{0.612074in}{1.740368in}}%
\pgfpathclose%
\pgfusepath{stroke,fill}%
\end{pgfscope}%
\begin{pgfscope}%
\pgfpathrectangle{\pgfqpoint{0.150000in}{0.150000in}}{\pgfqpoint{1.800000in}{1.800000in}}%
\pgfusepath{clip}%
\pgfsetbuttcap%
\pgfsetroundjoin%
\definecolor{currentfill}{rgb}{0.933333,0.800000,0.400000}%
\pgfsetfillcolor{currentfill}%
\pgfsetlinewidth{1.003750pt}%
\definecolor{currentstroke}{rgb}{0.600000,0.466667,0.000000}%
\pgfsetstrokecolor{currentstroke}%
\pgfsetdash{}{0pt}%
\pgfpathmoveto{\pgfqpoint{0.601697in}{1.688483in}}%
\pgfpathlineto{\pgfqpoint{0.612074in}{1.688483in}}%
\pgfpathlineto{\pgfqpoint{0.612074in}{1.740368in}}%
\pgfpathlineto{\pgfqpoint{0.601697in}{1.740368in}}%
\pgfpathlineto{\pgfqpoint{0.601697in}{1.688483in}}%
\pgfpathclose%
\pgfusepath{stroke,fill}%
\end{pgfscope}%
\begin{pgfscope}%
\pgfpathrectangle{\pgfqpoint{0.150000in}{0.150000in}}{\pgfqpoint{1.800000in}{1.800000in}}%
\pgfusepath{clip}%
\pgfsetbuttcap%
\pgfsetroundjoin%
\definecolor{currentfill}{rgb}{0.933333,0.800000,0.400000}%
\pgfsetfillcolor{currentfill}%
\pgfsetlinewidth{1.003750pt}%
\definecolor{currentstroke}{rgb}{0.600000,0.466667,0.000000}%
\pgfsetstrokecolor{currentstroke}%
\pgfsetdash{}{0pt}%
\pgfpathmoveto{\pgfqpoint{0.593207in}{1.646033in}}%
\pgfpathlineto{\pgfqpoint{0.601697in}{1.646033in}}%
\pgfpathlineto{\pgfqpoint{0.601697in}{1.688483in}}%
\pgfpathlineto{\pgfqpoint{0.593207in}{1.688483in}}%
\pgfpathlineto{\pgfqpoint{0.593207in}{1.646033in}}%
\pgfpathclose%
\pgfusepath{stroke,fill}%
\end{pgfscope}%
\begin{pgfscope}%
\pgfpathrectangle{\pgfqpoint{0.150000in}{0.150000in}}{\pgfqpoint{1.800000in}{1.800000in}}%
\pgfusepath{clip}%
\pgfsetbuttcap%
\pgfsetroundjoin%
\definecolor{currentfill}{rgb}{0.933333,0.800000,0.400000}%
\pgfsetfillcolor{currentfill}%
\pgfsetlinewidth{1.003750pt}%
\definecolor{currentstroke}{rgb}{0.600000,0.466667,0.000000}%
\pgfsetstrokecolor{currentstroke}%
\pgfsetdash{}{0pt}%
\pgfpathmoveto{\pgfqpoint{0.567393in}{1.516966in}}%
\pgfpathlineto{\pgfqpoint{0.577770in}{1.516966in}}%
\pgfpathlineto{\pgfqpoint{0.577770in}{1.568850in}}%
\pgfpathlineto{\pgfqpoint{0.567393in}{1.568850in}}%
\pgfpathlineto{\pgfqpoint{0.567393in}{1.516966in}}%
\pgfpathclose%
\pgfusepath{stroke,fill}%
\end{pgfscope}%
\begin{pgfscope}%
\pgfpathrectangle{\pgfqpoint{0.150000in}{0.150000in}}{\pgfqpoint{1.800000in}{1.800000in}}%
\pgfusepath{clip}%
\pgfsetbuttcap%
\pgfsetroundjoin%
\definecolor{currentfill}{rgb}{0.933333,0.800000,0.400000}%
\pgfsetfillcolor{currentfill}%
\pgfsetlinewidth{1.003750pt}%
\definecolor{currentstroke}{rgb}{0.600000,0.466667,0.000000}%
\pgfsetstrokecolor{currentstroke}%
\pgfsetdash{}{0pt}%
\pgfpathmoveto{\pgfqpoint{0.558903in}{1.474515in}}%
\pgfpathlineto{\pgfqpoint{0.567393in}{1.474515in}}%
\pgfpathlineto{\pgfqpoint{0.567393in}{1.516966in}}%
\pgfpathlineto{\pgfqpoint{0.558903in}{1.516966in}}%
\pgfpathlineto{\pgfqpoint{0.558903in}{1.474515in}}%
\pgfpathclose%
\pgfusepath{stroke,fill}%
\end{pgfscope}%
\begin{pgfscope}%
\pgfpathrectangle{\pgfqpoint{0.150000in}{0.150000in}}{\pgfqpoint{1.800000in}{1.800000in}}%
\pgfusepath{clip}%
\pgfsetbuttcap%
\pgfsetroundjoin%
\definecolor{currentfill}{rgb}{0.933333,0.800000,0.400000}%
\pgfsetfillcolor{currentfill}%
\pgfsetlinewidth{1.003750pt}%
\definecolor{currentstroke}{rgb}{0.600000,0.466667,0.000000}%
\pgfsetstrokecolor{currentstroke}%
\pgfsetdash{}{0pt}%
\pgfpathmoveto{\pgfqpoint{1.098704in}{1.098704in}}%
\pgfpathlineto{\pgfqpoint{1.158000in}{1.098704in}}%
\pgfpathlineto{\pgfqpoint{1.158000in}{1.158000in}}%
\pgfpathlineto{\pgfqpoint{1.098704in}{1.158000in}}%
\pgfpathlineto{\pgfqpoint{1.098704in}{1.098704in}}%
\pgfpathclose%
\pgfusepath{stroke,fill}%
\end{pgfscope}%
\begin{pgfscope}%
\pgfpathrectangle{\pgfqpoint{0.150000in}{0.150000in}}{\pgfqpoint{1.800000in}{1.800000in}}%
\pgfusepath{clip}%
\pgfsetbuttcap%
\pgfsetroundjoin%
\definecolor{currentfill}{rgb}{0.933333,0.800000,0.400000}%
\pgfsetfillcolor{currentfill}%
\pgfsetlinewidth{1.003750pt}%
\definecolor{currentstroke}{rgb}{0.600000,0.466667,0.000000}%
\pgfsetstrokecolor{currentstroke}%
\pgfsetdash{}{0pt}%
\pgfpathmoveto{\pgfqpoint{1.050189in}{1.050189in}}%
\pgfpathlineto{\pgfqpoint{1.098704in}{1.050189in}}%
\pgfpathlineto{\pgfqpoint{1.098704in}{1.098704in}}%
\pgfpathlineto{\pgfqpoint{1.050189in}{1.098704in}}%
\pgfpathlineto{\pgfqpoint{1.050189in}{1.050189in}}%
\pgfpathclose%
\pgfusepath{stroke,fill}%
\end{pgfscope}%
\begin{pgfscope}%
\pgfpathrectangle{\pgfqpoint{0.150000in}{0.150000in}}{\pgfqpoint{1.800000in}{1.800000in}}%
\pgfusepath{clip}%
\pgfsetbuttcap%
\pgfsetroundjoin%
\definecolor{currentfill}{rgb}{0.933333,0.800000,0.400000}%
\pgfsetfillcolor{currentfill}%
\pgfsetlinewidth{1.003750pt}%
\definecolor{currentstroke}{rgb}{0.600000,0.466667,0.000000}%
\pgfsetstrokecolor{currentstroke}%
\pgfsetdash{}{0pt}%
\pgfpathmoveto{\pgfqpoint{0.532445in}{1.085483in}}%
\pgfpathlineto{\pgfqpoint{0.558173in}{1.085483in}}%
\pgfpathlineto{\pgfqpoint{0.558173in}{1.162665in}}%
\pgfpathlineto{\pgfqpoint{0.532445in}{1.162665in}}%
\pgfpathlineto{\pgfqpoint{0.532445in}{1.085483in}}%
\pgfpathclose%
\pgfusepath{stroke,fill}%
\end{pgfscope}%
\begin{pgfscope}%
\pgfpathrectangle{\pgfqpoint{0.150000in}{0.150000in}}{\pgfqpoint{1.800000in}{1.800000in}}%
\pgfusepath{clip}%
\pgfsetbuttcap%
\pgfsetroundjoin%
\definecolor{currentfill}{rgb}{0.933333,0.800000,0.400000}%
\pgfsetfillcolor{currentfill}%
\pgfsetlinewidth{1.003750pt}%
\definecolor{currentstroke}{rgb}{0.600000,0.466667,0.000000}%
\pgfsetstrokecolor{currentstroke}%
\pgfsetdash{}{0pt}%
\pgfpathmoveto{\pgfqpoint{0.558173in}{1.008300in}}%
\pgfpathlineto{\pgfqpoint{0.583900in}{1.008300in}}%
\pgfpathlineto{\pgfqpoint{0.583900in}{1.085483in}}%
\pgfpathlineto{\pgfqpoint{0.558173in}{1.085483in}}%
\pgfpathlineto{\pgfqpoint{0.558173in}{1.008300in}}%
\pgfpathclose%
\pgfusepath{stroke,fill}%
\end{pgfscope}%
\begin{pgfscope}%
\pgfpathrectangle{\pgfqpoint{0.150000in}{0.150000in}}{\pgfqpoint{1.800000in}{1.800000in}}%
\pgfusepath{clip}%
\pgfsetbuttcap%
\pgfsetroundjoin%
\definecolor{currentfill}{rgb}{0.933333,0.800000,0.400000}%
\pgfsetfillcolor{currentfill}%
\pgfsetlinewidth{1.003750pt}%
\definecolor{currentstroke}{rgb}{0.600000,0.466667,0.000000}%
\pgfsetstrokecolor{currentstroke}%
\pgfsetdash{}{0pt}%
\pgfpathmoveto{\pgfqpoint{0.583900in}{0.945150in}}%
\pgfpathlineto{\pgfqpoint{0.604950in}{0.945150in}}%
\pgfpathlineto{\pgfqpoint{0.604950in}{1.008300in}}%
\pgfpathlineto{\pgfqpoint{0.583900in}{1.008300in}}%
\pgfpathlineto{\pgfqpoint{0.583900in}{0.945150in}}%
\pgfpathclose%
\pgfusepath{stroke,fill}%
\end{pgfscope}%
\begin{pgfscope}%
\pgfpathrectangle{\pgfqpoint{0.150000in}{0.150000in}}{\pgfqpoint{1.800000in}{1.800000in}}%
\pgfusepath{clip}%
\pgfsetbuttcap%
\pgfsetroundjoin%
\definecolor{currentfill}{rgb}{0.933333,0.800000,0.400000}%
\pgfsetfillcolor{currentfill}%
\pgfsetlinewidth{1.003750pt}%
\definecolor{currentstroke}{rgb}{0.600000,0.466667,0.000000}%
\pgfsetstrokecolor{currentstroke}%
\pgfsetdash{}{0pt}%
\pgfpathmoveto{\pgfqpoint{0.604950in}{0.867967in}}%
\pgfpathlineto{\pgfqpoint{0.630678in}{0.867967in}}%
\pgfpathlineto{\pgfqpoint{0.630678in}{0.945150in}}%
\pgfpathlineto{\pgfqpoint{0.604950in}{0.945150in}}%
\pgfpathlineto{\pgfqpoint{0.604950in}{0.867967in}}%
\pgfpathclose%
\pgfusepath{stroke,fill}%
\end{pgfscope}%
\begin{pgfscope}%
\pgfpathrectangle{\pgfqpoint{0.150000in}{0.150000in}}{\pgfqpoint{1.800000in}{1.800000in}}%
\pgfusepath{clip}%
\pgfsetbuttcap%
\pgfsetroundjoin%
\definecolor{currentfill}{rgb}{0.933333,0.800000,0.400000}%
\pgfsetfillcolor{currentfill}%
\pgfsetlinewidth{1.003750pt}%
\definecolor{currentstroke}{rgb}{0.600000,0.466667,0.000000}%
\pgfsetstrokecolor{currentstroke}%
\pgfsetdash{}{0pt}%
\pgfpathmoveto{\pgfqpoint{0.630678in}{0.804818in}}%
\pgfpathlineto{\pgfqpoint{0.651728in}{0.804818in}}%
\pgfpathlineto{\pgfqpoint{0.651728in}{0.867967in}}%
\pgfpathlineto{\pgfqpoint{0.630678in}{0.867967in}}%
\pgfpathlineto{\pgfqpoint{0.630678in}{0.804818in}}%
\pgfpathclose%
\pgfusepath{stroke,fill}%
\end{pgfscope}%
\begin{pgfscope}%
\pgfpathrectangle{\pgfqpoint{0.150000in}{0.150000in}}{\pgfqpoint{1.800000in}{1.800000in}}%
\pgfusepath{clip}%
\pgfsetbuttcap%
\pgfsetroundjoin%
\definecolor{currentfill}{rgb}{0.933333,0.800000,0.400000}%
\pgfsetfillcolor{currentfill}%
\pgfsetlinewidth{1.003750pt}%
\definecolor{currentstroke}{rgb}{0.600000,0.466667,0.000000}%
\pgfsetstrokecolor{currentstroke}%
\pgfsetdash{}{0pt}%
\pgfpathmoveto{\pgfqpoint{0.719170in}{0.719170in}}%
\pgfpathlineto{\pgfqpoint{0.801600in}{0.719170in}}%
\pgfpathlineto{\pgfqpoint{0.801600in}{0.801600in}}%
\pgfpathlineto{\pgfqpoint{0.719170in}{0.801600in}}%
\pgfpathlineto{\pgfqpoint{0.719170in}{0.719170in}}%
\pgfpathclose%
\pgfusepath{stroke,fill}%
\end{pgfscope}%
\begin{pgfscope}%
\pgfpathrectangle{\pgfqpoint{0.150000in}{0.150000in}}{\pgfqpoint{1.800000in}{1.800000in}}%
\pgfusepath{clip}%
\pgfsetbuttcap%
\pgfsetroundjoin%
\definecolor{currentfill}{rgb}{0.933333,0.800000,0.400000}%
\pgfsetfillcolor{currentfill}%
\pgfsetlinewidth{1.003750pt}%
\definecolor{currentstroke}{rgb}{0.600000,0.466667,0.000000}%
\pgfsetstrokecolor{currentstroke}%
\pgfsetdash{}{0pt}%
\pgfpathmoveto{\pgfqpoint{1.877527in}{1.877527in}}%
\pgfpathlineto{\pgfqpoint{1.950000in}{1.877527in}}%
\pgfpathlineto{\pgfqpoint{1.950000in}{1.950000in}}%
\pgfpathlineto{\pgfqpoint{1.877527in}{1.950000in}}%
\pgfpathlineto{\pgfqpoint{1.877527in}{1.877527in}}%
\pgfpathclose%
\pgfusepath{stroke,fill}%
\end{pgfscope}%
\begin{pgfscope}%
\pgfpathrectangle{\pgfqpoint{0.150000in}{0.150000in}}{\pgfqpoint{1.800000in}{1.800000in}}%
\pgfusepath{clip}%
\pgfsetbuttcap%
\pgfsetroundjoin%
\definecolor{currentfill}{rgb}{0.933333,0.800000,0.400000}%
\pgfsetfillcolor{currentfill}%
\pgfsetlinewidth{1.003750pt}%
\definecolor{currentstroke}{rgb}{0.600000,0.466667,0.000000}%
\pgfsetstrokecolor{currentstroke}%
\pgfsetdash{}{0pt}%
\pgfpathmoveto{\pgfqpoint{1.818231in}{1.818231in}}%
\pgfpathlineto{\pgfqpoint{1.877527in}{1.818231in}}%
\pgfpathlineto{\pgfqpoint{1.877527in}{1.877527in}}%
\pgfpathlineto{\pgfqpoint{1.818231in}{1.877527in}}%
\pgfpathlineto{\pgfqpoint{1.818231in}{1.818231in}}%
\pgfpathclose%
\pgfusepath{stroke,fill}%
\end{pgfscope}%
\begin{pgfscope}%
\pgfpathrectangle{\pgfqpoint{0.150000in}{0.150000in}}{\pgfqpoint{1.800000in}{1.800000in}}%
\pgfusepath{clip}%
\pgfsetbuttcap%
\pgfsetroundjoin%
\definecolor{currentfill}{rgb}{0.933333,0.800000,0.400000}%
\pgfsetfillcolor{currentfill}%
\pgfsetlinewidth{1.003750pt}%
\definecolor{currentstroke}{rgb}{0.600000,0.466667,0.000000}%
\pgfsetstrokecolor{currentstroke}%
\pgfsetdash{}{0pt}%
\pgfpathmoveto{\pgfqpoint{1.758935in}{1.758935in}}%
\pgfpathlineto{\pgfqpoint{1.818231in}{1.758935in}}%
\pgfpathlineto{\pgfqpoint{1.818231in}{1.818231in}}%
\pgfpathlineto{\pgfqpoint{1.758935in}{1.818231in}}%
\pgfpathlineto{\pgfqpoint{1.758935in}{1.758935in}}%
\pgfpathclose%
\pgfusepath{stroke,fill}%
\end{pgfscope}%
\begin{pgfscope}%
\pgfpathrectangle{\pgfqpoint{0.150000in}{0.150000in}}{\pgfqpoint{1.800000in}{1.800000in}}%
\pgfusepath{clip}%
\pgfsetbuttcap%
\pgfsetroundjoin%
\definecolor{currentfill}{rgb}{0.933333,0.800000,0.400000}%
\pgfsetfillcolor{currentfill}%
\pgfsetlinewidth{1.003750pt}%
\definecolor{currentstroke}{rgb}{0.600000,0.466667,0.000000}%
\pgfsetstrokecolor{currentstroke}%
\pgfsetdash{}{0pt}%
\pgfpathmoveto{\pgfqpoint{1.710420in}{1.710420in}}%
\pgfpathlineto{\pgfqpoint{1.758935in}{1.710420in}}%
\pgfpathlineto{\pgfqpoint{1.758935in}{1.758935in}}%
\pgfpathlineto{\pgfqpoint{1.710420in}{1.758935in}}%
\pgfpathlineto{\pgfqpoint{1.710420in}{1.710420in}}%
\pgfpathclose%
\pgfusepath{stroke,fill}%
\end{pgfscope}%
\begin{pgfscope}%
\pgfpathrectangle{\pgfqpoint{0.150000in}{0.150000in}}{\pgfqpoint{1.800000in}{1.800000in}}%
\pgfusepath{clip}%
\pgfsetbuttcap%
\pgfsetroundjoin%
\definecolor{currentfill}{rgb}{0.933333,0.800000,0.400000}%
\pgfsetfillcolor{currentfill}%
\pgfsetlinewidth{1.003750pt}%
\definecolor{currentstroke}{rgb}{0.600000,0.466667,0.000000}%
\pgfsetstrokecolor{currentstroke}%
\pgfsetdash{}{0pt}%
\pgfpathmoveto{\pgfqpoint{1.651124in}{1.651124in}}%
\pgfpathlineto{\pgfqpoint{1.710420in}{1.651124in}}%
\pgfpathlineto{\pgfqpoint{1.710420in}{1.710420in}}%
\pgfpathlineto{\pgfqpoint{1.651124in}{1.710420in}}%
\pgfpathlineto{\pgfqpoint{1.651124in}{1.651124in}}%
\pgfpathclose%
\pgfusepath{stroke,fill}%
\end{pgfscope}%
\begin{pgfscope}%
\pgfpathrectangle{\pgfqpoint{0.150000in}{0.150000in}}{\pgfqpoint{1.800000in}{1.800000in}}%
\pgfusepath{clip}%
\pgfsetbuttcap%
\pgfsetroundjoin%
\definecolor{currentfill}{rgb}{0.933333,0.800000,0.400000}%
\pgfsetfillcolor{currentfill}%
\pgfsetlinewidth{1.003750pt}%
\definecolor{currentstroke}{rgb}{0.600000,0.466667,0.000000}%
\pgfsetstrokecolor{currentstroke}%
\pgfsetdash{}{0pt}%
\pgfpathmoveto{\pgfqpoint{1.602609in}{1.602609in}}%
\pgfpathlineto{\pgfqpoint{1.651124in}{1.602609in}}%
\pgfpathlineto{\pgfqpoint{1.651124in}{1.651124in}}%
\pgfpathlineto{\pgfqpoint{1.602609in}{1.651124in}}%
\pgfpathlineto{\pgfqpoint{1.602609in}{1.602609in}}%
\pgfpathclose%
\pgfusepath{stroke,fill}%
\end{pgfscope}%
\begin{pgfscope}%
\pgfpathrectangle{\pgfqpoint{0.150000in}{0.150000in}}{\pgfqpoint{1.800000in}{1.800000in}}%
\pgfusepath{clip}%
\pgfsetbuttcap%
\pgfsetroundjoin%
\definecolor{currentfill}{rgb}{0.933333,0.800000,0.400000}%
\pgfsetfillcolor{currentfill}%
\pgfsetlinewidth{1.003750pt}%
\definecolor{currentstroke}{rgb}{0.600000,0.466667,0.000000}%
\pgfsetstrokecolor{currentstroke}%
\pgfsetdash{}{0pt}%
\pgfpathmoveto{\pgfqpoint{1.455104in}{1.455104in}}%
\pgfpathlineto{\pgfqpoint{1.514400in}{1.455104in}}%
\pgfpathlineto{\pgfqpoint{1.514400in}{1.514400in}}%
\pgfpathlineto{\pgfqpoint{1.455104in}{1.514400in}}%
\pgfpathlineto{\pgfqpoint{1.455104in}{1.455104in}}%
\pgfpathclose%
\pgfusepath{stroke,fill}%
\end{pgfscope}%
\begin{pgfscope}%
\pgfpathrectangle{\pgfqpoint{0.150000in}{0.150000in}}{\pgfqpoint{1.800000in}{1.800000in}}%
\pgfusepath{clip}%
\pgfsetbuttcap%
\pgfsetroundjoin%
\definecolor{currentfill}{rgb}{0.933333,0.800000,0.400000}%
\pgfsetfillcolor{currentfill}%
\pgfsetlinewidth{1.003750pt}%
\definecolor{currentstroke}{rgb}{0.600000,0.466667,0.000000}%
\pgfsetstrokecolor{currentstroke}%
\pgfsetdash{}{0pt}%
\pgfpathmoveto{\pgfqpoint{1.406589in}{1.406589in}}%
\pgfpathlineto{\pgfqpoint{1.455104in}{1.406589in}}%
\pgfpathlineto{\pgfqpoint{1.455104in}{1.455104in}}%
\pgfpathlineto{\pgfqpoint{1.406589in}{1.455104in}}%
\pgfpathlineto{\pgfqpoint{1.406589in}{1.406589in}}%
\pgfpathclose%
\pgfusepath{stroke,fill}%
\end{pgfscope}%
\begin{pgfscope}%
\pgfpathrectangle{\pgfqpoint{0.150000in}{0.150000in}}{\pgfqpoint{1.800000in}{1.800000in}}%
\pgfusepath{clip}%
\pgfsetbuttcap%
\pgfsetroundjoin%
\definecolor{currentfill}{rgb}{0.933333,0.800000,0.400000}%
\pgfsetfillcolor{currentfill}%
\pgfsetlinewidth{1.003750pt}%
\definecolor{currentstroke}{rgb}{0.600000,0.466667,0.000000}%
\pgfsetstrokecolor{currentstroke}%
\pgfsetdash{}{0pt}%
\pgfpathmoveto{\pgfqpoint{0.577770in}{1.568850in}}%
\pgfpathlineto{\pgfqpoint{0.593207in}{1.568850in}}%
\pgfpathlineto{\pgfqpoint{0.593207in}{1.646033in}}%
\pgfpathlineto{\pgfqpoint{0.577770in}{1.646033in}}%
\pgfpathlineto{\pgfqpoint{0.577770in}{1.568850in}}%
\pgfpathclose%
\pgfusepath{stroke,fill}%
\end{pgfscope}%
\begin{pgfscope}%
\pgfpathrectangle{\pgfqpoint{0.150000in}{0.150000in}}{\pgfqpoint{1.800000in}{1.800000in}}%
\pgfusepath{clip}%
\pgfsetbuttcap%
\pgfsetroundjoin%
\definecolor{currentfill}{rgb}{0.933333,0.800000,0.400000}%
\pgfsetfillcolor{currentfill}%
\pgfsetlinewidth{1.003750pt}%
\definecolor{currentstroke}{rgb}{0.600000,0.466667,0.000000}%
\pgfsetstrokecolor{currentstroke}%
\pgfsetdash{}{0pt}%
\pgfpathmoveto{\pgfqpoint{0.543467in}{1.397333in}}%
\pgfpathlineto{\pgfqpoint{0.558903in}{1.397333in}}%
\pgfpathlineto{\pgfqpoint{0.558903in}{1.474515in}}%
\pgfpathlineto{\pgfqpoint{0.543467in}{1.474515in}}%
\pgfpathlineto{\pgfqpoint{0.543467in}{1.397333in}}%
\pgfpathclose%
\pgfusepath{stroke,fill}%
\end{pgfscope}%
\begin{pgfscope}%
\pgfpathrectangle{\pgfqpoint{0.150000in}{0.150000in}}{\pgfqpoint{1.800000in}{1.800000in}}%
\pgfusepath{clip}%
\pgfsetbuttcap%
\pgfsetroundjoin%
\definecolor{currentfill}{rgb}{0.933333,0.800000,0.400000}%
\pgfsetfillcolor{currentfill}%
\pgfsetlinewidth{1.003750pt}%
\definecolor{currentstroke}{rgb}{0.600000,0.466667,0.000000}%
\pgfsetstrokecolor{currentstroke}%
\pgfsetdash{}{0pt}%
\pgfpathmoveto{\pgfqpoint{0.528030in}{1.320150in}}%
\pgfpathlineto{\pgfqpoint{0.543467in}{1.320150in}}%
\pgfpathlineto{\pgfqpoint{0.543467in}{1.397333in}}%
\pgfpathlineto{\pgfqpoint{0.528030in}{1.397333in}}%
\pgfpathlineto{\pgfqpoint{0.528030in}{1.320150in}}%
\pgfpathclose%
\pgfusepath{stroke,fill}%
\end{pgfscope}%
\begin{pgfscope}%
\pgfpathrectangle{\pgfqpoint{0.150000in}{0.150000in}}{\pgfqpoint{1.800000in}{1.800000in}}%
\pgfusepath{clip}%
\pgfsetbuttcap%
\pgfsetroundjoin%
\definecolor{currentfill}{rgb}{0.933333,0.800000,0.400000}%
\pgfsetfillcolor{currentfill}%
\pgfsetlinewidth{1.003750pt}%
\definecolor{currentstroke}{rgb}{0.600000,0.466667,0.000000}%
\pgfsetstrokecolor{currentstroke}%
\pgfsetdash{}{0pt}%
\pgfpathmoveto{\pgfqpoint{0.515400in}{1.257000in}}%
\pgfpathlineto{\pgfqpoint{0.528030in}{1.257000in}}%
\pgfpathlineto{\pgfqpoint{0.528030in}{1.320150in}}%
\pgfpathlineto{\pgfqpoint{0.515400in}{1.320150in}}%
\pgfpathlineto{\pgfqpoint{0.515400in}{1.257000in}}%
\pgfpathclose%
\pgfusepath{stroke,fill}%
\end{pgfscope}%
\begin{pgfscope}%
\pgfpathrectangle{\pgfqpoint{0.150000in}{0.150000in}}{\pgfqpoint{1.800000in}{1.800000in}}%
\pgfusepath{clip}%
\pgfsetbuttcap%
\pgfsetroundjoin%
\definecolor{currentfill}{rgb}{0.933333,0.800000,0.400000}%
\pgfsetfillcolor{currentfill}%
\pgfsetlinewidth{1.003750pt}%
\definecolor{currentstroke}{rgb}{0.600000,0.466667,0.000000}%
\pgfsetstrokecolor{currentstroke}%
\pgfsetdash{}{0pt}%
\pgfpathmoveto{\pgfqpoint{0.961980in}{0.961980in}}%
\pgfpathlineto{\pgfqpoint{1.050189in}{0.961980in}}%
\pgfpathlineto{\pgfqpoint{1.050189in}{1.050189in}}%
\pgfpathlineto{\pgfqpoint{0.961980in}{1.050189in}}%
\pgfpathlineto{\pgfqpoint{0.961980in}{0.961980in}}%
\pgfpathclose%
\pgfusepath{stroke,fill}%
\end{pgfscope}%
\begin{pgfscope}%
\pgfpathrectangle{\pgfqpoint{0.150000in}{0.150000in}}{\pgfqpoint{1.800000in}{1.800000in}}%
\pgfusepath{clip}%
\pgfsetbuttcap%
\pgfsetroundjoin%
\definecolor{currentfill}{rgb}{0.933333,0.800000,0.400000}%
\pgfsetfillcolor{currentfill}%
\pgfsetlinewidth{1.003750pt}%
\definecolor{currentstroke}{rgb}{0.600000,0.466667,0.000000}%
\pgfsetstrokecolor{currentstroke}%
\pgfsetdash{}{0pt}%
\pgfpathmoveto{\pgfqpoint{0.873771in}{0.873771in}}%
\pgfpathlineto{\pgfqpoint{0.961980in}{0.873771in}}%
\pgfpathlineto{\pgfqpoint{0.961980in}{0.961980in}}%
\pgfpathlineto{\pgfqpoint{0.873771in}{0.961980in}}%
\pgfpathlineto{\pgfqpoint{0.873771in}{0.873771in}}%
\pgfpathclose%
\pgfusepath{stroke,fill}%
\end{pgfscope}%
\begin{pgfscope}%
\pgfpathrectangle{\pgfqpoint{0.150000in}{0.150000in}}{\pgfqpoint{1.800000in}{1.800000in}}%
\pgfusepath{clip}%
\pgfsetbuttcap%
\pgfsetroundjoin%
\definecolor{currentfill}{rgb}{0.933333,0.800000,0.400000}%
\pgfsetfillcolor{currentfill}%
\pgfsetlinewidth{1.003750pt}%
\definecolor{currentstroke}{rgb}{0.600000,0.466667,0.000000}%
\pgfsetstrokecolor{currentstroke}%
\pgfsetdash{}{0pt}%
\pgfpathmoveto{\pgfqpoint{0.801600in}{0.801600in}}%
\pgfpathlineto{\pgfqpoint{0.873771in}{0.801600in}}%
\pgfpathlineto{\pgfqpoint{0.873771in}{0.873771in}}%
\pgfpathlineto{\pgfqpoint{0.801600in}{0.873771in}}%
\pgfpathlineto{\pgfqpoint{0.801600in}{0.801600in}}%
\pgfpathclose%
\pgfusepath{stroke,fill}%
\end{pgfscope}%
\begin{pgfscope}%
\pgfpathrectangle{\pgfqpoint{0.150000in}{0.150000in}}{\pgfqpoint{1.800000in}{1.800000in}}%
\pgfusepath{clip}%
\pgfsetbuttcap%
\pgfsetroundjoin%
\definecolor{currentfill}{rgb}{0.933333,0.800000,0.400000}%
\pgfsetfillcolor{currentfill}%
\pgfsetlinewidth{1.003750pt}%
\definecolor{currentstroke}{rgb}{0.600000,0.466667,0.000000}%
\pgfsetstrokecolor{currentstroke}%
\pgfsetdash{}{0pt}%
\pgfpathmoveto{\pgfqpoint{1.514400in}{1.514400in}}%
\pgfpathlineto{\pgfqpoint{1.602609in}{1.514400in}}%
\pgfpathlineto{\pgfqpoint{1.602609in}{1.602609in}}%
\pgfpathlineto{\pgfqpoint{1.514400in}{1.602609in}}%
\pgfpathlineto{\pgfqpoint{1.514400in}{1.514400in}}%
\pgfpathclose%
\pgfusepath{stroke,fill}%
\end{pgfscope}%
\begin{pgfscope}%
\pgfpathrectangle{\pgfqpoint{0.150000in}{0.150000in}}{\pgfqpoint{1.800000in}{1.800000in}}%
\pgfusepath{clip}%
\pgfsetbuttcap%
\pgfsetroundjoin%
\definecolor{currentfill}{rgb}{0.933333,0.800000,0.400000}%
\pgfsetfillcolor{currentfill}%
\pgfsetlinewidth{1.003750pt}%
\definecolor{currentstroke}{rgb}{0.600000,0.466667,0.000000}%
\pgfsetstrokecolor{currentstroke}%
\pgfsetdash{}{0pt}%
\pgfpathmoveto{\pgfqpoint{1.318380in}{1.318380in}}%
\pgfpathlineto{\pgfqpoint{1.406589in}{1.318380in}}%
\pgfpathlineto{\pgfqpoint{1.406589in}{1.406589in}}%
\pgfpathlineto{\pgfqpoint{1.318380in}{1.406589in}}%
\pgfpathlineto{\pgfqpoint{1.318380in}{1.318380in}}%
\pgfpathclose%
\pgfusepath{stroke,fill}%
\end{pgfscope}%
\begin{pgfscope}%
\pgfpathrectangle{\pgfqpoint{0.150000in}{0.150000in}}{\pgfqpoint{1.800000in}{1.800000in}}%
\pgfusepath{clip}%
\pgfsetbuttcap%
\pgfsetroundjoin%
\definecolor{currentfill}{rgb}{0.933333,0.800000,0.400000}%
\pgfsetfillcolor{currentfill}%
\pgfsetlinewidth{1.003750pt}%
\definecolor{currentstroke}{rgb}{0.600000,0.466667,0.000000}%
\pgfsetstrokecolor{currentstroke}%
\pgfsetdash{}{0pt}%
\pgfpathmoveto{\pgfqpoint{1.230171in}{1.230171in}}%
\pgfpathlineto{\pgfqpoint{1.318380in}{1.230171in}}%
\pgfpathlineto{\pgfqpoint{1.318380in}{1.318380in}}%
\pgfpathlineto{\pgfqpoint{1.230171in}{1.318380in}}%
\pgfpathlineto{\pgfqpoint{1.230171in}{1.230171in}}%
\pgfpathclose%
\pgfusepath{stroke,fill}%
\end{pgfscope}%
\begin{pgfscope}%
\pgfpathrectangle{\pgfqpoint{0.150000in}{0.150000in}}{\pgfqpoint{1.800000in}{1.800000in}}%
\pgfusepath{clip}%
\pgfsetbuttcap%
\pgfsetroundjoin%
\definecolor{currentfill}{rgb}{0.933333,0.800000,0.400000}%
\pgfsetfillcolor{currentfill}%
\pgfsetlinewidth{1.003750pt}%
\definecolor{currentstroke}{rgb}{0.600000,0.466667,0.000000}%
\pgfsetstrokecolor{currentstroke}%
\pgfsetdash{}{0pt}%
\pgfpathmoveto{\pgfqpoint{1.158000in}{1.158000in}}%
\pgfpathlineto{\pgfqpoint{1.230171in}{1.158000in}}%
\pgfpathlineto{\pgfqpoint{1.230171in}{1.230171in}}%
\pgfpathlineto{\pgfqpoint{1.158000in}{1.230171in}}%
\pgfpathlineto{\pgfqpoint{1.158000in}{1.158000in}}%
\pgfpathclose%
\pgfusepath{stroke,fill}%
\end{pgfscope}%
\begin{pgfscope}%
\pgfpathrectangle{\pgfqpoint{0.150000in}{0.150000in}}{\pgfqpoint{1.800000in}{1.800000in}}%
\pgfusepath{clip}%
\pgfsetbuttcap%
\pgfsetroundjoin%
\definecolor{currentfill}{rgb}{0.000000,0.000000,0.000000}%
\pgfsetfillcolor{currentfill}%
\pgfsetlinewidth{1.003750pt}%
\definecolor{currentstroke}{rgb}{0.000000,0.000000,0.000000}%
\pgfsetstrokecolor{currentstroke}%
\pgfsetdash{}{0pt}%
\pgfsys@defobject{currentmarker}{\pgfqpoint{-0.038036in}{-0.038036in}}{\pgfqpoint{0.038036in}{0.038036in}}{%
\pgfpathmoveto{\pgfqpoint{0.000000in}{-0.038036in}}%
\pgfpathcurveto{\pgfqpoint{0.010087in}{-0.038036in}}{\pgfqpoint{0.019763in}{-0.034029in}}{\pgfqpoint{0.026896in}{-0.026896in}}%
\pgfpathcurveto{\pgfqpoint{0.034029in}{-0.019763in}}{\pgfqpoint{0.038036in}{-0.010087in}}{\pgfqpoint{0.038036in}{0.000000in}}%
\pgfpathcurveto{\pgfqpoint{0.038036in}{0.010087in}}{\pgfqpoint{0.034029in}{0.019763in}}{\pgfqpoint{0.026896in}{0.026896in}}%
\pgfpathcurveto{\pgfqpoint{0.019763in}{0.034029in}}{\pgfqpoint{0.010087in}{0.038036in}}{\pgfqpoint{0.000000in}{0.038036in}}%
\pgfpathcurveto{\pgfqpoint{-0.010087in}{0.038036in}}{\pgfqpoint{-0.019763in}{0.034029in}}{\pgfqpoint{-0.026896in}{0.026896in}}%
\pgfpathcurveto{\pgfqpoint{-0.034029in}{0.019763in}}{\pgfqpoint{-0.038036in}{0.010087in}}{\pgfqpoint{-0.038036in}{0.000000in}}%
\pgfpathcurveto{\pgfqpoint{-0.038036in}{-0.010087in}}{\pgfqpoint{-0.034029in}{-0.019763in}}{\pgfqpoint{-0.026896in}{-0.026896in}}%
\pgfpathcurveto{\pgfqpoint{-0.019763in}{-0.034029in}}{\pgfqpoint{-0.010087in}{-0.038036in}}{\pgfqpoint{0.000000in}{-0.038036in}}%
\pgfpathlineto{\pgfqpoint{0.000000in}{-0.038036in}}%
\pgfpathclose%
\pgfusepath{stroke,fill}%
}%
\begin{pgfscope}%
\pgfsys@transformshift{0.330000in}{0.330000in}%
\pgfsys@useobject{currentmarker}{}%
\end{pgfscope}%
\end{pgfscope}%
\begin{pgfscope}%
\pgfpathrectangle{\pgfqpoint{0.150000in}{0.150000in}}{\pgfqpoint{1.800000in}{1.800000in}}%
\pgfusepath{clip}%
\pgfsetrectcap%
\pgfsetroundjoin%
\pgfsetlinewidth{2.007500pt}%
\definecolor{currentstroke}{rgb}{0.000000,0.000000,0.000000}%
\pgfsetstrokecolor{currentstroke}%
\pgfsetdash{}{0pt}%
\pgfpathmoveto{\pgfqpoint{0.690000in}{0.690000in}}%
\pgfpathlineto{\pgfqpoint{0.510000in}{1.230000in}}%
\pgfusepath{stroke}%
\end{pgfscope}%
\end{pgfpicture}%
\makeatother%
\endgroup%

				\subcaption{Visibility separator with a segment obstacle}
			\end{subfigure}%
			\hfill
			\begin{subfigure}[t]{.31\textwidth}
				\input{imgs/sepvisible_1.pgf}
				\subcaption{Visibility separator with a segment obstacle}
			\end{subfigure}%
			\hfill
			\begin{subfigure}[t]{.31\textwidth}
				%% Creator: Matplotlib, PGF backend
%%
%% To include the figure in your LaTeX document, write
%%   \input{<filename>.pgf}
%%
%% Make sure the required packages are loaded in your preamble
%%   \usepackage{pgf}
%%
%% Also ensure that all the required font packages are loaded; for instance,
%% the lmodern package is sometimes necessary when using math font.
%%   \usepackage{lmodern}
%%
%% Figures using additional raster images can only be included by \input if
%% they are in the same directory as the main LaTeX file. For loading figures
%% from other directories you can use the `import` package
%%   \usepackage{import}
%%
%% and then include the figures with
%%   \import{<path to file>}{<filename>.pgf}
%%
%% Matplotlib used the following preamble
%%
\begingroup%
\makeatletter%
\begin{pgfpicture}%
\pgfpathrectangle{\pgfpointorigin}{\pgfqpoint{2.100000in}{2.100000in}}%
\pgfusepath{use as bounding box, clip}%
\begin{pgfscope}%
\pgfsetbuttcap%
\pgfsetmiterjoin%
\definecolor{currentfill}{rgb}{1.000000,1.000000,1.000000}%
\pgfsetfillcolor{currentfill}%
\pgfsetlinewidth{0.000000pt}%
\definecolor{currentstroke}{rgb}{1.000000,1.000000,1.000000}%
\pgfsetstrokecolor{currentstroke}%
\pgfsetdash{}{0pt}%
\pgfpathmoveto{\pgfqpoint{0.000000in}{0.000000in}}%
\pgfpathlineto{\pgfqpoint{2.100000in}{0.000000in}}%
\pgfpathlineto{\pgfqpoint{2.100000in}{2.100000in}}%
\pgfpathlineto{\pgfqpoint{0.000000in}{2.100000in}}%
\pgfpathlineto{\pgfqpoint{0.000000in}{0.000000in}}%
\pgfpathclose%
\pgfusepath{fill}%
\end{pgfscope}%
\begin{pgfscope}%
\pgfpathrectangle{\pgfqpoint{0.150000in}{0.150000in}}{\pgfqpoint{1.800000in}{1.800000in}}%
\pgfusepath{clip}%
\pgfsetbuttcap%
\pgfsetroundjoin%
\definecolor{currentfill}{rgb}{0.933333,0.600000,0.666667}%
\pgfsetfillcolor{currentfill}%
\pgfsetlinewidth{1.003750pt}%
\definecolor{currentstroke}{rgb}{0.600000,0.266667,0.333333}%
\pgfsetstrokecolor{currentstroke}%
\pgfsetdash{}{0pt}%
\pgfpathmoveto{\pgfqpoint{1.455104in}{0.545318in}}%
\pgfpathlineto{\pgfqpoint{1.514400in}{0.545318in}}%
\pgfpathlineto{\pgfqpoint{1.514400in}{0.555021in}}%
\pgfpathlineto{\pgfqpoint{1.455104in}{0.555021in}}%
\pgfpathlineto{\pgfqpoint{1.455104in}{0.545318in}}%
\pgfpathclose%
\pgfusepath{stroke,fill}%
\end{pgfscope}%
\begin{pgfscope}%
\pgfpathrectangle{\pgfqpoint{0.150000in}{0.150000in}}{\pgfqpoint{1.800000in}{1.800000in}}%
\pgfusepath{clip}%
\pgfsetbuttcap%
\pgfsetroundjoin%
\definecolor{currentfill}{rgb}{0.933333,0.600000,0.666667}%
\pgfsetfillcolor{currentfill}%
\pgfsetlinewidth{1.003750pt}%
\definecolor{currentstroke}{rgb}{0.600000,0.266667,0.333333}%
\pgfsetstrokecolor{currentstroke}%
\pgfsetdash{}{0pt}%
\pgfpathmoveto{\pgfqpoint{0.529600in}{1.065186in}}%
\pgfpathlineto{\pgfqpoint{0.549036in}{1.065186in}}%
\pgfpathlineto{\pgfqpoint{0.549036in}{1.112892in}}%
\pgfpathlineto{\pgfqpoint{0.529600in}{1.112892in}}%
\pgfpathlineto{\pgfqpoint{0.529600in}{1.065186in}}%
\pgfpathclose%
\pgfusepath{stroke,fill}%
\end{pgfscope}%
\begin{pgfscope}%
\pgfpathrectangle{\pgfqpoint{0.150000in}{0.150000in}}{\pgfqpoint{1.800000in}{1.800000in}}%
\pgfusepath{clip}%
\pgfsetbuttcap%
\pgfsetroundjoin%
\definecolor{currentfill}{rgb}{0.933333,0.600000,0.666667}%
\pgfsetfillcolor{currentfill}%
\pgfsetlinewidth{1.003750pt}%
\definecolor{currentstroke}{rgb}{0.600000,0.266667,0.333333}%
\pgfsetstrokecolor{currentstroke}%
\pgfsetdash{}{0pt}%
\pgfpathmoveto{\pgfqpoint{0.750015in}{0.627790in}}%
\pgfpathlineto{\pgfqpoint{0.806992in}{0.627790in}}%
\pgfpathlineto{\pgfqpoint{0.806992in}{0.651003in}}%
\pgfpathlineto{\pgfqpoint{0.750015in}{0.651003in}}%
\pgfpathlineto{\pgfqpoint{0.750015in}{0.627790in}}%
\pgfpathclose%
\pgfusepath{stroke,fill}%
\end{pgfscope}%
\begin{pgfscope}%
\pgfpathrectangle{\pgfqpoint{0.150000in}{0.150000in}}{\pgfqpoint{1.800000in}{1.800000in}}%
\pgfusepath{clip}%
\pgfsetbuttcap%
\pgfsetroundjoin%
\definecolor{currentfill}{rgb}{0.933333,0.600000,0.666667}%
\pgfsetfillcolor{currentfill}%
\pgfsetlinewidth{1.003750pt}%
\definecolor{currentstroke}{rgb}{0.600000,0.266667,0.333333}%
\pgfsetstrokecolor{currentstroke}%
\pgfsetdash{}{0pt}%
\pgfpathmoveto{\pgfqpoint{1.877527in}{0.627646in}}%
\pgfpathlineto{\pgfqpoint{1.950000in}{0.627646in}}%
\pgfpathlineto{\pgfqpoint{1.950000in}{0.639505in}}%
\pgfpathlineto{\pgfqpoint{1.877527in}{0.639505in}}%
\pgfpathlineto{\pgfqpoint{1.877527in}{0.627646in}}%
\pgfpathclose%
\pgfusepath{stroke,fill}%
\end{pgfscope}%
\begin{pgfscope}%
\pgfpathrectangle{\pgfqpoint{0.150000in}{0.150000in}}{\pgfqpoint{1.800000in}{1.800000in}}%
\pgfusepath{clip}%
\pgfsetbuttcap%
\pgfsetroundjoin%
\definecolor{currentfill}{rgb}{0.933333,0.600000,0.666667}%
\pgfsetfillcolor{currentfill}%
\pgfsetlinewidth{1.003750pt}%
\definecolor{currentstroke}{rgb}{0.600000,0.266667,0.333333}%
\pgfsetstrokecolor{currentstroke}%
\pgfsetdash{}{0pt}%
\pgfpathmoveto{\pgfqpoint{1.758935in}{0.606084in}}%
\pgfpathlineto{\pgfqpoint{1.818231in}{0.606084in}}%
\pgfpathlineto{\pgfqpoint{1.818231in}{0.615787in}}%
\pgfpathlineto{\pgfqpoint{1.758935in}{0.615787in}}%
\pgfpathlineto{\pgfqpoint{1.758935in}{0.606084in}}%
\pgfpathclose%
\pgfusepath{stroke,fill}%
\end{pgfscope}%
\begin{pgfscope}%
\pgfpathrectangle{\pgfqpoint{0.150000in}{0.150000in}}{\pgfqpoint{1.800000in}{1.800000in}}%
\pgfusepath{clip}%
\pgfsetbuttcap%
\pgfsetroundjoin%
\definecolor{currentfill}{rgb}{0.933333,0.600000,0.666667}%
\pgfsetfillcolor{currentfill}%
\pgfsetlinewidth{1.003750pt}%
\definecolor{currentstroke}{rgb}{0.600000,0.266667,0.333333}%
\pgfsetstrokecolor{currentstroke}%
\pgfsetdash{}{0pt}%
\pgfpathmoveto{\pgfqpoint{1.651124in}{0.584522in}}%
\pgfpathlineto{\pgfqpoint{1.710420in}{0.584522in}}%
\pgfpathlineto{\pgfqpoint{1.710420in}{0.594225in}}%
\pgfpathlineto{\pgfqpoint{1.651124in}{0.594225in}}%
\pgfpathlineto{\pgfqpoint{1.651124in}{0.584522in}}%
\pgfpathclose%
\pgfusepath{stroke,fill}%
\end{pgfscope}%
\begin{pgfscope}%
\pgfpathrectangle{\pgfqpoint{0.150000in}{0.150000in}}{\pgfqpoint{1.800000in}{1.800000in}}%
\pgfusepath{clip}%
\pgfsetbuttcap%
\pgfsetroundjoin%
\definecolor{currentfill}{rgb}{0.933333,0.600000,0.666667}%
\pgfsetfillcolor{currentfill}%
\pgfsetlinewidth{1.003750pt}%
\definecolor{currentstroke}{rgb}{0.600000,0.266667,0.333333}%
\pgfsetstrokecolor{currentstroke}%
\pgfsetdash{}{0pt}%
\pgfpathmoveto{\pgfqpoint{1.406589in}{0.527676in}}%
\pgfpathlineto{\pgfqpoint{1.514400in}{0.527676in}}%
\pgfpathlineto{\pgfqpoint{1.514400in}{0.545318in}}%
\pgfpathlineto{\pgfqpoint{1.406589in}{0.545318in}}%
\pgfpathlineto{\pgfqpoint{1.406589in}{0.527676in}}%
\pgfpathclose%
\pgfusepath{stroke,fill}%
\end{pgfscope}%
\begin{pgfscope}%
\pgfpathrectangle{\pgfqpoint{0.150000in}{0.150000in}}{\pgfqpoint{1.800000in}{1.800000in}}%
\pgfusepath{clip}%
\pgfsetbuttcap%
\pgfsetroundjoin%
\definecolor{currentfill}{rgb}{0.933333,0.600000,0.666667}%
\pgfsetfillcolor{currentfill}%
\pgfsetlinewidth{1.003750pt}%
\definecolor{currentstroke}{rgb}{0.600000,0.266667,0.333333}%
\pgfsetstrokecolor{currentstroke}%
\pgfsetdash{}{0pt}%
\pgfpathmoveto{\pgfqpoint{1.230171in}{0.510000in}}%
\pgfpathlineto{\pgfqpoint{1.318380in}{0.510000in}}%
\pgfpathlineto{\pgfqpoint{1.318380in}{0.510034in}}%
\pgfpathlineto{\pgfqpoint{1.230171in}{0.510034in}}%
\pgfpathlineto{\pgfqpoint{1.230171in}{0.510000in}}%
\pgfpathclose%
\pgfusepath{stroke,fill}%
\end{pgfscope}%
\begin{pgfscope}%
\pgfpathrectangle{\pgfqpoint{0.150000in}{0.150000in}}{\pgfqpoint{1.800000in}{1.800000in}}%
\pgfusepath{clip}%
\pgfsetbuttcap%
\pgfsetroundjoin%
\definecolor{currentfill}{rgb}{0.933333,0.600000,0.666667}%
\pgfsetfillcolor{currentfill}%
\pgfsetlinewidth{1.003750pt}%
\definecolor{currentstroke}{rgb}{0.600000,0.266667,0.333333}%
\pgfsetstrokecolor{currentstroke}%
\pgfsetdash{}{0pt}%
\pgfpathmoveto{\pgfqpoint{0.628085in}{1.878735in}}%
\pgfpathlineto{\pgfqpoint{0.639747in}{1.878735in}}%
\pgfpathlineto{\pgfqpoint{0.639747in}{1.950000in}}%
\pgfpathlineto{\pgfqpoint{0.628085in}{1.950000in}}%
\pgfpathlineto{\pgfqpoint{0.628085in}{1.878735in}}%
\pgfpathclose%
\pgfusepath{stroke,fill}%
\end{pgfscope}%
\begin{pgfscope}%
\pgfpathrectangle{\pgfqpoint{0.150000in}{0.150000in}}{\pgfqpoint{1.800000in}{1.800000in}}%
\pgfusepath{clip}%
\pgfsetbuttcap%
\pgfsetroundjoin%
\definecolor{currentfill}{rgb}{0.933333,0.600000,0.666667}%
\pgfsetfillcolor{currentfill}%
\pgfsetlinewidth{1.003750pt}%
\definecolor{currentstroke}{rgb}{0.600000,0.266667,0.333333}%
\pgfsetstrokecolor{currentstroke}%
\pgfsetdash{}{0pt}%
\pgfpathmoveto{\pgfqpoint{0.606883in}{1.762119in}}%
\pgfpathlineto{\pgfqpoint{0.616424in}{1.762119in}}%
\pgfpathlineto{\pgfqpoint{0.616424in}{1.820427in}}%
\pgfpathlineto{\pgfqpoint{0.606883in}{1.820427in}}%
\pgfpathlineto{\pgfqpoint{0.606883in}{1.762119in}}%
\pgfpathclose%
\pgfusepath{stroke,fill}%
\end{pgfscope}%
\begin{pgfscope}%
\pgfpathrectangle{\pgfqpoint{0.150000in}{0.150000in}}{\pgfqpoint{1.800000in}{1.800000in}}%
\pgfusepath{clip}%
\pgfsetbuttcap%
\pgfsetroundjoin%
\definecolor{currentfill}{rgb}{0.933333,0.600000,0.666667}%
\pgfsetfillcolor{currentfill}%
\pgfsetlinewidth{1.003750pt}%
\definecolor{currentstroke}{rgb}{0.600000,0.266667,0.333333}%
\pgfsetstrokecolor{currentstroke}%
\pgfsetdash{}{0pt}%
\pgfpathmoveto{\pgfqpoint{0.585680in}{1.656105in}}%
\pgfpathlineto{\pgfqpoint{0.595221in}{1.656105in}}%
\pgfpathlineto{\pgfqpoint{0.595221in}{1.714413in}}%
\pgfpathlineto{\pgfqpoint{0.585680in}{1.714413in}}%
\pgfpathlineto{\pgfqpoint{0.585680in}{1.656105in}}%
\pgfpathclose%
\pgfusepath{stroke,fill}%
\end{pgfscope}%
\begin{pgfscope}%
\pgfpathrectangle{\pgfqpoint{0.150000in}{0.150000in}}{\pgfqpoint{1.800000in}{1.800000in}}%
\pgfusepath{clip}%
\pgfsetbuttcap%
\pgfsetroundjoin%
\definecolor{currentfill}{rgb}{0.933333,0.600000,0.666667}%
\pgfsetfillcolor{currentfill}%
\pgfsetlinewidth{1.003750pt}%
\definecolor{currentstroke}{rgb}{0.600000,0.266667,0.333333}%
\pgfsetstrokecolor{currentstroke}%
\pgfsetdash{}{0pt}%
\pgfpathmoveto{\pgfqpoint{0.547129in}{1.463352in}}%
\pgfpathlineto{\pgfqpoint{0.556670in}{1.463352in}}%
\pgfpathlineto{\pgfqpoint{0.556670in}{1.521660in}}%
\pgfpathlineto{\pgfqpoint{0.547129in}{1.521660in}}%
\pgfpathlineto{\pgfqpoint{0.547129in}{1.463352in}}%
\pgfpathclose%
\pgfusepath{stroke,fill}%
\end{pgfscope}%
\begin{pgfscope}%
\pgfpathrectangle{\pgfqpoint{0.150000in}{0.150000in}}{\pgfqpoint{1.800000in}{1.800000in}}%
\pgfusepath{clip}%
\pgfsetbuttcap%
\pgfsetroundjoin%
\definecolor{currentfill}{rgb}{0.933333,0.600000,0.666667}%
\pgfsetfillcolor{currentfill}%
\pgfsetlinewidth{1.003750pt}%
\definecolor{currentstroke}{rgb}{0.600000,0.266667,0.333333}%
\pgfsetstrokecolor{currentstroke}%
\pgfsetdash{}{0pt}%
\pgfpathmoveto{\pgfqpoint{0.529600in}{0.978447in}}%
\pgfpathlineto{\pgfqpoint{0.564938in}{0.978447in}}%
\pgfpathlineto{\pgfqpoint{0.564938in}{1.065186in}}%
\pgfpathlineto{\pgfqpoint{0.529600in}{1.065186in}}%
\pgfpathlineto{\pgfqpoint{0.529600in}{0.978447in}}%
\pgfpathclose%
\pgfusepath{stroke,fill}%
\end{pgfscope}%
\begin{pgfscope}%
\pgfpathrectangle{\pgfqpoint{0.150000in}{0.150000in}}{\pgfqpoint{1.800000in}{1.800000in}}%
\pgfusepath{clip}%
\pgfsetbuttcap%
\pgfsetroundjoin%
\definecolor{currentfill}{rgb}{0.933333,0.600000,0.666667}%
\pgfsetfillcolor{currentfill}%
\pgfsetlinewidth{1.003750pt}%
\definecolor{currentstroke}{rgb}{0.600000,0.266667,0.333333}%
\pgfsetstrokecolor{currentstroke}%
\pgfsetdash{}{0pt}%
\pgfpathmoveto{\pgfqpoint{0.593851in}{0.820740in}}%
\pgfpathlineto{\pgfqpoint{0.622764in}{0.820740in}}%
\pgfpathlineto{\pgfqpoint{0.622764in}{0.891708in}}%
\pgfpathlineto{\pgfqpoint{0.593851in}{0.891708in}}%
\pgfpathlineto{\pgfqpoint{0.593851in}{0.820740in}}%
\pgfpathclose%
\pgfusepath{stroke,fill}%
\end{pgfscope}%
\begin{pgfscope}%
\pgfpathrectangle{\pgfqpoint{0.150000in}{0.150000in}}{\pgfqpoint{1.800000in}{1.800000in}}%
\pgfusepath{clip}%
\pgfsetbuttcap%
\pgfsetroundjoin%
\definecolor{currentfill}{rgb}{0.933333,0.600000,0.666667}%
\pgfsetfillcolor{currentfill}%
\pgfsetlinewidth{1.003750pt}%
\definecolor{currentstroke}{rgb}{0.600000,0.266667,0.333333}%
\pgfsetstrokecolor{currentstroke}%
\pgfsetdash{}{0pt}%
\pgfpathmoveto{\pgfqpoint{1.003247in}{0.534000in}}%
\pgfpathlineto{\pgfqpoint{1.072886in}{0.534000in}}%
\pgfpathlineto{\pgfqpoint{1.072886in}{0.562371in}}%
\pgfpathlineto{\pgfqpoint{1.003247in}{0.562371in}}%
\pgfpathlineto{\pgfqpoint{1.003247in}{0.534000in}}%
\pgfpathclose%
\pgfusepath{stroke,fill}%
\end{pgfscope}%
\begin{pgfscope}%
\pgfpathrectangle{\pgfqpoint{0.150000in}{0.150000in}}{\pgfqpoint{1.800000in}{1.800000in}}%
\pgfusepath{clip}%
\pgfsetbuttcap%
\pgfsetroundjoin%
\definecolor{currentfill}{rgb}{0.933333,0.600000,0.666667}%
\pgfsetfillcolor{currentfill}%
\pgfsetlinewidth{1.003750pt}%
\definecolor{currentstroke}{rgb}{0.600000,0.266667,0.333333}%
\pgfsetstrokecolor{currentstroke}%
\pgfsetdash{}{0pt}%
\pgfpathmoveto{\pgfqpoint{0.876631in}{0.585584in}}%
\pgfpathlineto{\pgfqpoint{0.933608in}{0.585584in}}%
\pgfpathlineto{\pgfqpoint{0.933608in}{0.608797in}}%
\pgfpathlineto{\pgfqpoint{0.876631in}{0.608797in}}%
\pgfpathlineto{\pgfqpoint{0.876631in}{0.585584in}}%
\pgfpathclose%
\pgfusepath{stroke,fill}%
\end{pgfscope}%
\begin{pgfscope}%
\pgfpathrectangle{\pgfqpoint{0.150000in}{0.150000in}}{\pgfqpoint{1.800000in}{1.800000in}}%
\pgfusepath{clip}%
\pgfsetbuttcap%
\pgfsetroundjoin%
\definecolor{currentfill}{rgb}{0.933333,0.600000,0.666667}%
\pgfsetfillcolor{currentfill}%
\pgfsetlinewidth{1.003750pt}%
\definecolor{currentstroke}{rgb}{0.600000,0.266667,0.333333}%
\pgfsetstrokecolor{currentstroke}%
\pgfsetdash{}{0pt}%
\pgfpathmoveto{\pgfqpoint{0.646420in}{0.669995in}}%
\pgfpathlineto{\pgfqpoint{0.674057in}{0.669995in}}%
\pgfpathlineto{\pgfqpoint{0.674057in}{0.737830in}}%
\pgfpathlineto{\pgfqpoint{0.646420in}{0.737830in}}%
\pgfpathlineto{\pgfqpoint{0.646420in}{0.669995in}}%
\pgfpathclose%
\pgfusepath{stroke,fill}%
\end{pgfscope}%
\begin{pgfscope}%
\pgfpathrectangle{\pgfqpoint{0.150000in}{0.150000in}}{\pgfqpoint{1.800000in}{1.800000in}}%
\pgfusepath{clip}%
\pgfsetbuttcap%
\pgfsetroundjoin%
\definecolor{currentfill}{rgb}{0.933333,0.600000,0.666667}%
\pgfsetfillcolor{currentfill}%
\pgfsetlinewidth{1.003750pt}%
\definecolor{currentstroke}{rgb}{0.600000,0.266667,0.333333}%
\pgfsetstrokecolor{currentstroke}%
\pgfsetdash{}{0pt}%
\pgfpathmoveto{\pgfqpoint{1.818231in}{0.606084in}}%
\pgfpathlineto{\pgfqpoint{1.950000in}{0.606084in}}%
\pgfpathlineto{\pgfqpoint{1.950000in}{0.627646in}}%
\pgfpathlineto{\pgfqpoint{1.818231in}{0.627646in}}%
\pgfpathlineto{\pgfqpoint{1.818231in}{0.606084in}}%
\pgfpathclose%
\pgfusepath{stroke,fill}%
\end{pgfscope}%
\begin{pgfscope}%
\pgfpathrectangle{\pgfqpoint{0.150000in}{0.150000in}}{\pgfqpoint{1.800000in}{1.800000in}}%
\pgfusepath{clip}%
\pgfsetbuttcap%
\pgfsetroundjoin%
\definecolor{currentfill}{rgb}{0.933333,0.600000,0.666667}%
\pgfsetfillcolor{currentfill}%
\pgfsetlinewidth{1.003750pt}%
\definecolor{currentstroke}{rgb}{0.600000,0.266667,0.333333}%
\pgfsetstrokecolor{currentstroke}%
\pgfsetdash{}{0pt}%
\pgfpathmoveto{\pgfqpoint{1.602609in}{0.566880in}}%
\pgfpathlineto{\pgfqpoint{1.710420in}{0.566880in}}%
\pgfpathlineto{\pgfqpoint{1.710420in}{0.584522in}}%
\pgfpathlineto{\pgfqpoint{1.602609in}{0.584522in}}%
\pgfpathlineto{\pgfqpoint{1.602609in}{0.566880in}}%
\pgfpathclose%
\pgfusepath{stroke,fill}%
\end{pgfscope}%
\begin{pgfscope}%
\pgfpathrectangle{\pgfqpoint{0.150000in}{0.150000in}}{\pgfqpoint{1.800000in}{1.800000in}}%
\pgfusepath{clip}%
\pgfsetbuttcap%
\pgfsetroundjoin%
\definecolor{currentfill}{rgb}{0.933333,0.600000,0.666667}%
\pgfsetfillcolor{currentfill}%
\pgfsetlinewidth{1.003750pt}%
\definecolor{currentstroke}{rgb}{0.600000,0.266667,0.333333}%
\pgfsetstrokecolor{currentstroke}%
\pgfsetdash{}{0pt}%
\pgfpathmoveto{\pgfqpoint{1.318380in}{0.510000in}}%
\pgfpathlineto{\pgfqpoint{1.514400in}{0.510000in}}%
\pgfpathlineto{\pgfqpoint{1.514400in}{0.527676in}}%
\pgfpathlineto{\pgfqpoint{1.318380in}{0.527676in}}%
\pgfpathlineto{\pgfqpoint{1.318380in}{0.510000in}}%
\pgfpathclose%
\pgfusepath{stroke,fill}%
\end{pgfscope}%
\begin{pgfscope}%
\pgfpathrectangle{\pgfqpoint{0.150000in}{0.150000in}}{\pgfqpoint{1.800000in}{1.800000in}}%
\pgfusepath{clip}%
\pgfsetbuttcap%
\pgfsetroundjoin%
\definecolor{currentfill}{rgb}{0.933333,0.600000,0.666667}%
\pgfsetfillcolor{currentfill}%
\pgfsetlinewidth{1.003750pt}%
\definecolor{currentstroke}{rgb}{0.600000,0.266667,0.333333}%
\pgfsetstrokecolor{currentstroke}%
\pgfsetdash{}{0pt}%
\pgfpathmoveto{\pgfqpoint{0.606883in}{1.820427in}}%
\pgfpathlineto{\pgfqpoint{0.628085in}{1.820427in}}%
\pgfpathlineto{\pgfqpoint{0.628085in}{1.950000in}}%
\pgfpathlineto{\pgfqpoint{0.606883in}{1.950000in}}%
\pgfpathlineto{\pgfqpoint{0.606883in}{1.820427in}}%
\pgfpathclose%
\pgfusepath{stroke,fill}%
\end{pgfscope}%
\begin{pgfscope}%
\pgfpathrectangle{\pgfqpoint{0.150000in}{0.150000in}}{\pgfqpoint{1.800000in}{1.800000in}}%
\pgfusepath{clip}%
\pgfsetbuttcap%
\pgfsetroundjoin%
\definecolor{currentfill}{rgb}{0.933333,0.600000,0.666667}%
\pgfsetfillcolor{currentfill}%
\pgfsetlinewidth{1.003750pt}%
\definecolor{currentstroke}{rgb}{0.600000,0.266667,0.333333}%
\pgfsetstrokecolor{currentstroke}%
\pgfsetdash{}{0pt}%
\pgfpathmoveto{\pgfqpoint{0.568332in}{1.608399in}}%
\pgfpathlineto{\pgfqpoint{0.585680in}{1.608399in}}%
\pgfpathlineto{\pgfqpoint{0.585680in}{1.714413in}}%
\pgfpathlineto{\pgfqpoint{0.568332in}{1.714413in}}%
\pgfpathlineto{\pgfqpoint{0.568332in}{1.608399in}}%
\pgfpathclose%
\pgfusepath{stroke,fill}%
\end{pgfscope}%
\begin{pgfscope}%
\pgfpathrectangle{\pgfqpoint{0.150000in}{0.150000in}}{\pgfqpoint{1.800000in}{1.800000in}}%
\pgfusepath{clip}%
\pgfsetbuttcap%
\pgfsetroundjoin%
\definecolor{currentfill}{rgb}{0.933333,0.600000,0.666667}%
\pgfsetfillcolor{currentfill}%
\pgfsetlinewidth{1.003750pt}%
\definecolor{currentstroke}{rgb}{0.600000,0.266667,0.333333}%
\pgfsetstrokecolor{currentstroke}%
\pgfsetdash{}{0pt}%
\pgfpathmoveto{\pgfqpoint{0.529781in}{1.415646in}}%
\pgfpathlineto{\pgfqpoint{0.547129in}{1.415646in}}%
\pgfpathlineto{\pgfqpoint{0.547129in}{1.521660in}}%
\pgfpathlineto{\pgfqpoint{0.529781in}{1.521660in}}%
\pgfpathlineto{\pgfqpoint{0.529781in}{1.415646in}}%
\pgfpathclose%
\pgfusepath{stroke,fill}%
\end{pgfscope}%
\begin{pgfscope}%
\pgfpathrectangle{\pgfqpoint{0.150000in}{0.150000in}}{\pgfqpoint{1.800000in}{1.800000in}}%
\pgfusepath{clip}%
\pgfsetbuttcap%
\pgfsetroundjoin%
\definecolor{currentfill}{rgb}{0.933333,0.600000,0.666667}%
\pgfsetfillcolor{currentfill}%
\pgfsetlinewidth{1.003750pt}%
\definecolor{currentstroke}{rgb}{0.600000,0.266667,0.333333}%
\pgfsetstrokecolor{currentstroke}%
\pgfsetdash{}{0pt}%
\pgfpathmoveto{\pgfqpoint{0.510000in}{1.242168in}}%
\pgfpathlineto{\pgfqpoint{0.512434in}{1.242168in}}%
\pgfpathlineto{\pgfqpoint{0.512434in}{1.328907in}}%
\pgfpathlineto{\pgfqpoint{0.510000in}{1.328907in}}%
\pgfpathlineto{\pgfqpoint{0.510000in}{1.242168in}}%
\pgfpathclose%
\pgfusepath{stroke,fill}%
\end{pgfscope}%
\begin{pgfscope}%
\pgfpathrectangle{\pgfqpoint{0.150000in}{0.150000in}}{\pgfqpoint{1.800000in}{1.800000in}}%
\pgfusepath{clip}%
\pgfsetbuttcap%
\pgfsetroundjoin%
\definecolor{currentfill}{rgb}{0.933333,0.600000,0.666667}%
\pgfsetfillcolor{currentfill}%
\pgfsetlinewidth{1.003750pt}%
\definecolor{currentstroke}{rgb}{0.600000,0.266667,0.333333}%
\pgfsetstrokecolor{currentstroke}%
\pgfsetdash{}{0pt}%
\pgfpathmoveto{\pgfqpoint{0.529600in}{0.820740in}}%
\pgfpathlineto{\pgfqpoint{0.593851in}{0.820740in}}%
\pgfpathlineto{\pgfqpoint{0.593851in}{0.978447in}}%
\pgfpathlineto{\pgfqpoint{0.529600in}{0.978447in}}%
\pgfpathlineto{\pgfqpoint{0.529600in}{0.820740in}}%
\pgfpathclose%
\pgfusepath{stroke,fill}%
\end{pgfscope}%
\begin{pgfscope}%
\pgfpathrectangle{\pgfqpoint{0.150000in}{0.150000in}}{\pgfqpoint{1.800000in}{1.800000in}}%
\pgfusepath{clip}%
\pgfsetbuttcap%
\pgfsetroundjoin%
\definecolor{currentfill}{rgb}{0.933333,0.600000,0.666667}%
\pgfsetfillcolor{currentfill}%
\pgfsetlinewidth{1.003750pt}%
\definecolor{currentstroke}{rgb}{0.600000,0.266667,0.333333}%
\pgfsetstrokecolor{currentstroke}%
\pgfsetdash{}{0pt}%
\pgfpathmoveto{\pgfqpoint{0.876631in}{0.534000in}}%
\pgfpathlineto{\pgfqpoint{1.003247in}{0.534000in}}%
\pgfpathlineto{\pgfqpoint{1.003247in}{0.585584in}}%
\pgfpathlineto{\pgfqpoint{0.876631in}{0.585584in}}%
\pgfpathlineto{\pgfqpoint{0.876631in}{0.534000in}}%
\pgfpathclose%
\pgfusepath{stroke,fill}%
\end{pgfscope}%
\begin{pgfscope}%
\pgfpathrectangle{\pgfqpoint{0.150000in}{0.150000in}}{\pgfqpoint{1.800000in}{1.800000in}}%
\pgfusepath{clip}%
\pgfsetbuttcap%
\pgfsetroundjoin%
\definecolor{currentfill}{rgb}{0.933333,0.600000,0.666667}%
\pgfsetfillcolor{currentfill}%
\pgfsetlinewidth{1.003750pt}%
\definecolor{currentstroke}{rgb}{0.600000,0.266667,0.333333}%
\pgfsetstrokecolor{currentstroke}%
\pgfsetdash{}{0pt}%
\pgfpathmoveto{\pgfqpoint{0.646420in}{0.627790in}}%
\pgfpathlineto{\pgfqpoint{0.750015in}{0.627790in}}%
\pgfpathlineto{\pgfqpoint{0.750015in}{0.669995in}}%
\pgfpathlineto{\pgfqpoint{0.646420in}{0.669995in}}%
\pgfpathlineto{\pgfqpoint{0.646420in}{0.627790in}}%
\pgfpathclose%
\pgfusepath{stroke,fill}%
\end{pgfscope}%
\begin{pgfscope}%
\pgfpathrectangle{\pgfqpoint{0.150000in}{0.150000in}}{\pgfqpoint{1.800000in}{1.800000in}}%
\pgfusepath{clip}%
\pgfsetbuttcap%
\pgfsetroundjoin%
\definecolor{currentfill}{rgb}{0.933333,0.600000,0.666667}%
\pgfsetfillcolor{currentfill}%
\pgfsetlinewidth{1.003750pt}%
\definecolor{currentstroke}{rgb}{0.600000,0.266667,0.333333}%
\pgfsetstrokecolor{currentstroke}%
\pgfsetdash{}{0pt}%
\pgfpathmoveto{\pgfqpoint{1.710420in}{0.566880in}}%
\pgfpathlineto{\pgfqpoint{1.950000in}{0.566880in}}%
\pgfpathlineto{\pgfqpoint{1.950000in}{0.606084in}}%
\pgfpathlineto{\pgfqpoint{1.710420in}{0.606084in}}%
\pgfpathlineto{\pgfqpoint{1.710420in}{0.566880in}}%
\pgfpathclose%
\pgfusepath{stroke,fill}%
\end{pgfscope}%
\begin{pgfscope}%
\pgfpathrectangle{\pgfqpoint{0.150000in}{0.150000in}}{\pgfqpoint{1.800000in}{1.800000in}}%
\pgfusepath{clip}%
\pgfsetbuttcap%
\pgfsetroundjoin%
\definecolor{currentfill}{rgb}{0.933333,0.600000,0.666667}%
\pgfsetfillcolor{currentfill}%
\pgfsetlinewidth{1.003750pt}%
\definecolor{currentstroke}{rgb}{0.600000,0.266667,0.333333}%
\pgfsetstrokecolor{currentstroke}%
\pgfsetdash{}{0pt}%
\pgfpathmoveto{\pgfqpoint{0.568332in}{1.714413in}}%
\pgfpathlineto{\pgfqpoint{0.606883in}{1.714413in}}%
\pgfpathlineto{\pgfqpoint{0.606883in}{1.950000in}}%
\pgfpathlineto{\pgfqpoint{0.568332in}{1.950000in}}%
\pgfpathlineto{\pgfqpoint{0.568332in}{1.714413in}}%
\pgfpathclose%
\pgfusepath{stroke,fill}%
\end{pgfscope}%
\begin{pgfscope}%
\pgfpathrectangle{\pgfqpoint{0.150000in}{0.150000in}}{\pgfqpoint{1.800000in}{1.800000in}}%
\pgfusepath{clip}%
\pgfsetbuttcap%
\pgfsetroundjoin%
\definecolor{currentfill}{rgb}{0.933333,0.600000,0.666667}%
\pgfsetfillcolor{currentfill}%
\pgfsetlinewidth{1.003750pt}%
\definecolor{currentstroke}{rgb}{0.600000,0.266667,0.333333}%
\pgfsetstrokecolor{currentstroke}%
\pgfsetdash{}{0pt}%
\pgfpathmoveto{\pgfqpoint{0.510000in}{1.328907in}}%
\pgfpathlineto{\pgfqpoint{0.529781in}{1.328907in}}%
\pgfpathlineto{\pgfqpoint{0.529781in}{1.521660in}}%
\pgfpathlineto{\pgfqpoint{0.510000in}{1.521660in}}%
\pgfpathlineto{\pgfqpoint{0.510000in}{1.328907in}}%
\pgfpathclose%
\pgfusepath{stroke,fill}%
\end{pgfscope}%
\begin{pgfscope}%
\pgfpathrectangle{\pgfqpoint{0.150000in}{0.150000in}}{\pgfqpoint{1.800000in}{1.800000in}}%
\pgfusepath{clip}%
\pgfsetbuttcap%
\pgfsetroundjoin%
\definecolor{currentfill}{rgb}{0.933333,0.600000,0.666667}%
\pgfsetfillcolor{currentfill}%
\pgfsetlinewidth{1.003750pt}%
\definecolor{currentstroke}{rgb}{0.600000,0.266667,0.333333}%
\pgfsetstrokecolor{currentstroke}%
\pgfsetdash{}{0pt}%
\pgfpathmoveto{\pgfqpoint{0.646420in}{0.534000in}}%
\pgfpathlineto{\pgfqpoint{0.876631in}{0.534000in}}%
\pgfpathlineto{\pgfqpoint{0.876631in}{0.627790in}}%
\pgfpathlineto{\pgfqpoint{0.646420in}{0.627790in}}%
\pgfpathlineto{\pgfqpoint{0.646420in}{0.534000in}}%
\pgfpathclose%
\pgfusepath{stroke,fill}%
\end{pgfscope}%
\begin{pgfscope}%
\pgfpathrectangle{\pgfqpoint{0.150000in}{0.150000in}}{\pgfqpoint{1.800000in}{1.800000in}}%
\pgfusepath{clip}%
\pgfsetbuttcap%
\pgfsetroundjoin%
\definecolor{currentfill}{rgb}{0.933333,0.600000,0.666667}%
\pgfsetfillcolor{currentfill}%
\pgfsetlinewidth{1.003750pt}%
\definecolor{currentstroke}{rgb}{0.600000,0.266667,0.333333}%
\pgfsetstrokecolor{currentstroke}%
\pgfsetdash{}{0pt}%
\pgfpathmoveto{\pgfqpoint{1.514400in}{0.510000in}}%
\pgfpathlineto{\pgfqpoint{1.950000in}{0.510000in}}%
\pgfpathlineto{\pgfqpoint{1.950000in}{0.566880in}}%
\pgfpathlineto{\pgfqpoint{1.514400in}{0.566880in}}%
\pgfpathlineto{\pgfqpoint{1.514400in}{0.510000in}}%
\pgfpathclose%
\pgfusepath{stroke,fill}%
\end{pgfscope}%
\begin{pgfscope}%
\pgfpathrectangle{\pgfqpoint{0.150000in}{0.150000in}}{\pgfqpoint{1.800000in}{1.800000in}}%
\pgfusepath{clip}%
\pgfsetbuttcap%
\pgfsetroundjoin%
\definecolor{currentfill}{rgb}{0.933333,0.600000,0.666667}%
\pgfsetfillcolor{currentfill}%
\pgfsetlinewidth{1.003750pt}%
\definecolor{currentstroke}{rgb}{0.600000,0.266667,0.333333}%
\pgfsetstrokecolor{currentstroke}%
\pgfsetdash{}{0pt}%
\pgfpathmoveto{\pgfqpoint{0.510000in}{1.521660in}}%
\pgfpathlineto{\pgfqpoint{0.568332in}{1.521660in}}%
\pgfpathlineto{\pgfqpoint{0.568332in}{1.950000in}}%
\pgfpathlineto{\pgfqpoint{0.510000in}{1.950000in}}%
\pgfpathlineto{\pgfqpoint{0.510000in}{1.521660in}}%
\pgfpathclose%
\pgfusepath{stroke,fill}%
\end{pgfscope}%
\begin{pgfscope}%
\pgfpathrectangle{\pgfqpoint{0.150000in}{0.150000in}}{\pgfqpoint{1.800000in}{1.800000in}}%
\pgfusepath{clip}%
\pgfsetbuttcap%
\pgfsetroundjoin%
\definecolor{currentfill}{rgb}{0.933333,0.600000,0.666667}%
\pgfsetfillcolor{currentfill}%
\pgfsetlinewidth{1.003750pt}%
\definecolor{currentstroke}{rgb}{0.600000,0.266667,0.333333}%
\pgfsetstrokecolor{currentstroke}%
\pgfsetdash{}{0pt}%
\pgfpathmoveto{\pgfqpoint{0.529600in}{0.534000in}}%
\pgfpathlineto{\pgfqpoint{0.646420in}{0.534000in}}%
\pgfpathlineto{\pgfqpoint{0.646420in}{0.820740in}}%
\pgfpathlineto{\pgfqpoint{0.529600in}{0.820740in}}%
\pgfpathlineto{\pgfqpoint{0.529600in}{0.534000in}}%
\pgfpathclose%
\pgfusepath{stroke,fill}%
\end{pgfscope}%
\begin{pgfscope}%
\pgfpathrectangle{\pgfqpoint{0.150000in}{0.150000in}}{\pgfqpoint{1.800000in}{1.800000in}}%
\pgfusepath{clip}%
\pgfsetbuttcap%
\pgfsetroundjoin%
\definecolor{currentfill}{rgb}{0.933333,0.600000,0.666667}%
\pgfsetfillcolor{currentfill}%
\pgfsetlinewidth{1.003750pt}%
\definecolor{currentstroke}{rgb}{0.600000,0.266667,0.333333}%
\pgfsetstrokecolor{currentstroke}%
\pgfsetdash{}{0pt}%
\pgfpathmoveto{\pgfqpoint{0.510000in}{0.534000in}}%
\pgfpathlineto{\pgfqpoint{0.529600in}{0.534000in}}%
\pgfpathlineto{\pgfqpoint{0.529600in}{1.171200in}}%
\pgfpathlineto{\pgfqpoint{0.510000in}{1.171200in}}%
\pgfpathlineto{\pgfqpoint{0.510000in}{0.534000in}}%
\pgfpathclose%
\pgfusepath{stroke,fill}%
\end{pgfscope}%
\begin{pgfscope}%
\pgfpathrectangle{\pgfqpoint{0.150000in}{0.150000in}}{\pgfqpoint{1.800000in}{1.800000in}}%
\pgfusepath{clip}%
\pgfsetbuttcap%
\pgfsetroundjoin%
\definecolor{currentfill}{rgb}{0.933333,0.600000,0.666667}%
\pgfsetfillcolor{currentfill}%
\pgfsetlinewidth{1.003750pt}%
\definecolor{currentstroke}{rgb}{0.600000,0.266667,0.333333}%
\pgfsetstrokecolor{currentstroke}%
\pgfsetdash{}{0pt}%
\pgfpathmoveto{\pgfqpoint{0.510000in}{0.510000in}}%
\pgfpathlineto{\pgfqpoint{1.158000in}{0.510000in}}%
\pgfpathlineto{\pgfqpoint{1.158000in}{0.534000in}}%
\pgfpathlineto{\pgfqpoint{0.510000in}{0.534000in}}%
\pgfpathlineto{\pgfqpoint{0.510000in}{0.510000in}}%
\pgfpathclose%
\pgfusepath{stroke,fill}%
\end{pgfscope}%
\begin{pgfscope}%
\pgfpathrectangle{\pgfqpoint{0.150000in}{0.150000in}}{\pgfqpoint{1.800000in}{1.800000in}}%
\pgfusepath{clip}%
\pgfsetbuttcap%
\pgfsetroundjoin%
\definecolor{currentfill}{rgb}{0.933333,0.600000,0.666667}%
\pgfsetfillcolor{currentfill}%
\pgfsetlinewidth{1.003750pt}%
\definecolor{currentstroke}{rgb}{0.600000,0.266667,0.333333}%
\pgfsetstrokecolor{currentstroke}%
\pgfsetdash{}{0pt}%
\pgfpathmoveto{\pgfqpoint{0.510000in}{0.150000in}}%
\pgfpathlineto{\pgfqpoint{1.950000in}{0.150000in}}%
\pgfpathlineto{\pgfqpoint{1.950000in}{0.510000in}}%
\pgfpathlineto{\pgfqpoint{0.510000in}{0.510000in}}%
\pgfpathlineto{\pgfqpoint{0.510000in}{0.150000in}}%
\pgfpathclose%
\pgfusepath{stroke,fill}%
\end{pgfscope}%
\begin{pgfscope}%
\pgfpathrectangle{\pgfqpoint{0.150000in}{0.150000in}}{\pgfqpoint{1.800000in}{1.800000in}}%
\pgfusepath{clip}%
\pgfsetbuttcap%
\pgfsetroundjoin%
\definecolor{currentfill}{rgb}{0.933333,0.600000,0.666667}%
\pgfsetfillcolor{currentfill}%
\pgfsetlinewidth{1.003750pt}%
\definecolor{currentstroke}{rgb}{0.600000,0.266667,0.333333}%
\pgfsetstrokecolor{currentstroke}%
\pgfsetdash{}{0pt}%
\pgfpathmoveto{\pgfqpoint{0.150000in}{0.150000in}}%
\pgfpathlineto{\pgfqpoint{0.510000in}{0.150000in}}%
\pgfpathlineto{\pgfqpoint{0.510000in}{1.950000in}}%
\pgfpathlineto{\pgfqpoint{0.150000in}{1.950000in}}%
\pgfpathlineto{\pgfqpoint{0.150000in}{0.150000in}}%
\pgfpathclose%
\pgfusepath{stroke,fill}%
\end{pgfscope}%
\begin{pgfscope}%
\pgfpathrectangle{\pgfqpoint{0.150000in}{0.150000in}}{\pgfqpoint{1.800000in}{1.800000in}}%
\pgfusepath{clip}%
\pgfsetbuttcap%
\pgfsetroundjoin%
\definecolor{currentfill}{rgb}{0.400000,0.600000,0.800000}%
\pgfsetfillcolor{currentfill}%
\pgfsetlinewidth{1.003750pt}%
\definecolor{currentstroke}{rgb}{0.000000,0.266667,0.533333}%
\pgfsetstrokecolor{currentstroke}%
\pgfsetdash{}{0pt}%
\pgfpathmoveto{\pgfqpoint{1.406589in}{0.555021in}}%
\pgfpathlineto{\pgfqpoint{1.455104in}{0.555021in}}%
\pgfpathlineto{\pgfqpoint{1.455104in}{0.566880in}}%
\pgfpathlineto{\pgfqpoint{1.406589in}{0.566880in}}%
\pgfpathlineto{\pgfqpoint{1.406589in}{0.555021in}}%
\pgfpathclose%
\pgfusepath{stroke,fill}%
\end{pgfscope}%
\begin{pgfscope}%
\pgfpathrectangle{\pgfqpoint{0.150000in}{0.150000in}}{\pgfqpoint{1.800000in}{1.800000in}}%
\pgfusepath{clip}%
\pgfsetbuttcap%
\pgfsetroundjoin%
\definecolor{currentfill}{rgb}{0.400000,0.600000,0.800000}%
\pgfsetfillcolor{currentfill}%
\pgfsetlinewidth{1.003750pt}%
\definecolor{currentstroke}{rgb}{0.000000,0.266667,0.533333}%
\pgfsetstrokecolor{currentstroke}%
\pgfsetdash{}{0pt}%
\pgfpathmoveto{\pgfqpoint{1.101888in}{1.055979in}}%
\pgfpathlineto{\pgfqpoint{1.158000in}{1.055979in}}%
\pgfpathlineto{\pgfqpoint{1.158000in}{1.101888in}}%
\pgfpathlineto{\pgfqpoint{1.101888in}{1.101888in}}%
\pgfpathlineto{\pgfqpoint{1.101888in}{1.055979in}}%
\pgfpathclose%
\pgfusepath{stroke,fill}%
\end{pgfscope}%
\begin{pgfscope}%
\pgfpathrectangle{\pgfqpoint{0.150000in}{0.150000in}}{\pgfqpoint{1.800000in}{1.800000in}}%
\pgfusepath{clip}%
\pgfsetbuttcap%
\pgfsetroundjoin%
\definecolor{currentfill}{rgb}{0.400000,0.600000,0.800000}%
\pgfsetfillcolor{currentfill}%
\pgfsetlinewidth{1.003750pt}%
\definecolor{currentstroke}{rgb}{0.000000,0.266667,0.533333}%
\pgfsetstrokecolor{currentstroke}%
\pgfsetdash{}{0pt}%
\pgfpathmoveto{\pgfqpoint{1.055979in}{1.101888in}}%
\pgfpathlineto{\pgfqpoint{1.101888in}{1.101888in}}%
\pgfpathlineto{\pgfqpoint{1.101888in}{1.158000in}}%
\pgfpathlineto{\pgfqpoint{1.055979in}{1.158000in}}%
\pgfpathlineto{\pgfqpoint{1.055979in}{1.101888in}}%
\pgfpathclose%
\pgfusepath{stroke,fill}%
\end{pgfscope}%
\begin{pgfscope}%
\pgfpathrectangle{\pgfqpoint{0.150000in}{0.150000in}}{\pgfqpoint{1.800000in}{1.800000in}}%
\pgfusepath{clip}%
\pgfsetbuttcap%
\pgfsetroundjoin%
\definecolor{currentfill}{rgb}{0.400000,0.600000,0.800000}%
\pgfsetfillcolor{currentfill}%
\pgfsetlinewidth{1.003750pt}%
\definecolor{currentstroke}{rgb}{0.000000,0.266667,0.533333}%
\pgfsetstrokecolor{currentstroke}%
\pgfsetdash{}{0pt}%
\pgfpathmoveto{\pgfqpoint{0.549036in}{1.112892in}}%
\pgfpathlineto{\pgfqpoint{0.564938in}{1.112892in}}%
\pgfpathlineto{\pgfqpoint{0.564938in}{1.171200in}}%
\pgfpathlineto{\pgfqpoint{0.549036in}{1.171200in}}%
\pgfpathlineto{\pgfqpoint{0.549036in}{1.112892in}}%
\pgfpathclose%
\pgfusepath{stroke,fill}%
\end{pgfscope}%
\begin{pgfscope}%
\pgfpathrectangle{\pgfqpoint{0.150000in}{0.150000in}}{\pgfqpoint{1.800000in}{1.800000in}}%
\pgfusepath{clip}%
\pgfsetbuttcap%
\pgfsetroundjoin%
\definecolor{currentfill}{rgb}{0.400000,0.600000,0.800000}%
\pgfsetfillcolor{currentfill}%
\pgfsetlinewidth{1.003750pt}%
\definecolor{currentstroke}{rgb}{0.000000,0.266667,0.533333}%
\pgfsetstrokecolor{currentstroke}%
\pgfsetdash{}{0pt}%
\pgfpathmoveto{\pgfqpoint{0.806992in}{0.651003in}}%
\pgfpathlineto{\pgfqpoint{0.876631in}{0.651003in}}%
\pgfpathlineto{\pgfqpoint{0.876631in}{0.669995in}}%
\pgfpathlineto{\pgfqpoint{0.806992in}{0.669995in}}%
\pgfpathlineto{\pgfqpoint{0.806992in}{0.651003in}}%
\pgfpathclose%
\pgfusepath{stroke,fill}%
\end{pgfscope}%
\begin{pgfscope}%
\pgfpathrectangle{\pgfqpoint{0.150000in}{0.150000in}}{\pgfqpoint{1.800000in}{1.800000in}}%
\pgfusepath{clip}%
\pgfsetbuttcap%
\pgfsetroundjoin%
\definecolor{currentfill}{rgb}{0.400000,0.600000,0.800000}%
\pgfsetfillcolor{currentfill}%
\pgfsetlinewidth{1.003750pt}%
\definecolor{currentstroke}{rgb}{0.000000,0.266667,0.533333}%
\pgfsetstrokecolor{currentstroke}%
\pgfsetdash{}{0pt}%
\pgfpathmoveto{\pgfqpoint{0.737830in}{0.750015in}}%
\pgfpathlineto{\pgfqpoint{0.750015in}{0.750015in}}%
\pgfpathlineto{\pgfqpoint{0.750015in}{0.820740in}}%
\pgfpathlineto{\pgfqpoint{0.737830in}{0.820740in}}%
\pgfpathlineto{\pgfqpoint{0.737830in}{0.750015in}}%
\pgfpathclose%
\pgfusepath{stroke,fill}%
\end{pgfscope}%
\begin{pgfscope}%
\pgfpathrectangle{\pgfqpoint{0.150000in}{0.150000in}}{\pgfqpoint{1.800000in}{1.800000in}}%
\pgfusepath{clip}%
\pgfsetbuttcap%
\pgfsetroundjoin%
\definecolor{currentfill}{rgb}{0.400000,0.600000,0.800000}%
\pgfsetfillcolor{currentfill}%
\pgfsetlinewidth{1.003750pt}%
\definecolor{currentstroke}{rgb}{0.000000,0.266667,0.533333}%
\pgfsetstrokecolor{currentstroke}%
\pgfsetdash{}{0pt}%
\pgfpathmoveto{\pgfqpoint{0.693038in}{0.737830in}}%
\pgfpathlineto{\pgfqpoint{0.737830in}{0.737830in}}%
\pgfpathlineto{\pgfqpoint{0.737830in}{0.820740in}}%
\pgfpathlineto{\pgfqpoint{0.693038in}{0.820740in}}%
\pgfpathlineto{\pgfqpoint{0.693038in}{0.737830in}}%
\pgfpathclose%
\pgfusepath{stroke,fill}%
\end{pgfscope}%
\begin{pgfscope}%
\pgfpathrectangle{\pgfqpoint{0.150000in}{0.150000in}}{\pgfqpoint{1.800000in}{1.800000in}}%
\pgfusepath{clip}%
\pgfsetbuttcap%
\pgfsetroundjoin%
\definecolor{currentfill}{rgb}{0.400000,0.600000,0.800000}%
\pgfsetfillcolor{currentfill}%
\pgfsetlinewidth{1.003750pt}%
\definecolor{currentstroke}{rgb}{0.000000,0.266667,0.533333}%
\pgfsetstrokecolor{currentstroke}%
\pgfsetdash{}{0pt}%
\pgfpathmoveto{\pgfqpoint{0.674057in}{0.737830in}}%
\pgfpathlineto{\pgfqpoint{0.693038in}{0.737830in}}%
\pgfpathlineto{\pgfqpoint{0.693038in}{0.820740in}}%
\pgfpathlineto{\pgfqpoint{0.674057in}{0.820740in}}%
\pgfpathlineto{\pgfqpoint{0.674057in}{0.737830in}}%
\pgfpathclose%
\pgfusepath{stroke,fill}%
\end{pgfscope}%
\begin{pgfscope}%
\pgfpathrectangle{\pgfqpoint{0.150000in}{0.150000in}}{\pgfqpoint{1.800000in}{1.800000in}}%
\pgfusepath{clip}%
\pgfsetbuttcap%
\pgfsetroundjoin%
\definecolor{currentfill}{rgb}{0.400000,0.600000,0.800000}%
\pgfsetfillcolor{currentfill}%
\pgfsetlinewidth{1.003750pt}%
\definecolor{currentstroke}{rgb}{0.000000,0.266667,0.533333}%
\pgfsetstrokecolor{currentstroke}%
\pgfsetdash{}{0pt}%
\pgfpathmoveto{\pgfqpoint{1.818231in}{0.639505in}}%
\pgfpathlineto{\pgfqpoint{1.877527in}{0.639505in}}%
\pgfpathlineto{\pgfqpoint{1.877527in}{0.654000in}}%
\pgfpathlineto{\pgfqpoint{1.818231in}{0.654000in}}%
\pgfpathlineto{\pgfqpoint{1.818231in}{0.639505in}}%
\pgfpathclose%
\pgfusepath{stroke,fill}%
\end{pgfscope}%
\begin{pgfscope}%
\pgfpathrectangle{\pgfqpoint{0.150000in}{0.150000in}}{\pgfqpoint{1.800000in}{1.800000in}}%
\pgfusepath{clip}%
\pgfsetbuttcap%
\pgfsetroundjoin%
\definecolor{currentfill}{rgb}{0.400000,0.600000,0.800000}%
\pgfsetfillcolor{currentfill}%
\pgfsetlinewidth{1.003750pt}%
\definecolor{currentstroke}{rgb}{0.000000,0.266667,0.533333}%
\pgfsetstrokecolor{currentstroke}%
\pgfsetdash{}{0pt}%
\pgfpathmoveto{\pgfqpoint{1.710420in}{0.615787in}}%
\pgfpathlineto{\pgfqpoint{1.758935in}{0.615787in}}%
\pgfpathlineto{\pgfqpoint{1.758935in}{0.627646in}}%
\pgfpathlineto{\pgfqpoint{1.710420in}{0.627646in}}%
\pgfpathlineto{\pgfqpoint{1.710420in}{0.615787in}}%
\pgfpathclose%
\pgfusepath{stroke,fill}%
\end{pgfscope}%
\begin{pgfscope}%
\pgfpathrectangle{\pgfqpoint{0.150000in}{0.150000in}}{\pgfqpoint{1.800000in}{1.800000in}}%
\pgfusepath{clip}%
\pgfsetbuttcap%
\pgfsetroundjoin%
\definecolor{currentfill}{rgb}{0.400000,0.600000,0.800000}%
\pgfsetfillcolor{currentfill}%
\pgfsetlinewidth{1.003750pt}%
\definecolor{currentstroke}{rgb}{0.000000,0.266667,0.533333}%
\pgfsetstrokecolor{currentstroke}%
\pgfsetdash{}{0pt}%
\pgfpathmoveto{\pgfqpoint{1.602609in}{0.594225in}}%
\pgfpathlineto{\pgfqpoint{1.651124in}{0.594225in}}%
\pgfpathlineto{\pgfqpoint{1.651124in}{0.606084in}}%
\pgfpathlineto{\pgfqpoint{1.602609in}{0.606084in}}%
\pgfpathlineto{\pgfqpoint{1.602609in}{0.594225in}}%
\pgfpathclose%
\pgfusepath{stroke,fill}%
\end{pgfscope}%
\begin{pgfscope}%
\pgfpathrectangle{\pgfqpoint{0.150000in}{0.150000in}}{\pgfqpoint{1.800000in}{1.800000in}}%
\pgfusepath{clip}%
\pgfsetbuttcap%
\pgfsetroundjoin%
\definecolor{currentfill}{rgb}{0.400000,0.600000,0.800000}%
\pgfsetfillcolor{currentfill}%
\pgfsetlinewidth{1.003750pt}%
\definecolor{currentstroke}{rgb}{0.000000,0.266667,0.533333}%
\pgfsetstrokecolor{currentstroke}%
\pgfsetdash{}{0pt}%
\pgfpathmoveto{\pgfqpoint{1.318380in}{0.545318in}}%
\pgfpathlineto{\pgfqpoint{1.406589in}{0.545318in}}%
\pgfpathlineto{\pgfqpoint{1.406589in}{0.566880in}}%
\pgfpathlineto{\pgfqpoint{1.318380in}{0.566880in}}%
\pgfpathlineto{\pgfqpoint{1.318380in}{0.545318in}}%
\pgfpathclose%
\pgfusepath{stroke,fill}%
\end{pgfscope}%
\begin{pgfscope}%
\pgfpathrectangle{\pgfqpoint{0.150000in}{0.150000in}}{\pgfqpoint{1.800000in}{1.800000in}}%
\pgfusepath{clip}%
\pgfsetbuttcap%
\pgfsetroundjoin%
\definecolor{currentfill}{rgb}{0.400000,0.600000,0.800000}%
\pgfsetfillcolor{currentfill}%
\pgfsetlinewidth{1.003750pt}%
\definecolor{currentstroke}{rgb}{0.000000,0.266667,0.533333}%
\pgfsetstrokecolor{currentstroke}%
\pgfsetdash{}{0pt}%
\pgfpathmoveto{\pgfqpoint{1.230171in}{0.527676in}}%
\pgfpathlineto{\pgfqpoint{1.318380in}{0.527676in}}%
\pgfpathlineto{\pgfqpoint{1.318380in}{0.534000in}}%
\pgfpathlineto{\pgfqpoint{1.230171in}{0.534000in}}%
\pgfpathlineto{\pgfqpoint{1.230171in}{0.527676in}}%
\pgfpathclose%
\pgfusepath{stroke,fill}%
\end{pgfscope}%
\begin{pgfscope}%
\pgfpathrectangle{\pgfqpoint{0.150000in}{0.150000in}}{\pgfqpoint{1.800000in}{1.800000in}}%
\pgfusepath{clip}%
\pgfsetbuttcap%
\pgfsetroundjoin%
\definecolor{currentfill}{rgb}{0.400000,0.600000,0.800000}%
\pgfsetfillcolor{currentfill}%
\pgfsetlinewidth{1.003750pt}%
\definecolor{currentstroke}{rgb}{0.000000,0.266667,0.533333}%
\pgfsetstrokecolor{currentstroke}%
\pgfsetdash{}{0pt}%
\pgfpathmoveto{\pgfqpoint{0.639747in}{1.820427in}}%
\pgfpathlineto{\pgfqpoint{0.654000in}{1.820427in}}%
\pgfpathlineto{\pgfqpoint{0.654000in}{1.878735in}}%
\pgfpathlineto{\pgfqpoint{0.639747in}{1.878735in}}%
\pgfpathlineto{\pgfqpoint{0.639747in}{1.820427in}}%
\pgfpathclose%
\pgfusepath{stroke,fill}%
\end{pgfscope}%
\begin{pgfscope}%
\pgfpathrectangle{\pgfqpoint{0.150000in}{0.150000in}}{\pgfqpoint{1.800000in}{1.800000in}}%
\pgfusepath{clip}%
\pgfsetbuttcap%
\pgfsetroundjoin%
\definecolor{currentfill}{rgb}{0.400000,0.600000,0.800000}%
\pgfsetfillcolor{currentfill}%
\pgfsetlinewidth{1.003750pt}%
\definecolor{currentstroke}{rgb}{0.000000,0.266667,0.533333}%
\pgfsetstrokecolor{currentstroke}%
\pgfsetdash{}{0pt}%
\pgfpathmoveto{\pgfqpoint{0.616424in}{1.714413in}}%
\pgfpathlineto{\pgfqpoint{0.628085in}{1.714413in}}%
\pgfpathlineto{\pgfqpoint{0.628085in}{1.762119in}}%
\pgfpathlineto{\pgfqpoint{0.616424in}{1.762119in}}%
\pgfpathlineto{\pgfqpoint{0.616424in}{1.714413in}}%
\pgfpathclose%
\pgfusepath{stroke,fill}%
\end{pgfscope}%
\begin{pgfscope}%
\pgfpathrectangle{\pgfqpoint{0.150000in}{0.150000in}}{\pgfqpoint{1.800000in}{1.800000in}}%
\pgfusepath{clip}%
\pgfsetbuttcap%
\pgfsetroundjoin%
\definecolor{currentfill}{rgb}{0.400000,0.600000,0.800000}%
\pgfsetfillcolor{currentfill}%
\pgfsetlinewidth{1.003750pt}%
\definecolor{currentstroke}{rgb}{0.000000,0.266667,0.533333}%
\pgfsetstrokecolor{currentstroke}%
\pgfsetdash{}{0pt}%
\pgfpathmoveto{\pgfqpoint{0.595221in}{1.608399in}}%
\pgfpathlineto{\pgfqpoint{0.606883in}{1.608399in}}%
\pgfpathlineto{\pgfqpoint{0.606883in}{1.656105in}}%
\pgfpathlineto{\pgfqpoint{0.595221in}{1.656105in}}%
\pgfpathlineto{\pgfqpoint{0.595221in}{1.608399in}}%
\pgfpathclose%
\pgfusepath{stroke,fill}%
\end{pgfscope}%
\begin{pgfscope}%
\pgfpathrectangle{\pgfqpoint{0.150000in}{0.150000in}}{\pgfqpoint{1.800000in}{1.800000in}}%
\pgfusepath{clip}%
\pgfsetbuttcap%
\pgfsetroundjoin%
\definecolor{currentfill}{rgb}{0.400000,0.600000,0.800000}%
\pgfsetfillcolor{currentfill}%
\pgfsetlinewidth{1.003750pt}%
\definecolor{currentstroke}{rgb}{0.000000,0.266667,0.533333}%
\pgfsetstrokecolor{currentstroke}%
\pgfsetdash{}{0pt}%
\pgfpathmoveto{\pgfqpoint{0.556670in}{1.415646in}}%
\pgfpathlineto{\pgfqpoint{0.568332in}{1.415646in}}%
\pgfpathlineto{\pgfqpoint{0.568332in}{1.463352in}}%
\pgfpathlineto{\pgfqpoint{0.556670in}{1.463352in}}%
\pgfpathlineto{\pgfqpoint{0.556670in}{1.415646in}}%
\pgfpathclose%
\pgfusepath{stroke,fill}%
\end{pgfscope}%
\begin{pgfscope}%
\pgfpathrectangle{\pgfqpoint{0.150000in}{0.150000in}}{\pgfqpoint{1.800000in}{1.800000in}}%
\pgfusepath{clip}%
\pgfsetbuttcap%
\pgfsetroundjoin%
\definecolor{currentfill}{rgb}{0.400000,0.600000,0.800000}%
\pgfsetfillcolor{currentfill}%
\pgfsetlinewidth{1.003750pt}%
\definecolor{currentstroke}{rgb}{0.000000,0.266667,0.533333}%
\pgfsetstrokecolor{currentstroke}%
\pgfsetdash{}{0pt}%
\pgfpathmoveto{\pgfqpoint{1.055979in}{0.972507in}}%
\pgfpathlineto{\pgfqpoint{1.158000in}{0.972507in}}%
\pgfpathlineto{\pgfqpoint{1.158000in}{1.055979in}}%
\pgfpathlineto{\pgfqpoint{1.055979in}{1.055979in}}%
\pgfpathlineto{\pgfqpoint{1.055979in}{0.972507in}}%
\pgfpathclose%
\pgfusepath{stroke,fill}%
\end{pgfscope}%
\begin{pgfscope}%
\pgfpathrectangle{\pgfqpoint{0.150000in}{0.150000in}}{\pgfqpoint{1.800000in}{1.800000in}}%
\pgfusepath{clip}%
\pgfsetbuttcap%
\pgfsetroundjoin%
\definecolor{currentfill}{rgb}{0.400000,0.600000,0.800000}%
\pgfsetfillcolor{currentfill}%
\pgfsetlinewidth{1.003750pt}%
\definecolor{currentstroke}{rgb}{0.000000,0.266667,0.533333}%
\pgfsetstrokecolor{currentstroke}%
\pgfsetdash{}{0pt}%
\pgfpathmoveto{\pgfqpoint{0.972507in}{1.055979in}}%
\pgfpathlineto{\pgfqpoint{1.055979in}{1.055979in}}%
\pgfpathlineto{\pgfqpoint{1.055979in}{1.158000in}}%
\pgfpathlineto{\pgfqpoint{0.972507in}{1.158000in}}%
\pgfpathlineto{\pgfqpoint{0.972507in}{1.055979in}}%
\pgfpathclose%
\pgfusepath{stroke,fill}%
\end{pgfscope}%
\begin{pgfscope}%
\pgfpathrectangle{\pgfqpoint{0.150000in}{0.150000in}}{\pgfqpoint{1.800000in}{1.800000in}}%
\pgfusepath{clip}%
\pgfsetbuttcap%
\pgfsetroundjoin%
\definecolor{currentfill}{rgb}{0.400000,0.600000,0.800000}%
\pgfsetfillcolor{currentfill}%
\pgfsetlinewidth{1.003750pt}%
\definecolor{currentstroke}{rgb}{0.000000,0.266667,0.533333}%
\pgfsetstrokecolor{currentstroke}%
\pgfsetdash{}{0pt}%
\pgfpathmoveto{\pgfqpoint{0.889035in}{0.820740in}}%
\pgfpathlineto{\pgfqpoint{0.972507in}{0.820740in}}%
\pgfpathlineto{\pgfqpoint{0.972507in}{0.889035in}}%
\pgfpathlineto{\pgfqpoint{0.889035in}{0.889035in}}%
\pgfpathlineto{\pgfqpoint{0.889035in}{0.820740in}}%
\pgfpathclose%
\pgfusepath{stroke,fill}%
\end{pgfscope}%
\begin{pgfscope}%
\pgfpathrectangle{\pgfqpoint{0.150000in}{0.150000in}}{\pgfqpoint{1.800000in}{1.800000in}}%
\pgfusepath{clip}%
\pgfsetbuttcap%
\pgfsetroundjoin%
\definecolor{currentfill}{rgb}{0.400000,0.600000,0.800000}%
\pgfsetfillcolor{currentfill}%
\pgfsetlinewidth{1.003750pt}%
\definecolor{currentstroke}{rgb}{0.000000,0.266667,0.533333}%
\pgfsetstrokecolor{currentstroke}%
\pgfsetdash{}{0pt}%
\pgfpathmoveto{\pgfqpoint{0.820740in}{0.889035in}}%
\pgfpathlineto{\pgfqpoint{0.889035in}{0.889035in}}%
\pgfpathlineto{\pgfqpoint{0.889035in}{0.972507in}}%
\pgfpathlineto{\pgfqpoint{0.820740in}{0.972507in}}%
\pgfpathlineto{\pgfqpoint{0.820740in}{0.889035in}}%
\pgfpathclose%
\pgfusepath{stroke,fill}%
\end{pgfscope}%
\begin{pgfscope}%
\pgfpathrectangle{\pgfqpoint{0.150000in}{0.150000in}}{\pgfqpoint{1.800000in}{1.800000in}}%
\pgfusepath{clip}%
\pgfsetbuttcap%
\pgfsetroundjoin%
\definecolor{currentfill}{rgb}{0.400000,0.600000,0.800000}%
\pgfsetfillcolor{currentfill}%
\pgfsetlinewidth{1.003750pt}%
\definecolor{currentstroke}{rgb}{0.000000,0.266667,0.533333}%
\pgfsetstrokecolor{currentstroke}%
\pgfsetdash{}{0pt}%
\pgfpathmoveto{\pgfqpoint{0.564938in}{1.065186in}}%
\pgfpathlineto{\pgfqpoint{0.593851in}{1.065186in}}%
\pgfpathlineto{\pgfqpoint{0.593851in}{1.171200in}}%
\pgfpathlineto{\pgfqpoint{0.564938in}{1.171200in}}%
\pgfpathlineto{\pgfqpoint{0.564938in}{1.065186in}}%
\pgfpathclose%
\pgfusepath{stroke,fill}%
\end{pgfscope}%
\begin{pgfscope}%
\pgfpathrectangle{\pgfqpoint{0.150000in}{0.150000in}}{\pgfqpoint{1.800000in}{1.800000in}}%
\pgfusepath{clip}%
\pgfsetbuttcap%
\pgfsetroundjoin%
\definecolor{currentfill}{rgb}{0.400000,0.600000,0.800000}%
\pgfsetfillcolor{currentfill}%
\pgfsetlinewidth{1.003750pt}%
\definecolor{currentstroke}{rgb}{0.000000,0.266667,0.533333}%
\pgfsetstrokecolor{currentstroke}%
\pgfsetdash{}{0pt}%
\pgfpathmoveto{\pgfqpoint{0.622764in}{0.891708in}}%
\pgfpathlineto{\pgfqpoint{0.646420in}{0.891708in}}%
\pgfpathlineto{\pgfqpoint{0.646420in}{0.978447in}}%
\pgfpathlineto{\pgfqpoint{0.622764in}{0.978447in}}%
\pgfpathlineto{\pgfqpoint{0.622764in}{0.891708in}}%
\pgfpathclose%
\pgfusepath{stroke,fill}%
\end{pgfscope}%
\begin{pgfscope}%
\pgfpathrectangle{\pgfqpoint{0.150000in}{0.150000in}}{\pgfqpoint{1.800000in}{1.800000in}}%
\pgfusepath{clip}%
\pgfsetbuttcap%
\pgfsetroundjoin%
\definecolor{currentfill}{rgb}{0.400000,0.600000,0.800000}%
\pgfsetfillcolor{currentfill}%
\pgfsetlinewidth{1.003750pt}%
\definecolor{currentstroke}{rgb}{0.000000,0.266667,0.533333}%
\pgfsetstrokecolor{currentstroke}%
\pgfsetdash{}{0pt}%
\pgfpathmoveto{\pgfqpoint{1.072886in}{0.562371in}}%
\pgfpathlineto{\pgfqpoint{1.158000in}{0.562371in}}%
\pgfpathlineto{\pgfqpoint{1.158000in}{0.585584in}}%
\pgfpathlineto{\pgfqpoint{1.072886in}{0.585584in}}%
\pgfpathlineto{\pgfqpoint{1.072886in}{0.562371in}}%
\pgfpathclose%
\pgfusepath{stroke,fill}%
\end{pgfscope}%
\begin{pgfscope}%
\pgfpathrectangle{\pgfqpoint{0.150000in}{0.150000in}}{\pgfqpoint{1.800000in}{1.800000in}}%
\pgfusepath{clip}%
\pgfsetbuttcap%
\pgfsetroundjoin%
\definecolor{currentfill}{rgb}{0.400000,0.600000,0.800000}%
\pgfsetfillcolor{currentfill}%
\pgfsetlinewidth{1.003750pt}%
\definecolor{currentstroke}{rgb}{0.000000,0.266667,0.533333}%
\pgfsetstrokecolor{currentstroke}%
\pgfsetdash{}{0pt}%
\pgfpathmoveto{\pgfqpoint{0.933608in}{0.608797in}}%
\pgfpathlineto{\pgfqpoint{1.003247in}{0.608797in}}%
\pgfpathlineto{\pgfqpoint{1.003247in}{0.627790in}}%
\pgfpathlineto{\pgfqpoint{0.933608in}{0.627790in}}%
\pgfpathlineto{\pgfqpoint{0.933608in}{0.608797in}}%
\pgfpathclose%
\pgfusepath{stroke,fill}%
\end{pgfscope}%
\begin{pgfscope}%
\pgfpathrectangle{\pgfqpoint{0.150000in}{0.150000in}}{\pgfqpoint{1.800000in}{1.800000in}}%
\pgfusepath{clip}%
\pgfsetbuttcap%
\pgfsetroundjoin%
\definecolor{currentfill}{rgb}{0.400000,0.600000,0.800000}%
\pgfsetfillcolor{currentfill}%
\pgfsetlinewidth{1.003750pt}%
\definecolor{currentstroke}{rgb}{0.000000,0.266667,0.533333}%
\pgfsetstrokecolor{currentstroke}%
\pgfsetdash{}{0pt}%
\pgfpathmoveto{\pgfqpoint{0.750015in}{0.714617in}}%
\pgfpathlineto{\pgfqpoint{0.820740in}{0.714617in}}%
\pgfpathlineto{\pgfqpoint{0.820740in}{0.750015in}}%
\pgfpathlineto{\pgfqpoint{0.750015in}{0.750015in}}%
\pgfpathlineto{\pgfqpoint{0.750015in}{0.714617in}}%
\pgfpathclose%
\pgfusepath{stroke,fill}%
\end{pgfscope}%
\begin{pgfscope}%
\pgfpathrectangle{\pgfqpoint{0.150000in}{0.150000in}}{\pgfqpoint{1.800000in}{1.800000in}}%
\pgfusepath{clip}%
\pgfsetbuttcap%
\pgfsetroundjoin%
\definecolor{currentfill}{rgb}{0.400000,0.600000,0.800000}%
\pgfsetfillcolor{currentfill}%
\pgfsetlinewidth{1.003750pt}%
\definecolor{currentstroke}{rgb}{0.000000,0.266667,0.533333}%
\pgfsetstrokecolor{currentstroke}%
\pgfsetdash{}{0pt}%
\pgfpathmoveto{\pgfqpoint{0.820740in}{0.714617in}}%
\pgfpathlineto{\pgfqpoint{0.876631in}{0.714617in}}%
\pgfpathlineto{\pgfqpoint{0.876631in}{0.820740in}}%
\pgfpathlineto{\pgfqpoint{0.820740in}{0.820740in}}%
\pgfpathlineto{\pgfqpoint{0.820740in}{0.714617in}}%
\pgfpathclose%
\pgfusepath{stroke,fill}%
\end{pgfscope}%
\begin{pgfscope}%
\pgfpathrectangle{\pgfqpoint{0.150000in}{0.150000in}}{\pgfqpoint{1.800000in}{1.800000in}}%
\pgfusepath{clip}%
\pgfsetbuttcap%
\pgfsetroundjoin%
\definecolor{currentfill}{rgb}{0.400000,0.600000,0.800000}%
\pgfsetfillcolor{currentfill}%
\pgfsetlinewidth{1.003750pt}%
\definecolor{currentstroke}{rgb}{0.000000,0.266667,0.533333}%
\pgfsetstrokecolor{currentstroke}%
\pgfsetdash{}{0pt}%
\pgfpathmoveto{\pgfqpoint{0.750015in}{0.669995in}}%
\pgfpathlineto{\pgfqpoint{0.876631in}{0.669995in}}%
\pgfpathlineto{\pgfqpoint{0.876631in}{0.714617in}}%
\pgfpathlineto{\pgfqpoint{0.750015in}{0.714617in}}%
\pgfpathlineto{\pgfqpoint{0.750015in}{0.669995in}}%
\pgfpathclose%
\pgfusepath{stroke,fill}%
\end{pgfscope}%
\begin{pgfscope}%
\pgfpathrectangle{\pgfqpoint{0.150000in}{0.150000in}}{\pgfqpoint{1.800000in}{1.800000in}}%
\pgfusepath{clip}%
\pgfsetbuttcap%
\pgfsetroundjoin%
\definecolor{currentfill}{rgb}{0.400000,0.600000,0.800000}%
\pgfsetfillcolor{currentfill}%
\pgfsetlinewidth{1.003750pt}%
\definecolor{currentstroke}{rgb}{0.000000,0.266667,0.533333}%
\pgfsetstrokecolor{currentstroke}%
\pgfsetdash{}{0pt}%
\pgfpathmoveto{\pgfqpoint{1.400838in}{1.419162in}}%
\pgfpathlineto{\pgfqpoint{1.419162in}{1.419162in}}%
\pgfpathlineto{\pgfqpoint{1.419162in}{1.501620in}}%
\pgfpathlineto{\pgfqpoint{1.400838in}{1.501620in}}%
\pgfpathlineto{\pgfqpoint{1.400838in}{1.419162in}}%
\pgfpathclose%
\pgfusepath{stroke,fill}%
\end{pgfscope}%
\begin{pgfscope}%
\pgfpathrectangle{\pgfqpoint{0.150000in}{0.150000in}}{\pgfqpoint{1.800000in}{1.800000in}}%
\pgfusepath{clip}%
\pgfsetbuttcap%
\pgfsetroundjoin%
\definecolor{currentfill}{rgb}{0.400000,0.600000,0.800000}%
\pgfsetfillcolor{currentfill}%
\pgfsetlinewidth{1.003750pt}%
\definecolor{currentstroke}{rgb}{0.000000,0.266667,0.533333}%
\pgfsetstrokecolor{currentstroke}%
\pgfsetdash{}{0pt}%
\pgfpathmoveto{\pgfqpoint{1.400838in}{1.318380in}}%
\pgfpathlineto{\pgfqpoint{1.419162in}{1.318380in}}%
\pgfpathlineto{\pgfqpoint{1.419162in}{1.400838in}}%
\pgfpathlineto{\pgfqpoint{1.400838in}{1.400838in}}%
\pgfpathlineto{\pgfqpoint{1.400838in}{1.318380in}}%
\pgfpathclose%
\pgfusepath{stroke,fill}%
\end{pgfscope}%
\begin{pgfscope}%
\pgfpathrectangle{\pgfqpoint{0.150000in}{0.150000in}}{\pgfqpoint{1.800000in}{1.800000in}}%
\pgfusepath{clip}%
\pgfsetbuttcap%
\pgfsetroundjoin%
\definecolor{currentfill}{rgb}{0.400000,0.600000,0.800000}%
\pgfsetfillcolor{currentfill}%
\pgfsetlinewidth{1.003750pt}%
\definecolor{currentstroke}{rgb}{0.000000,0.266667,0.533333}%
\pgfsetstrokecolor{currentstroke}%
\pgfsetdash{}{0pt}%
\pgfpathmoveto{\pgfqpoint{1.419162in}{1.318380in}}%
\pgfpathlineto{\pgfqpoint{1.501620in}{1.318380in}}%
\pgfpathlineto{\pgfqpoint{1.501620in}{1.501620in}}%
\pgfpathlineto{\pgfqpoint{1.419162in}{1.501620in}}%
\pgfpathlineto{\pgfqpoint{1.419162in}{1.318380in}}%
\pgfpathclose%
\pgfusepath{stroke,fill}%
\end{pgfscope}%
\begin{pgfscope}%
\pgfpathrectangle{\pgfqpoint{0.150000in}{0.150000in}}{\pgfqpoint{1.800000in}{1.800000in}}%
\pgfusepath{clip}%
\pgfsetbuttcap%
\pgfsetroundjoin%
\definecolor{currentfill}{rgb}{0.400000,0.600000,0.800000}%
\pgfsetfillcolor{currentfill}%
\pgfsetlinewidth{1.003750pt}%
\definecolor{currentstroke}{rgb}{0.000000,0.266667,0.533333}%
\pgfsetstrokecolor{currentstroke}%
\pgfsetdash{}{0pt}%
\pgfpathmoveto{\pgfqpoint{1.318380in}{1.400838in}}%
\pgfpathlineto{\pgfqpoint{1.400838in}{1.400838in}}%
\pgfpathlineto{\pgfqpoint{1.400838in}{1.501620in}}%
\pgfpathlineto{\pgfqpoint{1.318380in}{1.501620in}}%
\pgfpathlineto{\pgfqpoint{1.318380in}{1.400838in}}%
\pgfpathclose%
\pgfusepath{stroke,fill}%
\end{pgfscope}%
\begin{pgfscope}%
\pgfpathrectangle{\pgfqpoint{0.150000in}{0.150000in}}{\pgfqpoint{1.800000in}{1.800000in}}%
\pgfusepath{clip}%
\pgfsetbuttcap%
\pgfsetroundjoin%
\definecolor{currentfill}{rgb}{0.400000,0.600000,0.800000}%
\pgfsetfillcolor{currentfill}%
\pgfsetlinewidth{1.003750pt}%
\definecolor{currentstroke}{rgb}{0.000000,0.266667,0.533333}%
\pgfsetstrokecolor{currentstroke}%
\pgfsetdash{}{0pt}%
\pgfpathmoveto{\pgfqpoint{1.230171in}{1.158000in}}%
\pgfpathlineto{\pgfqpoint{1.318380in}{1.158000in}}%
\pgfpathlineto{\pgfqpoint{1.318380in}{1.230171in}}%
\pgfpathlineto{\pgfqpoint{1.230171in}{1.230171in}}%
\pgfpathlineto{\pgfqpoint{1.230171in}{1.158000in}}%
\pgfpathclose%
\pgfusepath{stroke,fill}%
\end{pgfscope}%
\begin{pgfscope}%
\pgfpathrectangle{\pgfqpoint{0.150000in}{0.150000in}}{\pgfqpoint{1.800000in}{1.800000in}}%
\pgfusepath{clip}%
\pgfsetbuttcap%
\pgfsetroundjoin%
\definecolor{currentfill}{rgb}{0.400000,0.600000,0.800000}%
\pgfsetfillcolor{currentfill}%
\pgfsetlinewidth{1.003750pt}%
\definecolor{currentstroke}{rgb}{0.000000,0.266667,0.533333}%
\pgfsetstrokecolor{currentstroke}%
\pgfsetdash{}{0pt}%
\pgfpathmoveto{\pgfqpoint{1.158000in}{1.230171in}}%
\pgfpathlineto{\pgfqpoint{1.230171in}{1.230171in}}%
\pgfpathlineto{\pgfqpoint{1.230171in}{1.318380in}}%
\pgfpathlineto{\pgfqpoint{1.158000in}{1.318380in}}%
\pgfpathlineto{\pgfqpoint{1.158000in}{1.230171in}}%
\pgfpathclose%
\pgfusepath{stroke,fill}%
\end{pgfscope}%
\begin{pgfscope}%
\pgfpathrectangle{\pgfqpoint{0.150000in}{0.150000in}}{\pgfqpoint{1.800000in}{1.800000in}}%
\pgfusepath{clip}%
\pgfsetbuttcap%
\pgfsetroundjoin%
\definecolor{currentfill}{rgb}{0.400000,0.600000,0.800000}%
\pgfsetfillcolor{currentfill}%
\pgfsetlinewidth{1.003750pt}%
\definecolor{currentstroke}{rgb}{0.000000,0.266667,0.533333}%
\pgfsetstrokecolor{currentstroke}%
\pgfsetdash{}{0pt}%
\pgfpathmoveto{\pgfqpoint{1.710420in}{0.627646in}}%
\pgfpathlineto{\pgfqpoint{1.818231in}{0.627646in}}%
\pgfpathlineto{\pgfqpoint{1.818231in}{0.654000in}}%
\pgfpathlineto{\pgfqpoint{1.710420in}{0.654000in}}%
\pgfpathlineto{\pgfqpoint{1.710420in}{0.627646in}}%
\pgfpathclose%
\pgfusepath{stroke,fill}%
\end{pgfscope}%
\begin{pgfscope}%
\pgfpathrectangle{\pgfqpoint{0.150000in}{0.150000in}}{\pgfqpoint{1.800000in}{1.800000in}}%
\pgfusepath{clip}%
\pgfsetbuttcap%
\pgfsetroundjoin%
\definecolor{currentfill}{rgb}{0.400000,0.600000,0.800000}%
\pgfsetfillcolor{currentfill}%
\pgfsetlinewidth{1.003750pt}%
\definecolor{currentstroke}{rgb}{0.000000,0.266667,0.533333}%
\pgfsetstrokecolor{currentstroke}%
\pgfsetdash{}{0pt}%
\pgfpathmoveto{\pgfqpoint{1.514400in}{0.584522in}}%
\pgfpathlineto{\pgfqpoint{1.602609in}{0.584522in}}%
\pgfpathlineto{\pgfqpoint{1.602609in}{0.606084in}}%
\pgfpathlineto{\pgfqpoint{1.514400in}{0.606084in}}%
\pgfpathlineto{\pgfqpoint{1.514400in}{0.584522in}}%
\pgfpathclose%
\pgfusepath{stroke,fill}%
\end{pgfscope}%
\begin{pgfscope}%
\pgfpathrectangle{\pgfqpoint{0.150000in}{0.150000in}}{\pgfqpoint{1.800000in}{1.800000in}}%
\pgfusepath{clip}%
\pgfsetbuttcap%
\pgfsetroundjoin%
\definecolor{currentfill}{rgb}{0.400000,0.600000,0.800000}%
\pgfsetfillcolor{currentfill}%
\pgfsetlinewidth{1.003750pt}%
\definecolor{currentstroke}{rgb}{0.000000,0.266667,0.533333}%
\pgfsetstrokecolor{currentstroke}%
\pgfsetdash{}{0pt}%
\pgfpathmoveto{\pgfqpoint{1.158000in}{0.534000in}}%
\pgfpathlineto{\pgfqpoint{1.318380in}{0.534000in}}%
\pgfpathlineto{\pgfqpoint{1.318380in}{0.566880in}}%
\pgfpathlineto{\pgfqpoint{1.158000in}{0.566880in}}%
\pgfpathlineto{\pgfqpoint{1.158000in}{0.534000in}}%
\pgfpathclose%
\pgfusepath{stroke,fill}%
\end{pgfscope}%
\begin{pgfscope}%
\pgfpathrectangle{\pgfqpoint{0.150000in}{0.150000in}}{\pgfqpoint{1.800000in}{1.800000in}}%
\pgfusepath{clip}%
\pgfsetbuttcap%
\pgfsetroundjoin%
\definecolor{currentfill}{rgb}{0.400000,0.600000,0.800000}%
\pgfsetfillcolor{currentfill}%
\pgfsetlinewidth{1.003750pt}%
\definecolor{currentstroke}{rgb}{0.000000,0.266667,0.533333}%
\pgfsetstrokecolor{currentstroke}%
\pgfsetdash{}{0pt}%
\pgfpathmoveto{\pgfqpoint{0.628085in}{1.714413in}}%
\pgfpathlineto{\pgfqpoint{0.654000in}{1.714413in}}%
\pgfpathlineto{\pgfqpoint{0.654000in}{1.820427in}}%
\pgfpathlineto{\pgfqpoint{0.628085in}{1.820427in}}%
\pgfpathlineto{\pgfqpoint{0.628085in}{1.714413in}}%
\pgfpathclose%
\pgfusepath{stroke,fill}%
\end{pgfscope}%
\begin{pgfscope}%
\pgfpathrectangle{\pgfqpoint{0.150000in}{0.150000in}}{\pgfqpoint{1.800000in}{1.800000in}}%
\pgfusepath{clip}%
\pgfsetbuttcap%
\pgfsetroundjoin%
\definecolor{currentfill}{rgb}{0.400000,0.600000,0.800000}%
\pgfsetfillcolor{currentfill}%
\pgfsetlinewidth{1.003750pt}%
\definecolor{currentstroke}{rgb}{0.000000,0.266667,0.533333}%
\pgfsetstrokecolor{currentstroke}%
\pgfsetdash{}{0pt}%
\pgfpathmoveto{\pgfqpoint{0.585680in}{1.521660in}}%
\pgfpathlineto{\pgfqpoint{0.606883in}{1.521660in}}%
\pgfpathlineto{\pgfqpoint{0.606883in}{1.608399in}}%
\pgfpathlineto{\pgfqpoint{0.585680in}{1.608399in}}%
\pgfpathlineto{\pgfqpoint{0.585680in}{1.521660in}}%
\pgfpathclose%
\pgfusepath{stroke,fill}%
\end{pgfscope}%
\begin{pgfscope}%
\pgfpathrectangle{\pgfqpoint{0.150000in}{0.150000in}}{\pgfqpoint{1.800000in}{1.800000in}}%
\pgfusepath{clip}%
\pgfsetbuttcap%
\pgfsetroundjoin%
\definecolor{currentfill}{rgb}{0.400000,0.600000,0.800000}%
\pgfsetfillcolor{currentfill}%
\pgfsetlinewidth{1.003750pt}%
\definecolor{currentstroke}{rgb}{0.000000,0.266667,0.533333}%
\pgfsetstrokecolor{currentstroke}%
\pgfsetdash{}{0pt}%
\pgfpathmoveto{\pgfqpoint{0.547129in}{1.328907in}}%
\pgfpathlineto{\pgfqpoint{0.568332in}{1.328907in}}%
\pgfpathlineto{\pgfqpoint{0.568332in}{1.415646in}}%
\pgfpathlineto{\pgfqpoint{0.547129in}{1.415646in}}%
\pgfpathlineto{\pgfqpoint{0.547129in}{1.328907in}}%
\pgfpathclose%
\pgfusepath{stroke,fill}%
\end{pgfscope}%
\begin{pgfscope}%
\pgfpathrectangle{\pgfqpoint{0.150000in}{0.150000in}}{\pgfqpoint{1.800000in}{1.800000in}}%
\pgfusepath{clip}%
\pgfsetbuttcap%
\pgfsetroundjoin%
\definecolor{currentfill}{rgb}{0.400000,0.600000,0.800000}%
\pgfsetfillcolor{currentfill}%
\pgfsetlinewidth{1.003750pt}%
\definecolor{currentstroke}{rgb}{0.000000,0.266667,0.533333}%
\pgfsetstrokecolor{currentstroke}%
\pgfsetdash{}{0pt}%
\pgfpathmoveto{\pgfqpoint{0.529600in}{1.171200in}}%
\pgfpathlineto{\pgfqpoint{0.529781in}{1.171200in}}%
\pgfpathlineto{\pgfqpoint{0.529781in}{1.242168in}}%
\pgfpathlineto{\pgfqpoint{0.529600in}{1.242168in}}%
\pgfpathlineto{\pgfqpoint{0.529600in}{1.171200in}}%
\pgfpathclose%
\pgfusepath{stroke,fill}%
\end{pgfscope}%
\begin{pgfscope}%
\pgfpathrectangle{\pgfqpoint{0.150000in}{0.150000in}}{\pgfqpoint{1.800000in}{1.800000in}}%
\pgfusepath{clip}%
\pgfsetbuttcap%
\pgfsetroundjoin%
\definecolor{currentfill}{rgb}{0.400000,0.600000,0.800000}%
\pgfsetfillcolor{currentfill}%
\pgfsetlinewidth{1.003750pt}%
\definecolor{currentstroke}{rgb}{0.000000,0.266667,0.533333}%
\pgfsetstrokecolor{currentstroke}%
\pgfsetdash{}{0pt}%
\pgfpathmoveto{\pgfqpoint{0.972507in}{0.820740in}}%
\pgfpathlineto{\pgfqpoint{1.158000in}{0.820740in}}%
\pgfpathlineto{\pgfqpoint{1.158000in}{0.972507in}}%
\pgfpathlineto{\pgfqpoint{0.972507in}{0.972507in}}%
\pgfpathlineto{\pgfqpoint{0.972507in}{0.820740in}}%
\pgfpathclose%
\pgfusepath{stroke,fill}%
\end{pgfscope}%
\begin{pgfscope}%
\pgfpathrectangle{\pgfqpoint{0.150000in}{0.150000in}}{\pgfqpoint{1.800000in}{1.800000in}}%
\pgfusepath{clip}%
\pgfsetbuttcap%
\pgfsetroundjoin%
\definecolor{currentfill}{rgb}{0.400000,0.600000,0.800000}%
\pgfsetfillcolor{currentfill}%
\pgfsetlinewidth{1.003750pt}%
\definecolor{currentstroke}{rgb}{0.000000,0.266667,0.533333}%
\pgfsetstrokecolor{currentstroke}%
\pgfsetdash{}{0pt}%
\pgfpathmoveto{\pgfqpoint{0.820740in}{0.972507in}}%
\pgfpathlineto{\pgfqpoint{0.972507in}{0.972507in}}%
\pgfpathlineto{\pgfqpoint{0.972507in}{1.158000in}}%
\pgfpathlineto{\pgfqpoint{0.820740in}{1.158000in}}%
\pgfpathlineto{\pgfqpoint{0.820740in}{0.972507in}}%
\pgfpathclose%
\pgfusepath{stroke,fill}%
\end{pgfscope}%
\begin{pgfscope}%
\pgfpathrectangle{\pgfqpoint{0.150000in}{0.150000in}}{\pgfqpoint{1.800000in}{1.800000in}}%
\pgfusepath{clip}%
\pgfsetbuttcap%
\pgfsetroundjoin%
\definecolor{currentfill}{rgb}{0.400000,0.600000,0.800000}%
\pgfsetfillcolor{currentfill}%
\pgfsetlinewidth{1.003750pt}%
\definecolor{currentstroke}{rgb}{0.000000,0.266667,0.533333}%
\pgfsetstrokecolor{currentstroke}%
\pgfsetdash{}{0pt}%
\pgfpathmoveto{\pgfqpoint{0.593851in}{0.978447in}}%
\pgfpathlineto{\pgfqpoint{0.646420in}{0.978447in}}%
\pgfpathlineto{\pgfqpoint{0.646420in}{1.171200in}}%
\pgfpathlineto{\pgfqpoint{0.593851in}{1.171200in}}%
\pgfpathlineto{\pgfqpoint{0.593851in}{0.978447in}}%
\pgfpathclose%
\pgfusepath{stroke,fill}%
\end{pgfscope}%
\begin{pgfscope}%
\pgfpathrectangle{\pgfqpoint{0.150000in}{0.150000in}}{\pgfqpoint{1.800000in}{1.800000in}}%
\pgfusepath{clip}%
\pgfsetbuttcap%
\pgfsetroundjoin%
\definecolor{currentfill}{rgb}{0.400000,0.600000,0.800000}%
\pgfsetfillcolor{currentfill}%
\pgfsetlinewidth{1.003750pt}%
\definecolor{currentstroke}{rgb}{0.000000,0.266667,0.533333}%
\pgfsetstrokecolor{currentstroke}%
\pgfsetdash{}{0pt}%
\pgfpathmoveto{\pgfqpoint{1.003247in}{0.585584in}}%
\pgfpathlineto{\pgfqpoint{1.158000in}{0.585584in}}%
\pgfpathlineto{\pgfqpoint{1.158000in}{0.627790in}}%
\pgfpathlineto{\pgfqpoint{1.003247in}{0.627790in}}%
\pgfpathlineto{\pgfqpoint{1.003247in}{0.585584in}}%
\pgfpathclose%
\pgfusepath{stroke,fill}%
\end{pgfscope}%
\begin{pgfscope}%
\pgfpathrectangle{\pgfqpoint{0.150000in}{0.150000in}}{\pgfqpoint{1.800000in}{1.800000in}}%
\pgfusepath{clip}%
\pgfsetbuttcap%
\pgfsetroundjoin%
\definecolor{currentfill}{rgb}{0.400000,0.600000,0.800000}%
\pgfsetfillcolor{currentfill}%
\pgfsetlinewidth{1.003750pt}%
\definecolor{currentstroke}{rgb}{0.000000,0.266667,0.533333}%
\pgfsetstrokecolor{currentstroke}%
\pgfsetdash{}{0pt}%
\pgfpathmoveto{\pgfqpoint{1.318380in}{1.501620in}}%
\pgfpathlineto{\pgfqpoint{1.501620in}{1.501620in}}%
\pgfpathlineto{\pgfqpoint{1.501620in}{1.514400in}}%
\pgfpathlineto{\pgfqpoint{1.318380in}{1.514400in}}%
\pgfpathlineto{\pgfqpoint{1.318380in}{1.501620in}}%
\pgfpathclose%
\pgfusepath{stroke,fill}%
\end{pgfscope}%
\begin{pgfscope}%
\pgfpathrectangle{\pgfqpoint{0.150000in}{0.150000in}}{\pgfqpoint{1.800000in}{1.800000in}}%
\pgfusepath{clip}%
\pgfsetbuttcap%
\pgfsetroundjoin%
\definecolor{currentfill}{rgb}{0.400000,0.600000,0.800000}%
\pgfsetfillcolor{currentfill}%
\pgfsetlinewidth{1.003750pt}%
\definecolor{currentstroke}{rgb}{0.000000,0.266667,0.533333}%
\pgfsetstrokecolor{currentstroke}%
\pgfsetdash{}{0pt}%
\pgfpathmoveto{\pgfqpoint{1.318380in}{1.158000in}}%
\pgfpathlineto{\pgfqpoint{1.501620in}{1.158000in}}%
\pgfpathlineto{\pgfqpoint{1.501620in}{1.318380in}}%
\pgfpathlineto{\pgfqpoint{1.318380in}{1.318380in}}%
\pgfpathlineto{\pgfqpoint{1.318380in}{1.158000in}}%
\pgfpathclose%
\pgfusepath{stroke,fill}%
\end{pgfscope}%
\begin{pgfscope}%
\pgfpathrectangle{\pgfqpoint{0.150000in}{0.150000in}}{\pgfqpoint{1.800000in}{1.800000in}}%
\pgfusepath{clip}%
\pgfsetbuttcap%
\pgfsetroundjoin%
\definecolor{currentfill}{rgb}{0.400000,0.600000,0.800000}%
\pgfsetfillcolor{currentfill}%
\pgfsetlinewidth{1.003750pt}%
\definecolor{currentstroke}{rgb}{0.000000,0.266667,0.533333}%
\pgfsetstrokecolor{currentstroke}%
\pgfsetdash{}{0pt}%
\pgfpathmoveto{\pgfqpoint{1.501620in}{1.158000in}}%
\pgfpathlineto{\pgfqpoint{1.514400in}{1.158000in}}%
\pgfpathlineto{\pgfqpoint{1.514400in}{1.514400in}}%
\pgfpathlineto{\pgfqpoint{1.501620in}{1.514400in}}%
\pgfpathlineto{\pgfqpoint{1.501620in}{1.158000in}}%
\pgfpathclose%
\pgfusepath{stroke,fill}%
\end{pgfscope}%
\begin{pgfscope}%
\pgfpathrectangle{\pgfqpoint{0.150000in}{0.150000in}}{\pgfqpoint{1.800000in}{1.800000in}}%
\pgfusepath{clip}%
\pgfsetbuttcap%
\pgfsetroundjoin%
\definecolor{currentfill}{rgb}{0.400000,0.600000,0.800000}%
\pgfsetfillcolor{currentfill}%
\pgfsetlinewidth{1.003750pt}%
\definecolor{currentstroke}{rgb}{0.000000,0.266667,0.533333}%
\pgfsetstrokecolor{currentstroke}%
\pgfsetdash{}{0pt}%
\pgfpathmoveto{\pgfqpoint{1.158000in}{1.318380in}}%
\pgfpathlineto{\pgfqpoint{1.318380in}{1.318380in}}%
\pgfpathlineto{\pgfqpoint{1.318380in}{1.514400in}}%
\pgfpathlineto{\pgfqpoint{1.158000in}{1.514400in}}%
\pgfpathlineto{\pgfqpoint{1.158000in}{1.318380in}}%
\pgfpathclose%
\pgfusepath{stroke,fill}%
\end{pgfscope}%
\begin{pgfscope}%
\pgfpathrectangle{\pgfqpoint{0.150000in}{0.150000in}}{\pgfqpoint{1.800000in}{1.800000in}}%
\pgfusepath{clip}%
\pgfsetbuttcap%
\pgfsetroundjoin%
\definecolor{currentfill}{rgb}{0.400000,0.600000,0.800000}%
\pgfsetfillcolor{currentfill}%
\pgfsetlinewidth{1.003750pt}%
\definecolor{currentstroke}{rgb}{0.000000,0.266667,0.533333}%
\pgfsetstrokecolor{currentstroke}%
\pgfsetdash{}{0pt}%
\pgfpathmoveto{\pgfqpoint{1.514400in}{0.606084in}}%
\pgfpathlineto{\pgfqpoint{1.710420in}{0.606084in}}%
\pgfpathlineto{\pgfqpoint{1.710420in}{0.654000in}}%
\pgfpathlineto{\pgfqpoint{1.514400in}{0.654000in}}%
\pgfpathlineto{\pgfqpoint{1.514400in}{0.606084in}}%
\pgfpathclose%
\pgfusepath{stroke,fill}%
\end{pgfscope}%
\begin{pgfscope}%
\pgfpathrectangle{\pgfqpoint{0.150000in}{0.150000in}}{\pgfqpoint{1.800000in}{1.800000in}}%
\pgfusepath{clip}%
\pgfsetbuttcap%
\pgfsetroundjoin%
\definecolor{currentfill}{rgb}{0.400000,0.600000,0.800000}%
\pgfsetfillcolor{currentfill}%
\pgfsetlinewidth{1.003750pt}%
\definecolor{currentstroke}{rgb}{0.000000,0.266667,0.533333}%
\pgfsetstrokecolor{currentstroke}%
\pgfsetdash{}{0pt}%
\pgfpathmoveto{\pgfqpoint{1.158000in}{0.801600in}}%
\pgfpathlineto{\pgfqpoint{1.514400in}{0.801600in}}%
\pgfpathlineto{\pgfqpoint{1.514400in}{1.158000in}}%
\pgfpathlineto{\pgfqpoint{1.158000in}{1.158000in}}%
\pgfpathlineto{\pgfqpoint{1.158000in}{0.801600in}}%
\pgfpathclose%
\pgfusepath{stroke,fill}%
\end{pgfscope}%
\begin{pgfscope}%
\pgfpathrectangle{\pgfqpoint{0.150000in}{0.150000in}}{\pgfqpoint{1.800000in}{1.800000in}}%
\pgfusepath{clip}%
\pgfsetbuttcap%
\pgfsetroundjoin%
\definecolor{currentfill}{rgb}{0.400000,0.600000,0.800000}%
\pgfsetfillcolor{currentfill}%
\pgfsetlinewidth{1.003750pt}%
\definecolor{currentstroke}{rgb}{0.000000,0.266667,0.533333}%
\pgfsetstrokecolor{currentstroke}%
\pgfsetdash{}{0pt}%
\pgfpathmoveto{\pgfqpoint{1.158000in}{0.566880in}}%
\pgfpathlineto{\pgfqpoint{1.514400in}{0.566880in}}%
\pgfpathlineto{\pgfqpoint{1.514400in}{0.801600in}}%
\pgfpathlineto{\pgfqpoint{1.158000in}{0.801600in}}%
\pgfpathlineto{\pgfqpoint{1.158000in}{0.566880in}}%
\pgfpathclose%
\pgfusepath{stroke,fill}%
\end{pgfscope}%
\begin{pgfscope}%
\pgfpathrectangle{\pgfqpoint{0.150000in}{0.150000in}}{\pgfqpoint{1.800000in}{1.800000in}}%
\pgfusepath{clip}%
\pgfsetbuttcap%
\pgfsetroundjoin%
\definecolor{currentfill}{rgb}{0.400000,0.600000,0.800000}%
\pgfsetfillcolor{currentfill}%
\pgfsetlinewidth{1.003750pt}%
\definecolor{currentstroke}{rgb}{0.000000,0.266667,0.533333}%
\pgfsetstrokecolor{currentstroke}%
\pgfsetdash{}{0pt}%
\pgfpathmoveto{\pgfqpoint{0.606883in}{1.521660in}}%
\pgfpathlineto{\pgfqpoint{0.654000in}{1.521660in}}%
\pgfpathlineto{\pgfqpoint{0.654000in}{1.714413in}}%
\pgfpathlineto{\pgfqpoint{0.606883in}{1.714413in}}%
\pgfpathlineto{\pgfqpoint{0.606883in}{1.521660in}}%
\pgfpathclose%
\pgfusepath{stroke,fill}%
\end{pgfscope}%
\begin{pgfscope}%
\pgfpathrectangle{\pgfqpoint{0.150000in}{0.150000in}}{\pgfqpoint{1.800000in}{1.800000in}}%
\pgfusepath{clip}%
\pgfsetbuttcap%
\pgfsetroundjoin%
\definecolor{currentfill}{rgb}{0.400000,0.600000,0.800000}%
\pgfsetfillcolor{currentfill}%
\pgfsetlinewidth{1.003750pt}%
\definecolor{currentstroke}{rgb}{0.000000,0.266667,0.533333}%
\pgfsetstrokecolor{currentstroke}%
\pgfsetdash{}{0pt}%
\pgfpathmoveto{\pgfqpoint{0.529781in}{1.171200in}}%
\pgfpathlineto{\pgfqpoint{0.568332in}{1.171200in}}%
\pgfpathlineto{\pgfqpoint{0.568332in}{1.328907in}}%
\pgfpathlineto{\pgfqpoint{0.529781in}{1.328907in}}%
\pgfpathlineto{\pgfqpoint{0.529781in}{1.171200in}}%
\pgfpathclose%
\pgfusepath{stroke,fill}%
\end{pgfscope}%
\begin{pgfscope}%
\pgfpathrectangle{\pgfqpoint{0.150000in}{0.150000in}}{\pgfqpoint{1.800000in}{1.800000in}}%
\pgfusepath{clip}%
\pgfsetbuttcap%
\pgfsetroundjoin%
\definecolor{currentfill}{rgb}{0.400000,0.600000,0.800000}%
\pgfsetfillcolor{currentfill}%
\pgfsetlinewidth{1.003750pt}%
\definecolor{currentstroke}{rgb}{0.000000,0.266667,0.533333}%
\pgfsetstrokecolor{currentstroke}%
\pgfsetdash{}{0pt}%
\pgfpathmoveto{\pgfqpoint{0.820740in}{1.158000in}}%
\pgfpathlineto{\pgfqpoint{1.158000in}{1.158000in}}%
\pgfpathlineto{\pgfqpoint{1.158000in}{1.171200in}}%
\pgfpathlineto{\pgfqpoint{0.820740in}{1.171200in}}%
\pgfpathlineto{\pgfqpoint{0.820740in}{1.158000in}}%
\pgfpathclose%
\pgfusepath{stroke,fill}%
\end{pgfscope}%
\begin{pgfscope}%
\pgfpathrectangle{\pgfqpoint{0.150000in}{0.150000in}}{\pgfqpoint{1.800000in}{1.800000in}}%
\pgfusepath{clip}%
\pgfsetbuttcap%
\pgfsetroundjoin%
\definecolor{currentfill}{rgb}{0.400000,0.600000,0.800000}%
\pgfsetfillcolor{currentfill}%
\pgfsetlinewidth{1.003750pt}%
\definecolor{currentstroke}{rgb}{0.000000,0.266667,0.533333}%
\pgfsetstrokecolor{currentstroke}%
\pgfsetdash{}{0pt}%
\pgfpathmoveto{\pgfqpoint{0.812380in}{0.820740in}}%
\pgfpathlineto{\pgfqpoint{0.820740in}{0.820740in}}%
\pgfpathlineto{\pgfqpoint{0.820740in}{1.171200in}}%
\pgfpathlineto{\pgfqpoint{0.812380in}{1.171200in}}%
\pgfpathlineto{\pgfqpoint{0.812380in}{0.820740in}}%
\pgfpathclose%
\pgfusepath{stroke,fill}%
\end{pgfscope}%
\begin{pgfscope}%
\pgfpathrectangle{\pgfqpoint{0.150000in}{0.150000in}}{\pgfqpoint{1.800000in}{1.800000in}}%
\pgfusepath{clip}%
\pgfsetbuttcap%
\pgfsetroundjoin%
\definecolor{currentfill}{rgb}{0.400000,0.600000,0.800000}%
\pgfsetfillcolor{currentfill}%
\pgfsetlinewidth{1.003750pt}%
\definecolor{currentstroke}{rgb}{0.000000,0.266667,0.533333}%
\pgfsetstrokecolor{currentstroke}%
\pgfsetdash{}{0pt}%
\pgfpathmoveto{\pgfqpoint{0.646420in}{0.820740in}}%
\pgfpathlineto{\pgfqpoint{0.812380in}{0.820740in}}%
\pgfpathlineto{\pgfqpoint{0.812380in}{1.171200in}}%
\pgfpathlineto{\pgfqpoint{0.646420in}{1.171200in}}%
\pgfpathlineto{\pgfqpoint{0.646420in}{0.820740in}}%
\pgfpathclose%
\pgfusepath{stroke,fill}%
\end{pgfscope}%
\begin{pgfscope}%
\pgfpathrectangle{\pgfqpoint{0.150000in}{0.150000in}}{\pgfqpoint{1.800000in}{1.800000in}}%
\pgfusepath{clip}%
\pgfsetbuttcap%
\pgfsetroundjoin%
\definecolor{currentfill}{rgb}{0.400000,0.600000,0.800000}%
\pgfsetfillcolor{currentfill}%
\pgfsetlinewidth{1.003750pt}%
\definecolor{currentstroke}{rgb}{0.000000,0.266667,0.533333}%
\pgfsetstrokecolor{currentstroke}%
\pgfsetdash{}{0pt}%
\pgfpathmoveto{\pgfqpoint{0.876631in}{0.627790in}}%
\pgfpathlineto{\pgfqpoint{1.158000in}{0.627790in}}%
\pgfpathlineto{\pgfqpoint{1.158000in}{0.820740in}}%
\pgfpathlineto{\pgfqpoint{0.876631in}{0.820740in}}%
\pgfpathlineto{\pgfqpoint{0.876631in}{0.627790in}}%
\pgfpathclose%
\pgfusepath{stroke,fill}%
\end{pgfscope}%
\begin{pgfscope}%
\pgfpathrectangle{\pgfqpoint{0.150000in}{0.150000in}}{\pgfqpoint{1.800000in}{1.800000in}}%
\pgfusepath{clip}%
\pgfsetbuttcap%
\pgfsetroundjoin%
\definecolor{currentfill}{rgb}{0.400000,0.600000,0.800000}%
\pgfsetfillcolor{currentfill}%
\pgfsetlinewidth{1.003750pt}%
\definecolor{currentstroke}{rgb}{0.000000,0.266667,0.533333}%
\pgfsetstrokecolor{currentstroke}%
\pgfsetdash{}{0pt}%
\pgfpathmoveto{\pgfqpoint{1.514400in}{1.158000in}}%
\pgfpathlineto{\pgfqpoint{1.950000in}{1.158000in}}%
\pgfpathlineto{\pgfqpoint{1.950000in}{1.950000in}}%
\pgfpathlineto{\pgfqpoint{1.514400in}{1.950000in}}%
\pgfpathlineto{\pgfqpoint{1.514400in}{1.158000in}}%
\pgfpathclose%
\pgfusepath{stroke,fill}%
\end{pgfscope}%
\begin{pgfscope}%
\pgfpathrectangle{\pgfqpoint{0.150000in}{0.150000in}}{\pgfqpoint{1.800000in}{1.800000in}}%
\pgfusepath{clip}%
\pgfsetbuttcap%
\pgfsetroundjoin%
\definecolor{currentfill}{rgb}{0.400000,0.600000,0.800000}%
\pgfsetfillcolor{currentfill}%
\pgfsetlinewidth{1.003750pt}%
\definecolor{currentstroke}{rgb}{0.000000,0.266667,0.533333}%
\pgfsetstrokecolor{currentstroke}%
\pgfsetdash{}{0pt}%
\pgfpathmoveto{\pgfqpoint{1.158000in}{1.514400in}}%
\pgfpathlineto{\pgfqpoint{1.514400in}{1.514400in}}%
\pgfpathlineto{\pgfqpoint{1.514400in}{1.950000in}}%
\pgfpathlineto{\pgfqpoint{1.158000in}{1.950000in}}%
\pgfpathlineto{\pgfqpoint{1.158000in}{1.514400in}}%
\pgfpathclose%
\pgfusepath{stroke,fill}%
\end{pgfscope}%
\begin{pgfscope}%
\pgfpathrectangle{\pgfqpoint{0.150000in}{0.150000in}}{\pgfqpoint{1.800000in}{1.800000in}}%
\pgfusepath{clip}%
\pgfsetbuttcap%
\pgfsetroundjoin%
\definecolor{currentfill}{rgb}{0.400000,0.600000,0.800000}%
\pgfsetfillcolor{currentfill}%
\pgfsetlinewidth{1.003750pt}%
\definecolor{currentstroke}{rgb}{0.000000,0.266667,0.533333}%
\pgfsetstrokecolor{currentstroke}%
\pgfsetdash{}{0pt}%
\pgfpathmoveto{\pgfqpoint{1.514400in}{0.654000in}}%
\pgfpathlineto{\pgfqpoint{1.950000in}{0.654000in}}%
\pgfpathlineto{\pgfqpoint{1.950000in}{1.158000in}}%
\pgfpathlineto{\pgfqpoint{1.514400in}{1.158000in}}%
\pgfpathlineto{\pgfqpoint{1.514400in}{0.654000in}}%
\pgfpathclose%
\pgfusepath{stroke,fill}%
\end{pgfscope}%
\begin{pgfscope}%
\pgfpathrectangle{\pgfqpoint{0.150000in}{0.150000in}}{\pgfqpoint{1.800000in}{1.800000in}}%
\pgfusepath{clip}%
\pgfsetbuttcap%
\pgfsetroundjoin%
\definecolor{currentfill}{rgb}{0.400000,0.600000,0.800000}%
\pgfsetfillcolor{currentfill}%
\pgfsetlinewidth{1.003750pt}%
\definecolor{currentstroke}{rgb}{0.000000,0.266667,0.533333}%
\pgfsetstrokecolor{currentstroke}%
\pgfsetdash{}{0pt}%
\pgfpathmoveto{\pgfqpoint{0.568332in}{1.171200in}}%
\pgfpathlineto{\pgfqpoint{0.654000in}{1.171200in}}%
\pgfpathlineto{\pgfqpoint{0.654000in}{1.521660in}}%
\pgfpathlineto{\pgfqpoint{0.568332in}{1.521660in}}%
\pgfpathlineto{\pgfqpoint{0.568332in}{1.171200in}}%
\pgfpathclose%
\pgfusepath{stroke,fill}%
\end{pgfscope}%
\begin{pgfscope}%
\pgfpathrectangle{\pgfqpoint{0.150000in}{0.150000in}}{\pgfqpoint{1.800000in}{1.800000in}}%
\pgfusepath{clip}%
\pgfsetbuttcap%
\pgfsetroundjoin%
\definecolor{currentfill}{rgb}{0.400000,0.600000,0.800000}%
\pgfsetfillcolor{currentfill}%
\pgfsetlinewidth{1.003750pt}%
\definecolor{currentstroke}{rgb}{0.000000,0.266667,0.533333}%
\pgfsetstrokecolor{currentstroke}%
\pgfsetdash{}{0pt}%
\pgfpathmoveto{\pgfqpoint{0.654000in}{1.171200in}}%
\pgfpathlineto{\pgfqpoint{1.158000in}{1.171200in}}%
\pgfpathlineto{\pgfqpoint{1.158000in}{1.950000in}}%
\pgfpathlineto{\pgfqpoint{0.654000in}{1.950000in}}%
\pgfpathlineto{\pgfqpoint{0.654000in}{1.171200in}}%
\pgfpathclose%
\pgfusepath{stroke,fill}%
\end{pgfscope}%
\begin{pgfscope}%
\pgfpathrectangle{\pgfqpoint{0.150000in}{0.150000in}}{\pgfqpoint{1.800000in}{1.800000in}}%
\pgfusepath{clip}%
\pgfsetbuttcap%
\pgfsetroundjoin%
\definecolor{currentfill}{rgb}{0.933333,0.800000,0.400000}%
\pgfsetfillcolor{currentfill}%
\pgfsetlinewidth{1.003750pt}%
\definecolor{currentstroke}{rgb}{0.600000,0.466667,0.000000}%
\pgfsetstrokecolor{currentstroke}%
\pgfsetdash{}{0pt}%
\pgfpathmoveto{\pgfqpoint{1.455104in}{0.555021in}}%
\pgfpathlineto{\pgfqpoint{1.514400in}{0.555021in}}%
\pgfpathlineto{\pgfqpoint{1.514400in}{0.566880in}}%
\pgfpathlineto{\pgfqpoint{1.455104in}{0.566880in}}%
\pgfpathlineto{\pgfqpoint{1.455104in}{0.555021in}}%
\pgfpathclose%
\pgfusepath{stroke,fill}%
\end{pgfscope}%
\begin{pgfscope}%
\pgfpathrectangle{\pgfqpoint{0.150000in}{0.150000in}}{\pgfqpoint{1.800000in}{1.800000in}}%
\pgfusepath{clip}%
\pgfsetbuttcap%
\pgfsetroundjoin%
\definecolor{currentfill}{rgb}{0.933333,0.800000,0.400000}%
\pgfsetfillcolor{currentfill}%
\pgfsetlinewidth{1.003750pt}%
\definecolor{currentstroke}{rgb}{0.600000,0.466667,0.000000}%
\pgfsetstrokecolor{currentstroke}%
\pgfsetdash{}{0pt}%
\pgfpathmoveto{\pgfqpoint{1.406589in}{0.545318in}}%
\pgfpathlineto{\pgfqpoint{1.455104in}{0.545318in}}%
\pgfpathlineto{\pgfqpoint{1.455104in}{0.555021in}}%
\pgfpathlineto{\pgfqpoint{1.406589in}{0.555021in}}%
\pgfpathlineto{\pgfqpoint{1.406589in}{0.545318in}}%
\pgfpathclose%
\pgfusepath{stroke,fill}%
\end{pgfscope}%
\begin{pgfscope}%
\pgfpathrectangle{\pgfqpoint{0.150000in}{0.150000in}}{\pgfqpoint{1.800000in}{1.800000in}}%
\pgfusepath{clip}%
\pgfsetbuttcap%
\pgfsetroundjoin%
\definecolor{currentfill}{rgb}{0.933333,0.800000,0.400000}%
\pgfsetfillcolor{currentfill}%
\pgfsetlinewidth{1.003750pt}%
\definecolor{currentstroke}{rgb}{0.600000,0.466667,0.000000}%
\pgfsetstrokecolor{currentstroke}%
\pgfsetdash{}{0pt}%
\pgfpathmoveto{\pgfqpoint{1.101888in}{1.101888in}}%
\pgfpathlineto{\pgfqpoint{1.158000in}{1.101888in}}%
\pgfpathlineto{\pgfqpoint{1.158000in}{1.158000in}}%
\pgfpathlineto{\pgfqpoint{1.101888in}{1.158000in}}%
\pgfpathlineto{\pgfqpoint{1.101888in}{1.101888in}}%
\pgfpathclose%
\pgfusepath{stroke,fill}%
\end{pgfscope}%
\begin{pgfscope}%
\pgfpathrectangle{\pgfqpoint{0.150000in}{0.150000in}}{\pgfqpoint{1.800000in}{1.800000in}}%
\pgfusepath{clip}%
\pgfsetbuttcap%
\pgfsetroundjoin%
\definecolor{currentfill}{rgb}{0.933333,0.800000,0.400000}%
\pgfsetfillcolor{currentfill}%
\pgfsetlinewidth{1.003750pt}%
\definecolor{currentstroke}{rgb}{0.600000,0.466667,0.000000}%
\pgfsetstrokecolor{currentstroke}%
\pgfsetdash{}{0pt}%
\pgfpathmoveto{\pgfqpoint{1.055979in}{1.055979in}}%
\pgfpathlineto{\pgfqpoint{1.101888in}{1.055979in}}%
\pgfpathlineto{\pgfqpoint{1.101888in}{1.101888in}}%
\pgfpathlineto{\pgfqpoint{1.055979in}{1.101888in}}%
\pgfpathlineto{\pgfqpoint{1.055979in}{1.055979in}}%
\pgfpathclose%
\pgfusepath{stroke,fill}%
\end{pgfscope}%
\begin{pgfscope}%
\pgfpathrectangle{\pgfqpoint{0.150000in}{0.150000in}}{\pgfqpoint{1.800000in}{1.800000in}}%
\pgfusepath{clip}%
\pgfsetbuttcap%
\pgfsetroundjoin%
\definecolor{currentfill}{rgb}{0.933333,0.800000,0.400000}%
\pgfsetfillcolor{currentfill}%
\pgfsetlinewidth{1.003750pt}%
\definecolor{currentstroke}{rgb}{0.600000,0.466667,0.000000}%
\pgfsetstrokecolor{currentstroke}%
\pgfsetdash{}{0pt}%
\pgfpathmoveto{\pgfqpoint{0.529600in}{1.112892in}}%
\pgfpathlineto{\pgfqpoint{0.549036in}{1.112892in}}%
\pgfpathlineto{\pgfqpoint{0.549036in}{1.171200in}}%
\pgfpathlineto{\pgfqpoint{0.529600in}{1.171200in}}%
\pgfpathlineto{\pgfqpoint{0.529600in}{1.112892in}}%
\pgfpathclose%
\pgfusepath{stroke,fill}%
\end{pgfscope}%
\begin{pgfscope}%
\pgfpathrectangle{\pgfqpoint{0.150000in}{0.150000in}}{\pgfqpoint{1.800000in}{1.800000in}}%
\pgfusepath{clip}%
\pgfsetbuttcap%
\pgfsetroundjoin%
\definecolor{currentfill}{rgb}{0.933333,0.800000,0.400000}%
\pgfsetfillcolor{currentfill}%
\pgfsetlinewidth{1.003750pt}%
\definecolor{currentstroke}{rgb}{0.600000,0.466667,0.000000}%
\pgfsetstrokecolor{currentstroke}%
\pgfsetdash{}{0pt}%
\pgfpathmoveto{\pgfqpoint{0.549036in}{1.065186in}}%
\pgfpathlineto{\pgfqpoint{0.564938in}{1.065186in}}%
\pgfpathlineto{\pgfqpoint{0.564938in}{1.112892in}}%
\pgfpathlineto{\pgfqpoint{0.549036in}{1.112892in}}%
\pgfpathlineto{\pgfqpoint{0.549036in}{1.065186in}}%
\pgfpathclose%
\pgfusepath{stroke,fill}%
\end{pgfscope}%
\begin{pgfscope}%
\pgfpathrectangle{\pgfqpoint{0.150000in}{0.150000in}}{\pgfqpoint{1.800000in}{1.800000in}}%
\pgfusepath{clip}%
\pgfsetbuttcap%
\pgfsetroundjoin%
\definecolor{currentfill}{rgb}{0.933333,0.800000,0.400000}%
\pgfsetfillcolor{currentfill}%
\pgfsetlinewidth{1.003750pt}%
\definecolor{currentstroke}{rgb}{0.600000,0.466667,0.000000}%
\pgfsetstrokecolor{currentstroke}%
\pgfsetdash{}{0pt}%
\pgfpathmoveto{\pgfqpoint{0.806992in}{0.627790in}}%
\pgfpathlineto{\pgfqpoint{0.876631in}{0.627790in}}%
\pgfpathlineto{\pgfqpoint{0.876631in}{0.651003in}}%
\pgfpathlineto{\pgfqpoint{0.806992in}{0.651003in}}%
\pgfpathlineto{\pgfqpoint{0.806992in}{0.627790in}}%
\pgfpathclose%
\pgfusepath{stroke,fill}%
\end{pgfscope}%
\begin{pgfscope}%
\pgfpathrectangle{\pgfqpoint{0.150000in}{0.150000in}}{\pgfqpoint{1.800000in}{1.800000in}}%
\pgfusepath{clip}%
\pgfsetbuttcap%
\pgfsetroundjoin%
\definecolor{currentfill}{rgb}{0.933333,0.800000,0.400000}%
\pgfsetfillcolor{currentfill}%
\pgfsetlinewidth{1.003750pt}%
\definecolor{currentstroke}{rgb}{0.600000,0.466667,0.000000}%
\pgfsetstrokecolor{currentstroke}%
\pgfsetdash{}{0pt}%
\pgfpathmoveto{\pgfqpoint{0.750015in}{0.651003in}}%
\pgfpathlineto{\pgfqpoint{0.806992in}{0.651003in}}%
\pgfpathlineto{\pgfqpoint{0.806992in}{0.669995in}}%
\pgfpathlineto{\pgfqpoint{0.750015in}{0.669995in}}%
\pgfpathlineto{\pgfqpoint{0.750015in}{0.651003in}}%
\pgfpathclose%
\pgfusepath{stroke,fill}%
\end{pgfscope}%
\begin{pgfscope}%
\pgfpathrectangle{\pgfqpoint{0.150000in}{0.150000in}}{\pgfqpoint{1.800000in}{1.800000in}}%
\pgfusepath{clip}%
\pgfsetbuttcap%
\pgfsetroundjoin%
\definecolor{currentfill}{rgb}{0.933333,0.800000,0.400000}%
\pgfsetfillcolor{currentfill}%
\pgfsetlinewidth{1.003750pt}%
\definecolor{currentstroke}{rgb}{0.600000,0.466667,0.000000}%
\pgfsetstrokecolor{currentstroke}%
\pgfsetdash{}{0pt}%
\pgfpathmoveto{\pgfqpoint{0.737830in}{0.737830in}}%
\pgfpathlineto{\pgfqpoint{0.750015in}{0.737830in}}%
\pgfpathlineto{\pgfqpoint{0.750015in}{0.750015in}}%
\pgfpathlineto{\pgfqpoint{0.737830in}{0.750015in}}%
\pgfpathlineto{\pgfqpoint{0.737830in}{0.737830in}}%
\pgfpathclose%
\pgfusepath{stroke,fill}%
\end{pgfscope}%
\begin{pgfscope}%
\pgfpathrectangle{\pgfqpoint{0.150000in}{0.150000in}}{\pgfqpoint{1.800000in}{1.800000in}}%
\pgfusepath{clip}%
\pgfsetbuttcap%
\pgfsetroundjoin%
\definecolor{currentfill}{rgb}{0.933333,0.800000,0.400000}%
\pgfsetfillcolor{currentfill}%
\pgfsetlinewidth{1.003750pt}%
\definecolor{currentstroke}{rgb}{0.600000,0.466667,0.000000}%
\pgfsetstrokecolor{currentstroke}%
\pgfsetdash{}{0pt}%
\pgfpathmoveto{\pgfqpoint{0.646420in}{0.737830in}}%
\pgfpathlineto{\pgfqpoint{0.674057in}{0.737830in}}%
\pgfpathlineto{\pgfqpoint{0.674057in}{0.820740in}}%
\pgfpathlineto{\pgfqpoint{0.646420in}{0.820740in}}%
\pgfpathlineto{\pgfqpoint{0.646420in}{0.737830in}}%
\pgfpathclose%
\pgfusepath{stroke,fill}%
\end{pgfscope}%
\begin{pgfscope}%
\pgfpathrectangle{\pgfqpoint{0.150000in}{0.150000in}}{\pgfqpoint{1.800000in}{1.800000in}}%
\pgfusepath{clip}%
\pgfsetbuttcap%
\pgfsetroundjoin%
\definecolor{currentfill}{rgb}{0.933333,0.800000,0.400000}%
\pgfsetfillcolor{currentfill}%
\pgfsetlinewidth{1.003750pt}%
\definecolor{currentstroke}{rgb}{0.600000,0.466667,0.000000}%
\pgfsetstrokecolor{currentstroke}%
\pgfsetdash{}{0pt}%
\pgfpathmoveto{\pgfqpoint{1.877527in}{0.639505in}}%
\pgfpathlineto{\pgfqpoint{1.950000in}{0.639505in}}%
\pgfpathlineto{\pgfqpoint{1.950000in}{0.654000in}}%
\pgfpathlineto{\pgfqpoint{1.877527in}{0.654000in}}%
\pgfpathlineto{\pgfqpoint{1.877527in}{0.639505in}}%
\pgfpathclose%
\pgfusepath{stroke,fill}%
\end{pgfscope}%
\begin{pgfscope}%
\pgfpathrectangle{\pgfqpoint{0.150000in}{0.150000in}}{\pgfqpoint{1.800000in}{1.800000in}}%
\pgfusepath{clip}%
\pgfsetbuttcap%
\pgfsetroundjoin%
\definecolor{currentfill}{rgb}{0.933333,0.800000,0.400000}%
\pgfsetfillcolor{currentfill}%
\pgfsetlinewidth{1.003750pt}%
\definecolor{currentstroke}{rgb}{0.600000,0.466667,0.000000}%
\pgfsetstrokecolor{currentstroke}%
\pgfsetdash{}{0pt}%
\pgfpathmoveto{\pgfqpoint{1.818231in}{0.627646in}}%
\pgfpathlineto{\pgfqpoint{1.877527in}{0.627646in}}%
\pgfpathlineto{\pgfqpoint{1.877527in}{0.639505in}}%
\pgfpathlineto{\pgfqpoint{1.818231in}{0.639505in}}%
\pgfpathlineto{\pgfqpoint{1.818231in}{0.627646in}}%
\pgfpathclose%
\pgfusepath{stroke,fill}%
\end{pgfscope}%
\begin{pgfscope}%
\pgfpathrectangle{\pgfqpoint{0.150000in}{0.150000in}}{\pgfqpoint{1.800000in}{1.800000in}}%
\pgfusepath{clip}%
\pgfsetbuttcap%
\pgfsetroundjoin%
\definecolor{currentfill}{rgb}{0.933333,0.800000,0.400000}%
\pgfsetfillcolor{currentfill}%
\pgfsetlinewidth{1.003750pt}%
\definecolor{currentstroke}{rgb}{0.600000,0.466667,0.000000}%
\pgfsetstrokecolor{currentstroke}%
\pgfsetdash{}{0pt}%
\pgfpathmoveto{\pgfqpoint{1.758935in}{0.615787in}}%
\pgfpathlineto{\pgfqpoint{1.818231in}{0.615787in}}%
\pgfpathlineto{\pgfqpoint{1.818231in}{0.627646in}}%
\pgfpathlineto{\pgfqpoint{1.758935in}{0.627646in}}%
\pgfpathlineto{\pgfqpoint{1.758935in}{0.615787in}}%
\pgfpathclose%
\pgfusepath{stroke,fill}%
\end{pgfscope}%
\begin{pgfscope}%
\pgfpathrectangle{\pgfqpoint{0.150000in}{0.150000in}}{\pgfqpoint{1.800000in}{1.800000in}}%
\pgfusepath{clip}%
\pgfsetbuttcap%
\pgfsetroundjoin%
\definecolor{currentfill}{rgb}{0.933333,0.800000,0.400000}%
\pgfsetfillcolor{currentfill}%
\pgfsetlinewidth{1.003750pt}%
\definecolor{currentstroke}{rgb}{0.600000,0.466667,0.000000}%
\pgfsetstrokecolor{currentstroke}%
\pgfsetdash{}{0pt}%
\pgfpathmoveto{\pgfqpoint{1.710420in}{0.606084in}}%
\pgfpathlineto{\pgfqpoint{1.758935in}{0.606084in}}%
\pgfpathlineto{\pgfqpoint{1.758935in}{0.615787in}}%
\pgfpathlineto{\pgfqpoint{1.710420in}{0.615787in}}%
\pgfpathlineto{\pgfqpoint{1.710420in}{0.606084in}}%
\pgfpathclose%
\pgfusepath{stroke,fill}%
\end{pgfscope}%
\begin{pgfscope}%
\pgfpathrectangle{\pgfqpoint{0.150000in}{0.150000in}}{\pgfqpoint{1.800000in}{1.800000in}}%
\pgfusepath{clip}%
\pgfsetbuttcap%
\pgfsetroundjoin%
\definecolor{currentfill}{rgb}{0.933333,0.800000,0.400000}%
\pgfsetfillcolor{currentfill}%
\pgfsetlinewidth{1.003750pt}%
\definecolor{currentstroke}{rgb}{0.600000,0.466667,0.000000}%
\pgfsetstrokecolor{currentstroke}%
\pgfsetdash{}{0pt}%
\pgfpathmoveto{\pgfqpoint{1.651124in}{0.594225in}}%
\pgfpathlineto{\pgfqpoint{1.710420in}{0.594225in}}%
\pgfpathlineto{\pgfqpoint{1.710420in}{0.606084in}}%
\pgfpathlineto{\pgfqpoint{1.651124in}{0.606084in}}%
\pgfpathlineto{\pgfqpoint{1.651124in}{0.594225in}}%
\pgfpathclose%
\pgfusepath{stroke,fill}%
\end{pgfscope}%
\begin{pgfscope}%
\pgfpathrectangle{\pgfqpoint{0.150000in}{0.150000in}}{\pgfqpoint{1.800000in}{1.800000in}}%
\pgfusepath{clip}%
\pgfsetbuttcap%
\pgfsetroundjoin%
\definecolor{currentfill}{rgb}{0.933333,0.800000,0.400000}%
\pgfsetfillcolor{currentfill}%
\pgfsetlinewidth{1.003750pt}%
\definecolor{currentstroke}{rgb}{0.600000,0.466667,0.000000}%
\pgfsetstrokecolor{currentstroke}%
\pgfsetdash{}{0pt}%
\pgfpathmoveto{\pgfqpoint{1.602609in}{0.584522in}}%
\pgfpathlineto{\pgfqpoint{1.651124in}{0.584522in}}%
\pgfpathlineto{\pgfqpoint{1.651124in}{0.594225in}}%
\pgfpathlineto{\pgfqpoint{1.602609in}{0.594225in}}%
\pgfpathlineto{\pgfqpoint{1.602609in}{0.584522in}}%
\pgfpathclose%
\pgfusepath{stroke,fill}%
\end{pgfscope}%
\begin{pgfscope}%
\pgfpathrectangle{\pgfqpoint{0.150000in}{0.150000in}}{\pgfqpoint{1.800000in}{1.800000in}}%
\pgfusepath{clip}%
\pgfsetbuttcap%
\pgfsetroundjoin%
\definecolor{currentfill}{rgb}{0.933333,0.800000,0.400000}%
\pgfsetfillcolor{currentfill}%
\pgfsetlinewidth{1.003750pt}%
\definecolor{currentstroke}{rgb}{0.600000,0.466667,0.000000}%
\pgfsetstrokecolor{currentstroke}%
\pgfsetdash{}{0pt}%
\pgfpathmoveto{\pgfqpoint{1.318380in}{0.527676in}}%
\pgfpathlineto{\pgfqpoint{1.406589in}{0.527676in}}%
\pgfpathlineto{\pgfqpoint{1.406589in}{0.545318in}}%
\pgfpathlineto{\pgfqpoint{1.318380in}{0.545318in}}%
\pgfpathlineto{\pgfqpoint{1.318380in}{0.527676in}}%
\pgfpathclose%
\pgfusepath{stroke,fill}%
\end{pgfscope}%
\begin{pgfscope}%
\pgfpathrectangle{\pgfqpoint{0.150000in}{0.150000in}}{\pgfqpoint{1.800000in}{1.800000in}}%
\pgfusepath{clip}%
\pgfsetbuttcap%
\pgfsetroundjoin%
\definecolor{currentfill}{rgb}{0.933333,0.800000,0.400000}%
\pgfsetfillcolor{currentfill}%
\pgfsetlinewidth{1.003750pt}%
\definecolor{currentstroke}{rgb}{0.600000,0.466667,0.000000}%
\pgfsetstrokecolor{currentstroke}%
\pgfsetdash{}{0pt}%
\pgfpathmoveto{\pgfqpoint{1.230171in}{0.510034in}}%
\pgfpathlineto{\pgfqpoint{1.318380in}{0.510034in}}%
\pgfpathlineto{\pgfqpoint{1.318380in}{0.527676in}}%
\pgfpathlineto{\pgfqpoint{1.230171in}{0.527676in}}%
\pgfpathlineto{\pgfqpoint{1.230171in}{0.510034in}}%
\pgfpathclose%
\pgfusepath{stroke,fill}%
\end{pgfscope}%
\begin{pgfscope}%
\pgfpathrectangle{\pgfqpoint{0.150000in}{0.150000in}}{\pgfqpoint{1.800000in}{1.800000in}}%
\pgfusepath{clip}%
\pgfsetbuttcap%
\pgfsetroundjoin%
\definecolor{currentfill}{rgb}{0.933333,0.800000,0.400000}%
\pgfsetfillcolor{currentfill}%
\pgfsetlinewidth{1.003750pt}%
\definecolor{currentstroke}{rgb}{0.600000,0.466667,0.000000}%
\pgfsetstrokecolor{currentstroke}%
\pgfsetdash{}{0pt}%
\pgfpathmoveto{\pgfqpoint{1.158000in}{0.510000in}}%
\pgfpathlineto{\pgfqpoint{1.230171in}{0.510000in}}%
\pgfpathlineto{\pgfqpoint{1.230171in}{0.534000in}}%
\pgfpathlineto{\pgfqpoint{1.158000in}{0.534000in}}%
\pgfpathlineto{\pgfqpoint{1.158000in}{0.510000in}}%
\pgfpathclose%
\pgfusepath{stroke,fill}%
\end{pgfscope}%
\begin{pgfscope}%
\pgfpathrectangle{\pgfqpoint{0.150000in}{0.150000in}}{\pgfqpoint{1.800000in}{1.800000in}}%
\pgfusepath{clip}%
\pgfsetbuttcap%
\pgfsetroundjoin%
\definecolor{currentfill}{rgb}{0.933333,0.800000,0.400000}%
\pgfsetfillcolor{currentfill}%
\pgfsetlinewidth{1.003750pt}%
\definecolor{currentstroke}{rgb}{0.600000,0.466667,0.000000}%
\pgfsetstrokecolor{currentstroke}%
\pgfsetdash{}{0pt}%
\pgfpathmoveto{\pgfqpoint{0.639747in}{1.878735in}}%
\pgfpathlineto{\pgfqpoint{0.654000in}{1.878735in}}%
\pgfpathlineto{\pgfqpoint{0.654000in}{1.950000in}}%
\pgfpathlineto{\pgfqpoint{0.639747in}{1.950000in}}%
\pgfpathlineto{\pgfqpoint{0.639747in}{1.878735in}}%
\pgfpathclose%
\pgfusepath{stroke,fill}%
\end{pgfscope}%
\begin{pgfscope}%
\pgfpathrectangle{\pgfqpoint{0.150000in}{0.150000in}}{\pgfqpoint{1.800000in}{1.800000in}}%
\pgfusepath{clip}%
\pgfsetbuttcap%
\pgfsetroundjoin%
\definecolor{currentfill}{rgb}{0.933333,0.800000,0.400000}%
\pgfsetfillcolor{currentfill}%
\pgfsetlinewidth{1.003750pt}%
\definecolor{currentstroke}{rgb}{0.600000,0.466667,0.000000}%
\pgfsetstrokecolor{currentstroke}%
\pgfsetdash{}{0pt}%
\pgfpathmoveto{\pgfqpoint{0.628085in}{1.820427in}}%
\pgfpathlineto{\pgfqpoint{0.639747in}{1.820427in}}%
\pgfpathlineto{\pgfqpoint{0.639747in}{1.878735in}}%
\pgfpathlineto{\pgfqpoint{0.628085in}{1.878735in}}%
\pgfpathlineto{\pgfqpoint{0.628085in}{1.820427in}}%
\pgfpathclose%
\pgfusepath{stroke,fill}%
\end{pgfscope}%
\begin{pgfscope}%
\pgfpathrectangle{\pgfqpoint{0.150000in}{0.150000in}}{\pgfqpoint{1.800000in}{1.800000in}}%
\pgfusepath{clip}%
\pgfsetbuttcap%
\pgfsetroundjoin%
\definecolor{currentfill}{rgb}{0.933333,0.800000,0.400000}%
\pgfsetfillcolor{currentfill}%
\pgfsetlinewidth{1.003750pt}%
\definecolor{currentstroke}{rgb}{0.600000,0.466667,0.000000}%
\pgfsetstrokecolor{currentstroke}%
\pgfsetdash{}{0pt}%
\pgfpathmoveto{\pgfqpoint{0.616424in}{1.762119in}}%
\pgfpathlineto{\pgfqpoint{0.628085in}{1.762119in}}%
\pgfpathlineto{\pgfqpoint{0.628085in}{1.820427in}}%
\pgfpathlineto{\pgfqpoint{0.616424in}{1.820427in}}%
\pgfpathlineto{\pgfqpoint{0.616424in}{1.762119in}}%
\pgfpathclose%
\pgfusepath{stroke,fill}%
\end{pgfscope}%
\begin{pgfscope}%
\pgfpathrectangle{\pgfqpoint{0.150000in}{0.150000in}}{\pgfqpoint{1.800000in}{1.800000in}}%
\pgfusepath{clip}%
\pgfsetbuttcap%
\pgfsetroundjoin%
\definecolor{currentfill}{rgb}{0.933333,0.800000,0.400000}%
\pgfsetfillcolor{currentfill}%
\pgfsetlinewidth{1.003750pt}%
\definecolor{currentstroke}{rgb}{0.600000,0.466667,0.000000}%
\pgfsetstrokecolor{currentstroke}%
\pgfsetdash{}{0pt}%
\pgfpathmoveto{\pgfqpoint{0.606883in}{1.714413in}}%
\pgfpathlineto{\pgfqpoint{0.616424in}{1.714413in}}%
\pgfpathlineto{\pgfqpoint{0.616424in}{1.762119in}}%
\pgfpathlineto{\pgfqpoint{0.606883in}{1.762119in}}%
\pgfpathlineto{\pgfqpoint{0.606883in}{1.714413in}}%
\pgfpathclose%
\pgfusepath{stroke,fill}%
\end{pgfscope}%
\begin{pgfscope}%
\pgfpathrectangle{\pgfqpoint{0.150000in}{0.150000in}}{\pgfqpoint{1.800000in}{1.800000in}}%
\pgfusepath{clip}%
\pgfsetbuttcap%
\pgfsetroundjoin%
\definecolor{currentfill}{rgb}{0.933333,0.800000,0.400000}%
\pgfsetfillcolor{currentfill}%
\pgfsetlinewidth{1.003750pt}%
\definecolor{currentstroke}{rgb}{0.600000,0.466667,0.000000}%
\pgfsetstrokecolor{currentstroke}%
\pgfsetdash{}{0pt}%
\pgfpathmoveto{\pgfqpoint{0.595221in}{1.656105in}}%
\pgfpathlineto{\pgfqpoint{0.606883in}{1.656105in}}%
\pgfpathlineto{\pgfqpoint{0.606883in}{1.714413in}}%
\pgfpathlineto{\pgfqpoint{0.595221in}{1.714413in}}%
\pgfpathlineto{\pgfqpoint{0.595221in}{1.656105in}}%
\pgfpathclose%
\pgfusepath{stroke,fill}%
\end{pgfscope}%
\begin{pgfscope}%
\pgfpathrectangle{\pgfqpoint{0.150000in}{0.150000in}}{\pgfqpoint{1.800000in}{1.800000in}}%
\pgfusepath{clip}%
\pgfsetbuttcap%
\pgfsetroundjoin%
\definecolor{currentfill}{rgb}{0.933333,0.800000,0.400000}%
\pgfsetfillcolor{currentfill}%
\pgfsetlinewidth{1.003750pt}%
\definecolor{currentstroke}{rgb}{0.600000,0.466667,0.000000}%
\pgfsetstrokecolor{currentstroke}%
\pgfsetdash{}{0pt}%
\pgfpathmoveto{\pgfqpoint{0.585680in}{1.608399in}}%
\pgfpathlineto{\pgfqpoint{0.595221in}{1.608399in}}%
\pgfpathlineto{\pgfqpoint{0.595221in}{1.656105in}}%
\pgfpathlineto{\pgfqpoint{0.585680in}{1.656105in}}%
\pgfpathlineto{\pgfqpoint{0.585680in}{1.608399in}}%
\pgfpathclose%
\pgfusepath{stroke,fill}%
\end{pgfscope}%
\begin{pgfscope}%
\pgfpathrectangle{\pgfqpoint{0.150000in}{0.150000in}}{\pgfqpoint{1.800000in}{1.800000in}}%
\pgfusepath{clip}%
\pgfsetbuttcap%
\pgfsetroundjoin%
\definecolor{currentfill}{rgb}{0.933333,0.800000,0.400000}%
\pgfsetfillcolor{currentfill}%
\pgfsetlinewidth{1.003750pt}%
\definecolor{currentstroke}{rgb}{0.600000,0.466667,0.000000}%
\pgfsetstrokecolor{currentstroke}%
\pgfsetdash{}{0pt}%
\pgfpathmoveto{\pgfqpoint{0.556670in}{1.463352in}}%
\pgfpathlineto{\pgfqpoint{0.568332in}{1.463352in}}%
\pgfpathlineto{\pgfqpoint{0.568332in}{1.521660in}}%
\pgfpathlineto{\pgfqpoint{0.556670in}{1.521660in}}%
\pgfpathlineto{\pgfqpoint{0.556670in}{1.463352in}}%
\pgfpathclose%
\pgfusepath{stroke,fill}%
\end{pgfscope}%
\begin{pgfscope}%
\pgfpathrectangle{\pgfqpoint{0.150000in}{0.150000in}}{\pgfqpoint{1.800000in}{1.800000in}}%
\pgfusepath{clip}%
\pgfsetbuttcap%
\pgfsetroundjoin%
\definecolor{currentfill}{rgb}{0.933333,0.800000,0.400000}%
\pgfsetfillcolor{currentfill}%
\pgfsetlinewidth{1.003750pt}%
\definecolor{currentstroke}{rgb}{0.600000,0.466667,0.000000}%
\pgfsetstrokecolor{currentstroke}%
\pgfsetdash{}{0pt}%
\pgfpathmoveto{\pgfqpoint{0.547129in}{1.415646in}}%
\pgfpathlineto{\pgfqpoint{0.556670in}{1.415646in}}%
\pgfpathlineto{\pgfqpoint{0.556670in}{1.463352in}}%
\pgfpathlineto{\pgfqpoint{0.547129in}{1.463352in}}%
\pgfpathlineto{\pgfqpoint{0.547129in}{1.415646in}}%
\pgfpathclose%
\pgfusepath{stroke,fill}%
\end{pgfscope}%
\begin{pgfscope}%
\pgfpathrectangle{\pgfqpoint{0.150000in}{0.150000in}}{\pgfqpoint{1.800000in}{1.800000in}}%
\pgfusepath{clip}%
\pgfsetbuttcap%
\pgfsetroundjoin%
\definecolor{currentfill}{rgb}{0.933333,0.800000,0.400000}%
\pgfsetfillcolor{currentfill}%
\pgfsetlinewidth{1.003750pt}%
\definecolor{currentstroke}{rgb}{0.600000,0.466667,0.000000}%
\pgfsetstrokecolor{currentstroke}%
\pgfsetdash{}{0pt}%
\pgfpathmoveto{\pgfqpoint{0.972507in}{0.972507in}}%
\pgfpathlineto{\pgfqpoint{1.055979in}{0.972507in}}%
\pgfpathlineto{\pgfqpoint{1.055979in}{1.055979in}}%
\pgfpathlineto{\pgfqpoint{0.972507in}{1.055979in}}%
\pgfpathlineto{\pgfqpoint{0.972507in}{0.972507in}}%
\pgfpathclose%
\pgfusepath{stroke,fill}%
\end{pgfscope}%
\begin{pgfscope}%
\pgfpathrectangle{\pgfqpoint{0.150000in}{0.150000in}}{\pgfqpoint{1.800000in}{1.800000in}}%
\pgfusepath{clip}%
\pgfsetbuttcap%
\pgfsetroundjoin%
\definecolor{currentfill}{rgb}{0.933333,0.800000,0.400000}%
\pgfsetfillcolor{currentfill}%
\pgfsetlinewidth{1.003750pt}%
\definecolor{currentstroke}{rgb}{0.600000,0.466667,0.000000}%
\pgfsetstrokecolor{currentstroke}%
\pgfsetdash{}{0pt}%
\pgfpathmoveto{\pgfqpoint{0.889035in}{0.889035in}}%
\pgfpathlineto{\pgfqpoint{0.972507in}{0.889035in}}%
\pgfpathlineto{\pgfqpoint{0.972507in}{0.972507in}}%
\pgfpathlineto{\pgfqpoint{0.889035in}{0.972507in}}%
\pgfpathlineto{\pgfqpoint{0.889035in}{0.889035in}}%
\pgfpathclose%
\pgfusepath{stroke,fill}%
\end{pgfscope}%
\begin{pgfscope}%
\pgfpathrectangle{\pgfqpoint{0.150000in}{0.150000in}}{\pgfqpoint{1.800000in}{1.800000in}}%
\pgfusepath{clip}%
\pgfsetbuttcap%
\pgfsetroundjoin%
\definecolor{currentfill}{rgb}{0.933333,0.800000,0.400000}%
\pgfsetfillcolor{currentfill}%
\pgfsetlinewidth{1.003750pt}%
\definecolor{currentstroke}{rgb}{0.600000,0.466667,0.000000}%
\pgfsetstrokecolor{currentstroke}%
\pgfsetdash{}{0pt}%
\pgfpathmoveto{\pgfqpoint{0.820740in}{0.820740in}}%
\pgfpathlineto{\pgfqpoint{0.889035in}{0.820740in}}%
\pgfpathlineto{\pgfqpoint{0.889035in}{0.889035in}}%
\pgfpathlineto{\pgfqpoint{0.820740in}{0.889035in}}%
\pgfpathlineto{\pgfqpoint{0.820740in}{0.820740in}}%
\pgfpathclose%
\pgfusepath{stroke,fill}%
\end{pgfscope}%
\begin{pgfscope}%
\pgfpathrectangle{\pgfqpoint{0.150000in}{0.150000in}}{\pgfqpoint{1.800000in}{1.800000in}}%
\pgfusepath{clip}%
\pgfsetbuttcap%
\pgfsetroundjoin%
\definecolor{currentfill}{rgb}{0.933333,0.800000,0.400000}%
\pgfsetfillcolor{currentfill}%
\pgfsetlinewidth{1.003750pt}%
\definecolor{currentstroke}{rgb}{0.600000,0.466667,0.000000}%
\pgfsetstrokecolor{currentstroke}%
\pgfsetdash{}{0pt}%
\pgfpathmoveto{\pgfqpoint{0.564938in}{0.978447in}}%
\pgfpathlineto{\pgfqpoint{0.593851in}{0.978447in}}%
\pgfpathlineto{\pgfqpoint{0.593851in}{1.065186in}}%
\pgfpathlineto{\pgfqpoint{0.564938in}{1.065186in}}%
\pgfpathlineto{\pgfqpoint{0.564938in}{0.978447in}}%
\pgfpathclose%
\pgfusepath{stroke,fill}%
\end{pgfscope}%
\begin{pgfscope}%
\pgfpathrectangle{\pgfqpoint{0.150000in}{0.150000in}}{\pgfqpoint{1.800000in}{1.800000in}}%
\pgfusepath{clip}%
\pgfsetbuttcap%
\pgfsetroundjoin%
\definecolor{currentfill}{rgb}{0.933333,0.800000,0.400000}%
\pgfsetfillcolor{currentfill}%
\pgfsetlinewidth{1.003750pt}%
\definecolor{currentstroke}{rgb}{0.600000,0.466667,0.000000}%
\pgfsetstrokecolor{currentstroke}%
\pgfsetdash{}{0pt}%
\pgfpathmoveto{\pgfqpoint{0.593851in}{0.891708in}}%
\pgfpathlineto{\pgfqpoint{0.622764in}{0.891708in}}%
\pgfpathlineto{\pgfqpoint{0.622764in}{0.978447in}}%
\pgfpathlineto{\pgfqpoint{0.593851in}{0.978447in}}%
\pgfpathlineto{\pgfqpoint{0.593851in}{0.891708in}}%
\pgfpathclose%
\pgfusepath{stroke,fill}%
\end{pgfscope}%
\begin{pgfscope}%
\pgfpathrectangle{\pgfqpoint{0.150000in}{0.150000in}}{\pgfqpoint{1.800000in}{1.800000in}}%
\pgfusepath{clip}%
\pgfsetbuttcap%
\pgfsetroundjoin%
\definecolor{currentfill}{rgb}{0.933333,0.800000,0.400000}%
\pgfsetfillcolor{currentfill}%
\pgfsetlinewidth{1.003750pt}%
\definecolor{currentstroke}{rgb}{0.600000,0.466667,0.000000}%
\pgfsetstrokecolor{currentstroke}%
\pgfsetdash{}{0pt}%
\pgfpathmoveto{\pgfqpoint{0.622764in}{0.820740in}}%
\pgfpathlineto{\pgfqpoint{0.646420in}{0.820740in}}%
\pgfpathlineto{\pgfqpoint{0.646420in}{0.891708in}}%
\pgfpathlineto{\pgfqpoint{0.622764in}{0.891708in}}%
\pgfpathlineto{\pgfqpoint{0.622764in}{0.820740in}}%
\pgfpathclose%
\pgfusepath{stroke,fill}%
\end{pgfscope}%
\begin{pgfscope}%
\pgfpathrectangle{\pgfqpoint{0.150000in}{0.150000in}}{\pgfqpoint{1.800000in}{1.800000in}}%
\pgfusepath{clip}%
\pgfsetbuttcap%
\pgfsetroundjoin%
\definecolor{currentfill}{rgb}{0.933333,0.800000,0.400000}%
\pgfsetfillcolor{currentfill}%
\pgfsetlinewidth{1.003750pt}%
\definecolor{currentstroke}{rgb}{0.600000,0.466667,0.000000}%
\pgfsetstrokecolor{currentstroke}%
\pgfsetdash{}{0pt}%
\pgfpathmoveto{\pgfqpoint{1.072886in}{0.534000in}}%
\pgfpathlineto{\pgfqpoint{1.158000in}{0.534000in}}%
\pgfpathlineto{\pgfqpoint{1.158000in}{0.562371in}}%
\pgfpathlineto{\pgfqpoint{1.072886in}{0.562371in}}%
\pgfpathlineto{\pgfqpoint{1.072886in}{0.534000in}}%
\pgfpathclose%
\pgfusepath{stroke,fill}%
\end{pgfscope}%
\begin{pgfscope}%
\pgfpathrectangle{\pgfqpoint{0.150000in}{0.150000in}}{\pgfqpoint{1.800000in}{1.800000in}}%
\pgfusepath{clip}%
\pgfsetbuttcap%
\pgfsetroundjoin%
\definecolor{currentfill}{rgb}{0.933333,0.800000,0.400000}%
\pgfsetfillcolor{currentfill}%
\pgfsetlinewidth{1.003750pt}%
\definecolor{currentstroke}{rgb}{0.600000,0.466667,0.000000}%
\pgfsetstrokecolor{currentstroke}%
\pgfsetdash{}{0pt}%
\pgfpathmoveto{\pgfqpoint{1.003247in}{0.562371in}}%
\pgfpathlineto{\pgfqpoint{1.072886in}{0.562371in}}%
\pgfpathlineto{\pgfqpoint{1.072886in}{0.585584in}}%
\pgfpathlineto{\pgfqpoint{1.003247in}{0.585584in}}%
\pgfpathlineto{\pgfqpoint{1.003247in}{0.562371in}}%
\pgfpathclose%
\pgfusepath{stroke,fill}%
\end{pgfscope}%
\begin{pgfscope}%
\pgfpathrectangle{\pgfqpoint{0.150000in}{0.150000in}}{\pgfqpoint{1.800000in}{1.800000in}}%
\pgfusepath{clip}%
\pgfsetbuttcap%
\pgfsetroundjoin%
\definecolor{currentfill}{rgb}{0.933333,0.800000,0.400000}%
\pgfsetfillcolor{currentfill}%
\pgfsetlinewidth{1.003750pt}%
\definecolor{currentstroke}{rgb}{0.600000,0.466667,0.000000}%
\pgfsetstrokecolor{currentstroke}%
\pgfsetdash{}{0pt}%
\pgfpathmoveto{\pgfqpoint{0.933608in}{0.585584in}}%
\pgfpathlineto{\pgfqpoint{1.003247in}{0.585584in}}%
\pgfpathlineto{\pgfqpoint{1.003247in}{0.608797in}}%
\pgfpathlineto{\pgfqpoint{0.933608in}{0.608797in}}%
\pgfpathlineto{\pgfqpoint{0.933608in}{0.585584in}}%
\pgfpathclose%
\pgfusepath{stroke,fill}%
\end{pgfscope}%
\begin{pgfscope}%
\pgfpathrectangle{\pgfqpoint{0.150000in}{0.150000in}}{\pgfqpoint{1.800000in}{1.800000in}}%
\pgfusepath{clip}%
\pgfsetbuttcap%
\pgfsetroundjoin%
\definecolor{currentfill}{rgb}{0.933333,0.800000,0.400000}%
\pgfsetfillcolor{currentfill}%
\pgfsetlinewidth{1.003750pt}%
\definecolor{currentstroke}{rgb}{0.600000,0.466667,0.000000}%
\pgfsetstrokecolor{currentstroke}%
\pgfsetdash{}{0pt}%
\pgfpathmoveto{\pgfqpoint{0.876631in}{0.608797in}}%
\pgfpathlineto{\pgfqpoint{0.933608in}{0.608797in}}%
\pgfpathlineto{\pgfqpoint{0.933608in}{0.627790in}}%
\pgfpathlineto{\pgfqpoint{0.876631in}{0.627790in}}%
\pgfpathlineto{\pgfqpoint{0.876631in}{0.608797in}}%
\pgfpathclose%
\pgfusepath{stroke,fill}%
\end{pgfscope}%
\begin{pgfscope}%
\pgfpathrectangle{\pgfqpoint{0.150000in}{0.150000in}}{\pgfqpoint{1.800000in}{1.800000in}}%
\pgfusepath{clip}%
\pgfsetbuttcap%
\pgfsetroundjoin%
\definecolor{currentfill}{rgb}{0.933333,0.800000,0.400000}%
\pgfsetfillcolor{currentfill}%
\pgfsetlinewidth{1.003750pt}%
\definecolor{currentstroke}{rgb}{0.600000,0.466667,0.000000}%
\pgfsetstrokecolor{currentstroke}%
\pgfsetdash{}{0pt}%
\pgfpathmoveto{\pgfqpoint{0.750015in}{0.750015in}}%
\pgfpathlineto{\pgfqpoint{0.820740in}{0.750015in}}%
\pgfpathlineto{\pgfqpoint{0.820740in}{0.820740in}}%
\pgfpathlineto{\pgfqpoint{0.750015in}{0.820740in}}%
\pgfpathlineto{\pgfqpoint{0.750015in}{0.750015in}}%
\pgfpathclose%
\pgfusepath{stroke,fill}%
\end{pgfscope}%
\begin{pgfscope}%
\pgfpathrectangle{\pgfqpoint{0.150000in}{0.150000in}}{\pgfqpoint{1.800000in}{1.800000in}}%
\pgfusepath{clip}%
\pgfsetbuttcap%
\pgfsetroundjoin%
\definecolor{currentfill}{rgb}{0.933333,0.800000,0.400000}%
\pgfsetfillcolor{currentfill}%
\pgfsetlinewidth{1.003750pt}%
\definecolor{currentstroke}{rgb}{0.600000,0.466667,0.000000}%
\pgfsetstrokecolor{currentstroke}%
\pgfsetdash{}{0pt}%
\pgfpathmoveto{\pgfqpoint{0.674057in}{0.669995in}}%
\pgfpathlineto{\pgfqpoint{0.750015in}{0.669995in}}%
\pgfpathlineto{\pgfqpoint{0.750015in}{0.737830in}}%
\pgfpathlineto{\pgfqpoint{0.674057in}{0.737830in}}%
\pgfpathlineto{\pgfqpoint{0.674057in}{0.669995in}}%
\pgfpathclose%
\pgfusepath{stroke,fill}%
\end{pgfscope}%
\begin{pgfscope}%
\pgfpathrectangle{\pgfqpoint{0.150000in}{0.150000in}}{\pgfqpoint{1.800000in}{1.800000in}}%
\pgfusepath{clip}%
\pgfsetbuttcap%
\pgfsetroundjoin%
\definecolor{currentfill}{rgb}{0.933333,0.800000,0.400000}%
\pgfsetfillcolor{currentfill}%
\pgfsetlinewidth{1.003750pt}%
\definecolor{currentstroke}{rgb}{0.600000,0.466667,0.000000}%
\pgfsetstrokecolor{currentstroke}%
\pgfsetdash{}{0pt}%
\pgfpathmoveto{\pgfqpoint{1.400838in}{1.400838in}}%
\pgfpathlineto{\pgfqpoint{1.419162in}{1.400838in}}%
\pgfpathlineto{\pgfqpoint{1.419162in}{1.419162in}}%
\pgfpathlineto{\pgfqpoint{1.400838in}{1.419162in}}%
\pgfpathlineto{\pgfqpoint{1.400838in}{1.400838in}}%
\pgfpathclose%
\pgfusepath{stroke,fill}%
\end{pgfscope}%
\begin{pgfscope}%
\pgfpathrectangle{\pgfqpoint{0.150000in}{0.150000in}}{\pgfqpoint{1.800000in}{1.800000in}}%
\pgfusepath{clip}%
\pgfsetbuttcap%
\pgfsetroundjoin%
\definecolor{currentfill}{rgb}{0.933333,0.800000,0.400000}%
\pgfsetfillcolor{currentfill}%
\pgfsetlinewidth{1.003750pt}%
\definecolor{currentstroke}{rgb}{0.600000,0.466667,0.000000}%
\pgfsetstrokecolor{currentstroke}%
\pgfsetdash{}{0pt}%
\pgfpathmoveto{\pgfqpoint{1.318380in}{1.318380in}}%
\pgfpathlineto{\pgfqpoint{1.400838in}{1.318380in}}%
\pgfpathlineto{\pgfqpoint{1.400838in}{1.400838in}}%
\pgfpathlineto{\pgfqpoint{1.318380in}{1.400838in}}%
\pgfpathlineto{\pgfqpoint{1.318380in}{1.318380in}}%
\pgfpathclose%
\pgfusepath{stroke,fill}%
\end{pgfscope}%
\begin{pgfscope}%
\pgfpathrectangle{\pgfqpoint{0.150000in}{0.150000in}}{\pgfqpoint{1.800000in}{1.800000in}}%
\pgfusepath{clip}%
\pgfsetbuttcap%
\pgfsetroundjoin%
\definecolor{currentfill}{rgb}{0.933333,0.800000,0.400000}%
\pgfsetfillcolor{currentfill}%
\pgfsetlinewidth{1.003750pt}%
\definecolor{currentstroke}{rgb}{0.600000,0.466667,0.000000}%
\pgfsetstrokecolor{currentstroke}%
\pgfsetdash{}{0pt}%
\pgfpathmoveto{\pgfqpoint{1.230171in}{1.230171in}}%
\pgfpathlineto{\pgfqpoint{1.318380in}{1.230171in}}%
\pgfpathlineto{\pgfqpoint{1.318380in}{1.318380in}}%
\pgfpathlineto{\pgfqpoint{1.230171in}{1.318380in}}%
\pgfpathlineto{\pgfqpoint{1.230171in}{1.230171in}}%
\pgfpathclose%
\pgfusepath{stroke,fill}%
\end{pgfscope}%
\begin{pgfscope}%
\pgfpathrectangle{\pgfqpoint{0.150000in}{0.150000in}}{\pgfqpoint{1.800000in}{1.800000in}}%
\pgfusepath{clip}%
\pgfsetbuttcap%
\pgfsetroundjoin%
\definecolor{currentfill}{rgb}{0.933333,0.800000,0.400000}%
\pgfsetfillcolor{currentfill}%
\pgfsetlinewidth{1.003750pt}%
\definecolor{currentstroke}{rgb}{0.600000,0.466667,0.000000}%
\pgfsetstrokecolor{currentstroke}%
\pgfsetdash{}{0pt}%
\pgfpathmoveto{\pgfqpoint{1.158000in}{1.158000in}}%
\pgfpathlineto{\pgfqpoint{1.230171in}{1.158000in}}%
\pgfpathlineto{\pgfqpoint{1.230171in}{1.230171in}}%
\pgfpathlineto{\pgfqpoint{1.158000in}{1.230171in}}%
\pgfpathlineto{\pgfqpoint{1.158000in}{1.158000in}}%
\pgfpathclose%
\pgfusepath{stroke,fill}%
\end{pgfscope}%
\begin{pgfscope}%
\pgfpathrectangle{\pgfqpoint{0.150000in}{0.150000in}}{\pgfqpoint{1.800000in}{1.800000in}}%
\pgfusepath{clip}%
\pgfsetbuttcap%
\pgfsetroundjoin%
\definecolor{currentfill}{rgb}{0.933333,0.800000,0.400000}%
\pgfsetfillcolor{currentfill}%
\pgfsetlinewidth{1.003750pt}%
\definecolor{currentstroke}{rgb}{0.600000,0.466667,0.000000}%
\pgfsetstrokecolor{currentstroke}%
\pgfsetdash{}{0pt}%
\pgfpathmoveto{\pgfqpoint{1.514400in}{0.566880in}}%
\pgfpathlineto{\pgfqpoint{1.602609in}{0.566880in}}%
\pgfpathlineto{\pgfqpoint{1.602609in}{0.584522in}}%
\pgfpathlineto{\pgfqpoint{1.514400in}{0.584522in}}%
\pgfpathlineto{\pgfqpoint{1.514400in}{0.566880in}}%
\pgfpathclose%
\pgfusepath{stroke,fill}%
\end{pgfscope}%
\begin{pgfscope}%
\pgfpathrectangle{\pgfqpoint{0.150000in}{0.150000in}}{\pgfqpoint{1.800000in}{1.800000in}}%
\pgfusepath{clip}%
\pgfsetbuttcap%
\pgfsetroundjoin%
\definecolor{currentfill}{rgb}{0.933333,0.800000,0.400000}%
\pgfsetfillcolor{currentfill}%
\pgfsetlinewidth{1.003750pt}%
\definecolor{currentstroke}{rgb}{0.600000,0.466667,0.000000}%
\pgfsetstrokecolor{currentstroke}%
\pgfsetdash{}{0pt}%
\pgfpathmoveto{\pgfqpoint{0.568332in}{1.521660in}}%
\pgfpathlineto{\pgfqpoint{0.585680in}{1.521660in}}%
\pgfpathlineto{\pgfqpoint{0.585680in}{1.608399in}}%
\pgfpathlineto{\pgfqpoint{0.568332in}{1.608399in}}%
\pgfpathlineto{\pgfqpoint{0.568332in}{1.521660in}}%
\pgfpathclose%
\pgfusepath{stroke,fill}%
\end{pgfscope}%
\begin{pgfscope}%
\pgfpathrectangle{\pgfqpoint{0.150000in}{0.150000in}}{\pgfqpoint{1.800000in}{1.800000in}}%
\pgfusepath{clip}%
\pgfsetbuttcap%
\pgfsetroundjoin%
\definecolor{currentfill}{rgb}{0.933333,0.800000,0.400000}%
\pgfsetfillcolor{currentfill}%
\pgfsetlinewidth{1.003750pt}%
\definecolor{currentstroke}{rgb}{0.600000,0.466667,0.000000}%
\pgfsetstrokecolor{currentstroke}%
\pgfsetdash{}{0pt}%
\pgfpathmoveto{\pgfqpoint{0.529781in}{1.328907in}}%
\pgfpathlineto{\pgfqpoint{0.547129in}{1.328907in}}%
\pgfpathlineto{\pgfqpoint{0.547129in}{1.415646in}}%
\pgfpathlineto{\pgfqpoint{0.529781in}{1.415646in}}%
\pgfpathlineto{\pgfqpoint{0.529781in}{1.328907in}}%
\pgfpathclose%
\pgfusepath{stroke,fill}%
\end{pgfscope}%
\begin{pgfscope}%
\pgfpathrectangle{\pgfqpoint{0.150000in}{0.150000in}}{\pgfqpoint{1.800000in}{1.800000in}}%
\pgfusepath{clip}%
\pgfsetbuttcap%
\pgfsetroundjoin%
\definecolor{currentfill}{rgb}{0.933333,0.800000,0.400000}%
\pgfsetfillcolor{currentfill}%
\pgfsetlinewidth{1.003750pt}%
\definecolor{currentstroke}{rgb}{0.600000,0.466667,0.000000}%
\pgfsetstrokecolor{currentstroke}%
\pgfsetdash{}{0pt}%
\pgfpathmoveto{\pgfqpoint{0.512434in}{1.242168in}}%
\pgfpathlineto{\pgfqpoint{0.529781in}{1.242168in}}%
\pgfpathlineto{\pgfqpoint{0.529781in}{1.328907in}}%
\pgfpathlineto{\pgfqpoint{0.512434in}{1.328907in}}%
\pgfpathlineto{\pgfqpoint{0.512434in}{1.242168in}}%
\pgfpathclose%
\pgfusepath{stroke,fill}%
\end{pgfscope}%
\begin{pgfscope}%
\pgfpathrectangle{\pgfqpoint{0.150000in}{0.150000in}}{\pgfqpoint{1.800000in}{1.800000in}}%
\pgfusepath{clip}%
\pgfsetbuttcap%
\pgfsetroundjoin%
\definecolor{currentfill}{rgb}{0.933333,0.800000,0.400000}%
\pgfsetfillcolor{currentfill}%
\pgfsetlinewidth{1.003750pt}%
\definecolor{currentstroke}{rgb}{0.600000,0.466667,0.000000}%
\pgfsetstrokecolor{currentstroke}%
\pgfsetdash{}{0pt}%
\pgfpathmoveto{\pgfqpoint{0.510000in}{1.171200in}}%
\pgfpathlineto{\pgfqpoint{0.529600in}{1.171200in}}%
\pgfpathlineto{\pgfqpoint{0.529600in}{1.242168in}}%
\pgfpathlineto{\pgfqpoint{0.510000in}{1.242168in}}%
\pgfpathlineto{\pgfqpoint{0.510000in}{1.171200in}}%
\pgfpathclose%
\pgfusepath{stroke,fill}%
\end{pgfscope}%
\begin{pgfscope}%
\pgfpathrectangle{\pgfqpoint{0.150000in}{0.150000in}}{\pgfqpoint{1.800000in}{1.800000in}}%
\pgfusepath{clip}%
\pgfsetbuttcap%
\pgfsetroundjoin%
\definecolor{currentfill}{rgb}{0.933333,0.800000,0.400000}%
\pgfsetfillcolor{currentfill}%
\pgfsetlinewidth{1.003750pt}%
\definecolor{currentstroke}{rgb}{0.600000,0.466667,0.000000}%
\pgfsetstrokecolor{currentstroke}%
\pgfsetdash{}{0pt}%
\pgfpathmoveto{\pgfqpoint{1.158000in}{1.158000in}}%
\pgfpathlineto{\pgfqpoint{1.158000in}{1.158000in}}%
\pgfpathlineto{\pgfqpoint{1.158000in}{1.158000in}}%
\pgfpathlineto{\pgfqpoint{1.158000in}{1.158000in}}%
\pgfpathlineto{\pgfqpoint{1.158000in}{1.158000in}}%
\pgfpathclose%
\pgfusepath{stroke,fill}%
\end{pgfscope}%
\begin{pgfscope}%
\pgfpathrectangle{\pgfqpoint{0.150000in}{0.150000in}}{\pgfqpoint{1.800000in}{1.800000in}}%
\pgfusepath{clip}%
\pgfsetbuttcap%
\pgfsetroundjoin%
\definecolor{currentfill}{rgb}{0.000000,0.000000,0.000000}%
\pgfsetfillcolor{currentfill}%
\pgfsetlinewidth{1.003750pt}%
\definecolor{currentstroke}{rgb}{0.000000,0.000000,0.000000}%
\pgfsetstrokecolor{currentstroke}%
\pgfsetdash{}{0pt}%
\pgfsys@defobject{currentmarker}{\pgfqpoint{-0.038036in}{-0.038036in}}{\pgfqpoint{0.038036in}{0.038036in}}{%
\pgfpathmoveto{\pgfqpoint{0.000000in}{-0.038036in}}%
\pgfpathcurveto{\pgfqpoint{0.010087in}{-0.038036in}}{\pgfqpoint{0.019763in}{-0.034029in}}{\pgfqpoint{0.026896in}{-0.026896in}}%
\pgfpathcurveto{\pgfqpoint{0.034029in}{-0.019763in}}{\pgfqpoint{0.038036in}{-0.010087in}}{\pgfqpoint{0.038036in}{0.000000in}}%
\pgfpathcurveto{\pgfqpoint{0.038036in}{0.010087in}}{\pgfqpoint{0.034029in}{0.019763in}}{\pgfqpoint{0.026896in}{0.026896in}}%
\pgfpathcurveto{\pgfqpoint{0.019763in}{0.034029in}}{\pgfqpoint{0.010087in}{0.038036in}}{\pgfqpoint{0.000000in}{0.038036in}}%
\pgfpathcurveto{\pgfqpoint{-0.010087in}{0.038036in}}{\pgfqpoint{-0.019763in}{0.034029in}}{\pgfqpoint{-0.026896in}{0.026896in}}%
\pgfpathcurveto{\pgfqpoint{-0.034029in}{0.019763in}}{\pgfqpoint{-0.038036in}{0.010087in}}{\pgfqpoint{-0.038036in}{0.000000in}}%
\pgfpathcurveto{\pgfqpoint{-0.038036in}{-0.010087in}}{\pgfqpoint{-0.034029in}{-0.019763in}}{\pgfqpoint{-0.026896in}{-0.026896in}}%
\pgfpathcurveto{\pgfqpoint{-0.019763in}{-0.034029in}}{\pgfqpoint{-0.010087in}{-0.038036in}}{\pgfqpoint{0.000000in}{-0.038036in}}%
\pgfpathlineto{\pgfqpoint{0.000000in}{-0.038036in}}%
\pgfpathclose%
\pgfusepath{stroke,fill}%
}%
\begin{pgfscope}%
\pgfsys@transformshift{0.330000in}{0.330000in}%
\pgfsys@useobject{currentmarker}{}%
\end{pgfscope}%
\end{pgfscope}%
\begin{pgfscope}%
\pgfpathrectangle{\pgfqpoint{0.150000in}{0.150000in}}{\pgfqpoint{1.800000in}{1.800000in}}%
\pgfusepath{clip}%
\pgfsetrectcap%
\pgfsetroundjoin%
\pgfsetlinewidth{2.007500pt}%
\definecolor{currentstroke}{rgb}{0.000000,0.000000,0.000000}%
\pgfsetstrokecolor{currentstroke}%
\pgfsetdash{}{0pt}%
\pgfpathmoveto{\pgfqpoint{1.230000in}{0.510000in}}%
\pgfpathlineto{\pgfqpoint{0.690000in}{0.690000in}}%
\pgfpathlineto{\pgfqpoint{0.510000in}{1.230000in}}%
\pgfpathlineto{\pgfqpoint{1.230000in}{1.590000in}}%
\pgfpathlineto{\pgfqpoint{1.590000in}{1.230000in}}%
\pgfpathlineto{\pgfqpoint{1.230000in}{0.510000in}}%
\pgfusepath{stroke}%
\end{pgfscope}%
\end{pgfpicture}%
\makeatother%
\endgroup%

				\subcaption{Visibility separator with a polygon obstacle}
			\end{subfigure}
			\caption{Visibility separator for a point relative to obstacles. In red the set visible from the point, in blue the invisible set, in yellow the uncertain set.}
			\label{fig:sepvisible}
		\end{figure}

	\subsection*{Main results}
		Solutions can be found for geometric contractors to prevent the apparition of this common boundary. These solutions rely on contractor-specific solutions to avoid the use of contractors union when building more complex contractors, and some of these will be presented.
		
		However, this problem lays the foundations for a larger issue of the characterization of the union of two adjacent contractors in the general case, for which a solution has not yet been found. Figure~\ref{fig:general_case} shows a paving of the union of adjacent contractors on rings, on which the common boundary is appearing.

		\begin{figure}[!htb]
			\centering
			\begin{subfigure}[t]{.48\textwidth}
				\input{imgs/circle_discontinuity.pgf}
				\subcaption{Union of three adjacent ring separators}
			\end{subfigure}%
			\hfill
			\begin{subfigure}[t]{.48\textwidth}
				%% Creator: Matplotlib, PGF backend
%%
%% To include the figure in your LaTeX document, write
%%   \input{<filename>.pgf}
%%
%% Make sure the required packages are loaded in your preamble
%%   \usepackage{pgf}
%%
%% Also ensure that all the required font packages are loaded; for instance,
%% the lmodern package is sometimes necessary when using math font.
%%   \usepackage{lmodern}
%%
%% Figures using additional raster images can only be included by \input if
%% they are in the same directory as the main LaTeX file. For loading figures
%% from other directories you can use the `import` package
%%   \usepackage{import}
%%
%% and then include the figures with
%%   \import{<path to file>}{<filename>.pgf}
%%
%% Matplotlib used the following preamble
%%
\begingroup%
\makeatletter%
\begin{pgfpicture}%
\pgfpathrectangle{\pgfpointorigin}{\pgfqpoint{2.000000in}{2.000000in}}%
\pgfusepath{use as bounding box, clip}%
\begin{pgfscope}%
\pgfsetbuttcap%
\pgfsetmiterjoin%
\definecolor{currentfill}{rgb}{1.000000,1.000000,1.000000}%
\pgfsetfillcolor{currentfill}%
\pgfsetlinewidth{0.000000pt}%
\definecolor{currentstroke}{rgb}{1.000000,1.000000,1.000000}%
\pgfsetstrokecolor{currentstroke}%
\pgfsetdash{}{0pt}%
\pgfpathmoveto{\pgfqpoint{0.000000in}{0.000000in}}%
\pgfpathlineto{\pgfqpoint{2.000000in}{0.000000in}}%
\pgfpathlineto{\pgfqpoint{2.000000in}{2.000000in}}%
\pgfpathlineto{\pgfqpoint{0.000000in}{2.000000in}}%
\pgfpathlineto{\pgfqpoint{0.000000in}{0.000000in}}%
\pgfpathclose%
\pgfusepath{fill}%
\end{pgfscope}%
\begin{pgfscope}%
\pgfpathrectangle{\pgfqpoint{0.150000in}{0.150000in}}{\pgfqpoint{1.700000in}{1.700000in}}%
\pgfusepath{clip}%
\pgfsetbuttcap%
\pgfsetroundjoin%
\definecolor{currentfill}{rgb}{0.933333,0.600000,0.666667}%
\pgfsetfillcolor{currentfill}%
\pgfsetlinewidth{1.003750pt}%
\definecolor{currentstroke}{rgb}{0.600000,0.266667,0.333333}%
\pgfsetstrokecolor{currentstroke}%
\pgfsetdash{}{0pt}%
\pgfpathmoveto{\pgfqpoint{1.466315in}{1.423721in}}%
\pgfpathlineto{\pgfqpoint{1.515791in}{1.423721in}}%
\pgfpathlineto{\pgfqpoint{1.515791in}{1.464356in}}%
\pgfpathlineto{\pgfqpoint{1.466315in}{1.464356in}}%
\pgfpathlineto{\pgfqpoint{1.466315in}{1.423721in}}%
\pgfpathclose%
\pgfusepath{stroke,fill}%
\end{pgfscope}%
\begin{pgfscope}%
\pgfpathrectangle{\pgfqpoint{0.150000in}{0.150000in}}{\pgfqpoint{1.700000in}{1.700000in}}%
\pgfusepath{clip}%
\pgfsetbuttcap%
\pgfsetroundjoin%
\definecolor{currentfill}{rgb}{0.933333,0.600000,0.666667}%
\pgfsetfillcolor{currentfill}%
\pgfsetlinewidth{1.003750pt}%
\definecolor{currentstroke}{rgb}{0.600000,0.266667,0.333333}%
\pgfsetstrokecolor{currentstroke}%
\pgfsetdash{}{0pt}%
\pgfpathmoveto{\pgfqpoint{1.549661in}{1.327483in}}%
\pgfpathlineto{\pgfqpoint{1.586670in}{1.327483in}}%
\pgfpathlineto{\pgfqpoint{1.586670in}{1.370790in}}%
\pgfpathlineto{\pgfqpoint{1.549661in}{1.370790in}}%
\pgfpathlineto{\pgfqpoint{1.549661in}{1.327483in}}%
\pgfpathclose%
\pgfusepath{stroke,fill}%
\end{pgfscope}%
\begin{pgfscope}%
\pgfpathrectangle{\pgfqpoint{0.150000in}{0.150000in}}{\pgfqpoint{1.700000in}{1.700000in}}%
\pgfusepath{clip}%
\pgfsetbuttcap%
\pgfsetroundjoin%
\definecolor{currentfill}{rgb}{0.933333,0.600000,0.666667}%
\pgfsetfillcolor{currentfill}%
\pgfsetlinewidth{1.003750pt}%
\definecolor{currentstroke}{rgb}{0.600000,0.266667,0.333333}%
\pgfsetstrokecolor{currentstroke}%
\pgfsetdash{}{0pt}%
\pgfpathmoveto{\pgfqpoint{1.611900in}{1.248742in}}%
\pgfpathlineto{\pgfqpoint{1.633175in}{1.248742in}}%
\pgfpathlineto{\pgfqpoint{1.633175in}{1.284175in}}%
\pgfpathlineto{\pgfqpoint{1.611900in}{1.284175in}}%
\pgfpathlineto{\pgfqpoint{1.611900in}{1.248742in}}%
\pgfpathclose%
\pgfusepath{stroke,fill}%
\end{pgfscope}%
\begin{pgfscope}%
\pgfpathrectangle{\pgfqpoint{0.150000in}{0.150000in}}{\pgfqpoint{1.700000in}{1.700000in}}%
\pgfusepath{clip}%
\pgfsetbuttcap%
\pgfsetroundjoin%
\definecolor{currentfill}{rgb}{0.933333,0.600000,0.666667}%
\pgfsetfillcolor{currentfill}%
\pgfsetlinewidth{1.003750pt}%
\definecolor{currentstroke}{rgb}{0.600000,0.266667,0.333333}%
\pgfsetstrokecolor{currentstroke}%
\pgfsetdash{}{0pt}%
\pgfpathmoveto{\pgfqpoint{1.647915in}{1.152503in}}%
\pgfpathlineto{\pgfqpoint{1.665826in}{1.152503in}}%
\pgfpathlineto{\pgfqpoint{1.665826in}{1.195811in}}%
\pgfpathlineto{\pgfqpoint{1.647915in}{1.195811in}}%
\pgfpathlineto{\pgfqpoint{1.647915in}{1.152503in}}%
\pgfpathclose%
\pgfusepath{stroke,fill}%
\end{pgfscope}%
\begin{pgfscope}%
\pgfpathrectangle{\pgfqpoint{0.150000in}{0.150000in}}{\pgfqpoint{1.700000in}{1.700000in}}%
\pgfusepath{clip}%
\pgfsetbuttcap%
\pgfsetroundjoin%
\definecolor{currentfill}{rgb}{0.933333,0.600000,0.666667}%
\pgfsetfillcolor{currentfill}%
\pgfsetlinewidth{1.003750pt}%
\definecolor{currentstroke}{rgb}{0.600000,0.266667,0.333333}%
\pgfsetstrokecolor{currentstroke}%
\pgfsetdash{}{0pt}%
\pgfpathmoveto{\pgfqpoint{1.677059in}{1.073763in}}%
\pgfpathlineto{\pgfqpoint{1.685378in}{1.073763in}}%
\pgfpathlineto{\pgfqpoint{1.685378in}{1.109196in}}%
\pgfpathlineto{\pgfqpoint{1.677059in}{1.109196in}}%
\pgfpathlineto{\pgfqpoint{1.677059in}{1.073763in}}%
\pgfpathclose%
\pgfusepath{stroke,fill}%
\end{pgfscope}%
\begin{pgfscope}%
\pgfpathrectangle{\pgfqpoint{0.150000in}{0.150000in}}{\pgfqpoint{1.700000in}{1.700000in}}%
\pgfusepath{clip}%
\pgfsetbuttcap%
\pgfsetroundjoin%
\definecolor{currentfill}{rgb}{0.933333,0.600000,0.666667}%
\pgfsetfillcolor{currentfill}%
\pgfsetlinewidth{1.003750pt}%
\definecolor{currentstroke}{rgb}{0.600000,0.266667,0.333333}%
\pgfsetstrokecolor{currentstroke}%
\pgfsetdash{}{0pt}%
\pgfpathmoveto{\pgfqpoint{1.690091in}{0.995022in}}%
\pgfpathlineto{\pgfqpoint{1.693354in}{0.995022in}}%
\pgfpathlineto{\pgfqpoint{1.693354in}{1.030455in}}%
\pgfpathlineto{\pgfqpoint{1.690091in}{1.030455in}}%
\pgfpathlineto{\pgfqpoint{1.690091in}{0.995022in}}%
\pgfpathclose%
\pgfusepath{stroke,fill}%
\end{pgfscope}%
\begin{pgfscope}%
\pgfpathrectangle{\pgfqpoint{0.150000in}{0.150000in}}{\pgfqpoint{1.700000in}{1.700000in}}%
\pgfusepath{clip}%
\pgfsetbuttcap%
\pgfsetroundjoin%
\definecolor{currentfill}{rgb}{0.933333,0.600000,0.666667}%
\pgfsetfillcolor{currentfill}%
\pgfsetlinewidth{1.003750pt}%
\definecolor{currentstroke}{rgb}{0.600000,0.266667,0.333333}%
\pgfsetstrokecolor{currentstroke}%
\pgfsetdash{}{0pt}%
\pgfpathmoveto{\pgfqpoint{1.349591in}{1.195811in}}%
\pgfpathlineto{\pgfqpoint{1.370538in}{1.195811in}}%
\pgfpathlineto{\pgfqpoint{1.370538in}{1.248742in}}%
\pgfpathlineto{\pgfqpoint{1.349591in}{1.248742in}}%
\pgfpathlineto{\pgfqpoint{1.349591in}{1.195811in}}%
\pgfpathclose%
\pgfusepath{stroke,fill}%
\end{pgfscope}%
\begin{pgfscope}%
\pgfpathrectangle{\pgfqpoint{0.150000in}{0.150000in}}{\pgfqpoint{1.700000in}{1.700000in}}%
\pgfusepath{clip}%
\pgfsetbuttcap%
\pgfsetroundjoin%
\definecolor{currentfill}{rgb}{0.933333,0.600000,0.666667}%
\pgfsetfillcolor{currentfill}%
\pgfsetlinewidth{1.003750pt}%
\definecolor{currentstroke}{rgb}{0.600000,0.266667,0.333333}%
\pgfsetstrokecolor{currentstroke}%
\pgfsetdash{}{0pt}%
\pgfpathmoveto{\pgfqpoint{1.385528in}{1.109196in}}%
\pgfpathlineto{\pgfqpoint{1.393846in}{1.109196in}}%
\pgfpathlineto{\pgfqpoint{1.393846in}{1.152503in}}%
\pgfpathlineto{\pgfqpoint{1.385528in}{1.152503in}}%
\pgfpathlineto{\pgfqpoint{1.385528in}{1.109196in}}%
\pgfpathclose%
\pgfusepath{stroke,fill}%
\end{pgfscope}%
\begin{pgfscope}%
\pgfpathrectangle{\pgfqpoint{0.150000in}{0.150000in}}{\pgfqpoint{1.700000in}{1.700000in}}%
\pgfusepath{clip}%
\pgfsetbuttcap%
\pgfsetroundjoin%
\definecolor{currentfill}{rgb}{0.933333,0.600000,0.666667}%
\pgfsetfillcolor{currentfill}%
\pgfsetlinewidth{1.003750pt}%
\definecolor{currentstroke}{rgb}{0.600000,0.266667,0.333333}%
\pgfsetstrokecolor{currentstroke}%
\pgfsetdash{}{0pt}%
\pgfpathmoveto{\pgfqpoint{1.399535in}{1.030455in}}%
\pgfpathlineto{\pgfqpoint{1.400694in}{1.030455in}}%
\pgfpathlineto{\pgfqpoint{1.400694in}{1.073763in}}%
\pgfpathlineto{\pgfqpoint{1.399535in}{1.073763in}}%
\pgfpathlineto{\pgfqpoint{1.399535in}{1.030455in}}%
\pgfpathclose%
\pgfusepath{stroke,fill}%
\end{pgfscope}%
\begin{pgfscope}%
\pgfpathrectangle{\pgfqpoint{0.150000in}{0.150000in}}{\pgfqpoint{1.700000in}{1.700000in}}%
\pgfusepath{clip}%
\pgfsetbuttcap%
\pgfsetroundjoin%
\definecolor{currentfill}{rgb}{0.933333,0.600000,0.666667}%
\pgfsetfillcolor{currentfill}%
\pgfsetlinewidth{1.003750pt}%
\definecolor{currentstroke}{rgb}{0.600000,0.266667,0.333333}%
\pgfsetstrokecolor{currentstroke}%
\pgfsetdash{}{0pt}%
\pgfpathmoveto{\pgfqpoint{1.674356in}{0.878544in}}%
\pgfpathlineto{\pgfqpoint{1.683312in}{0.878544in}}%
\pgfpathlineto{\pgfqpoint{1.683312in}{0.930598in}}%
\pgfpathlineto{\pgfqpoint{1.674356in}{0.930598in}}%
\pgfpathlineto{\pgfqpoint{1.674356in}{0.878544in}}%
\pgfpathclose%
\pgfusepath{stroke,fill}%
\end{pgfscope}%
\begin{pgfscope}%
\pgfpathrectangle{\pgfqpoint{0.150000in}{0.150000in}}{\pgfqpoint{1.700000in}{1.700000in}}%
\pgfusepath{clip}%
\pgfsetbuttcap%
\pgfsetroundjoin%
\definecolor{currentfill}{rgb}{0.933333,0.600000,0.666667}%
\pgfsetfillcolor{currentfill}%
\pgfsetlinewidth{1.003750pt}%
\definecolor{currentstroke}{rgb}{0.600000,0.266667,0.333333}%
\pgfsetstrokecolor{currentstroke}%
\pgfsetdash{}{0pt}%
\pgfpathmoveto{\pgfqpoint{1.650656in}{0.793365in}}%
\pgfpathlineto{\pgfqpoint{1.662547in}{0.793365in}}%
\pgfpathlineto{\pgfqpoint{1.662547in}{0.835955in}}%
\pgfpathlineto{\pgfqpoint{1.650656in}{0.835955in}}%
\pgfpathlineto{\pgfqpoint{1.650656in}{0.793365in}}%
\pgfpathclose%
\pgfusepath{stroke,fill}%
\end{pgfscope}%
\begin{pgfscope}%
\pgfpathrectangle{\pgfqpoint{0.150000in}{0.150000in}}{\pgfqpoint{1.700000in}{1.700000in}}%
\pgfusepath{clip}%
\pgfsetbuttcap%
\pgfsetroundjoin%
\definecolor{currentfill}{rgb}{0.933333,0.600000,0.666667}%
\pgfsetfillcolor{currentfill}%
\pgfsetlinewidth{1.003750pt}%
\definecolor{currentstroke}{rgb}{0.600000,0.266667,0.333333}%
\pgfsetstrokecolor{currentstroke}%
\pgfsetdash{}{0pt}%
\pgfpathmoveto{\pgfqpoint{1.616409in}{0.715930in}}%
\pgfpathlineto{\pgfqpoint{1.633223in}{0.715930in}}%
\pgfpathlineto{\pgfqpoint{1.633223in}{0.758520in}}%
\pgfpathlineto{\pgfqpoint{1.616409in}{0.758520in}}%
\pgfpathlineto{\pgfqpoint{1.616409in}{0.715930in}}%
\pgfpathclose%
\pgfusepath{stroke,fill}%
\end{pgfscope}%
\begin{pgfscope}%
\pgfpathrectangle{\pgfqpoint{0.150000in}{0.150000in}}{\pgfqpoint{1.700000in}{1.700000in}}%
\pgfusepath{clip}%
\pgfsetbuttcap%
\pgfsetroundjoin%
\definecolor{currentfill}{rgb}{0.933333,0.600000,0.666667}%
\pgfsetfillcolor{currentfill}%
\pgfsetlinewidth{1.003750pt}%
\definecolor{currentstroke}{rgb}{0.600000,0.266667,0.333333}%
\pgfsetstrokecolor{currentstroke}%
\pgfsetdash{}{0pt}%
\pgfpathmoveto{\pgfqpoint{0.878552in}{1.674359in}}%
\pgfpathlineto{\pgfqpoint{0.930598in}{1.674359in}}%
\pgfpathlineto{\pgfqpoint{0.930598in}{1.683313in}}%
\pgfpathlineto{\pgfqpoint{0.878552in}{1.683313in}}%
\pgfpathlineto{\pgfqpoint{0.878552in}{1.674359in}}%
\pgfpathclose%
\pgfusepath{stroke,fill}%
\end{pgfscope}%
\begin{pgfscope}%
\pgfpathrectangle{\pgfqpoint{0.150000in}{0.150000in}}{\pgfqpoint{1.700000in}{1.700000in}}%
\pgfusepath{clip}%
\pgfsetbuttcap%
\pgfsetroundjoin%
\definecolor{currentfill}{rgb}{0.933333,0.600000,0.666667}%
\pgfsetfillcolor{currentfill}%
\pgfsetlinewidth{1.003750pt}%
\definecolor{currentstroke}{rgb}{0.600000,0.266667,0.333333}%
\pgfsetstrokecolor{currentstroke}%
\pgfsetdash{}{0pt}%
\pgfpathmoveto{\pgfqpoint{0.793385in}{1.650666in}}%
\pgfpathlineto{\pgfqpoint{0.835968in}{1.650666in}}%
\pgfpathlineto{\pgfqpoint{0.835968in}{1.662553in}}%
\pgfpathlineto{\pgfqpoint{0.793385in}{1.662553in}}%
\pgfpathlineto{\pgfqpoint{0.793385in}{1.650666in}}%
\pgfpathclose%
\pgfusepath{stroke,fill}%
\end{pgfscope}%
\begin{pgfscope}%
\pgfpathrectangle{\pgfqpoint{0.150000in}{0.150000in}}{\pgfqpoint{1.700000in}{1.700000in}}%
\pgfusepath{clip}%
\pgfsetbuttcap%
\pgfsetroundjoin%
\definecolor{currentfill}{rgb}{0.933333,0.600000,0.666667}%
\pgfsetfillcolor{currentfill}%
\pgfsetlinewidth{1.003750pt}%
\definecolor{currentstroke}{rgb}{0.600000,0.266667,0.333333}%
\pgfsetstrokecolor{currentstroke}%
\pgfsetdash{}{0pt}%
\pgfpathmoveto{\pgfqpoint{0.715961in}{1.616427in}}%
\pgfpathlineto{\pgfqpoint{0.758544in}{1.616427in}}%
\pgfpathlineto{\pgfqpoint{0.758544in}{1.633237in}}%
\pgfpathlineto{\pgfqpoint{0.715961in}{1.633237in}}%
\pgfpathlineto{\pgfqpoint{0.715961in}{1.616427in}}%
\pgfpathclose%
\pgfusepath{stroke,fill}%
\end{pgfscope}%
\begin{pgfscope}%
\pgfpathrectangle{\pgfqpoint{0.150000in}{0.150000in}}{\pgfqpoint{1.700000in}{1.700000in}}%
\pgfusepath{clip}%
\pgfsetbuttcap%
\pgfsetroundjoin%
\definecolor{currentfill}{rgb}{0.933333,0.600000,0.666667}%
\pgfsetfillcolor{currentfill}%
\pgfsetlinewidth{1.003750pt}%
\definecolor{currentstroke}{rgb}{0.600000,0.266667,0.333333}%
\pgfsetstrokecolor{currentstroke}%
\pgfsetdash{}{0pt}%
\pgfpathmoveto{\pgfqpoint{0.856418in}{0.605362in}}%
\pgfpathlineto{\pgfqpoint{0.889799in}{0.605362in}}%
\pgfpathlineto{\pgfqpoint{0.889799in}{0.614758in}}%
\pgfpathlineto{\pgfqpoint{0.856418in}{0.614758in}}%
\pgfpathlineto{\pgfqpoint{0.856418in}{0.605362in}}%
\pgfpathclose%
\pgfusepath{stroke,fill}%
\end{pgfscope}%
\begin{pgfscope}%
\pgfpathrectangle{\pgfqpoint{0.150000in}{0.150000in}}{\pgfqpoint{1.700000in}{1.700000in}}%
\pgfusepath{clip}%
\pgfsetbuttcap%
\pgfsetroundjoin%
\definecolor{currentfill}{rgb}{0.933333,0.600000,0.666667}%
\pgfsetfillcolor{currentfill}%
\pgfsetlinewidth{1.003750pt}%
\definecolor{currentstroke}{rgb}{0.600000,0.266667,0.333333}%
\pgfsetstrokecolor{currentstroke}%
\pgfsetdash{}{0pt}%
\pgfpathmoveto{\pgfqpoint{1.398124in}{1.423721in}}%
\pgfpathlineto{\pgfqpoint{1.466315in}{1.423721in}}%
\pgfpathlineto{\pgfqpoint{1.466315in}{1.514020in}}%
\pgfpathlineto{\pgfqpoint{1.398124in}{1.514020in}}%
\pgfpathlineto{\pgfqpoint{1.398124in}{1.423721in}}%
\pgfpathclose%
\pgfusepath{stroke,fill}%
\end{pgfscope}%
\begin{pgfscope}%
\pgfpathrectangle{\pgfqpoint{0.150000in}{0.150000in}}{\pgfqpoint{1.700000in}{1.700000in}}%
\pgfusepath{clip}%
\pgfsetbuttcap%
\pgfsetroundjoin%
\definecolor{currentfill}{rgb}{0.933333,0.600000,0.666667}%
\pgfsetfillcolor{currentfill}%
\pgfsetlinewidth{1.003750pt}%
\definecolor{currentstroke}{rgb}{0.600000,0.266667,0.333333}%
\pgfsetstrokecolor{currentstroke}%
\pgfsetdash{}{0pt}%
\pgfpathmoveto{\pgfqpoint{1.274139in}{1.568475in}}%
\pgfpathlineto{\pgfqpoint{1.329932in}{1.568475in}}%
\pgfpathlineto{\pgfqpoint{1.329932in}{1.610583in}}%
\pgfpathlineto{\pgfqpoint{1.274139in}{1.610583in}}%
\pgfpathlineto{\pgfqpoint{1.274139in}{1.568475in}}%
\pgfpathclose%
\pgfusepath{stroke,fill}%
\end{pgfscope}%
\begin{pgfscope}%
\pgfpathrectangle{\pgfqpoint{0.150000in}{0.150000in}}{\pgfqpoint{1.700000in}{1.700000in}}%
\pgfusepath{clip}%
\pgfsetbuttcap%
\pgfsetroundjoin%
\definecolor{currentfill}{rgb}{0.933333,0.600000,0.666667}%
\pgfsetfillcolor{currentfill}%
\pgfsetlinewidth{1.003750pt}%
\definecolor{currentstroke}{rgb}{0.600000,0.266667,0.333333}%
\pgfsetstrokecolor{currentstroke}%
\pgfsetdash{}{0pt}%
\pgfpathmoveto{\pgfqpoint{1.549661in}{1.248742in}}%
\pgfpathlineto{\pgfqpoint{1.611900in}{1.248742in}}%
\pgfpathlineto{\pgfqpoint{1.611900in}{1.327483in}}%
\pgfpathlineto{\pgfqpoint{1.549661in}{1.327483in}}%
\pgfpathlineto{\pgfqpoint{1.549661in}{1.248742in}}%
\pgfpathclose%
\pgfusepath{stroke,fill}%
\end{pgfscope}%
\begin{pgfscope}%
\pgfpathrectangle{\pgfqpoint{0.150000in}{0.150000in}}{\pgfqpoint{1.700000in}{1.700000in}}%
\pgfusepath{clip}%
\pgfsetbuttcap%
\pgfsetroundjoin%
\definecolor{currentfill}{rgb}{0.933333,0.600000,0.666667}%
\pgfsetfillcolor{currentfill}%
\pgfsetlinewidth{1.003750pt}%
\definecolor{currentstroke}{rgb}{0.600000,0.266667,0.333333}%
\pgfsetstrokecolor{currentstroke}%
\pgfsetdash{}{0pt}%
\pgfpathmoveto{\pgfqpoint{1.647915in}{1.073763in}}%
\pgfpathlineto{\pgfqpoint{1.677059in}{1.073763in}}%
\pgfpathlineto{\pgfqpoint{1.677059in}{1.152503in}}%
\pgfpathlineto{\pgfqpoint{1.647915in}{1.152503in}}%
\pgfpathlineto{\pgfqpoint{1.647915in}{1.073763in}}%
\pgfpathclose%
\pgfusepath{stroke,fill}%
\end{pgfscope}%
\begin{pgfscope}%
\pgfpathrectangle{\pgfqpoint{0.150000in}{0.150000in}}{\pgfqpoint{1.700000in}{1.700000in}}%
\pgfusepath{clip}%
\pgfsetbuttcap%
\pgfsetroundjoin%
\definecolor{currentfill}{rgb}{0.933333,0.600000,0.666667}%
\pgfsetfillcolor{currentfill}%
\pgfsetlinewidth{1.003750pt}%
\definecolor{currentstroke}{rgb}{0.600000,0.266667,0.333333}%
\pgfsetstrokecolor{currentstroke}%
\pgfsetdash{}{0pt}%
\pgfpathmoveto{\pgfqpoint{1.690091in}{0.930598in}}%
\pgfpathlineto{\pgfqpoint{1.690543in}{0.930598in}}%
\pgfpathlineto{\pgfqpoint{1.690543in}{0.995022in}}%
\pgfpathlineto{\pgfqpoint{1.690091in}{0.995022in}}%
\pgfpathlineto{\pgfqpoint{1.690091in}{0.930598in}}%
\pgfpathclose%
\pgfusepath{stroke,fill}%
\end{pgfscope}%
\begin{pgfscope}%
\pgfpathrectangle{\pgfqpoint{0.150000in}{0.150000in}}{\pgfqpoint{1.700000in}{1.700000in}}%
\pgfusepath{clip}%
\pgfsetbuttcap%
\pgfsetroundjoin%
\definecolor{currentfill}{rgb}{0.933333,0.600000,0.666667}%
\pgfsetfillcolor{currentfill}%
\pgfsetlinewidth{1.003750pt}%
\definecolor{currentstroke}{rgb}{0.600000,0.266667,0.333333}%
\pgfsetstrokecolor{currentstroke}%
\pgfsetdash{}{0pt}%
\pgfpathmoveto{\pgfqpoint{1.370538in}{1.152503in}}%
\pgfpathlineto{\pgfqpoint{1.393846in}{1.152503in}}%
\pgfpathlineto{\pgfqpoint{1.393846in}{1.248742in}}%
\pgfpathlineto{\pgfqpoint{1.370538in}{1.248742in}}%
\pgfpathlineto{\pgfqpoint{1.370538in}{1.152503in}}%
\pgfpathclose%
\pgfusepath{stroke,fill}%
\end{pgfscope}%
\begin{pgfscope}%
\pgfpathrectangle{\pgfqpoint{0.150000in}{0.150000in}}{\pgfqpoint{1.700000in}{1.700000in}}%
\pgfusepath{clip}%
\pgfsetbuttcap%
\pgfsetroundjoin%
\definecolor{currentfill}{rgb}{0.933333,0.600000,0.666667}%
\pgfsetfillcolor{currentfill}%
\pgfsetlinewidth{1.003750pt}%
\definecolor{currentstroke}{rgb}{0.600000,0.266667,0.333333}%
\pgfsetstrokecolor{currentstroke}%
\pgfsetdash{}{0pt}%
\pgfpathmoveto{\pgfqpoint{1.400663in}{0.930598in}}%
\pgfpathlineto{\pgfqpoint{1.400694in}{0.930598in}}%
\pgfpathlineto{\pgfqpoint{1.400694in}{0.995022in}}%
\pgfpathlineto{\pgfqpoint{1.400663in}{0.995022in}}%
\pgfpathlineto{\pgfqpoint{1.400663in}{0.930598in}}%
\pgfpathclose%
\pgfusepath{stroke,fill}%
\end{pgfscope}%
\begin{pgfscope}%
\pgfpathrectangle{\pgfqpoint{0.150000in}{0.150000in}}{\pgfqpoint{1.700000in}{1.700000in}}%
\pgfusepath{clip}%
\pgfsetbuttcap%
\pgfsetroundjoin%
\definecolor{currentfill}{rgb}{0.933333,0.600000,0.666667}%
\pgfsetfillcolor{currentfill}%
\pgfsetlinewidth{1.003750pt}%
\definecolor{currentstroke}{rgb}{0.600000,0.266667,0.333333}%
\pgfsetstrokecolor{currentstroke}%
\pgfsetdash{}{0pt}%
\pgfpathmoveto{\pgfqpoint{1.170218in}{1.637585in}}%
\pgfpathlineto{\pgfqpoint{1.216982in}{1.637585in}}%
\pgfpathlineto{\pgfqpoint{1.216982in}{1.659231in}}%
\pgfpathlineto{\pgfqpoint{1.170218in}{1.659231in}}%
\pgfpathlineto{\pgfqpoint{1.170218in}{1.637585in}}%
\pgfpathclose%
\pgfusepath{stroke,fill}%
\end{pgfscope}%
\begin{pgfscope}%
\pgfpathrectangle{\pgfqpoint{0.150000in}{0.150000in}}{\pgfqpoint{1.700000in}{1.700000in}}%
\pgfusepath{clip}%
\pgfsetbuttcap%
\pgfsetroundjoin%
\definecolor{currentfill}{rgb}{0.933333,0.600000,0.666667}%
\pgfsetfillcolor{currentfill}%
\pgfsetlinewidth{1.003750pt}%
\definecolor{currentstroke}{rgb}{0.600000,0.266667,0.333333}%
\pgfsetstrokecolor{currentstroke}%
\pgfsetdash{}{0pt}%
\pgfpathmoveto{\pgfqpoint{1.085191in}{1.672824in}}%
\pgfpathlineto{\pgfqpoint{1.123453in}{1.672824in}}%
\pgfpathlineto{\pgfqpoint{1.123453in}{1.682954in}}%
\pgfpathlineto{\pgfqpoint{1.085191in}{1.682954in}}%
\pgfpathlineto{\pgfqpoint{1.085191in}{1.672824in}}%
\pgfpathclose%
\pgfusepath{stroke,fill}%
\end{pgfscope}%
\begin{pgfscope}%
\pgfpathrectangle{\pgfqpoint{0.150000in}{0.150000in}}{\pgfqpoint{1.700000in}{1.700000in}}%
\pgfusepath{clip}%
\pgfsetbuttcap%
\pgfsetroundjoin%
\definecolor{currentfill}{rgb}{0.933333,0.600000,0.666667}%
\pgfsetfillcolor{currentfill}%
\pgfsetlinewidth{1.003750pt}%
\definecolor{currentstroke}{rgb}{0.600000,0.266667,0.333333}%
\pgfsetstrokecolor{currentstroke}%
\pgfsetdash{}{0pt}%
\pgfpathmoveto{\pgfqpoint{1.000165in}{1.688774in}}%
\pgfpathlineto{\pgfqpoint{1.038427in}{1.688774in}}%
\pgfpathlineto{\pgfqpoint{1.038427in}{1.692957in}}%
\pgfpathlineto{\pgfqpoint{1.000165in}{1.692957in}}%
\pgfpathlineto{\pgfqpoint{1.000165in}{1.688774in}}%
\pgfpathclose%
\pgfusepath{stroke,fill}%
\end{pgfscope}%
\begin{pgfscope}%
\pgfpathrectangle{\pgfqpoint{0.150000in}{0.150000in}}{\pgfqpoint{1.700000in}{1.700000in}}%
\pgfusepath{clip}%
\pgfsetbuttcap%
\pgfsetroundjoin%
\definecolor{currentfill}{rgb}{0.933333,0.600000,0.666667}%
\pgfsetfillcolor{currentfill}%
\pgfsetlinewidth{1.003750pt}%
\definecolor{currentstroke}{rgb}{0.600000,0.266667,0.333333}%
\pgfsetstrokecolor{currentstroke}%
\pgfsetdash{}{0pt}%
\pgfpathmoveto{\pgfqpoint{1.216982in}{1.336859in}}%
\pgfpathlineto{\pgfqpoint{1.274139in}{1.336859in}}%
\pgfpathlineto{\pgfqpoint{1.274139in}{1.362742in}}%
\pgfpathlineto{\pgfqpoint{1.216982in}{1.362742in}}%
\pgfpathlineto{\pgfqpoint{1.216982in}{1.336859in}}%
\pgfpathclose%
\pgfusepath{stroke,fill}%
\end{pgfscope}%
\begin{pgfscope}%
\pgfpathrectangle{\pgfqpoint{0.150000in}{0.150000in}}{\pgfqpoint{1.700000in}{1.700000in}}%
\pgfusepath{clip}%
\pgfsetbuttcap%
\pgfsetroundjoin%
\definecolor{currentfill}{rgb}{0.933333,0.600000,0.666667}%
\pgfsetfillcolor{currentfill}%
\pgfsetlinewidth{1.003750pt}%
\definecolor{currentstroke}{rgb}{0.600000,0.266667,0.333333}%
\pgfsetstrokecolor{currentstroke}%
\pgfsetdash{}{0pt}%
\pgfpathmoveto{\pgfqpoint{1.123453in}{1.381202in}}%
\pgfpathlineto{\pgfqpoint{1.170218in}{1.381202in}}%
\pgfpathlineto{\pgfqpoint{1.170218in}{1.391533in}}%
\pgfpathlineto{\pgfqpoint{1.123453in}{1.391533in}}%
\pgfpathlineto{\pgfqpoint{1.123453in}{1.381202in}}%
\pgfpathclose%
\pgfusepath{stroke,fill}%
\end{pgfscope}%
\begin{pgfscope}%
\pgfpathrectangle{\pgfqpoint{0.150000in}{0.150000in}}{\pgfqpoint{1.700000in}{1.700000in}}%
\pgfusepath{clip}%
\pgfsetbuttcap%
\pgfsetroundjoin%
\definecolor{currentfill}{rgb}{0.933333,0.600000,0.666667}%
\pgfsetfillcolor{currentfill}%
\pgfsetlinewidth{1.003750pt}%
\definecolor{currentstroke}{rgb}{0.600000,0.266667,0.333333}%
\pgfsetstrokecolor{currentstroke}%
\pgfsetdash{}{0pt}%
\pgfpathmoveto{\pgfqpoint{1.038427in}{1.398847in}}%
\pgfpathlineto{\pgfqpoint{1.085191in}{1.398847in}}%
\pgfpathlineto{\pgfqpoint{1.085191in}{1.400694in}}%
\pgfpathlineto{\pgfqpoint{1.038427in}{1.400694in}}%
\pgfpathlineto{\pgfqpoint{1.038427in}{1.398847in}}%
\pgfpathclose%
\pgfusepath{stroke,fill}%
\end{pgfscope}%
\begin{pgfscope}%
\pgfpathrectangle{\pgfqpoint{0.150000in}{0.150000in}}{\pgfqpoint{1.700000in}{1.700000in}}%
\pgfusepath{clip}%
\pgfsetbuttcap%
\pgfsetroundjoin%
\definecolor{currentfill}{rgb}{0.933333,0.600000,0.666667}%
\pgfsetfillcolor{currentfill}%
\pgfsetlinewidth{1.003750pt}%
\definecolor{currentstroke}{rgb}{0.600000,0.266667,0.333333}%
\pgfsetstrokecolor{currentstroke}%
\pgfsetdash{}{0pt}%
\pgfpathmoveto{\pgfqpoint{1.650656in}{0.835955in}}%
\pgfpathlineto{\pgfqpoint{1.674356in}{0.835955in}}%
\pgfpathlineto{\pgfqpoint{1.674356in}{0.930598in}}%
\pgfpathlineto{\pgfqpoint{1.650656in}{0.930598in}}%
\pgfpathlineto{\pgfqpoint{1.650656in}{0.835955in}}%
\pgfpathclose%
\pgfusepath{stroke,fill}%
\end{pgfscope}%
\begin{pgfscope}%
\pgfpathrectangle{\pgfqpoint{0.150000in}{0.150000in}}{\pgfqpoint{1.700000in}{1.700000in}}%
\pgfusepath{clip}%
\pgfsetbuttcap%
\pgfsetroundjoin%
\definecolor{currentfill}{rgb}{0.933333,0.600000,0.666667}%
\pgfsetfillcolor{currentfill}%
\pgfsetlinewidth{1.003750pt}%
\definecolor{currentstroke}{rgb}{0.600000,0.266667,0.333333}%
\pgfsetstrokecolor{currentstroke}%
\pgfsetdash{}{0pt}%
\pgfpathmoveto{\pgfqpoint{1.579254in}{0.681084in}}%
\pgfpathlineto{\pgfqpoint{1.616409in}{0.681084in}}%
\pgfpathlineto{\pgfqpoint{1.616409in}{0.758520in}}%
\pgfpathlineto{\pgfqpoint{1.579254in}{0.758520in}}%
\pgfpathlineto{\pgfqpoint{1.579254in}{0.681084in}}%
\pgfpathclose%
\pgfusepath{stroke,fill}%
\end{pgfscope}%
\begin{pgfscope}%
\pgfpathrectangle{\pgfqpoint{0.150000in}{0.150000in}}{\pgfqpoint{1.700000in}{1.700000in}}%
\pgfusepath{clip}%
\pgfsetbuttcap%
\pgfsetroundjoin%
\definecolor{currentfill}{rgb}{0.933333,0.600000,0.666667}%
\pgfsetfillcolor{currentfill}%
\pgfsetlinewidth{1.003750pt}%
\definecolor{currentstroke}{rgb}{0.600000,0.266667,0.333333}%
\pgfsetstrokecolor{currentstroke}%
\pgfsetdash{}{0pt}%
\pgfpathmoveto{\pgfqpoint{1.376454in}{0.807234in}}%
\pgfpathlineto{\pgfqpoint{1.394638in}{0.807234in}}%
\pgfpathlineto{\pgfqpoint{1.394638in}{0.862748in}}%
\pgfpathlineto{\pgfqpoint{1.376454in}{0.862748in}}%
\pgfpathlineto{\pgfqpoint{1.376454in}{0.807234in}}%
\pgfpathclose%
\pgfusepath{stroke,fill}%
\end{pgfscope}%
\begin{pgfscope}%
\pgfpathrectangle{\pgfqpoint{0.150000in}{0.150000in}}{\pgfqpoint{1.700000in}{1.700000in}}%
\pgfusepath{clip}%
\pgfsetbuttcap%
\pgfsetroundjoin%
\definecolor{currentfill}{rgb}{0.933333,0.600000,0.666667}%
\pgfsetfillcolor{currentfill}%
\pgfsetlinewidth{1.003750pt}%
\definecolor{currentstroke}{rgb}{0.600000,0.266667,0.333333}%
\pgfsetstrokecolor{currentstroke}%
\pgfsetdash{}{0pt}%
\pgfpathmoveto{\pgfqpoint{1.314504in}{0.706301in}}%
\pgfpathlineto{\pgfqpoint{1.351279in}{0.706301in}}%
\pgfpathlineto{\pgfqpoint{1.351279in}{0.751721in}}%
\pgfpathlineto{\pgfqpoint{1.314504in}{0.751721in}}%
\pgfpathlineto{\pgfqpoint{1.314504in}{0.706301in}}%
\pgfpathclose%
\pgfusepath{stroke,fill}%
\end{pgfscope}%
\begin{pgfscope}%
\pgfpathrectangle{\pgfqpoint{0.150000in}{0.150000in}}{\pgfqpoint{1.700000in}{1.700000in}}%
\pgfusepath{clip}%
\pgfsetbuttcap%
\pgfsetroundjoin%
\definecolor{currentfill}{rgb}{0.933333,0.600000,0.666667}%
\pgfsetfillcolor{currentfill}%
\pgfsetlinewidth{1.003750pt}%
\definecolor{currentstroke}{rgb}{0.600000,0.266667,0.333333}%
\pgfsetstrokecolor{currentstroke}%
\pgfsetdash{}{0pt}%
\pgfpathmoveto{\pgfqpoint{1.501483in}{0.564103in}}%
\pgfpathlineto{\pgfqpoint{1.540056in}{0.564103in}}%
\pgfpathlineto{\pgfqpoint{1.540056in}{0.617728in}}%
\pgfpathlineto{\pgfqpoint{1.501483in}{0.617728in}}%
\pgfpathlineto{\pgfqpoint{1.501483in}{0.564103in}}%
\pgfpathclose%
\pgfusepath{stroke,fill}%
\end{pgfscope}%
\begin{pgfscope}%
\pgfpathrectangle{\pgfqpoint{0.150000in}{0.150000in}}{\pgfqpoint{1.700000in}{1.700000in}}%
\pgfusepath{clip}%
\pgfsetbuttcap%
\pgfsetroundjoin%
\definecolor{currentfill}{rgb}{0.933333,0.600000,0.666667}%
\pgfsetfillcolor{currentfill}%
\pgfsetlinewidth{1.003750pt}%
\definecolor{currentstroke}{rgb}{0.600000,0.266667,0.333333}%
\pgfsetstrokecolor{currentstroke}%
\pgfsetdash{}{0pt}%
\pgfpathmoveto{\pgfqpoint{1.410580in}{0.472907in}}%
\pgfpathlineto{\pgfqpoint{1.451486in}{0.472907in}}%
\pgfpathlineto{\pgfqpoint{1.451486in}{0.520228in}}%
\pgfpathlineto{\pgfqpoint{1.410580in}{0.520228in}}%
\pgfpathlineto{\pgfqpoint{1.410580in}{0.472907in}}%
\pgfpathclose%
\pgfusepath{stroke,fill}%
\end{pgfscope}%
\begin{pgfscope}%
\pgfpathrectangle{\pgfqpoint{0.150000in}{0.150000in}}{\pgfqpoint{1.700000in}{1.700000in}}%
\pgfusepath{clip}%
\pgfsetbuttcap%
\pgfsetroundjoin%
\definecolor{currentfill}{rgb}{0.933333,0.600000,0.666667}%
\pgfsetfillcolor{currentfill}%
\pgfsetlinewidth{1.003750pt}%
\definecolor{currentstroke}{rgb}{0.600000,0.266667,0.333333}%
\pgfsetstrokecolor{currentstroke}%
\pgfsetdash{}{0pt}%
\pgfpathmoveto{\pgfqpoint{1.334676in}{0.412098in}}%
\pgfpathlineto{\pgfqpoint{1.368833in}{0.412098in}}%
\pgfpathlineto{\pgfqpoint{1.368833in}{0.440455in}}%
\pgfpathlineto{\pgfqpoint{1.334676in}{0.440455in}}%
\pgfpathlineto{\pgfqpoint{1.334676in}{0.412098in}}%
\pgfpathclose%
\pgfusepath{stroke,fill}%
\end{pgfscope}%
\begin{pgfscope}%
\pgfpathrectangle{\pgfqpoint{0.150000in}{0.150000in}}{\pgfqpoint{1.700000in}{1.700000in}}%
\pgfusepath{clip}%
\pgfsetbuttcap%
\pgfsetroundjoin%
\definecolor{currentfill}{rgb}{0.933333,0.600000,0.666667}%
\pgfsetfillcolor{currentfill}%
\pgfsetlinewidth{1.003750pt}%
\definecolor{currentstroke}{rgb}{0.600000,0.266667,0.333333}%
\pgfsetstrokecolor{currentstroke}%
\pgfsetdash{}{0pt}%
\pgfpathmoveto{\pgfqpoint{1.215677in}{0.636748in}}%
\pgfpathlineto{\pgfqpoint{1.272573in}{0.636748in}}%
\pgfpathlineto{\pgfqpoint{1.272573in}{0.662303in}}%
\pgfpathlineto{\pgfqpoint{1.215677in}{0.662303in}}%
\pgfpathlineto{\pgfqpoint{1.215677in}{0.636748in}}%
\pgfpathclose%
\pgfusepath{stroke,fill}%
\end{pgfscope}%
\begin{pgfscope}%
\pgfpathrectangle{\pgfqpoint{0.150000in}{0.150000in}}{\pgfqpoint{1.700000in}{1.700000in}}%
\pgfusepath{clip}%
\pgfsetbuttcap%
\pgfsetroundjoin%
\definecolor{currentfill}{rgb}{0.933333,0.600000,0.666667}%
\pgfsetfillcolor{currentfill}%
\pgfsetlinewidth{1.003750pt}%
\definecolor{currentstroke}{rgb}{0.600000,0.266667,0.333333}%
\pgfsetstrokecolor{currentstroke}%
\pgfsetdash{}{0pt}%
\pgfpathmoveto{\pgfqpoint{1.122574in}{0.608314in}}%
\pgfpathlineto{\pgfqpoint{1.169126in}{0.608314in}}%
\pgfpathlineto{\pgfqpoint{1.169126in}{0.618515in}}%
\pgfpathlineto{\pgfqpoint{1.122574in}{0.618515in}}%
\pgfpathlineto{\pgfqpoint{1.122574in}{0.608314in}}%
\pgfpathclose%
\pgfusepath{stroke,fill}%
\end{pgfscope}%
\begin{pgfscope}%
\pgfpathrectangle{\pgfqpoint{0.150000in}{0.150000in}}{\pgfqpoint{1.700000in}{1.700000in}}%
\pgfusepath{clip}%
\pgfsetbuttcap%
\pgfsetroundjoin%
\definecolor{currentfill}{rgb}{0.933333,0.600000,0.666667}%
\pgfsetfillcolor{currentfill}%
\pgfsetlinewidth{1.003750pt}%
\definecolor{currentstroke}{rgb}{0.600000,0.266667,0.333333}%
\pgfsetstrokecolor{currentstroke}%
\pgfsetdash{}{0pt}%
\pgfpathmoveto{\pgfqpoint{1.037935in}{0.599306in}}%
\pgfpathlineto{\pgfqpoint{1.084487in}{0.599306in}}%
\pgfpathlineto{\pgfqpoint{1.084487in}{0.601106in}}%
\pgfpathlineto{\pgfqpoint{1.037935in}{0.601106in}}%
\pgfpathlineto{\pgfqpoint{1.037935in}{0.599306in}}%
\pgfpathclose%
\pgfusepath{stroke,fill}%
\end{pgfscope}%
\begin{pgfscope}%
\pgfpathrectangle{\pgfqpoint{0.150000in}{0.150000in}}{\pgfqpoint{1.700000in}{1.700000in}}%
\pgfusepath{clip}%
\pgfsetbuttcap%
\pgfsetroundjoin%
\definecolor{currentfill}{rgb}{0.933333,0.600000,0.666667}%
\pgfsetfillcolor{currentfill}%
\pgfsetlinewidth{1.003750pt}%
\definecolor{currentstroke}{rgb}{0.600000,0.266667,0.333333}%
\pgfsetstrokecolor{currentstroke}%
\pgfsetdash{}{0pt}%
\pgfpathmoveto{\pgfqpoint{1.169126in}{0.340341in}}%
\pgfpathlineto{\pgfqpoint{1.215677in}{0.340341in}}%
\pgfpathlineto{\pgfqpoint{1.215677in}{0.361744in}}%
\pgfpathlineto{\pgfqpoint{1.169126in}{0.361744in}}%
\pgfpathlineto{\pgfqpoint{1.169126in}{0.340341in}}%
\pgfpathclose%
\pgfusepath{stroke,fill}%
\end{pgfscope}%
\begin{pgfscope}%
\pgfpathrectangle{\pgfqpoint{0.150000in}{0.150000in}}{\pgfqpoint{1.700000in}{1.700000in}}%
\pgfusepath{clip}%
\pgfsetbuttcap%
\pgfsetroundjoin%
\definecolor{currentfill}{rgb}{0.933333,0.600000,0.666667}%
\pgfsetfillcolor{currentfill}%
\pgfsetlinewidth{1.003750pt}%
\definecolor{currentstroke}{rgb}{0.600000,0.266667,0.333333}%
\pgfsetstrokecolor{currentstroke}%
\pgfsetdash{}{0pt}%
\pgfpathmoveto{\pgfqpoint{1.084487in}{0.316888in}}%
\pgfpathlineto{\pgfqpoint{1.122574in}{0.316888in}}%
\pgfpathlineto{\pgfqpoint{1.122574in}{0.326900in}}%
\pgfpathlineto{\pgfqpoint{1.084487in}{0.326900in}}%
\pgfpathlineto{\pgfqpoint{1.084487in}{0.316888in}}%
\pgfpathclose%
\pgfusepath{stroke,fill}%
\end{pgfscope}%
\begin{pgfscope}%
\pgfpathrectangle{\pgfqpoint{0.150000in}{0.150000in}}{\pgfqpoint{1.700000in}{1.700000in}}%
\pgfusepath{clip}%
\pgfsetbuttcap%
\pgfsetroundjoin%
\definecolor{currentfill}{rgb}{0.933333,0.600000,0.666667}%
\pgfsetfillcolor{currentfill}%
\pgfsetlinewidth{1.003750pt}%
\definecolor{currentstroke}{rgb}{0.600000,0.266667,0.333333}%
\pgfsetstrokecolor{currentstroke}%
\pgfsetdash{}{0pt}%
\pgfpathmoveto{\pgfqpoint{0.999848in}{0.307015in}}%
\pgfpathlineto{\pgfqpoint{1.037935in}{0.307015in}}%
\pgfpathlineto{\pgfqpoint{1.037935in}{0.311140in}}%
\pgfpathlineto{\pgfqpoint{0.999848in}{0.311140in}}%
\pgfpathlineto{\pgfqpoint{0.999848in}{0.307015in}}%
\pgfpathclose%
\pgfusepath{stroke,fill}%
\end{pgfscope}%
\begin{pgfscope}%
\pgfpathrectangle{\pgfqpoint{0.150000in}{0.150000in}}{\pgfqpoint{1.700000in}{1.700000in}}%
\pgfusepath{clip}%
\pgfsetbuttcap%
\pgfsetroundjoin%
\definecolor{currentfill}{rgb}{0.933333,0.600000,0.666667}%
\pgfsetfillcolor{currentfill}%
\pgfsetlinewidth{1.003750pt}%
\definecolor{currentstroke}{rgb}{0.600000,0.266667,0.333333}%
\pgfsetstrokecolor{currentstroke}%
\pgfsetdash{}{0pt}%
\pgfpathmoveto{\pgfqpoint{0.835968in}{1.650666in}}%
\pgfpathlineto{\pgfqpoint{0.930598in}{1.650666in}}%
\pgfpathlineto{\pgfqpoint{0.930598in}{1.674359in}}%
\pgfpathlineto{\pgfqpoint{0.835968in}{1.674359in}}%
\pgfpathlineto{\pgfqpoint{0.835968in}{1.650666in}}%
\pgfpathclose%
\pgfusepath{stroke,fill}%
\end{pgfscope}%
\begin{pgfscope}%
\pgfpathrectangle{\pgfqpoint{0.150000in}{0.150000in}}{\pgfqpoint{1.700000in}{1.700000in}}%
\pgfusepath{clip}%
\pgfsetbuttcap%
\pgfsetroundjoin%
\definecolor{currentfill}{rgb}{0.933333,0.600000,0.666667}%
\pgfsetfillcolor{currentfill}%
\pgfsetlinewidth{1.003750pt}%
\definecolor{currentstroke}{rgb}{0.600000,0.266667,0.333333}%
\pgfsetstrokecolor{currentstroke}%
\pgfsetdash{}{0pt}%
\pgfpathmoveto{\pgfqpoint{0.681120in}{1.579284in}}%
\pgfpathlineto{\pgfqpoint{0.758544in}{1.579284in}}%
\pgfpathlineto{\pgfqpoint{0.758544in}{1.616427in}}%
\pgfpathlineto{\pgfqpoint{0.681120in}{1.616427in}}%
\pgfpathlineto{\pgfqpoint{0.681120in}{1.579284in}}%
\pgfpathclose%
\pgfusepath{stroke,fill}%
\end{pgfscope}%
\begin{pgfscope}%
\pgfpathrectangle{\pgfqpoint{0.150000in}{0.150000in}}{\pgfqpoint{1.700000in}{1.700000in}}%
\pgfusepath{clip}%
\pgfsetbuttcap%
\pgfsetroundjoin%
\definecolor{currentfill}{rgb}{0.933333,0.600000,0.666667}%
\pgfsetfillcolor{currentfill}%
\pgfsetlinewidth{1.003750pt}%
\definecolor{currentstroke}{rgb}{0.600000,0.266667,0.333333}%
\pgfsetstrokecolor{currentstroke}%
\pgfsetdash{}{0pt}%
\pgfpathmoveto{\pgfqpoint{0.807332in}{1.376473in}}%
\pgfpathlineto{\pgfqpoint{0.862802in}{1.376473in}}%
\pgfpathlineto{\pgfqpoint{0.862802in}{1.394638in}}%
\pgfpathlineto{\pgfqpoint{0.807332in}{1.394638in}}%
\pgfpathlineto{\pgfqpoint{0.807332in}{1.376473in}}%
\pgfpathclose%
\pgfusepath{stroke,fill}%
\end{pgfscope}%
\begin{pgfscope}%
\pgfpathrectangle{\pgfqpoint{0.150000in}{0.150000in}}{\pgfqpoint{1.700000in}{1.700000in}}%
\pgfusepath{clip}%
\pgfsetbuttcap%
\pgfsetroundjoin%
\definecolor{currentfill}{rgb}{0.933333,0.600000,0.666667}%
\pgfsetfillcolor{currentfill}%
\pgfsetlinewidth{1.003750pt}%
\definecolor{currentstroke}{rgb}{0.600000,0.266667,0.333333}%
\pgfsetstrokecolor{currentstroke}%
\pgfsetdash{}{0pt}%
\pgfpathmoveto{\pgfqpoint{0.706478in}{1.314616in}}%
\pgfpathlineto{\pgfqpoint{0.751862in}{1.314616in}}%
\pgfpathlineto{\pgfqpoint{0.751862in}{1.351333in}}%
\pgfpathlineto{\pgfqpoint{0.706478in}{1.351333in}}%
\pgfpathlineto{\pgfqpoint{0.706478in}{1.314616in}}%
\pgfpathclose%
\pgfusepath{stroke,fill}%
\end{pgfscope}%
\begin{pgfscope}%
\pgfpathrectangle{\pgfqpoint{0.150000in}{0.150000in}}{\pgfqpoint{1.700000in}{1.700000in}}%
\pgfusepath{clip}%
\pgfsetbuttcap%
\pgfsetroundjoin%
\definecolor{currentfill}{rgb}{0.933333,0.600000,0.666667}%
\pgfsetfillcolor{currentfill}%
\pgfsetlinewidth{1.003750pt}%
\definecolor{currentstroke}{rgb}{0.600000,0.266667,0.333333}%
\pgfsetstrokecolor{currentstroke}%
\pgfsetdash{}{0pt}%
\pgfpathmoveto{\pgfqpoint{0.564161in}{1.501548in}}%
\pgfpathlineto{\pgfqpoint{0.617773in}{1.501548in}}%
\pgfpathlineto{\pgfqpoint{0.617773in}{1.540103in}}%
\pgfpathlineto{\pgfqpoint{0.564161in}{1.540103in}}%
\pgfpathlineto{\pgfqpoint{0.564161in}{1.501548in}}%
\pgfpathclose%
\pgfusepath{stroke,fill}%
\end{pgfscope}%
\begin{pgfscope}%
\pgfpathrectangle{\pgfqpoint{0.150000in}{0.150000in}}{\pgfqpoint{1.700000in}{1.700000in}}%
\pgfusepath{clip}%
\pgfsetbuttcap%
\pgfsetroundjoin%
\definecolor{currentfill}{rgb}{0.933333,0.600000,0.666667}%
\pgfsetfillcolor{currentfill}%
\pgfsetlinewidth{1.003750pt}%
\definecolor{currentstroke}{rgb}{0.600000,0.266667,0.333333}%
\pgfsetstrokecolor{currentstroke}%
\pgfsetdash{}{0pt}%
\pgfpathmoveto{\pgfqpoint{0.472988in}{1.410698in}}%
\pgfpathlineto{\pgfqpoint{0.520296in}{1.410698in}}%
\pgfpathlineto{\pgfqpoint{0.520296in}{1.451581in}}%
\pgfpathlineto{\pgfqpoint{0.472988in}{1.451581in}}%
\pgfpathlineto{\pgfqpoint{0.472988in}{1.410698in}}%
\pgfpathclose%
\pgfusepath{stroke,fill}%
\end{pgfscope}%
\begin{pgfscope}%
\pgfpathrectangle{\pgfqpoint{0.150000in}{0.150000in}}{\pgfqpoint{1.700000in}{1.700000in}}%
\pgfusepath{clip}%
\pgfsetbuttcap%
\pgfsetroundjoin%
\definecolor{currentfill}{rgb}{0.933333,0.600000,0.666667}%
\pgfsetfillcolor{currentfill}%
\pgfsetlinewidth{1.003750pt}%
\definecolor{currentstroke}{rgb}{0.600000,0.266667,0.333333}%
\pgfsetstrokecolor{currentstroke}%
\pgfsetdash{}{0pt}%
\pgfpathmoveto{\pgfqpoint{0.412186in}{1.334835in}}%
\pgfpathlineto{\pgfqpoint{0.440542in}{1.334835in}}%
\pgfpathlineto{\pgfqpoint{0.440542in}{1.368973in}}%
\pgfpathlineto{\pgfqpoint{0.412186in}{1.368973in}}%
\pgfpathlineto{\pgfqpoint{0.412186in}{1.334835in}}%
\pgfpathclose%
\pgfusepath{stroke,fill}%
\end{pgfscope}%
\begin{pgfscope}%
\pgfpathrectangle{\pgfqpoint{0.150000in}{0.150000in}}{\pgfqpoint{1.700000in}{1.700000in}}%
\pgfusepath{clip}%
\pgfsetbuttcap%
\pgfsetroundjoin%
\definecolor{currentfill}{rgb}{0.933333,0.600000,0.666667}%
\pgfsetfillcolor{currentfill}%
\pgfsetlinewidth{1.003750pt}%
\definecolor{currentstroke}{rgb}{0.600000,0.266667,0.333333}%
\pgfsetstrokecolor{currentstroke}%
\pgfsetdash{}{0pt}%
\pgfpathmoveto{\pgfqpoint{0.636859in}{1.215894in}}%
\pgfpathlineto{\pgfqpoint{0.662442in}{1.215894in}}%
\pgfpathlineto{\pgfqpoint{0.662442in}{1.272765in}}%
\pgfpathlineto{\pgfqpoint{0.636859in}{1.272765in}}%
\pgfpathlineto{\pgfqpoint{0.636859in}{1.215894in}}%
\pgfpathclose%
\pgfusepath{stroke,fill}%
\end{pgfscope}%
\begin{pgfscope}%
\pgfpathrectangle{\pgfqpoint{0.150000in}{0.150000in}}{\pgfqpoint{1.700000in}{1.700000in}}%
\pgfusepath{clip}%
\pgfsetbuttcap%
\pgfsetroundjoin%
\definecolor{currentfill}{rgb}{0.933333,0.600000,0.666667}%
\pgfsetfillcolor{currentfill}%
\pgfsetlinewidth{1.003750pt}%
\definecolor{currentstroke}{rgb}{0.600000,0.266667,0.333333}%
\pgfsetstrokecolor{currentstroke}%
\pgfsetdash{}{0pt}%
\pgfpathmoveto{\pgfqpoint{0.608374in}{1.122834in}}%
\pgfpathlineto{\pgfqpoint{0.618598in}{1.122834in}}%
\pgfpathlineto{\pgfqpoint{0.618598in}{1.169364in}}%
\pgfpathlineto{\pgfqpoint{0.608374in}{1.169364in}}%
\pgfpathlineto{\pgfqpoint{0.608374in}{1.122834in}}%
\pgfpathclose%
\pgfusepath{stroke,fill}%
\end{pgfscope}%
\begin{pgfscope}%
\pgfpathrectangle{\pgfqpoint{0.150000in}{0.150000in}}{\pgfqpoint{1.700000in}{1.700000in}}%
\pgfusepath{clip}%
\pgfsetbuttcap%
\pgfsetroundjoin%
\definecolor{currentfill}{rgb}{0.933333,0.600000,0.666667}%
\pgfsetfillcolor{currentfill}%
\pgfsetlinewidth{1.003750pt}%
\definecolor{currentstroke}{rgb}{0.600000,0.266667,0.333333}%
\pgfsetstrokecolor{currentstroke}%
\pgfsetdash{}{0pt}%
\pgfpathmoveto{\pgfqpoint{0.599306in}{1.038234in}}%
\pgfpathlineto{\pgfqpoint{0.601134in}{1.038234in}}%
\pgfpathlineto{\pgfqpoint{0.601134in}{1.084764in}}%
\pgfpathlineto{\pgfqpoint{0.599306in}{1.084764in}}%
\pgfpathlineto{\pgfqpoint{0.599306in}{1.038234in}}%
\pgfpathclose%
\pgfusepath{stroke,fill}%
\end{pgfscope}%
\begin{pgfscope}%
\pgfpathrectangle{\pgfqpoint{0.150000in}{0.150000in}}{\pgfqpoint{1.700000in}{1.700000in}}%
\pgfusepath{clip}%
\pgfsetbuttcap%
\pgfsetroundjoin%
\definecolor{currentfill}{rgb}{0.933333,0.600000,0.666667}%
\pgfsetfillcolor{currentfill}%
\pgfsetlinewidth{1.003750pt}%
\definecolor{currentstroke}{rgb}{0.600000,0.266667,0.333333}%
\pgfsetstrokecolor{currentstroke}%
\pgfsetdash{}{0pt}%
\pgfpathmoveto{\pgfqpoint{0.340412in}{1.169364in}}%
\pgfpathlineto{\pgfqpoint{0.361826in}{1.169364in}}%
\pgfpathlineto{\pgfqpoint{0.361826in}{1.215894in}}%
\pgfpathlineto{\pgfqpoint{0.340412in}{1.215894in}}%
\pgfpathlineto{\pgfqpoint{0.340412in}{1.169364in}}%
\pgfpathclose%
\pgfusepath{stroke,fill}%
\end{pgfscope}%
\begin{pgfscope}%
\pgfpathrectangle{\pgfqpoint{0.150000in}{0.150000in}}{\pgfqpoint{1.700000in}{1.700000in}}%
\pgfusepath{clip}%
\pgfsetbuttcap%
\pgfsetroundjoin%
\definecolor{currentfill}{rgb}{0.933333,0.600000,0.666667}%
\pgfsetfillcolor{currentfill}%
\pgfsetlinewidth{1.003750pt}%
\definecolor{currentstroke}{rgb}{0.600000,0.266667,0.333333}%
\pgfsetstrokecolor{currentstroke}%
\pgfsetdash{}{0pt}%
\pgfpathmoveto{\pgfqpoint{0.316935in}{1.084764in}}%
\pgfpathlineto{\pgfqpoint{0.326960in}{1.084764in}}%
\pgfpathlineto{\pgfqpoint{0.326960in}{1.122834in}}%
\pgfpathlineto{\pgfqpoint{0.316935in}{1.122834in}}%
\pgfpathlineto{\pgfqpoint{0.316935in}{1.084764in}}%
\pgfpathclose%
\pgfusepath{stroke,fill}%
\end{pgfscope}%
\begin{pgfscope}%
\pgfpathrectangle{\pgfqpoint{0.150000in}{0.150000in}}{\pgfqpoint{1.700000in}{1.700000in}}%
\pgfusepath{clip}%
\pgfsetbuttcap%
\pgfsetroundjoin%
\definecolor{currentfill}{rgb}{0.933333,0.600000,0.666667}%
\pgfsetfillcolor{currentfill}%
\pgfsetlinewidth{1.003750pt}%
\definecolor{currentstroke}{rgb}{0.600000,0.266667,0.333333}%
\pgfsetstrokecolor{currentstroke}%
\pgfsetdash{}{0pt}%
\pgfpathmoveto{\pgfqpoint{0.307032in}{1.000164in}}%
\pgfpathlineto{\pgfqpoint{0.311174in}{1.000164in}}%
\pgfpathlineto{\pgfqpoint{0.311174in}{1.038234in}}%
\pgfpathlineto{\pgfqpoint{0.307032in}{1.038234in}}%
\pgfpathlineto{\pgfqpoint{0.307032in}{1.000164in}}%
\pgfpathclose%
\pgfusepath{stroke,fill}%
\end{pgfscope}%
\begin{pgfscope}%
\pgfpathrectangle{\pgfqpoint{0.150000in}{0.150000in}}{\pgfqpoint{1.700000in}{1.700000in}}%
\pgfusepath{clip}%
\pgfsetbuttcap%
\pgfsetroundjoin%
\definecolor{currentfill}{rgb}{0.933333,0.600000,0.666667}%
\pgfsetfillcolor{currentfill}%
\pgfsetlinewidth{1.003750pt}%
\definecolor{currentstroke}{rgb}{0.600000,0.266667,0.333333}%
\pgfsetstrokecolor{currentstroke}%
\pgfsetdash{}{0pt}%
\pgfpathmoveto{\pgfqpoint{0.605301in}{0.832457in}}%
\pgfpathlineto{\pgfqpoint{0.618724in}{0.832457in}}%
\pgfpathlineto{\pgfqpoint{0.618724in}{0.876777in}}%
\pgfpathlineto{\pgfqpoint{0.605301in}{0.876777in}}%
\pgfpathlineto{\pgfqpoint{0.605301in}{0.832457in}}%
\pgfpathclose%
\pgfusepath{stroke,fill}%
\end{pgfscope}%
\begin{pgfscope}%
\pgfpathrectangle{\pgfqpoint{0.150000in}{0.150000in}}{\pgfqpoint{1.700000in}{1.700000in}}%
\pgfusepath{clip}%
\pgfsetbuttcap%
\pgfsetroundjoin%
\definecolor{currentfill}{rgb}{0.933333,0.600000,0.666667}%
\pgfsetfillcolor{currentfill}%
\pgfsetlinewidth{1.003750pt}%
\definecolor{currentstroke}{rgb}{0.600000,0.266667,0.333333}%
\pgfsetstrokecolor{currentstroke}%
\pgfsetdash{}{0pt}%
\pgfpathmoveto{\pgfqpoint{0.636015in}{0.751875in}}%
\pgfpathlineto{\pgfqpoint{0.659898in}{0.751875in}}%
\pgfpathlineto{\pgfqpoint{0.659898in}{0.788137in}}%
\pgfpathlineto{\pgfqpoint{0.636015in}{0.788137in}}%
\pgfpathlineto{\pgfqpoint{0.636015in}{0.751875in}}%
\pgfpathclose%
\pgfusepath{stroke,fill}%
\end{pgfscope}%
\begin{pgfscope}%
\pgfpathrectangle{\pgfqpoint{0.150000in}{0.150000in}}{\pgfqpoint{1.700000in}{1.700000in}}%
\pgfusepath{clip}%
\pgfsetbuttcap%
\pgfsetroundjoin%
\definecolor{currentfill}{rgb}{0.933333,0.600000,0.666667}%
\pgfsetfillcolor{currentfill}%
\pgfsetlinewidth{1.003750pt}%
\definecolor{currentstroke}{rgb}{0.600000,0.266667,0.333333}%
\pgfsetstrokecolor{currentstroke}%
\pgfsetdash{}{0pt}%
\pgfpathmoveto{\pgfqpoint{0.795725in}{0.605362in}}%
\pgfpathlineto{\pgfqpoint{0.856418in}{0.605362in}}%
\pgfpathlineto{\pgfqpoint{0.856418in}{0.625915in}}%
\pgfpathlineto{\pgfqpoint{0.795725in}{0.625915in}}%
\pgfpathlineto{\pgfqpoint{0.795725in}{0.605362in}}%
\pgfpathclose%
\pgfusepath{stroke,fill}%
\end{pgfscope}%
\begin{pgfscope}%
\pgfpathrectangle{\pgfqpoint{0.150000in}{0.150000in}}{\pgfqpoint{1.700000in}{1.700000in}}%
\pgfusepath{clip}%
\pgfsetbuttcap%
\pgfsetroundjoin%
\definecolor{currentfill}{rgb}{0.933333,0.600000,0.666667}%
\pgfsetfillcolor{currentfill}%
\pgfsetlinewidth{1.003750pt}%
\definecolor{currentstroke}{rgb}{0.600000,0.266667,0.333333}%
\pgfsetstrokecolor{currentstroke}%
\pgfsetdash{}{0pt}%
\pgfpathmoveto{\pgfqpoint{0.685374in}{0.655287in}}%
\pgfpathlineto{\pgfqpoint{0.735032in}{0.655287in}}%
\pgfpathlineto{\pgfqpoint{0.735032in}{0.699421in}}%
\pgfpathlineto{\pgfqpoint{0.685374in}{0.699421in}}%
\pgfpathlineto{\pgfqpoint{0.685374in}{0.655287in}}%
\pgfpathclose%
\pgfusepath{stroke,fill}%
\end{pgfscope}%
\begin{pgfscope}%
\pgfpathrectangle{\pgfqpoint{0.150000in}{0.150000in}}{\pgfqpoint{1.700000in}{1.700000in}}%
\pgfusepath{clip}%
\pgfsetbuttcap%
\pgfsetroundjoin%
\definecolor{currentfill}{rgb}{0.933333,0.600000,0.666667}%
\pgfsetfillcolor{currentfill}%
\pgfsetlinewidth{1.003750pt}%
\definecolor{currentstroke}{rgb}{0.600000,0.266667,0.333333}%
\pgfsetstrokecolor{currentstroke}%
\pgfsetdash{}{0pt}%
\pgfpathmoveto{\pgfqpoint{0.317497in}{0.874076in}}%
\pgfpathlineto{\pgfqpoint{0.327746in}{0.874076in}}%
\pgfpathlineto{\pgfqpoint{0.327746in}{0.930946in}}%
\pgfpathlineto{\pgfqpoint{0.317497in}{0.930946in}}%
\pgfpathlineto{\pgfqpoint{0.317497in}{0.874076in}}%
\pgfpathclose%
\pgfusepath{stroke,fill}%
\end{pgfscope}%
\begin{pgfscope}%
\pgfpathrectangle{\pgfqpoint{0.150000in}{0.150000in}}{\pgfqpoint{1.700000in}{1.700000in}}%
\pgfusepath{clip}%
\pgfsetbuttcap%
\pgfsetroundjoin%
\definecolor{currentfill}{rgb}{0.933333,0.600000,0.666667}%
\pgfsetfillcolor{currentfill}%
\pgfsetlinewidth{1.003750pt}%
\definecolor{currentstroke}{rgb}{0.600000,0.266667,0.333333}%
\pgfsetstrokecolor{currentstroke}%
\pgfsetdash{}{0pt}%
\pgfpathmoveto{\pgfqpoint{0.341432in}{0.781015in}}%
\pgfpathlineto{\pgfqpoint{0.355338in}{0.781015in}}%
\pgfpathlineto{\pgfqpoint{0.355338in}{0.827545in}}%
\pgfpathlineto{\pgfqpoint{0.341432in}{0.827545in}}%
\pgfpathlineto{\pgfqpoint{0.341432in}{0.781015in}}%
\pgfpathclose%
\pgfusepath{stroke,fill}%
\end{pgfscope}%
\begin{pgfscope}%
\pgfpathrectangle{\pgfqpoint{0.150000in}{0.150000in}}{\pgfqpoint{1.700000in}{1.700000in}}%
\pgfusepath{clip}%
\pgfsetbuttcap%
\pgfsetroundjoin%
\definecolor{currentfill}{rgb}{0.933333,0.600000,0.666667}%
\pgfsetfillcolor{currentfill}%
\pgfsetlinewidth{1.003750pt}%
\definecolor{currentstroke}{rgb}{0.600000,0.266667,0.333333}%
\pgfsetstrokecolor{currentstroke}%
\pgfsetdash{}{0pt}%
\pgfpathmoveto{\pgfqpoint{0.375898in}{0.696415in}}%
\pgfpathlineto{\pgfqpoint{0.395899in}{0.696415in}}%
\pgfpathlineto{\pgfqpoint{0.395899in}{0.742945in}}%
\pgfpathlineto{\pgfqpoint{0.375898in}{0.742945in}}%
\pgfpathlineto{\pgfqpoint{0.375898in}{0.696415in}}%
\pgfpathclose%
\pgfusepath{stroke,fill}%
\end{pgfscope}%
\begin{pgfscope}%
\pgfpathrectangle{\pgfqpoint{0.150000in}{0.150000in}}{\pgfqpoint{1.700000in}{1.700000in}}%
\pgfusepath{clip}%
\pgfsetbuttcap%
\pgfsetroundjoin%
\definecolor{currentfill}{rgb}{0.933333,0.600000,0.666667}%
\pgfsetfillcolor{currentfill}%
\pgfsetlinewidth{1.003750pt}%
\definecolor{currentstroke}{rgb}{0.600000,0.266667,0.333333}%
\pgfsetstrokecolor{currentstroke}%
\pgfsetdash{}{0pt}%
\pgfpathmoveto{\pgfqpoint{0.885766in}{0.315444in}}%
\pgfpathlineto{\pgfqpoint{0.930598in}{0.315444in}}%
\pgfpathlineto{\pgfqpoint{0.930598in}{0.322585in}}%
\pgfpathlineto{\pgfqpoint{0.885766in}{0.322585in}}%
\pgfpathlineto{\pgfqpoint{0.885766in}{0.315444in}}%
\pgfpathclose%
\pgfusepath{stroke,fill}%
\end{pgfscope}%
\begin{pgfscope}%
\pgfpathrectangle{\pgfqpoint{0.150000in}{0.150000in}}{\pgfqpoint{1.700000in}{1.700000in}}%
\pgfusepath{clip}%
\pgfsetbuttcap%
\pgfsetroundjoin%
\definecolor{currentfill}{rgb}{0.933333,0.600000,0.666667}%
\pgfsetfillcolor{currentfill}%
\pgfsetlinewidth{1.003750pt}%
\definecolor{currentstroke}{rgb}{0.600000,0.266667,0.333333}%
\pgfsetstrokecolor{currentstroke}%
\pgfsetdash{}{0pt}%
\pgfpathmoveto{\pgfqpoint{1.274139in}{1.423721in}}%
\pgfpathlineto{\pgfqpoint{1.398124in}{1.423721in}}%
\pgfpathlineto{\pgfqpoint{1.398124in}{1.568475in}}%
\pgfpathlineto{\pgfqpoint{1.274139in}{1.568475in}}%
\pgfpathlineto{\pgfqpoint{1.274139in}{1.423721in}}%
\pgfpathclose%
\pgfusepath{stroke,fill}%
\end{pgfscope}%
\begin{pgfscope}%
\pgfpathrectangle{\pgfqpoint{0.150000in}{0.150000in}}{\pgfqpoint{1.700000in}{1.700000in}}%
\pgfusepath{clip}%
\pgfsetbuttcap%
\pgfsetroundjoin%
\definecolor{currentfill}{rgb}{0.933333,0.600000,0.666667}%
\pgfsetfillcolor{currentfill}%
\pgfsetlinewidth{1.003750pt}%
\definecolor{currentstroke}{rgb}{0.600000,0.266667,0.333333}%
\pgfsetstrokecolor{currentstroke}%
\pgfsetdash{}{0pt}%
\pgfpathmoveto{\pgfqpoint{1.442338in}{1.248742in}}%
\pgfpathlineto{\pgfqpoint{1.549661in}{1.248742in}}%
\pgfpathlineto{\pgfqpoint{1.549661in}{1.423721in}}%
\pgfpathlineto{\pgfqpoint{1.442338in}{1.423721in}}%
\pgfpathlineto{\pgfqpoint{1.442338in}{1.248742in}}%
\pgfpathclose%
\pgfusepath{stroke,fill}%
\end{pgfscope}%
\begin{pgfscope}%
\pgfpathrectangle{\pgfqpoint{0.150000in}{0.150000in}}{\pgfqpoint{1.700000in}{1.700000in}}%
\pgfusepath{clip}%
\pgfsetbuttcap%
\pgfsetroundjoin%
\definecolor{currentfill}{rgb}{0.933333,0.600000,0.666667}%
\pgfsetfillcolor{currentfill}%
\pgfsetlinewidth{1.003750pt}%
\definecolor{currentstroke}{rgb}{0.600000,0.266667,0.333333}%
\pgfsetstrokecolor{currentstroke}%
\pgfsetdash{}{0pt}%
\pgfpathmoveto{\pgfqpoint{1.274139in}{1.292239in}}%
\pgfpathlineto{\pgfqpoint{1.314138in}{1.292239in}}%
\pgfpathlineto{\pgfqpoint{1.314138in}{1.423721in}}%
\pgfpathlineto{\pgfqpoint{1.274139in}{1.423721in}}%
\pgfpathlineto{\pgfqpoint{1.274139in}{1.292239in}}%
\pgfpathclose%
\pgfusepath{stroke,fill}%
\end{pgfscope}%
\begin{pgfscope}%
\pgfpathrectangle{\pgfqpoint{0.150000in}{0.150000in}}{\pgfqpoint{1.700000in}{1.700000in}}%
\pgfusepath{clip}%
\pgfsetbuttcap%
\pgfsetroundjoin%
\definecolor{currentfill}{rgb}{0.933333,0.600000,0.666667}%
\pgfsetfillcolor{currentfill}%
\pgfsetlinewidth{1.003750pt}%
\definecolor{currentstroke}{rgb}{0.600000,0.266667,0.333333}%
\pgfsetstrokecolor{currentstroke}%
\pgfsetdash{}{0pt}%
\pgfpathmoveto{\pgfqpoint{1.314138in}{1.248742in}}%
\pgfpathlineto{\pgfqpoint{1.442338in}{1.248742in}}%
\pgfpathlineto{\pgfqpoint{1.442338in}{1.423721in}}%
\pgfpathlineto{\pgfqpoint{1.314138in}{1.423721in}}%
\pgfpathlineto{\pgfqpoint{1.314138in}{1.248742in}}%
\pgfpathclose%
\pgfusepath{stroke,fill}%
\end{pgfscope}%
\begin{pgfscope}%
\pgfpathrectangle{\pgfqpoint{0.150000in}{0.150000in}}{\pgfqpoint{1.700000in}{1.700000in}}%
\pgfusepath{clip}%
\pgfsetbuttcap%
\pgfsetroundjoin%
\definecolor{currentfill}{rgb}{0.933333,0.600000,0.666667}%
\pgfsetfillcolor{currentfill}%
\pgfsetlinewidth{1.003750pt}%
\definecolor{currentstroke}{rgb}{0.600000,0.266667,0.333333}%
\pgfsetstrokecolor{currentstroke}%
\pgfsetdash{}{0pt}%
\pgfpathmoveto{\pgfqpoint{1.647915in}{0.930598in}}%
\pgfpathlineto{\pgfqpoint{1.690091in}{0.930598in}}%
\pgfpathlineto{\pgfqpoint{1.690091in}{1.073763in}}%
\pgfpathlineto{\pgfqpoint{1.647915in}{1.073763in}}%
\pgfpathlineto{\pgfqpoint{1.647915in}{0.930598in}}%
\pgfpathclose%
\pgfusepath{stroke,fill}%
\end{pgfscope}%
\begin{pgfscope}%
\pgfpathrectangle{\pgfqpoint{0.150000in}{0.150000in}}{\pgfqpoint{1.700000in}{1.700000in}}%
\pgfusepath{clip}%
\pgfsetbuttcap%
\pgfsetroundjoin%
\definecolor{currentfill}{rgb}{0.933333,0.600000,0.666667}%
\pgfsetfillcolor{currentfill}%
\pgfsetlinewidth{1.003750pt}%
\definecolor{currentstroke}{rgb}{0.600000,0.266667,0.333333}%
\pgfsetstrokecolor{currentstroke}%
\pgfsetdash{}{0pt}%
\pgfpathmoveto{\pgfqpoint{1.393846in}{1.073763in}}%
\pgfpathlineto{\pgfqpoint{1.400694in}{1.073763in}}%
\pgfpathlineto{\pgfqpoint{1.400694in}{1.248742in}}%
\pgfpathlineto{\pgfqpoint{1.393846in}{1.248742in}}%
\pgfpathlineto{\pgfqpoint{1.393846in}{1.073763in}}%
\pgfpathclose%
\pgfusepath{stroke,fill}%
\end{pgfscope}%
\begin{pgfscope}%
\pgfpathrectangle{\pgfqpoint{0.150000in}{0.150000in}}{\pgfqpoint{1.700000in}{1.700000in}}%
\pgfusepath{clip}%
\pgfsetbuttcap%
\pgfsetroundjoin%
\definecolor{currentfill}{rgb}{0.933333,0.600000,0.666667}%
\pgfsetfillcolor{currentfill}%
\pgfsetlinewidth{1.003750pt}%
\definecolor{currentstroke}{rgb}{0.600000,0.266667,0.333333}%
\pgfsetstrokecolor{currentstroke}%
\pgfsetdash{}{0pt}%
\pgfpathmoveto{\pgfqpoint{1.085191in}{1.637585in}}%
\pgfpathlineto{\pgfqpoint{1.170218in}{1.637585in}}%
\pgfpathlineto{\pgfqpoint{1.170218in}{1.672824in}}%
\pgfpathlineto{\pgfqpoint{1.085191in}{1.672824in}}%
\pgfpathlineto{\pgfqpoint{1.085191in}{1.637585in}}%
\pgfpathclose%
\pgfusepath{stroke,fill}%
\end{pgfscope}%
\begin{pgfscope}%
\pgfpathrectangle{\pgfqpoint{0.150000in}{0.150000in}}{\pgfqpoint{1.700000in}{1.700000in}}%
\pgfusepath{clip}%
\pgfsetbuttcap%
\pgfsetroundjoin%
\definecolor{currentfill}{rgb}{0.933333,0.600000,0.666667}%
\pgfsetfillcolor{currentfill}%
\pgfsetlinewidth{1.003750pt}%
\definecolor{currentstroke}{rgb}{0.600000,0.266667,0.333333}%
\pgfsetstrokecolor{currentstroke}%
\pgfsetdash{}{0pt}%
\pgfpathmoveto{\pgfqpoint{0.930598in}{1.688774in}}%
\pgfpathlineto{\pgfqpoint{1.000165in}{1.688774in}}%
\pgfpathlineto{\pgfqpoint{1.000165in}{1.690543in}}%
\pgfpathlineto{\pgfqpoint{0.930598in}{1.690543in}}%
\pgfpathlineto{\pgfqpoint{0.930598in}{1.688774in}}%
\pgfpathclose%
\pgfusepath{stroke,fill}%
\end{pgfscope}%
\begin{pgfscope}%
\pgfpathrectangle{\pgfqpoint{0.150000in}{0.150000in}}{\pgfqpoint{1.700000in}{1.700000in}}%
\pgfusepath{clip}%
\pgfsetbuttcap%
\pgfsetroundjoin%
\definecolor{currentfill}{rgb}{0.933333,0.600000,0.666667}%
\pgfsetfillcolor{currentfill}%
\pgfsetlinewidth{1.003750pt}%
\definecolor{currentstroke}{rgb}{0.600000,0.266667,0.333333}%
\pgfsetstrokecolor{currentstroke}%
\pgfsetdash{}{0pt}%
\pgfpathmoveto{\pgfqpoint{1.170218in}{1.362742in}}%
\pgfpathlineto{\pgfqpoint{1.274139in}{1.362742in}}%
\pgfpathlineto{\pgfqpoint{1.274139in}{1.391533in}}%
\pgfpathlineto{\pgfqpoint{1.170218in}{1.391533in}}%
\pgfpathlineto{\pgfqpoint{1.170218in}{1.362742in}}%
\pgfpathclose%
\pgfusepath{stroke,fill}%
\end{pgfscope}%
\begin{pgfscope}%
\pgfpathrectangle{\pgfqpoint{0.150000in}{0.150000in}}{\pgfqpoint{1.700000in}{1.700000in}}%
\pgfusepath{clip}%
\pgfsetbuttcap%
\pgfsetroundjoin%
\definecolor{currentfill}{rgb}{0.933333,0.600000,0.666667}%
\pgfsetfillcolor{currentfill}%
\pgfsetlinewidth{1.003750pt}%
\definecolor{currentstroke}{rgb}{0.600000,0.266667,0.333333}%
\pgfsetstrokecolor{currentstroke}%
\pgfsetdash{}{0pt}%
\pgfpathmoveto{\pgfqpoint{1.000165in}{1.400694in}}%
\pgfpathlineto{\pgfqpoint{1.085191in}{1.400694in}}%
\pgfpathlineto{\pgfqpoint{1.085191in}{1.400694in}}%
\pgfpathlineto{\pgfqpoint{1.000165in}{1.400694in}}%
\pgfpathlineto{\pgfqpoint{1.000165in}{1.400694in}}%
\pgfpathclose%
\pgfusepath{stroke,fill}%
\end{pgfscope}%
\begin{pgfscope}%
\pgfpathrectangle{\pgfqpoint{0.150000in}{0.150000in}}{\pgfqpoint{1.700000in}{1.700000in}}%
\pgfusepath{clip}%
\pgfsetbuttcap%
\pgfsetroundjoin%
\definecolor{currentfill}{rgb}{0.933333,0.600000,0.666667}%
\pgfsetfillcolor{currentfill}%
\pgfsetlinewidth{1.003750pt}%
\definecolor{currentstroke}{rgb}{0.600000,0.266667,0.333333}%
\pgfsetstrokecolor{currentstroke}%
\pgfsetdash{}{0pt}%
\pgfpathmoveto{\pgfqpoint{1.579254in}{0.758520in}}%
\pgfpathlineto{\pgfqpoint{1.650656in}{0.758520in}}%
\pgfpathlineto{\pgfqpoint{1.650656in}{0.930598in}}%
\pgfpathlineto{\pgfqpoint{1.579254in}{0.930598in}}%
\pgfpathlineto{\pgfqpoint{1.579254in}{0.758520in}}%
\pgfpathclose%
\pgfusepath{stroke,fill}%
\end{pgfscope}%
\begin{pgfscope}%
\pgfpathrectangle{\pgfqpoint{0.150000in}{0.150000in}}{\pgfqpoint{1.700000in}{1.700000in}}%
\pgfusepath{clip}%
\pgfsetbuttcap%
\pgfsetroundjoin%
\definecolor{currentfill}{rgb}{0.933333,0.600000,0.666667}%
\pgfsetfillcolor{currentfill}%
\pgfsetlinewidth{1.003750pt}%
\definecolor{currentstroke}{rgb}{0.600000,0.266667,0.333333}%
\pgfsetstrokecolor{currentstroke}%
\pgfsetdash{}{0pt}%
\pgfpathmoveto{\pgfqpoint{1.351279in}{0.706301in}}%
\pgfpathlineto{\pgfqpoint{1.394638in}{0.706301in}}%
\pgfpathlineto{\pgfqpoint{1.394638in}{0.807234in}}%
\pgfpathlineto{\pgfqpoint{1.351279in}{0.807234in}}%
\pgfpathlineto{\pgfqpoint{1.351279in}{0.706301in}}%
\pgfpathclose%
\pgfusepath{stroke,fill}%
\end{pgfscope}%
\begin{pgfscope}%
\pgfpathrectangle{\pgfqpoint{0.150000in}{0.150000in}}{\pgfqpoint{1.700000in}{1.700000in}}%
\pgfusepath{clip}%
\pgfsetbuttcap%
\pgfsetroundjoin%
\definecolor{currentfill}{rgb}{0.933333,0.600000,0.666667}%
\pgfsetfillcolor{currentfill}%
\pgfsetlinewidth{1.003750pt}%
\definecolor{currentstroke}{rgb}{0.600000,0.266667,0.333333}%
\pgfsetstrokecolor{currentstroke}%
\pgfsetdash{}{0pt}%
\pgfpathmoveto{\pgfqpoint{1.410580in}{0.520228in}}%
\pgfpathlineto{\pgfqpoint{1.501483in}{0.520228in}}%
\pgfpathlineto{\pgfqpoint{1.501483in}{0.617728in}}%
\pgfpathlineto{\pgfqpoint{1.410580in}{0.617728in}}%
\pgfpathlineto{\pgfqpoint{1.410580in}{0.520228in}}%
\pgfpathclose%
\pgfusepath{stroke,fill}%
\end{pgfscope}%
\begin{pgfscope}%
\pgfpathrectangle{\pgfqpoint{0.150000in}{0.150000in}}{\pgfqpoint{1.700000in}{1.700000in}}%
\pgfusepath{clip}%
\pgfsetbuttcap%
\pgfsetroundjoin%
\definecolor{currentfill}{rgb}{0.933333,0.600000,0.666667}%
\pgfsetfillcolor{currentfill}%
\pgfsetlinewidth{1.003750pt}%
\definecolor{currentstroke}{rgb}{0.600000,0.266667,0.333333}%
\pgfsetstrokecolor{currentstroke}%
\pgfsetdash{}{0pt}%
\pgfpathmoveto{\pgfqpoint{1.272573in}{0.392004in}}%
\pgfpathlineto{\pgfqpoint{1.334676in}{0.392004in}}%
\pgfpathlineto{\pgfqpoint{1.334676in}{0.440455in}}%
\pgfpathlineto{\pgfqpoint{1.272573in}{0.440455in}}%
\pgfpathlineto{\pgfqpoint{1.272573in}{0.392004in}}%
\pgfpathclose%
\pgfusepath{stroke,fill}%
\end{pgfscope}%
\begin{pgfscope}%
\pgfpathrectangle{\pgfqpoint{0.150000in}{0.150000in}}{\pgfqpoint{1.700000in}{1.700000in}}%
\pgfusepath{clip}%
\pgfsetbuttcap%
\pgfsetroundjoin%
\definecolor{currentfill}{rgb}{0.933333,0.600000,0.666667}%
\pgfsetfillcolor{currentfill}%
\pgfsetlinewidth{1.003750pt}%
\definecolor{currentstroke}{rgb}{0.600000,0.266667,0.333333}%
\pgfsetstrokecolor{currentstroke}%
\pgfsetdash{}{0pt}%
\pgfpathmoveto{\pgfqpoint{1.169126in}{0.608314in}}%
\pgfpathlineto{\pgfqpoint{1.272573in}{0.608314in}}%
\pgfpathlineto{\pgfqpoint{1.272573in}{0.636748in}}%
\pgfpathlineto{\pgfqpoint{1.169126in}{0.636748in}}%
\pgfpathlineto{\pgfqpoint{1.169126in}{0.608314in}}%
\pgfpathclose%
\pgfusepath{stroke,fill}%
\end{pgfscope}%
\begin{pgfscope}%
\pgfpathrectangle{\pgfqpoint{0.150000in}{0.150000in}}{\pgfqpoint{1.700000in}{1.700000in}}%
\pgfusepath{clip}%
\pgfsetbuttcap%
\pgfsetroundjoin%
\definecolor{currentfill}{rgb}{0.933333,0.600000,0.666667}%
\pgfsetfillcolor{currentfill}%
\pgfsetlinewidth{1.003750pt}%
\definecolor{currentstroke}{rgb}{0.600000,0.266667,0.333333}%
\pgfsetstrokecolor{currentstroke}%
\pgfsetdash{}{0pt}%
\pgfpathmoveto{\pgfqpoint{0.930598in}{0.599306in}}%
\pgfpathlineto{\pgfqpoint{0.999848in}{0.599306in}}%
\pgfpathlineto{\pgfqpoint{0.999848in}{0.599306in}}%
\pgfpathlineto{\pgfqpoint{0.930598in}{0.599306in}}%
\pgfpathlineto{\pgfqpoint{0.930598in}{0.599306in}}%
\pgfpathclose%
\pgfusepath{stroke,fill}%
\end{pgfscope}%
\begin{pgfscope}%
\pgfpathrectangle{\pgfqpoint{0.150000in}{0.150000in}}{\pgfqpoint{1.700000in}{1.700000in}}%
\pgfusepath{clip}%
\pgfsetbuttcap%
\pgfsetroundjoin%
\definecolor{currentfill}{rgb}{0.933333,0.600000,0.666667}%
\pgfsetfillcolor{currentfill}%
\pgfsetlinewidth{1.003750pt}%
\definecolor{currentstroke}{rgb}{0.600000,0.266667,0.333333}%
\pgfsetstrokecolor{currentstroke}%
\pgfsetdash{}{0pt}%
\pgfpathmoveto{\pgfqpoint{1.084487in}{0.326900in}}%
\pgfpathlineto{\pgfqpoint{1.169126in}{0.326900in}}%
\pgfpathlineto{\pgfqpoint{1.169126in}{0.361744in}}%
\pgfpathlineto{\pgfqpoint{1.084487in}{0.361744in}}%
\pgfpathlineto{\pgfqpoint{1.084487in}{0.326900in}}%
\pgfpathclose%
\pgfusepath{stroke,fill}%
\end{pgfscope}%
\begin{pgfscope}%
\pgfpathrectangle{\pgfqpoint{0.150000in}{0.150000in}}{\pgfqpoint{1.700000in}{1.700000in}}%
\pgfusepath{clip}%
\pgfsetbuttcap%
\pgfsetroundjoin%
\definecolor{currentfill}{rgb}{0.933333,0.600000,0.666667}%
\pgfsetfillcolor{currentfill}%
\pgfsetlinewidth{1.003750pt}%
\definecolor{currentstroke}{rgb}{0.600000,0.266667,0.333333}%
\pgfsetstrokecolor{currentstroke}%
\pgfsetdash{}{0pt}%
\pgfpathmoveto{\pgfqpoint{0.930598in}{0.309457in}}%
\pgfpathlineto{\pgfqpoint{0.999848in}{0.309457in}}%
\pgfpathlineto{\pgfqpoint{0.999848in}{0.311140in}}%
\pgfpathlineto{\pgfqpoint{0.930598in}{0.311140in}}%
\pgfpathlineto{\pgfqpoint{0.930598in}{0.309457in}}%
\pgfpathclose%
\pgfusepath{stroke,fill}%
\end{pgfscope}%
\begin{pgfscope}%
\pgfpathrectangle{\pgfqpoint{0.150000in}{0.150000in}}{\pgfqpoint{1.700000in}{1.700000in}}%
\pgfusepath{clip}%
\pgfsetbuttcap%
\pgfsetroundjoin%
\definecolor{currentfill}{rgb}{0.933333,0.600000,0.666667}%
\pgfsetfillcolor{currentfill}%
\pgfsetlinewidth{1.003750pt}%
\definecolor{currentstroke}{rgb}{0.600000,0.266667,0.333333}%
\pgfsetstrokecolor{currentstroke}%
\pgfsetdash{}{0pt}%
\pgfpathmoveto{\pgfqpoint{0.758544in}{1.579284in}}%
\pgfpathlineto{\pgfqpoint{0.930598in}{1.579284in}}%
\pgfpathlineto{\pgfqpoint{0.930598in}{1.650666in}}%
\pgfpathlineto{\pgfqpoint{0.758544in}{1.650666in}}%
\pgfpathlineto{\pgfqpoint{0.758544in}{1.579284in}}%
\pgfpathclose%
\pgfusepath{stroke,fill}%
\end{pgfscope}%
\begin{pgfscope}%
\pgfpathrectangle{\pgfqpoint{0.150000in}{0.150000in}}{\pgfqpoint{1.700000in}{1.700000in}}%
\pgfusepath{clip}%
\pgfsetbuttcap%
\pgfsetroundjoin%
\definecolor{currentfill}{rgb}{0.933333,0.600000,0.666667}%
\pgfsetfillcolor{currentfill}%
\pgfsetlinewidth{1.003750pt}%
\definecolor{currentstroke}{rgb}{0.600000,0.266667,0.333333}%
\pgfsetstrokecolor{currentstroke}%
\pgfsetdash{}{0pt}%
\pgfpathmoveto{\pgfqpoint{0.706478in}{1.351333in}}%
\pgfpathlineto{\pgfqpoint{0.807332in}{1.351333in}}%
\pgfpathlineto{\pgfqpoint{0.807332in}{1.394638in}}%
\pgfpathlineto{\pgfqpoint{0.706478in}{1.394638in}}%
\pgfpathlineto{\pgfqpoint{0.706478in}{1.351333in}}%
\pgfpathclose%
\pgfusepath{stroke,fill}%
\end{pgfscope}%
\begin{pgfscope}%
\pgfpathrectangle{\pgfqpoint{0.150000in}{0.150000in}}{\pgfqpoint{1.700000in}{1.700000in}}%
\pgfusepath{clip}%
\pgfsetbuttcap%
\pgfsetroundjoin%
\definecolor{currentfill}{rgb}{0.933333,0.600000,0.666667}%
\pgfsetfillcolor{currentfill}%
\pgfsetlinewidth{1.003750pt}%
\definecolor{currentstroke}{rgb}{0.600000,0.266667,0.333333}%
\pgfsetstrokecolor{currentstroke}%
\pgfsetdash{}{0pt}%
\pgfpathmoveto{\pgfqpoint{0.520296in}{1.410698in}}%
\pgfpathlineto{\pgfqpoint{0.617773in}{1.410698in}}%
\pgfpathlineto{\pgfqpoint{0.617773in}{1.501548in}}%
\pgfpathlineto{\pgfqpoint{0.520296in}{1.501548in}}%
\pgfpathlineto{\pgfqpoint{0.520296in}{1.410698in}}%
\pgfpathclose%
\pgfusepath{stroke,fill}%
\end{pgfscope}%
\begin{pgfscope}%
\pgfpathrectangle{\pgfqpoint{0.150000in}{0.150000in}}{\pgfqpoint{1.700000in}{1.700000in}}%
\pgfusepath{clip}%
\pgfsetbuttcap%
\pgfsetroundjoin%
\definecolor{currentfill}{rgb}{0.933333,0.600000,0.666667}%
\pgfsetfillcolor{currentfill}%
\pgfsetlinewidth{1.003750pt}%
\definecolor{currentstroke}{rgb}{0.600000,0.266667,0.333333}%
\pgfsetstrokecolor{currentstroke}%
\pgfsetdash{}{0pt}%
\pgfpathmoveto{\pgfqpoint{0.392092in}{1.272765in}}%
\pgfpathlineto{\pgfqpoint{0.440542in}{1.272765in}}%
\pgfpathlineto{\pgfqpoint{0.440542in}{1.334835in}}%
\pgfpathlineto{\pgfqpoint{0.392092in}{1.334835in}}%
\pgfpathlineto{\pgfqpoint{0.392092in}{1.272765in}}%
\pgfpathclose%
\pgfusepath{stroke,fill}%
\end{pgfscope}%
\begin{pgfscope}%
\pgfpathrectangle{\pgfqpoint{0.150000in}{0.150000in}}{\pgfqpoint{1.700000in}{1.700000in}}%
\pgfusepath{clip}%
\pgfsetbuttcap%
\pgfsetroundjoin%
\definecolor{currentfill}{rgb}{0.933333,0.600000,0.666667}%
\pgfsetfillcolor{currentfill}%
\pgfsetlinewidth{1.003750pt}%
\definecolor{currentstroke}{rgb}{0.600000,0.266667,0.333333}%
\pgfsetstrokecolor{currentstroke}%
\pgfsetdash{}{0pt}%
\pgfpathmoveto{\pgfqpoint{0.608374in}{1.169364in}}%
\pgfpathlineto{\pgfqpoint{0.636859in}{1.169364in}}%
\pgfpathlineto{\pgfqpoint{0.636859in}{1.272765in}}%
\pgfpathlineto{\pgfqpoint{0.608374in}{1.272765in}}%
\pgfpathlineto{\pgfqpoint{0.608374in}{1.169364in}}%
\pgfpathclose%
\pgfusepath{stroke,fill}%
\end{pgfscope}%
\begin{pgfscope}%
\pgfpathrectangle{\pgfqpoint{0.150000in}{0.150000in}}{\pgfqpoint{1.700000in}{1.700000in}}%
\pgfusepath{clip}%
\pgfsetbuttcap%
\pgfsetroundjoin%
\definecolor{currentfill}{rgb}{0.933333,0.600000,0.666667}%
\pgfsetfillcolor{currentfill}%
\pgfsetlinewidth{1.003750pt}%
\definecolor{currentstroke}{rgb}{0.600000,0.266667,0.333333}%
\pgfsetstrokecolor{currentstroke}%
\pgfsetdash{}{0pt}%
\pgfpathmoveto{\pgfqpoint{0.599306in}{1.000164in}}%
\pgfpathlineto{\pgfqpoint{0.599306in}{1.000164in}}%
\pgfpathlineto{\pgfqpoint{0.599306in}{1.084764in}}%
\pgfpathlineto{\pgfqpoint{0.599306in}{1.084764in}}%
\pgfpathlineto{\pgfqpoint{0.599306in}{1.000164in}}%
\pgfpathclose%
\pgfusepath{stroke,fill}%
\end{pgfscope}%
\begin{pgfscope}%
\pgfpathrectangle{\pgfqpoint{0.150000in}{0.150000in}}{\pgfqpoint{1.700000in}{1.700000in}}%
\pgfusepath{clip}%
\pgfsetbuttcap%
\pgfsetroundjoin%
\definecolor{currentfill}{rgb}{0.933333,0.600000,0.666667}%
\pgfsetfillcolor{currentfill}%
\pgfsetlinewidth{1.003750pt}%
\definecolor{currentstroke}{rgb}{0.600000,0.266667,0.333333}%
\pgfsetstrokecolor{currentstroke}%
\pgfsetdash{}{0pt}%
\pgfpathmoveto{\pgfqpoint{0.326960in}{1.084764in}}%
\pgfpathlineto{\pgfqpoint{0.361826in}{1.084764in}}%
\pgfpathlineto{\pgfqpoint{0.361826in}{1.169364in}}%
\pgfpathlineto{\pgfqpoint{0.326960in}{1.169364in}}%
\pgfpathlineto{\pgfqpoint{0.326960in}{1.084764in}}%
\pgfpathclose%
\pgfusepath{stroke,fill}%
\end{pgfscope}%
\begin{pgfscope}%
\pgfpathrectangle{\pgfqpoint{0.150000in}{0.150000in}}{\pgfqpoint{1.700000in}{1.700000in}}%
\pgfusepath{clip}%
\pgfsetbuttcap%
\pgfsetroundjoin%
\definecolor{currentfill}{rgb}{0.933333,0.600000,0.666667}%
\pgfsetfillcolor{currentfill}%
\pgfsetlinewidth{1.003750pt}%
\definecolor{currentstroke}{rgb}{0.600000,0.266667,0.333333}%
\pgfsetstrokecolor{currentstroke}%
\pgfsetdash{}{0pt}%
\pgfpathmoveto{\pgfqpoint{0.309422in}{0.930946in}}%
\pgfpathlineto{\pgfqpoint{0.311174in}{0.930946in}}%
\pgfpathlineto{\pgfqpoint{0.311174in}{1.000164in}}%
\pgfpathlineto{\pgfqpoint{0.309422in}{1.000164in}}%
\pgfpathlineto{\pgfqpoint{0.309422in}{0.930946in}}%
\pgfpathclose%
\pgfusepath{stroke,fill}%
\end{pgfscope}%
\begin{pgfscope}%
\pgfpathrectangle{\pgfqpoint{0.150000in}{0.150000in}}{\pgfqpoint{1.700000in}{1.700000in}}%
\pgfusepath{clip}%
\pgfsetbuttcap%
\pgfsetroundjoin%
\definecolor{currentfill}{rgb}{0.933333,0.600000,0.666667}%
\pgfsetfillcolor{currentfill}%
\pgfsetlinewidth{1.003750pt}%
\definecolor{currentstroke}{rgb}{0.600000,0.266667,0.333333}%
\pgfsetstrokecolor{currentstroke}%
\pgfsetdash{}{0pt}%
\pgfpathmoveto{\pgfqpoint{0.605301in}{0.751875in}}%
\pgfpathlineto{\pgfqpoint{0.636015in}{0.751875in}}%
\pgfpathlineto{\pgfqpoint{0.636015in}{0.832457in}}%
\pgfpathlineto{\pgfqpoint{0.605301in}{0.832457in}}%
\pgfpathlineto{\pgfqpoint{0.605301in}{0.751875in}}%
\pgfpathclose%
\pgfusepath{stroke,fill}%
\end{pgfscope}%
\begin{pgfscope}%
\pgfpathrectangle{\pgfqpoint{0.150000in}{0.150000in}}{\pgfqpoint{1.700000in}{1.700000in}}%
\pgfusepath{clip}%
\pgfsetbuttcap%
\pgfsetroundjoin%
\definecolor{currentfill}{rgb}{0.933333,0.600000,0.666667}%
\pgfsetfillcolor{currentfill}%
\pgfsetlinewidth{1.003750pt}%
\definecolor{currentstroke}{rgb}{0.600000,0.266667,0.333333}%
\pgfsetstrokecolor{currentstroke}%
\pgfsetdash{}{0pt}%
\pgfpathmoveto{\pgfqpoint{0.685374in}{0.605362in}}%
\pgfpathlineto{\pgfqpoint{0.795725in}{0.605362in}}%
\pgfpathlineto{\pgfqpoint{0.795725in}{0.655287in}}%
\pgfpathlineto{\pgfqpoint{0.685374in}{0.655287in}}%
\pgfpathlineto{\pgfqpoint{0.685374in}{0.605362in}}%
\pgfpathclose%
\pgfusepath{stroke,fill}%
\end{pgfscope}%
\begin{pgfscope}%
\pgfpathrectangle{\pgfqpoint{0.150000in}{0.150000in}}{\pgfqpoint{1.700000in}{1.700000in}}%
\pgfusepath{clip}%
\pgfsetbuttcap%
\pgfsetroundjoin%
\definecolor{currentfill}{rgb}{0.933333,0.600000,0.666667}%
\pgfsetfillcolor{currentfill}%
\pgfsetlinewidth{1.003750pt}%
\definecolor{currentstroke}{rgb}{0.600000,0.266667,0.333333}%
\pgfsetstrokecolor{currentstroke}%
\pgfsetdash{}{0pt}%
\pgfpathmoveto{\pgfqpoint{0.327746in}{0.827545in}}%
\pgfpathlineto{\pgfqpoint{0.355338in}{0.827545in}}%
\pgfpathlineto{\pgfqpoint{0.355338in}{0.930946in}}%
\pgfpathlineto{\pgfqpoint{0.327746in}{0.930946in}}%
\pgfpathlineto{\pgfqpoint{0.327746in}{0.827545in}}%
\pgfpathclose%
\pgfusepath{stroke,fill}%
\end{pgfscope}%
\begin{pgfscope}%
\pgfpathrectangle{\pgfqpoint{0.150000in}{0.150000in}}{\pgfqpoint{1.700000in}{1.700000in}}%
\pgfusepath{clip}%
\pgfsetbuttcap%
\pgfsetroundjoin%
\definecolor{currentfill}{rgb}{0.933333,0.600000,0.666667}%
\pgfsetfillcolor{currentfill}%
\pgfsetlinewidth{1.003750pt}%
\definecolor{currentstroke}{rgb}{0.600000,0.266667,0.333333}%
\pgfsetstrokecolor{currentstroke}%
\pgfsetdash{}{0pt}%
\pgfpathmoveto{\pgfqpoint{0.395899in}{0.658345in}}%
\pgfpathlineto{\pgfqpoint{0.440670in}{0.658345in}}%
\pgfpathlineto{\pgfqpoint{0.440670in}{0.742945in}}%
\pgfpathlineto{\pgfqpoint{0.395899in}{0.742945in}}%
\pgfpathlineto{\pgfqpoint{0.395899in}{0.658345in}}%
\pgfpathclose%
\pgfusepath{stroke,fill}%
\end{pgfscope}%
\begin{pgfscope}%
\pgfpathrectangle{\pgfqpoint{0.150000in}{0.150000in}}{\pgfqpoint{1.700000in}{1.700000in}}%
\pgfusepath{clip}%
\pgfsetbuttcap%
\pgfsetroundjoin%
\definecolor{currentfill}{rgb}{0.933333,0.600000,0.666667}%
\pgfsetfillcolor{currentfill}%
\pgfsetlinewidth{1.003750pt}%
\definecolor{currentstroke}{rgb}{0.600000,0.266667,0.333333}%
\pgfsetstrokecolor{currentstroke}%
\pgfsetdash{}{0pt}%
\pgfpathmoveto{\pgfqpoint{0.849086in}{0.322585in}}%
\pgfpathlineto{\pgfqpoint{0.930598in}{0.322585in}}%
\pgfpathlineto{\pgfqpoint{0.930598in}{0.340975in}}%
\pgfpathlineto{\pgfqpoint{0.849086in}{0.340975in}}%
\pgfpathlineto{\pgfqpoint{0.849086in}{0.322585in}}%
\pgfpathclose%
\pgfusepath{stroke,fill}%
\end{pgfscope}%
\begin{pgfscope}%
\pgfpathrectangle{\pgfqpoint{0.150000in}{0.150000in}}{\pgfqpoint{1.700000in}{1.700000in}}%
\pgfusepath{clip}%
\pgfsetbuttcap%
\pgfsetroundjoin%
\definecolor{currentfill}{rgb}{0.933333,0.600000,0.666667}%
\pgfsetfillcolor{currentfill}%
\pgfsetlinewidth{1.003750pt}%
\definecolor{currentstroke}{rgb}{0.600000,0.266667,0.333333}%
\pgfsetstrokecolor{currentstroke}%
\pgfsetdash{}{0pt}%
\pgfpathmoveto{\pgfqpoint{0.715703in}{0.366879in}}%
\pgfpathlineto{\pgfqpoint{0.782395in}{0.366879in}}%
\pgfpathlineto{\pgfqpoint{0.782395in}{0.394328in}}%
\pgfpathlineto{\pgfqpoint{0.715703in}{0.394328in}}%
\pgfpathlineto{\pgfqpoint{0.715703in}{0.366879in}}%
\pgfpathclose%
\pgfusepath{stroke,fill}%
\end{pgfscope}%
\begin{pgfscope}%
\pgfpathrectangle{\pgfqpoint{0.150000in}{0.150000in}}{\pgfqpoint{1.700000in}{1.700000in}}%
\pgfusepath{clip}%
\pgfsetbuttcap%
\pgfsetroundjoin%
\definecolor{currentfill}{rgb}{0.933333,0.600000,0.666667}%
\pgfsetfillcolor{currentfill}%
\pgfsetlinewidth{1.003750pt}%
\definecolor{currentstroke}{rgb}{0.600000,0.266667,0.333333}%
\pgfsetstrokecolor{currentstroke}%
\pgfsetdash{}{0pt}%
\pgfpathmoveto{\pgfqpoint{0.594446in}{0.436801in}}%
\pgfpathlineto{\pgfqpoint{0.661138in}{0.436801in}}%
\pgfpathlineto{\pgfqpoint{0.661138in}{0.480426in}}%
\pgfpathlineto{\pgfqpoint{0.594446in}{0.480426in}}%
\pgfpathlineto{\pgfqpoint{0.594446in}{0.436801in}}%
\pgfpathclose%
\pgfusepath{stroke,fill}%
\end{pgfscope}%
\begin{pgfscope}%
\pgfpathrectangle{\pgfqpoint{0.150000in}{0.150000in}}{\pgfqpoint{1.700000in}{1.700000in}}%
\pgfusepath{clip}%
\pgfsetbuttcap%
\pgfsetroundjoin%
\definecolor{currentfill}{rgb}{0.933333,0.600000,0.666667}%
\pgfsetfillcolor{currentfill}%
\pgfsetlinewidth{1.003750pt}%
\definecolor{currentstroke}{rgb}{0.600000,0.266667,0.333333}%
\pgfsetstrokecolor{currentstroke}%
\pgfsetdash{}{0pt}%
\pgfpathmoveto{\pgfqpoint{0.489954in}{0.529342in}}%
\pgfpathlineto{\pgfqpoint{0.539881in}{0.529342in}}%
\pgfpathlineto{\pgfqpoint{0.539881in}{0.589127in}}%
\pgfpathlineto{\pgfqpoint{0.489954in}{0.589127in}}%
\pgfpathlineto{\pgfqpoint{0.489954in}{0.529342in}}%
\pgfpathclose%
\pgfusepath{stroke,fill}%
\end{pgfscope}%
\begin{pgfscope}%
\pgfpathrectangle{\pgfqpoint{0.150000in}{0.150000in}}{\pgfqpoint{1.700000in}{1.700000in}}%
\pgfusepath{clip}%
\pgfsetbuttcap%
\pgfsetroundjoin%
\definecolor{currentfill}{rgb}{0.933333,0.600000,0.666667}%
\pgfsetfillcolor{currentfill}%
\pgfsetlinewidth{1.003750pt}%
\definecolor{currentstroke}{rgb}{0.600000,0.266667,0.333333}%
\pgfsetstrokecolor{currentstroke}%
\pgfsetdash{}{0pt}%
\pgfpathmoveto{\pgfqpoint{1.485086in}{0.930598in}}%
\pgfpathlineto{\pgfqpoint{1.647915in}{0.930598in}}%
\pgfpathlineto{\pgfqpoint{1.647915in}{1.248742in}}%
\pgfpathlineto{\pgfqpoint{1.485086in}{1.248742in}}%
\pgfpathlineto{\pgfqpoint{1.485086in}{0.930598in}}%
\pgfpathclose%
\pgfusepath{stroke,fill}%
\end{pgfscope}%
\begin{pgfscope}%
\pgfpathrectangle{\pgfqpoint{0.150000in}{0.150000in}}{\pgfqpoint{1.700000in}{1.700000in}}%
\pgfusepath{clip}%
\pgfsetbuttcap%
\pgfsetroundjoin%
\definecolor{currentfill}{rgb}{0.933333,0.600000,0.666667}%
\pgfsetfillcolor{currentfill}%
\pgfsetlinewidth{1.003750pt}%
\definecolor{currentstroke}{rgb}{0.600000,0.266667,0.333333}%
\pgfsetstrokecolor{currentstroke}%
\pgfsetdash{}{0pt}%
\pgfpathmoveto{\pgfqpoint{1.400694in}{0.930598in}}%
\pgfpathlineto{\pgfqpoint{1.485086in}{0.930598in}}%
\pgfpathlineto{\pgfqpoint{1.485086in}{1.248742in}}%
\pgfpathlineto{\pgfqpoint{1.400694in}{1.248742in}}%
\pgfpathlineto{\pgfqpoint{1.400694in}{0.930598in}}%
\pgfpathclose%
\pgfusepath{stroke,fill}%
\end{pgfscope}%
\begin{pgfscope}%
\pgfpathrectangle{\pgfqpoint{0.150000in}{0.150000in}}{\pgfqpoint{1.700000in}{1.700000in}}%
\pgfusepath{clip}%
\pgfsetbuttcap%
\pgfsetroundjoin%
\definecolor{currentfill}{rgb}{0.933333,0.600000,0.666667}%
\pgfsetfillcolor{currentfill}%
\pgfsetlinewidth{1.003750pt}%
\definecolor{currentstroke}{rgb}{0.600000,0.266667,0.333333}%
\pgfsetstrokecolor{currentstroke}%
\pgfsetdash{}{0pt}%
\pgfpathmoveto{\pgfqpoint{0.930598in}{1.637585in}}%
\pgfpathlineto{\pgfqpoint{1.085191in}{1.637585in}}%
\pgfpathlineto{\pgfqpoint{1.085191in}{1.688774in}}%
\pgfpathlineto{\pgfqpoint{0.930598in}{1.688774in}}%
\pgfpathlineto{\pgfqpoint{0.930598in}{1.637585in}}%
\pgfpathclose%
\pgfusepath{stroke,fill}%
\end{pgfscope}%
\begin{pgfscope}%
\pgfpathrectangle{\pgfqpoint{0.150000in}{0.150000in}}{\pgfqpoint{1.700000in}{1.700000in}}%
\pgfusepath{clip}%
\pgfsetbuttcap%
\pgfsetroundjoin%
\definecolor{currentfill}{rgb}{0.933333,0.600000,0.666667}%
\pgfsetfillcolor{currentfill}%
\pgfsetlinewidth{1.003750pt}%
\definecolor{currentstroke}{rgb}{0.600000,0.266667,0.333333}%
\pgfsetstrokecolor{currentstroke}%
\pgfsetdash{}{0pt}%
\pgfpathmoveto{\pgfqpoint{1.085191in}{1.391533in}}%
\pgfpathlineto{\pgfqpoint{1.274139in}{1.391533in}}%
\pgfpathlineto{\pgfqpoint{1.274139in}{1.400694in}}%
\pgfpathlineto{\pgfqpoint{1.085191in}{1.400694in}}%
\pgfpathlineto{\pgfqpoint{1.085191in}{1.391533in}}%
\pgfpathclose%
\pgfusepath{stroke,fill}%
\end{pgfscope}%
\begin{pgfscope}%
\pgfpathrectangle{\pgfqpoint{0.150000in}{0.150000in}}{\pgfqpoint{1.700000in}{1.700000in}}%
\pgfusepath{clip}%
\pgfsetbuttcap%
\pgfsetroundjoin%
\definecolor{currentfill}{rgb}{0.933333,0.600000,0.666667}%
\pgfsetfillcolor{currentfill}%
\pgfsetlinewidth{1.003750pt}%
\definecolor{currentstroke}{rgb}{0.600000,0.266667,0.333333}%
\pgfsetstrokecolor{currentstroke}%
\pgfsetdash{}{0pt}%
\pgfpathmoveto{\pgfqpoint{1.460660in}{0.617728in}}%
\pgfpathlineto{\pgfqpoint{1.579254in}{0.617728in}}%
\pgfpathlineto{\pgfqpoint{1.579254in}{0.930598in}}%
\pgfpathlineto{\pgfqpoint{1.460660in}{0.930598in}}%
\pgfpathlineto{\pgfqpoint{1.460660in}{0.617728in}}%
\pgfpathclose%
\pgfusepath{stroke,fill}%
\end{pgfscope}%
\begin{pgfscope}%
\pgfpathrectangle{\pgfqpoint{0.150000in}{0.150000in}}{\pgfqpoint{1.700000in}{1.700000in}}%
\pgfusepath{clip}%
\pgfsetbuttcap%
\pgfsetroundjoin%
\definecolor{currentfill}{rgb}{0.933333,0.600000,0.666667}%
\pgfsetfillcolor{currentfill}%
\pgfsetlinewidth{1.003750pt}%
\definecolor{currentstroke}{rgb}{0.600000,0.266667,0.333333}%
\pgfsetstrokecolor{currentstroke}%
\pgfsetdash{}{0pt}%
\pgfpathmoveto{\pgfqpoint{1.272573in}{0.617728in}}%
\pgfpathlineto{\pgfqpoint{1.394638in}{0.617728in}}%
\pgfpathlineto{\pgfqpoint{1.394638in}{0.706301in}}%
\pgfpathlineto{\pgfqpoint{1.272573in}{0.706301in}}%
\pgfpathlineto{\pgfqpoint{1.272573in}{0.617728in}}%
\pgfpathclose%
\pgfusepath{stroke,fill}%
\end{pgfscope}%
\begin{pgfscope}%
\pgfpathrectangle{\pgfqpoint{0.150000in}{0.150000in}}{\pgfqpoint{1.700000in}{1.700000in}}%
\pgfusepath{clip}%
\pgfsetbuttcap%
\pgfsetroundjoin%
\definecolor{currentfill}{rgb}{0.933333,0.600000,0.666667}%
\pgfsetfillcolor{currentfill}%
\pgfsetlinewidth{1.003750pt}%
\definecolor{currentstroke}{rgb}{0.600000,0.266667,0.333333}%
\pgfsetstrokecolor{currentstroke}%
\pgfsetdash{}{0pt}%
\pgfpathmoveto{\pgfqpoint{1.394638in}{0.617728in}}%
\pgfpathlineto{\pgfqpoint{1.460660in}{0.617728in}}%
\pgfpathlineto{\pgfqpoint{1.460660in}{0.930598in}}%
\pgfpathlineto{\pgfqpoint{1.394638in}{0.930598in}}%
\pgfpathlineto{\pgfqpoint{1.394638in}{0.617728in}}%
\pgfpathclose%
\pgfusepath{stroke,fill}%
\end{pgfscope}%
\begin{pgfscope}%
\pgfpathrectangle{\pgfqpoint{0.150000in}{0.150000in}}{\pgfqpoint{1.700000in}{1.700000in}}%
\pgfusepath{clip}%
\pgfsetbuttcap%
\pgfsetroundjoin%
\definecolor{currentfill}{rgb}{0.933333,0.600000,0.666667}%
\pgfsetfillcolor{currentfill}%
\pgfsetlinewidth{1.003750pt}%
\definecolor{currentstroke}{rgb}{0.600000,0.266667,0.333333}%
\pgfsetstrokecolor{currentstroke}%
\pgfsetdash{}{0pt}%
\pgfpathmoveto{\pgfqpoint{1.272573in}{0.440455in}}%
\pgfpathlineto{\pgfqpoint{1.410580in}{0.440455in}}%
\pgfpathlineto{\pgfqpoint{1.410580in}{0.617728in}}%
\pgfpathlineto{\pgfqpoint{1.272573in}{0.617728in}}%
\pgfpathlineto{\pgfqpoint{1.272573in}{0.440455in}}%
\pgfpathclose%
\pgfusepath{stroke,fill}%
\end{pgfscope}%
\begin{pgfscope}%
\pgfpathrectangle{\pgfqpoint{0.150000in}{0.150000in}}{\pgfqpoint{1.700000in}{1.700000in}}%
\pgfusepath{clip}%
\pgfsetbuttcap%
\pgfsetroundjoin%
\definecolor{currentfill}{rgb}{0.933333,0.600000,0.666667}%
\pgfsetfillcolor{currentfill}%
\pgfsetlinewidth{1.003750pt}%
\definecolor{currentstroke}{rgb}{0.600000,0.266667,0.333333}%
\pgfsetstrokecolor{currentstroke}%
\pgfsetdash{}{0pt}%
\pgfpathmoveto{\pgfqpoint{1.084487in}{0.599306in}}%
\pgfpathlineto{\pgfqpoint{1.272573in}{0.599306in}}%
\pgfpathlineto{\pgfqpoint{1.272573in}{0.608314in}}%
\pgfpathlineto{\pgfqpoint{1.084487in}{0.608314in}}%
\pgfpathlineto{\pgfqpoint{1.084487in}{0.599306in}}%
\pgfpathclose%
\pgfusepath{stroke,fill}%
\end{pgfscope}%
\begin{pgfscope}%
\pgfpathrectangle{\pgfqpoint{0.150000in}{0.150000in}}{\pgfqpoint{1.700000in}{1.700000in}}%
\pgfusepath{clip}%
\pgfsetbuttcap%
\pgfsetroundjoin%
\definecolor{currentfill}{rgb}{0.933333,0.600000,0.666667}%
\pgfsetfillcolor{currentfill}%
\pgfsetlinewidth{1.003750pt}%
\definecolor{currentstroke}{rgb}{0.600000,0.266667,0.333333}%
\pgfsetstrokecolor{currentstroke}%
\pgfsetdash{}{0pt}%
\pgfpathmoveto{\pgfqpoint{0.930598in}{0.311140in}}%
\pgfpathlineto{\pgfqpoint{1.084487in}{0.311140in}}%
\pgfpathlineto{\pgfqpoint{1.084487in}{0.361744in}}%
\pgfpathlineto{\pgfqpoint{0.930598in}{0.361744in}}%
\pgfpathlineto{\pgfqpoint{0.930598in}{0.311140in}}%
\pgfpathclose%
\pgfusepath{stroke,fill}%
\end{pgfscope}%
\begin{pgfscope}%
\pgfpathrectangle{\pgfqpoint{0.150000in}{0.150000in}}{\pgfqpoint{1.700000in}{1.700000in}}%
\pgfusepath{clip}%
\pgfsetbuttcap%
\pgfsetroundjoin%
\definecolor{currentfill}{rgb}{0.933333,0.600000,0.666667}%
\pgfsetfillcolor{currentfill}%
\pgfsetlinewidth{1.003750pt}%
\definecolor{currentstroke}{rgb}{0.600000,0.266667,0.333333}%
\pgfsetstrokecolor{currentstroke}%
\pgfsetdash{}{0pt}%
\pgfpathmoveto{\pgfqpoint{0.617773in}{1.460765in}}%
\pgfpathlineto{\pgfqpoint{0.930598in}{1.460765in}}%
\pgfpathlineto{\pgfqpoint{0.930598in}{1.579284in}}%
\pgfpathlineto{\pgfqpoint{0.617773in}{1.579284in}}%
\pgfpathlineto{\pgfqpoint{0.617773in}{1.460765in}}%
\pgfpathclose%
\pgfusepath{stroke,fill}%
\end{pgfscope}%
\begin{pgfscope}%
\pgfpathrectangle{\pgfqpoint{0.150000in}{0.150000in}}{\pgfqpoint{1.700000in}{1.700000in}}%
\pgfusepath{clip}%
\pgfsetbuttcap%
\pgfsetroundjoin%
\definecolor{currentfill}{rgb}{0.933333,0.600000,0.666667}%
\pgfsetfillcolor{currentfill}%
\pgfsetlinewidth{1.003750pt}%
\definecolor{currentstroke}{rgb}{0.600000,0.266667,0.333333}%
\pgfsetstrokecolor{currentstroke}%
\pgfsetdash{}{0pt}%
\pgfpathmoveto{\pgfqpoint{0.706478in}{1.394638in}}%
\pgfpathlineto{\pgfqpoint{0.930598in}{1.394638in}}%
\pgfpathlineto{\pgfqpoint{0.930598in}{1.460765in}}%
\pgfpathlineto{\pgfqpoint{0.706478in}{1.460765in}}%
\pgfpathlineto{\pgfqpoint{0.706478in}{1.394638in}}%
\pgfpathclose%
\pgfusepath{stroke,fill}%
\end{pgfscope}%
\begin{pgfscope}%
\pgfpathrectangle{\pgfqpoint{0.150000in}{0.150000in}}{\pgfqpoint{1.700000in}{1.700000in}}%
\pgfusepath{clip}%
\pgfsetbuttcap%
\pgfsetroundjoin%
\definecolor{currentfill}{rgb}{0.933333,0.600000,0.666667}%
\pgfsetfillcolor{currentfill}%
\pgfsetlinewidth{1.003750pt}%
\definecolor{currentstroke}{rgb}{0.600000,0.266667,0.333333}%
\pgfsetstrokecolor{currentstroke}%
\pgfsetdash{}{0pt}%
\pgfpathmoveto{\pgfqpoint{0.617773in}{1.272765in}}%
\pgfpathlineto{\pgfqpoint{0.706478in}{1.272765in}}%
\pgfpathlineto{\pgfqpoint{0.706478in}{1.460765in}}%
\pgfpathlineto{\pgfqpoint{0.617773in}{1.460765in}}%
\pgfpathlineto{\pgfqpoint{0.617773in}{1.272765in}}%
\pgfpathclose%
\pgfusepath{stroke,fill}%
\end{pgfscope}%
\begin{pgfscope}%
\pgfpathrectangle{\pgfqpoint{0.150000in}{0.150000in}}{\pgfqpoint{1.700000in}{1.700000in}}%
\pgfusepath{clip}%
\pgfsetbuttcap%
\pgfsetroundjoin%
\definecolor{currentfill}{rgb}{0.933333,0.600000,0.666667}%
\pgfsetfillcolor{currentfill}%
\pgfsetlinewidth{1.003750pt}%
\definecolor{currentstroke}{rgb}{0.600000,0.266667,0.333333}%
\pgfsetstrokecolor{currentstroke}%
\pgfsetdash{}{0pt}%
\pgfpathmoveto{\pgfqpoint{0.440542in}{1.272765in}}%
\pgfpathlineto{\pgfqpoint{0.617773in}{1.272765in}}%
\pgfpathlineto{\pgfqpoint{0.617773in}{1.410698in}}%
\pgfpathlineto{\pgfqpoint{0.440542in}{1.410698in}}%
\pgfpathlineto{\pgfqpoint{0.440542in}{1.272765in}}%
\pgfpathclose%
\pgfusepath{stroke,fill}%
\end{pgfscope}%
\begin{pgfscope}%
\pgfpathrectangle{\pgfqpoint{0.150000in}{0.150000in}}{\pgfqpoint{1.700000in}{1.700000in}}%
\pgfusepath{clip}%
\pgfsetbuttcap%
\pgfsetroundjoin%
\definecolor{currentfill}{rgb}{0.933333,0.600000,0.666667}%
\pgfsetfillcolor{currentfill}%
\pgfsetlinewidth{1.003750pt}%
\definecolor{currentstroke}{rgb}{0.600000,0.266667,0.333333}%
\pgfsetstrokecolor{currentstroke}%
\pgfsetdash{}{0pt}%
\pgfpathmoveto{\pgfqpoint{0.599306in}{1.084764in}}%
\pgfpathlineto{\pgfqpoint{0.608374in}{1.084764in}}%
\pgfpathlineto{\pgfqpoint{0.608374in}{1.272765in}}%
\pgfpathlineto{\pgfqpoint{0.599306in}{1.272765in}}%
\pgfpathlineto{\pgfqpoint{0.599306in}{1.084764in}}%
\pgfpathclose%
\pgfusepath{stroke,fill}%
\end{pgfscope}%
\begin{pgfscope}%
\pgfpathrectangle{\pgfqpoint{0.150000in}{0.150000in}}{\pgfqpoint{1.700000in}{1.700000in}}%
\pgfusepath{clip}%
\pgfsetbuttcap%
\pgfsetroundjoin%
\definecolor{currentfill}{rgb}{0.933333,0.600000,0.666667}%
\pgfsetfillcolor{currentfill}%
\pgfsetlinewidth{1.003750pt}%
\definecolor{currentstroke}{rgb}{0.600000,0.266667,0.333333}%
\pgfsetstrokecolor{currentstroke}%
\pgfsetdash{}{0pt}%
\pgfpathmoveto{\pgfqpoint{0.311174in}{0.930946in}}%
\pgfpathlineto{\pgfqpoint{0.361826in}{0.930946in}}%
\pgfpathlineto{\pgfqpoint{0.361826in}{1.084764in}}%
\pgfpathlineto{\pgfqpoint{0.311174in}{1.084764in}}%
\pgfpathlineto{\pgfqpoint{0.311174in}{0.930946in}}%
\pgfpathclose%
\pgfusepath{stroke,fill}%
\end{pgfscope}%
\begin{pgfscope}%
\pgfpathrectangle{\pgfqpoint{0.150000in}{0.150000in}}{\pgfqpoint{1.700000in}{1.700000in}}%
\pgfusepath{clip}%
\pgfsetbuttcap%
\pgfsetroundjoin%
\definecolor{currentfill}{rgb}{0.933333,0.600000,0.666667}%
\pgfsetfillcolor{currentfill}%
\pgfsetlinewidth{1.003750pt}%
\definecolor{currentstroke}{rgb}{0.600000,0.266667,0.333333}%
\pgfsetstrokecolor{currentstroke}%
\pgfsetdash{}{0pt}%
\pgfpathmoveto{\pgfqpoint{0.605301in}{0.605362in}}%
\pgfpathlineto{\pgfqpoint{0.685374in}{0.605362in}}%
\pgfpathlineto{\pgfqpoint{0.685374in}{0.751875in}}%
\pgfpathlineto{\pgfqpoint{0.605301in}{0.751875in}}%
\pgfpathlineto{\pgfqpoint{0.605301in}{0.605362in}}%
\pgfpathclose%
\pgfusepath{stroke,fill}%
\end{pgfscope}%
\begin{pgfscope}%
\pgfpathrectangle{\pgfqpoint{0.150000in}{0.150000in}}{\pgfqpoint{1.700000in}{1.700000in}}%
\pgfusepath{clip}%
\pgfsetbuttcap%
\pgfsetroundjoin%
\definecolor{currentfill}{rgb}{0.933333,0.600000,0.666667}%
\pgfsetfillcolor{currentfill}%
\pgfsetlinewidth{1.003750pt}%
\definecolor{currentstroke}{rgb}{0.600000,0.266667,0.333333}%
\pgfsetstrokecolor{currentstroke}%
\pgfsetdash{}{0pt}%
\pgfpathmoveto{\pgfqpoint{0.355338in}{0.742945in}}%
\pgfpathlineto{\pgfqpoint{0.440670in}{0.742945in}}%
\pgfpathlineto{\pgfqpoint{0.440670in}{0.930946in}}%
\pgfpathlineto{\pgfqpoint{0.355338in}{0.930946in}}%
\pgfpathlineto{\pgfqpoint{0.355338in}{0.742945in}}%
\pgfpathclose%
\pgfusepath{stroke,fill}%
\end{pgfscope}%
\begin{pgfscope}%
\pgfpathrectangle{\pgfqpoint{0.150000in}{0.150000in}}{\pgfqpoint{1.700000in}{1.700000in}}%
\pgfusepath{clip}%
\pgfsetbuttcap%
\pgfsetroundjoin%
\definecolor{currentfill}{rgb}{0.933333,0.600000,0.666667}%
\pgfsetfillcolor{currentfill}%
\pgfsetlinewidth{1.003750pt}%
\definecolor{currentstroke}{rgb}{0.600000,0.266667,0.333333}%
\pgfsetstrokecolor{currentstroke}%
\pgfsetdash{}{0pt}%
\pgfpathmoveto{\pgfqpoint{0.782395in}{0.340975in}}%
\pgfpathlineto{\pgfqpoint{0.930598in}{0.340975in}}%
\pgfpathlineto{\pgfqpoint{0.930598in}{0.394328in}}%
\pgfpathlineto{\pgfqpoint{0.782395in}{0.394328in}}%
\pgfpathlineto{\pgfqpoint{0.782395in}{0.340975in}}%
\pgfpathclose%
\pgfusepath{stroke,fill}%
\end{pgfscope}%
\begin{pgfscope}%
\pgfpathrectangle{\pgfqpoint{0.150000in}{0.150000in}}{\pgfqpoint{1.700000in}{1.700000in}}%
\pgfusepath{clip}%
\pgfsetbuttcap%
\pgfsetroundjoin%
\definecolor{currentfill}{rgb}{0.933333,0.600000,0.666667}%
\pgfsetfillcolor{currentfill}%
\pgfsetlinewidth{1.003750pt}%
\definecolor{currentstroke}{rgb}{0.600000,0.266667,0.333333}%
\pgfsetstrokecolor{currentstroke}%
\pgfsetdash{}{0pt}%
\pgfpathmoveto{\pgfqpoint{0.539881in}{0.480426in}}%
\pgfpathlineto{\pgfqpoint{0.661138in}{0.480426in}}%
\pgfpathlineto{\pgfqpoint{0.661138in}{0.589127in}}%
\pgfpathlineto{\pgfqpoint{0.539881in}{0.589127in}}%
\pgfpathlineto{\pgfqpoint{0.539881in}{0.480426in}}%
\pgfpathclose%
\pgfusepath{stroke,fill}%
\end{pgfscope}%
\begin{pgfscope}%
\pgfpathrectangle{\pgfqpoint{0.150000in}{0.150000in}}{\pgfqpoint{1.700000in}{1.700000in}}%
\pgfusepath{clip}%
\pgfsetbuttcap%
\pgfsetroundjoin%
\definecolor{currentfill}{rgb}{0.933333,0.600000,0.666667}%
\pgfsetfillcolor{currentfill}%
\pgfsetlinewidth{1.003750pt}%
\definecolor{currentstroke}{rgb}{0.600000,0.266667,0.333333}%
\pgfsetstrokecolor{currentstroke}%
\pgfsetdash{}{0pt}%
\pgfpathmoveto{\pgfqpoint{0.930598in}{1.473041in}}%
\pgfpathlineto{\pgfqpoint{1.274139in}{1.473041in}}%
\pgfpathlineto{\pgfqpoint{1.274139in}{1.637585in}}%
\pgfpathlineto{\pgfqpoint{0.930598in}{1.637585in}}%
\pgfpathlineto{\pgfqpoint{0.930598in}{1.473041in}}%
\pgfpathclose%
\pgfusepath{stroke,fill}%
\end{pgfscope}%
\begin{pgfscope}%
\pgfpathrectangle{\pgfqpoint{0.150000in}{0.150000in}}{\pgfqpoint{1.700000in}{1.700000in}}%
\pgfusepath{clip}%
\pgfsetbuttcap%
\pgfsetroundjoin%
\definecolor{currentfill}{rgb}{0.933333,0.600000,0.666667}%
\pgfsetfillcolor{currentfill}%
\pgfsetlinewidth{1.003750pt}%
\definecolor{currentstroke}{rgb}{0.600000,0.266667,0.333333}%
\pgfsetstrokecolor{currentstroke}%
\pgfsetdash{}{0pt}%
\pgfpathmoveto{\pgfqpoint{0.930598in}{1.400694in}}%
\pgfpathlineto{\pgfqpoint{1.274139in}{1.400694in}}%
\pgfpathlineto{\pgfqpoint{1.274139in}{1.473041in}}%
\pgfpathlineto{\pgfqpoint{0.930598in}{1.473041in}}%
\pgfpathlineto{\pgfqpoint{0.930598in}{1.400694in}}%
\pgfpathclose%
\pgfusepath{stroke,fill}%
\end{pgfscope}%
\begin{pgfscope}%
\pgfpathrectangle{\pgfqpoint{0.150000in}{0.150000in}}{\pgfqpoint{1.700000in}{1.700000in}}%
\pgfusepath{clip}%
\pgfsetbuttcap%
\pgfsetroundjoin%
\definecolor{currentfill}{rgb}{0.933333,0.600000,0.666667}%
\pgfsetfillcolor{currentfill}%
\pgfsetlinewidth{1.003750pt}%
\definecolor{currentstroke}{rgb}{0.600000,0.266667,0.333333}%
\pgfsetstrokecolor{currentstroke}%
\pgfsetdash{}{0pt}%
\pgfpathmoveto{\pgfqpoint{0.930598in}{0.486123in}}%
\pgfpathlineto{\pgfqpoint{1.272573in}{0.486123in}}%
\pgfpathlineto{\pgfqpoint{1.272573in}{0.599306in}}%
\pgfpathlineto{\pgfqpoint{0.930598in}{0.599306in}}%
\pgfpathlineto{\pgfqpoint{0.930598in}{0.486123in}}%
\pgfpathclose%
\pgfusepath{stroke,fill}%
\end{pgfscope}%
\begin{pgfscope}%
\pgfpathrectangle{\pgfqpoint{0.150000in}{0.150000in}}{\pgfqpoint{1.700000in}{1.700000in}}%
\pgfusepath{clip}%
\pgfsetbuttcap%
\pgfsetroundjoin%
\definecolor{currentfill}{rgb}{0.933333,0.600000,0.666667}%
\pgfsetfillcolor{currentfill}%
\pgfsetlinewidth{1.003750pt}%
\definecolor{currentstroke}{rgb}{0.600000,0.266667,0.333333}%
\pgfsetstrokecolor{currentstroke}%
\pgfsetdash{}{0pt}%
\pgfpathmoveto{\pgfqpoint{0.930598in}{0.361744in}}%
\pgfpathlineto{\pgfqpoint{1.272573in}{0.361744in}}%
\pgfpathlineto{\pgfqpoint{1.272573in}{0.486123in}}%
\pgfpathlineto{\pgfqpoint{0.930598in}{0.486123in}}%
\pgfpathlineto{\pgfqpoint{0.930598in}{0.361744in}}%
\pgfpathclose%
\pgfusepath{stroke,fill}%
\end{pgfscope}%
\begin{pgfscope}%
\pgfpathrectangle{\pgfqpoint{0.150000in}{0.150000in}}{\pgfqpoint{1.700000in}{1.700000in}}%
\pgfusepath{clip}%
\pgfsetbuttcap%
\pgfsetroundjoin%
\definecolor{currentfill}{rgb}{0.933333,0.600000,0.666667}%
\pgfsetfillcolor{currentfill}%
\pgfsetlinewidth{1.003750pt}%
\definecolor{currentstroke}{rgb}{0.600000,0.266667,0.333333}%
\pgfsetstrokecolor{currentstroke}%
\pgfsetdash{}{0pt}%
\pgfpathmoveto{\pgfqpoint{0.486203in}{0.930946in}}%
\pgfpathlineto{\pgfqpoint{0.599306in}{0.930946in}}%
\pgfpathlineto{\pgfqpoint{0.599306in}{1.272765in}}%
\pgfpathlineto{\pgfqpoint{0.486203in}{1.272765in}}%
\pgfpathlineto{\pgfqpoint{0.486203in}{0.930946in}}%
\pgfpathclose%
\pgfusepath{stroke,fill}%
\end{pgfscope}%
\begin{pgfscope}%
\pgfpathrectangle{\pgfqpoint{0.150000in}{0.150000in}}{\pgfqpoint{1.700000in}{1.700000in}}%
\pgfusepath{clip}%
\pgfsetbuttcap%
\pgfsetroundjoin%
\definecolor{currentfill}{rgb}{0.933333,0.600000,0.666667}%
\pgfsetfillcolor{currentfill}%
\pgfsetlinewidth{1.003750pt}%
\definecolor{currentstroke}{rgb}{0.600000,0.266667,0.333333}%
\pgfsetstrokecolor{currentstroke}%
\pgfsetdash{}{0pt}%
\pgfpathmoveto{\pgfqpoint{0.361826in}{0.930946in}}%
\pgfpathlineto{\pgfqpoint{0.486203in}{0.930946in}}%
\pgfpathlineto{\pgfqpoint{0.486203in}{1.272765in}}%
\pgfpathlineto{\pgfqpoint{0.361826in}{1.272765in}}%
\pgfpathlineto{\pgfqpoint{0.361826in}{0.930946in}}%
\pgfpathclose%
\pgfusepath{stroke,fill}%
\end{pgfscope}%
\begin{pgfscope}%
\pgfpathrectangle{\pgfqpoint{0.150000in}{0.150000in}}{\pgfqpoint{1.700000in}{1.700000in}}%
\pgfusepath{clip}%
\pgfsetbuttcap%
\pgfsetroundjoin%
\definecolor{currentfill}{rgb}{0.933333,0.600000,0.666667}%
\pgfsetfillcolor{currentfill}%
\pgfsetlinewidth{1.003750pt}%
\definecolor{currentstroke}{rgb}{0.600000,0.266667,0.333333}%
\pgfsetstrokecolor{currentstroke}%
\pgfsetdash{}{0pt}%
\pgfpathmoveto{\pgfqpoint{0.605301in}{0.589127in}}%
\pgfpathlineto{\pgfqpoint{0.930598in}{0.589127in}}%
\pgfpathlineto{\pgfqpoint{0.930598in}{0.605362in}}%
\pgfpathlineto{\pgfqpoint{0.605301in}{0.605362in}}%
\pgfpathlineto{\pgfqpoint{0.605301in}{0.589127in}}%
\pgfpathclose%
\pgfusepath{stroke,fill}%
\end{pgfscope}%
\begin{pgfscope}%
\pgfpathrectangle{\pgfqpoint{0.150000in}{0.150000in}}{\pgfqpoint{1.700000in}{1.700000in}}%
\pgfusepath{clip}%
\pgfsetbuttcap%
\pgfsetroundjoin%
\definecolor{currentfill}{rgb}{0.933333,0.600000,0.666667}%
\pgfsetfillcolor{currentfill}%
\pgfsetlinewidth{1.003750pt}%
\definecolor{currentstroke}{rgb}{0.600000,0.266667,0.333333}%
\pgfsetstrokecolor{currentstroke}%
\pgfsetdash{}{0pt}%
\pgfpathmoveto{\pgfqpoint{0.588951in}{0.589127in}}%
\pgfpathlineto{\pgfqpoint{0.605301in}{0.589127in}}%
\pgfpathlineto{\pgfqpoint{0.605301in}{0.930946in}}%
\pgfpathlineto{\pgfqpoint{0.588951in}{0.930946in}}%
\pgfpathlineto{\pgfqpoint{0.588951in}{0.589127in}}%
\pgfpathclose%
\pgfusepath{stroke,fill}%
\end{pgfscope}%
\begin{pgfscope}%
\pgfpathrectangle{\pgfqpoint{0.150000in}{0.150000in}}{\pgfqpoint{1.700000in}{1.700000in}}%
\pgfusepath{clip}%
\pgfsetbuttcap%
\pgfsetroundjoin%
\definecolor{currentfill}{rgb}{0.933333,0.600000,0.666667}%
\pgfsetfillcolor{currentfill}%
\pgfsetlinewidth{1.003750pt}%
\definecolor{currentstroke}{rgb}{0.600000,0.266667,0.333333}%
\pgfsetstrokecolor{currentstroke}%
\pgfsetdash{}{0pt}%
\pgfpathmoveto{\pgfqpoint{0.440670in}{0.589127in}}%
\pgfpathlineto{\pgfqpoint{0.588951in}{0.589127in}}%
\pgfpathlineto{\pgfqpoint{0.588951in}{0.930946in}}%
\pgfpathlineto{\pgfqpoint{0.440670in}{0.930946in}}%
\pgfpathlineto{\pgfqpoint{0.440670in}{0.589127in}}%
\pgfpathclose%
\pgfusepath{stroke,fill}%
\end{pgfscope}%
\begin{pgfscope}%
\pgfpathrectangle{\pgfqpoint{0.150000in}{0.150000in}}{\pgfqpoint{1.700000in}{1.700000in}}%
\pgfusepath{clip}%
\pgfsetbuttcap%
\pgfsetroundjoin%
\definecolor{currentfill}{rgb}{0.933333,0.600000,0.666667}%
\pgfsetfillcolor{currentfill}%
\pgfsetlinewidth{1.003750pt}%
\definecolor{currentstroke}{rgb}{0.600000,0.266667,0.333333}%
\pgfsetstrokecolor{currentstroke}%
\pgfsetdash{}{0pt}%
\pgfpathmoveto{\pgfqpoint{0.661138in}{0.394328in}}%
\pgfpathlineto{\pgfqpoint{0.930598in}{0.394328in}}%
\pgfpathlineto{\pgfqpoint{0.930598in}{0.589127in}}%
\pgfpathlineto{\pgfqpoint{0.661138in}{0.589127in}}%
\pgfpathlineto{\pgfqpoint{0.661138in}{0.394328in}}%
\pgfpathclose%
\pgfusepath{stroke,fill}%
\end{pgfscope}%
\begin{pgfscope}%
\pgfpathrectangle{\pgfqpoint{0.150000in}{0.150000in}}{\pgfqpoint{1.700000in}{1.700000in}}%
\pgfusepath{clip}%
\pgfsetbuttcap%
\pgfsetroundjoin%
\definecolor{currentfill}{rgb}{0.400000,0.600000,0.800000}%
\pgfsetfillcolor{currentfill}%
\pgfsetlinewidth{1.003750pt}%
\definecolor{currentstroke}{rgb}{0.000000,0.266667,0.533333}%
\pgfsetstrokecolor{currentstroke}%
\pgfsetdash{}{0pt}%
\pgfpathmoveto{\pgfqpoint{1.515791in}{1.464356in}}%
\pgfpathlineto{\pgfqpoint{1.549661in}{1.464356in}}%
\pgfpathlineto{\pgfqpoint{1.549661in}{1.514020in}}%
\pgfpathlineto{\pgfqpoint{1.515791in}{1.514020in}}%
\pgfpathlineto{\pgfqpoint{1.515791in}{1.464356in}}%
\pgfpathclose%
\pgfusepath{stroke,fill}%
\end{pgfscope}%
\begin{pgfscope}%
\pgfpathrectangle{\pgfqpoint{0.150000in}{0.150000in}}{\pgfqpoint{1.700000in}{1.700000in}}%
\pgfusepath{clip}%
\pgfsetbuttcap%
\pgfsetroundjoin%
\definecolor{currentfill}{rgb}{0.400000,0.600000,0.800000}%
\pgfsetfillcolor{currentfill}%
\pgfsetlinewidth{1.003750pt}%
\definecolor{currentstroke}{rgb}{0.000000,0.266667,0.533333}%
\pgfsetstrokecolor{currentstroke}%
\pgfsetdash{}{0pt}%
\pgfpathmoveto{\pgfqpoint{1.586670in}{1.370790in}}%
\pgfpathlineto{\pgfqpoint{1.611900in}{1.370790in}}%
\pgfpathlineto{\pgfqpoint{1.611900in}{1.423721in}}%
\pgfpathlineto{\pgfqpoint{1.586670in}{1.423721in}}%
\pgfpathlineto{\pgfqpoint{1.586670in}{1.370790in}}%
\pgfpathclose%
\pgfusepath{stroke,fill}%
\end{pgfscope}%
\begin{pgfscope}%
\pgfpathrectangle{\pgfqpoint{0.150000in}{0.150000in}}{\pgfqpoint{1.700000in}{1.700000in}}%
\pgfusepath{clip}%
\pgfsetbuttcap%
\pgfsetroundjoin%
\definecolor{currentfill}{rgb}{0.400000,0.600000,0.800000}%
\pgfsetfillcolor{currentfill}%
\pgfsetlinewidth{1.003750pt}%
\definecolor{currentstroke}{rgb}{0.000000,0.266667,0.533333}%
\pgfsetstrokecolor{currentstroke}%
\pgfsetdash{}{0pt}%
\pgfpathmoveto{\pgfqpoint{1.633175in}{1.284175in}}%
\pgfpathlineto{\pgfqpoint{1.647915in}{1.284175in}}%
\pgfpathlineto{\pgfqpoint{1.647915in}{1.327483in}}%
\pgfpathlineto{\pgfqpoint{1.633175in}{1.327483in}}%
\pgfpathlineto{\pgfqpoint{1.633175in}{1.284175in}}%
\pgfpathclose%
\pgfusepath{stroke,fill}%
\end{pgfscope}%
\begin{pgfscope}%
\pgfpathrectangle{\pgfqpoint{0.150000in}{0.150000in}}{\pgfqpoint{1.700000in}{1.700000in}}%
\pgfusepath{clip}%
\pgfsetbuttcap%
\pgfsetroundjoin%
\definecolor{currentfill}{rgb}{0.400000,0.600000,0.800000}%
\pgfsetfillcolor{currentfill}%
\pgfsetlinewidth{1.003750pt}%
\definecolor{currentstroke}{rgb}{0.000000,0.266667,0.533333}%
\pgfsetstrokecolor{currentstroke}%
\pgfsetdash{}{0pt}%
\pgfpathmoveto{\pgfqpoint{1.665826in}{1.195811in}}%
\pgfpathlineto{\pgfqpoint{1.677059in}{1.195811in}}%
\pgfpathlineto{\pgfqpoint{1.677059in}{1.248742in}}%
\pgfpathlineto{\pgfqpoint{1.665826in}{1.248742in}}%
\pgfpathlineto{\pgfqpoint{1.665826in}{1.195811in}}%
\pgfpathclose%
\pgfusepath{stroke,fill}%
\end{pgfscope}%
\begin{pgfscope}%
\pgfpathrectangle{\pgfqpoint{0.150000in}{0.150000in}}{\pgfqpoint{1.700000in}{1.700000in}}%
\pgfusepath{clip}%
\pgfsetbuttcap%
\pgfsetroundjoin%
\definecolor{currentfill}{rgb}{0.400000,0.600000,0.800000}%
\pgfsetfillcolor{currentfill}%
\pgfsetlinewidth{1.003750pt}%
\definecolor{currentstroke}{rgb}{0.000000,0.266667,0.533333}%
\pgfsetstrokecolor{currentstroke}%
\pgfsetdash{}{0pt}%
\pgfpathmoveto{\pgfqpoint{1.685378in}{1.109196in}}%
\pgfpathlineto{\pgfqpoint{1.690091in}{1.109196in}}%
\pgfpathlineto{\pgfqpoint{1.690091in}{1.152503in}}%
\pgfpathlineto{\pgfqpoint{1.685378in}{1.152503in}}%
\pgfpathlineto{\pgfqpoint{1.685378in}{1.109196in}}%
\pgfpathclose%
\pgfusepath{stroke,fill}%
\end{pgfscope}%
\begin{pgfscope}%
\pgfpathrectangle{\pgfqpoint{0.150000in}{0.150000in}}{\pgfqpoint{1.700000in}{1.700000in}}%
\pgfusepath{clip}%
\pgfsetbuttcap%
\pgfsetroundjoin%
\definecolor{currentfill}{rgb}{0.400000,0.600000,0.800000}%
\pgfsetfillcolor{currentfill}%
\pgfsetlinewidth{1.003750pt}%
\definecolor{currentstroke}{rgb}{0.000000,0.266667,0.533333}%
\pgfsetstrokecolor{currentstroke}%
\pgfsetdash{}{0pt}%
\pgfpathmoveto{\pgfqpoint{1.693354in}{1.030455in}}%
\pgfpathlineto{\pgfqpoint{1.694022in}{1.030455in}}%
\pgfpathlineto{\pgfqpoint{1.694022in}{1.073763in}}%
\pgfpathlineto{\pgfqpoint{1.693354in}{1.073763in}}%
\pgfpathlineto{\pgfqpoint{1.693354in}{1.030455in}}%
\pgfpathclose%
\pgfusepath{stroke,fill}%
\end{pgfscope}%
\begin{pgfscope}%
\pgfpathrectangle{\pgfqpoint{0.150000in}{0.150000in}}{\pgfqpoint{1.700000in}{1.700000in}}%
\pgfusepath{clip}%
\pgfsetbuttcap%
\pgfsetroundjoin%
\definecolor{currentfill}{rgb}{0.400000,0.600000,0.800000}%
\pgfsetfillcolor{currentfill}%
\pgfsetlinewidth{1.003750pt}%
\definecolor{currentstroke}{rgb}{0.000000,0.266667,0.533333}%
\pgfsetstrokecolor{currentstroke}%
\pgfsetdash{}{0pt}%
\pgfpathmoveto{\pgfqpoint{1.314138in}{1.152503in}}%
\pgfpathlineto{\pgfqpoint{1.349591in}{1.152503in}}%
\pgfpathlineto{\pgfqpoint{1.349591in}{1.195811in}}%
\pgfpathlineto{\pgfqpoint{1.314138in}{1.195811in}}%
\pgfpathlineto{\pgfqpoint{1.314138in}{1.152503in}}%
\pgfpathclose%
\pgfusepath{stroke,fill}%
\end{pgfscope}%
\begin{pgfscope}%
\pgfpathrectangle{\pgfqpoint{0.150000in}{0.150000in}}{\pgfqpoint{1.700000in}{1.700000in}}%
\pgfusepath{clip}%
\pgfsetbuttcap%
\pgfsetroundjoin%
\definecolor{currentfill}{rgb}{0.400000,0.600000,0.800000}%
\pgfsetfillcolor{currentfill}%
\pgfsetlinewidth{1.003750pt}%
\definecolor{currentstroke}{rgb}{0.000000,0.266667,0.533333}%
\pgfsetstrokecolor{currentstroke}%
\pgfsetdash{}{0pt}%
\pgfpathmoveto{\pgfqpoint{1.370538in}{1.073763in}}%
\pgfpathlineto{\pgfqpoint{1.385528in}{1.073763in}}%
\pgfpathlineto{\pgfqpoint{1.385528in}{1.109196in}}%
\pgfpathlineto{\pgfqpoint{1.370538in}{1.109196in}}%
\pgfpathlineto{\pgfqpoint{1.370538in}{1.073763in}}%
\pgfpathclose%
\pgfusepath{stroke,fill}%
\end{pgfscope}%
\begin{pgfscope}%
\pgfpathrectangle{\pgfqpoint{0.150000in}{0.150000in}}{\pgfqpoint{1.700000in}{1.700000in}}%
\pgfusepath{clip}%
\pgfsetbuttcap%
\pgfsetroundjoin%
\definecolor{currentfill}{rgb}{0.400000,0.600000,0.800000}%
\pgfsetfillcolor{currentfill}%
\pgfsetlinewidth{1.003750pt}%
\definecolor{currentstroke}{rgb}{0.000000,0.266667,0.533333}%
\pgfsetstrokecolor{currentstroke}%
\pgfsetdash{}{0pt}%
\pgfpathmoveto{\pgfqpoint{1.393846in}{0.995022in}}%
\pgfpathlineto{\pgfqpoint{1.399535in}{0.995022in}}%
\pgfpathlineto{\pgfqpoint{1.399535in}{1.030455in}}%
\pgfpathlineto{\pgfqpoint{1.393846in}{1.030455in}}%
\pgfpathlineto{\pgfqpoint{1.393846in}{0.995022in}}%
\pgfpathclose%
\pgfusepath{stroke,fill}%
\end{pgfscope}%
\begin{pgfscope}%
\pgfpathrectangle{\pgfqpoint{0.150000in}{0.150000in}}{\pgfqpoint{1.700000in}{1.700000in}}%
\pgfusepath{clip}%
\pgfsetbuttcap%
\pgfsetroundjoin%
\definecolor{currentfill}{rgb}{0.400000,0.600000,0.800000}%
\pgfsetfillcolor{currentfill}%
\pgfsetlinewidth{1.003750pt}%
\definecolor{currentstroke}{rgb}{0.000000,0.266667,0.533333}%
\pgfsetstrokecolor{currentstroke}%
\pgfsetdash{}{0pt}%
\pgfpathmoveto{\pgfqpoint{1.683312in}{0.835955in}}%
\pgfpathlineto{\pgfqpoint{1.690543in}{0.835955in}}%
\pgfpathlineto{\pgfqpoint{1.690543in}{0.878544in}}%
\pgfpathlineto{\pgfqpoint{1.683312in}{0.878544in}}%
\pgfpathlineto{\pgfqpoint{1.683312in}{0.835955in}}%
\pgfpathclose%
\pgfusepath{stroke,fill}%
\end{pgfscope}%
\begin{pgfscope}%
\pgfpathrectangle{\pgfqpoint{0.150000in}{0.150000in}}{\pgfqpoint{1.700000in}{1.700000in}}%
\pgfusepath{clip}%
\pgfsetbuttcap%
\pgfsetroundjoin%
\definecolor{currentfill}{rgb}{0.400000,0.600000,0.800000}%
\pgfsetfillcolor{currentfill}%
\pgfsetlinewidth{1.003750pt}%
\definecolor{currentstroke}{rgb}{0.000000,0.266667,0.533333}%
\pgfsetstrokecolor{currentstroke}%
\pgfsetdash{}{0pt}%
\pgfpathmoveto{\pgfqpoint{1.662547in}{0.758520in}}%
\pgfpathlineto{\pgfqpoint{1.674356in}{0.758520in}}%
\pgfpathlineto{\pgfqpoint{1.674356in}{0.793365in}}%
\pgfpathlineto{\pgfqpoint{1.662547in}{0.793365in}}%
\pgfpathlineto{\pgfqpoint{1.662547in}{0.758520in}}%
\pgfpathclose%
\pgfusepath{stroke,fill}%
\end{pgfscope}%
\begin{pgfscope}%
\pgfpathrectangle{\pgfqpoint{0.150000in}{0.150000in}}{\pgfqpoint{1.700000in}{1.700000in}}%
\pgfusepath{clip}%
\pgfsetbuttcap%
\pgfsetroundjoin%
\definecolor{currentfill}{rgb}{0.400000,0.600000,0.800000}%
\pgfsetfillcolor{currentfill}%
\pgfsetlinewidth{1.003750pt}%
\definecolor{currentstroke}{rgb}{0.000000,0.266667,0.533333}%
\pgfsetstrokecolor{currentstroke}%
\pgfsetdash{}{0pt}%
\pgfpathmoveto{\pgfqpoint{1.633223in}{0.681084in}}%
\pgfpathlineto{\pgfqpoint{1.650656in}{0.681084in}}%
\pgfpathlineto{\pgfqpoint{1.650656in}{0.715930in}}%
\pgfpathlineto{\pgfqpoint{1.633223in}{0.715930in}}%
\pgfpathlineto{\pgfqpoint{1.633223in}{0.681084in}}%
\pgfpathclose%
\pgfusepath{stroke,fill}%
\end{pgfscope}%
\begin{pgfscope}%
\pgfpathrectangle{\pgfqpoint{0.150000in}{0.150000in}}{\pgfqpoint{1.700000in}{1.700000in}}%
\pgfusepath{clip}%
\pgfsetbuttcap%
\pgfsetroundjoin%
\definecolor{currentfill}{rgb}{0.400000,0.600000,0.800000}%
\pgfsetfillcolor{currentfill}%
\pgfsetlinewidth{1.003750pt}%
\definecolor{currentstroke}{rgb}{0.000000,0.266667,0.533333}%
\pgfsetstrokecolor{currentstroke}%
\pgfsetdash{}{0pt}%
\pgfpathmoveto{\pgfqpoint{0.835968in}{1.683313in}}%
\pgfpathlineto{\pgfqpoint{0.878552in}{1.683313in}}%
\pgfpathlineto{\pgfqpoint{0.878552in}{1.690543in}}%
\pgfpathlineto{\pgfqpoint{0.835968in}{1.690543in}}%
\pgfpathlineto{\pgfqpoint{0.835968in}{1.683313in}}%
\pgfpathclose%
\pgfusepath{stroke,fill}%
\end{pgfscope}%
\begin{pgfscope}%
\pgfpathrectangle{\pgfqpoint{0.150000in}{0.150000in}}{\pgfqpoint{1.700000in}{1.700000in}}%
\pgfusepath{clip}%
\pgfsetbuttcap%
\pgfsetroundjoin%
\definecolor{currentfill}{rgb}{0.400000,0.600000,0.800000}%
\pgfsetfillcolor{currentfill}%
\pgfsetlinewidth{1.003750pt}%
\definecolor{currentstroke}{rgb}{0.000000,0.266667,0.533333}%
\pgfsetstrokecolor{currentstroke}%
\pgfsetdash{}{0pt}%
\pgfpathmoveto{\pgfqpoint{0.758544in}{1.662553in}}%
\pgfpathlineto{\pgfqpoint{0.793385in}{1.662553in}}%
\pgfpathlineto{\pgfqpoint{0.793385in}{1.674359in}}%
\pgfpathlineto{\pgfqpoint{0.758544in}{1.674359in}}%
\pgfpathlineto{\pgfqpoint{0.758544in}{1.662553in}}%
\pgfpathclose%
\pgfusepath{stroke,fill}%
\end{pgfscope}%
\begin{pgfscope}%
\pgfpathrectangle{\pgfqpoint{0.150000in}{0.150000in}}{\pgfqpoint{1.700000in}{1.700000in}}%
\pgfusepath{clip}%
\pgfsetbuttcap%
\pgfsetroundjoin%
\definecolor{currentfill}{rgb}{0.400000,0.600000,0.800000}%
\pgfsetfillcolor{currentfill}%
\pgfsetlinewidth{1.003750pt}%
\definecolor{currentstroke}{rgb}{0.000000,0.266667,0.533333}%
\pgfsetstrokecolor{currentstroke}%
\pgfsetdash{}{0pt}%
\pgfpathmoveto{\pgfqpoint{0.681120in}{1.633237in}}%
\pgfpathlineto{\pgfqpoint{0.715961in}{1.633237in}}%
\pgfpathlineto{\pgfqpoint{0.715961in}{1.650666in}}%
\pgfpathlineto{\pgfqpoint{0.681120in}{1.650666in}}%
\pgfpathlineto{\pgfqpoint{0.681120in}{1.633237in}}%
\pgfpathclose%
\pgfusepath{stroke,fill}%
\end{pgfscope}%
\begin{pgfscope}%
\pgfpathrectangle{\pgfqpoint{0.150000in}{0.150000in}}{\pgfqpoint{1.700000in}{1.700000in}}%
\pgfusepath{clip}%
\pgfsetbuttcap%
\pgfsetroundjoin%
\definecolor{currentfill}{rgb}{0.400000,0.600000,0.800000}%
\pgfsetfillcolor{currentfill}%
\pgfsetlinewidth{1.003750pt}%
\definecolor{currentstroke}{rgb}{0.000000,0.266667,0.533333}%
\pgfsetstrokecolor{currentstroke}%
\pgfsetdash{}{0pt}%
\pgfpathmoveto{\pgfqpoint{0.889799in}{0.614758in}}%
\pgfpathlineto{\pgfqpoint{0.930598in}{0.614758in}}%
\pgfpathlineto{\pgfqpoint{0.930598in}{0.625915in}}%
\pgfpathlineto{\pgfqpoint{0.889799in}{0.625915in}}%
\pgfpathlineto{\pgfqpoint{0.889799in}{0.614758in}}%
\pgfpathclose%
\pgfusepath{stroke,fill}%
\end{pgfscope}%
\begin{pgfscope}%
\pgfpathrectangle{\pgfqpoint{0.150000in}{0.150000in}}{\pgfqpoint{1.700000in}{1.700000in}}%
\pgfusepath{clip}%
\pgfsetbuttcap%
\pgfsetroundjoin%
\definecolor{currentfill}{rgb}{0.400000,0.600000,0.800000}%
\pgfsetfillcolor{currentfill}%
\pgfsetlinewidth{1.003750pt}%
\definecolor{currentstroke}{rgb}{0.000000,0.266667,0.533333}%
\pgfsetstrokecolor{currentstroke}%
\pgfsetdash{}{0pt}%
\pgfpathmoveto{\pgfqpoint{1.466315in}{1.514020in}}%
\pgfpathlineto{\pgfqpoint{1.549661in}{1.514020in}}%
\pgfpathlineto{\pgfqpoint{1.549661in}{1.568475in}}%
\pgfpathlineto{\pgfqpoint{1.466315in}{1.568475in}}%
\pgfpathlineto{\pgfqpoint{1.466315in}{1.514020in}}%
\pgfpathclose%
\pgfusepath{stroke,fill}%
\end{pgfscope}%
\begin{pgfscope}%
\pgfpathrectangle{\pgfqpoint{0.150000in}{0.150000in}}{\pgfqpoint{1.700000in}{1.700000in}}%
\pgfusepath{clip}%
\pgfsetbuttcap%
\pgfsetroundjoin%
\definecolor{currentfill}{rgb}{0.400000,0.600000,0.800000}%
\pgfsetfillcolor{currentfill}%
\pgfsetlinewidth{1.003750pt}%
\definecolor{currentstroke}{rgb}{0.000000,0.266667,0.533333}%
\pgfsetstrokecolor{currentstroke}%
\pgfsetdash{}{0pt}%
\pgfpathmoveto{\pgfqpoint{1.329932in}{1.610583in}}%
\pgfpathlineto{\pgfqpoint{1.398124in}{1.610583in}}%
\pgfpathlineto{\pgfqpoint{1.398124in}{1.637585in}}%
\pgfpathlineto{\pgfqpoint{1.329932in}{1.637585in}}%
\pgfpathlineto{\pgfqpoint{1.329932in}{1.610583in}}%
\pgfpathclose%
\pgfusepath{stroke,fill}%
\end{pgfscope}%
\begin{pgfscope}%
\pgfpathrectangle{\pgfqpoint{0.150000in}{0.150000in}}{\pgfqpoint{1.700000in}{1.700000in}}%
\pgfusepath{clip}%
\pgfsetbuttcap%
\pgfsetroundjoin%
\definecolor{currentfill}{rgb}{0.400000,0.600000,0.800000}%
\pgfsetfillcolor{currentfill}%
\pgfsetlinewidth{1.003750pt}%
\definecolor{currentstroke}{rgb}{0.000000,0.266667,0.533333}%
\pgfsetstrokecolor{currentstroke}%
\pgfsetdash{}{0pt}%
\pgfpathmoveto{\pgfqpoint{1.611900in}{1.327483in}}%
\pgfpathlineto{\pgfqpoint{1.647915in}{1.327483in}}%
\pgfpathlineto{\pgfqpoint{1.647915in}{1.423721in}}%
\pgfpathlineto{\pgfqpoint{1.611900in}{1.423721in}}%
\pgfpathlineto{\pgfqpoint{1.611900in}{1.327483in}}%
\pgfpathclose%
\pgfusepath{stroke,fill}%
\end{pgfscope}%
\begin{pgfscope}%
\pgfpathrectangle{\pgfqpoint{0.150000in}{0.150000in}}{\pgfqpoint{1.700000in}{1.700000in}}%
\pgfusepath{clip}%
\pgfsetbuttcap%
\pgfsetroundjoin%
\definecolor{currentfill}{rgb}{0.400000,0.600000,0.800000}%
\pgfsetfillcolor{currentfill}%
\pgfsetlinewidth{1.003750pt}%
\definecolor{currentstroke}{rgb}{0.000000,0.266667,0.533333}%
\pgfsetstrokecolor{currentstroke}%
\pgfsetdash{}{0pt}%
\pgfpathmoveto{\pgfqpoint{1.677059in}{1.152503in}}%
\pgfpathlineto{\pgfqpoint{1.690091in}{1.152503in}}%
\pgfpathlineto{\pgfqpoint{1.690091in}{1.248742in}}%
\pgfpathlineto{\pgfqpoint{1.677059in}{1.248742in}}%
\pgfpathlineto{\pgfqpoint{1.677059in}{1.152503in}}%
\pgfpathclose%
\pgfusepath{stroke,fill}%
\end{pgfscope}%
\begin{pgfscope}%
\pgfpathrectangle{\pgfqpoint{0.150000in}{0.150000in}}{\pgfqpoint{1.700000in}{1.700000in}}%
\pgfusepath{clip}%
\pgfsetbuttcap%
\pgfsetroundjoin%
\definecolor{currentfill}{rgb}{0.400000,0.600000,0.800000}%
\pgfsetfillcolor{currentfill}%
\pgfsetlinewidth{1.003750pt}%
\definecolor{currentstroke}{rgb}{0.000000,0.266667,0.533333}%
\pgfsetstrokecolor{currentstroke}%
\pgfsetdash{}{0pt}%
\pgfpathmoveto{\pgfqpoint{1.694004in}{0.930598in}}%
\pgfpathlineto{\pgfqpoint{1.694022in}{0.930598in}}%
\pgfpathlineto{\pgfqpoint{1.694022in}{0.995022in}}%
\pgfpathlineto{\pgfqpoint{1.694004in}{0.995022in}}%
\pgfpathlineto{\pgfqpoint{1.694004in}{0.930598in}}%
\pgfpathclose%
\pgfusepath{stroke,fill}%
\end{pgfscope}%
\begin{pgfscope}%
\pgfpathrectangle{\pgfqpoint{0.150000in}{0.150000in}}{\pgfqpoint{1.700000in}{1.700000in}}%
\pgfusepath{clip}%
\pgfsetbuttcap%
\pgfsetroundjoin%
\definecolor{currentfill}{rgb}{0.400000,0.600000,0.800000}%
\pgfsetfillcolor{currentfill}%
\pgfsetlinewidth{1.003750pt}%
\definecolor{currentstroke}{rgb}{0.000000,0.266667,0.533333}%
\pgfsetstrokecolor{currentstroke}%
\pgfsetdash{}{0pt}%
\pgfpathmoveto{\pgfqpoint{1.314138in}{1.073763in}}%
\pgfpathlineto{\pgfqpoint{1.370538in}{1.073763in}}%
\pgfpathlineto{\pgfqpoint{1.370538in}{1.152503in}}%
\pgfpathlineto{\pgfqpoint{1.314138in}{1.152503in}}%
\pgfpathlineto{\pgfqpoint{1.314138in}{1.073763in}}%
\pgfpathclose%
\pgfusepath{stroke,fill}%
\end{pgfscope}%
\begin{pgfscope}%
\pgfpathrectangle{\pgfqpoint{0.150000in}{0.150000in}}{\pgfqpoint{1.700000in}{1.700000in}}%
\pgfusepath{clip}%
\pgfsetbuttcap%
\pgfsetroundjoin%
\definecolor{currentfill}{rgb}{0.400000,0.600000,0.800000}%
\pgfsetfillcolor{currentfill}%
\pgfsetlinewidth{1.003750pt}%
\definecolor{currentstroke}{rgb}{0.000000,0.266667,0.533333}%
\pgfsetstrokecolor{currentstroke}%
\pgfsetdash{}{0pt}%
\pgfpathmoveto{\pgfqpoint{1.393846in}{0.930598in}}%
\pgfpathlineto{\pgfqpoint{1.394638in}{0.930598in}}%
\pgfpathlineto{\pgfqpoint{1.394638in}{0.995022in}}%
\pgfpathlineto{\pgfqpoint{1.393846in}{0.995022in}}%
\pgfpathlineto{\pgfqpoint{1.393846in}{0.930598in}}%
\pgfpathclose%
\pgfusepath{stroke,fill}%
\end{pgfscope}%
\begin{pgfscope}%
\pgfpathrectangle{\pgfqpoint{0.150000in}{0.150000in}}{\pgfqpoint{1.700000in}{1.700000in}}%
\pgfusepath{clip}%
\pgfsetbuttcap%
\pgfsetroundjoin%
\definecolor{currentfill}{rgb}{0.400000,0.600000,0.800000}%
\pgfsetfillcolor{currentfill}%
\pgfsetlinewidth{1.003750pt}%
\definecolor{currentstroke}{rgb}{0.000000,0.266667,0.533333}%
\pgfsetstrokecolor{currentstroke}%
\pgfsetdash{}{0pt}%
\pgfpathmoveto{\pgfqpoint{1.216982in}{1.659231in}}%
\pgfpathlineto{\pgfqpoint{1.274139in}{1.659231in}}%
\pgfpathlineto{\pgfqpoint{1.274139in}{1.672824in}}%
\pgfpathlineto{\pgfqpoint{1.216982in}{1.672824in}}%
\pgfpathlineto{\pgfqpoint{1.216982in}{1.659231in}}%
\pgfpathclose%
\pgfusepath{stroke,fill}%
\end{pgfscope}%
\begin{pgfscope}%
\pgfpathrectangle{\pgfqpoint{0.150000in}{0.150000in}}{\pgfqpoint{1.700000in}{1.700000in}}%
\pgfusepath{clip}%
\pgfsetbuttcap%
\pgfsetroundjoin%
\definecolor{currentfill}{rgb}{0.400000,0.600000,0.800000}%
\pgfsetfillcolor{currentfill}%
\pgfsetlinewidth{1.003750pt}%
\definecolor{currentstroke}{rgb}{0.000000,0.266667,0.533333}%
\pgfsetstrokecolor{currentstroke}%
\pgfsetdash{}{0pt}%
\pgfpathmoveto{\pgfqpoint{1.123453in}{1.682954in}}%
\pgfpathlineto{\pgfqpoint{1.170218in}{1.682954in}}%
\pgfpathlineto{\pgfqpoint{1.170218in}{1.688774in}}%
\pgfpathlineto{\pgfqpoint{1.123453in}{1.688774in}}%
\pgfpathlineto{\pgfqpoint{1.123453in}{1.682954in}}%
\pgfpathclose%
\pgfusepath{stroke,fill}%
\end{pgfscope}%
\begin{pgfscope}%
\pgfpathrectangle{\pgfqpoint{0.150000in}{0.150000in}}{\pgfqpoint{1.700000in}{1.700000in}}%
\pgfusepath{clip}%
\pgfsetbuttcap%
\pgfsetroundjoin%
\definecolor{currentfill}{rgb}{0.400000,0.600000,0.800000}%
\pgfsetfillcolor{currentfill}%
\pgfsetlinewidth{1.003750pt}%
\definecolor{currentstroke}{rgb}{0.000000,0.266667,0.533333}%
\pgfsetstrokecolor{currentstroke}%
\pgfsetdash{}{0pt}%
\pgfpathmoveto{\pgfqpoint{1.038427in}{1.692957in}}%
\pgfpathlineto{\pgfqpoint{1.085191in}{1.692957in}}%
\pgfpathlineto{\pgfqpoint{1.085191in}{1.694022in}}%
\pgfpathlineto{\pgfqpoint{1.038427in}{1.694022in}}%
\pgfpathlineto{\pgfqpoint{1.038427in}{1.692957in}}%
\pgfpathclose%
\pgfusepath{stroke,fill}%
\end{pgfscope}%
\begin{pgfscope}%
\pgfpathrectangle{\pgfqpoint{0.150000in}{0.150000in}}{\pgfqpoint{1.700000in}{1.700000in}}%
\pgfusepath{clip}%
\pgfsetbuttcap%
\pgfsetroundjoin%
\definecolor{currentfill}{rgb}{0.400000,0.600000,0.800000}%
\pgfsetfillcolor{currentfill}%
\pgfsetlinewidth{1.003750pt}%
\definecolor{currentstroke}{rgb}{0.000000,0.266667,0.533333}%
\pgfsetstrokecolor{currentstroke}%
\pgfsetdash{}{0pt}%
\pgfpathmoveto{\pgfqpoint{1.170218in}{1.292239in}}%
\pgfpathlineto{\pgfqpoint{1.216982in}{1.292239in}}%
\pgfpathlineto{\pgfqpoint{1.216982in}{1.336859in}}%
\pgfpathlineto{\pgfqpoint{1.170218in}{1.336859in}}%
\pgfpathlineto{\pgfqpoint{1.170218in}{1.292239in}}%
\pgfpathclose%
\pgfusepath{stroke,fill}%
\end{pgfscope}%
\begin{pgfscope}%
\pgfpathrectangle{\pgfqpoint{0.150000in}{0.150000in}}{\pgfqpoint{1.700000in}{1.700000in}}%
\pgfusepath{clip}%
\pgfsetbuttcap%
\pgfsetroundjoin%
\definecolor{currentfill}{rgb}{0.400000,0.600000,0.800000}%
\pgfsetfillcolor{currentfill}%
\pgfsetlinewidth{1.003750pt}%
\definecolor{currentstroke}{rgb}{0.000000,0.266667,0.533333}%
\pgfsetstrokecolor{currentstroke}%
\pgfsetdash{}{0pt}%
\pgfpathmoveto{\pgfqpoint{1.085191in}{1.362742in}}%
\pgfpathlineto{\pgfqpoint{1.123453in}{1.362742in}}%
\pgfpathlineto{\pgfqpoint{1.123453in}{1.381202in}}%
\pgfpathlineto{\pgfqpoint{1.085191in}{1.381202in}}%
\pgfpathlineto{\pgfqpoint{1.085191in}{1.362742in}}%
\pgfpathclose%
\pgfusepath{stroke,fill}%
\end{pgfscope}%
\begin{pgfscope}%
\pgfpathrectangle{\pgfqpoint{0.150000in}{0.150000in}}{\pgfqpoint{1.700000in}{1.700000in}}%
\pgfusepath{clip}%
\pgfsetbuttcap%
\pgfsetroundjoin%
\definecolor{currentfill}{rgb}{0.400000,0.600000,0.800000}%
\pgfsetfillcolor{currentfill}%
\pgfsetlinewidth{1.003750pt}%
\definecolor{currentstroke}{rgb}{0.000000,0.266667,0.533333}%
\pgfsetstrokecolor{currentstroke}%
\pgfsetdash{}{0pt}%
\pgfpathmoveto{\pgfqpoint{1.000165in}{1.391533in}}%
\pgfpathlineto{\pgfqpoint{1.038427in}{1.391533in}}%
\pgfpathlineto{\pgfqpoint{1.038427in}{1.398847in}}%
\pgfpathlineto{\pgfqpoint{1.000165in}{1.398847in}}%
\pgfpathlineto{\pgfqpoint{1.000165in}{1.391533in}}%
\pgfpathclose%
\pgfusepath{stroke,fill}%
\end{pgfscope}%
\begin{pgfscope}%
\pgfpathrectangle{\pgfqpoint{0.150000in}{0.150000in}}{\pgfqpoint{1.700000in}{1.700000in}}%
\pgfusepath{clip}%
\pgfsetbuttcap%
\pgfsetroundjoin%
\definecolor{currentfill}{rgb}{0.400000,0.600000,0.800000}%
\pgfsetfillcolor{currentfill}%
\pgfsetlinewidth{1.003750pt}%
\definecolor{currentstroke}{rgb}{0.000000,0.266667,0.533333}%
\pgfsetstrokecolor{currentstroke}%
\pgfsetdash{}{0pt}%
\pgfpathmoveto{\pgfqpoint{1.674356in}{0.758520in}}%
\pgfpathlineto{\pgfqpoint{1.690543in}{0.758520in}}%
\pgfpathlineto{\pgfqpoint{1.690543in}{0.835955in}}%
\pgfpathlineto{\pgfqpoint{1.674356in}{0.835955in}}%
\pgfpathlineto{\pgfqpoint{1.674356in}{0.758520in}}%
\pgfpathclose%
\pgfusepath{stroke,fill}%
\end{pgfscope}%
\begin{pgfscope}%
\pgfpathrectangle{\pgfqpoint{0.150000in}{0.150000in}}{\pgfqpoint{1.700000in}{1.700000in}}%
\pgfusepath{clip}%
\pgfsetbuttcap%
\pgfsetroundjoin%
\definecolor{currentfill}{rgb}{0.400000,0.600000,0.800000}%
\pgfsetfillcolor{currentfill}%
\pgfsetlinewidth{1.003750pt}%
\definecolor{currentstroke}{rgb}{0.000000,0.266667,0.533333}%
\pgfsetstrokecolor{currentstroke}%
\pgfsetdash{}{0pt}%
\pgfpathmoveto{\pgfqpoint{1.616409in}{0.617728in}}%
\pgfpathlineto{\pgfqpoint{1.650656in}{0.617728in}}%
\pgfpathlineto{\pgfqpoint{1.650656in}{0.681084in}}%
\pgfpathlineto{\pgfqpoint{1.616409in}{0.681084in}}%
\pgfpathlineto{\pgfqpoint{1.616409in}{0.617728in}}%
\pgfpathclose%
\pgfusepath{stroke,fill}%
\end{pgfscope}%
\begin{pgfscope}%
\pgfpathrectangle{\pgfqpoint{0.150000in}{0.150000in}}{\pgfqpoint{1.700000in}{1.700000in}}%
\pgfusepath{clip}%
\pgfsetbuttcap%
\pgfsetroundjoin%
\definecolor{currentfill}{rgb}{0.400000,0.600000,0.800000}%
\pgfsetfillcolor{currentfill}%
\pgfsetlinewidth{1.003750pt}%
\definecolor{currentstroke}{rgb}{0.000000,0.266667,0.533333}%
\pgfsetstrokecolor{currentstroke}%
\pgfsetdash{}{0pt}%
\pgfpathmoveto{\pgfqpoint{1.351279in}{0.862748in}}%
\pgfpathlineto{\pgfqpoint{1.376454in}{0.862748in}}%
\pgfpathlineto{\pgfqpoint{1.376454in}{0.930598in}}%
\pgfpathlineto{\pgfqpoint{1.351279in}{0.930598in}}%
\pgfpathlineto{\pgfqpoint{1.351279in}{0.862748in}}%
\pgfpathclose%
\pgfusepath{stroke,fill}%
\end{pgfscope}%
\begin{pgfscope}%
\pgfpathrectangle{\pgfqpoint{0.150000in}{0.150000in}}{\pgfqpoint{1.700000in}{1.700000in}}%
\pgfusepath{clip}%
\pgfsetbuttcap%
\pgfsetroundjoin%
\definecolor{currentfill}{rgb}{0.400000,0.600000,0.800000}%
\pgfsetfillcolor{currentfill}%
\pgfsetlinewidth{1.003750pt}%
\definecolor{currentstroke}{rgb}{0.000000,0.266667,0.533333}%
\pgfsetstrokecolor{currentstroke}%
\pgfsetdash{}{0pt}%
\pgfpathmoveto{\pgfqpoint{1.272573in}{0.751721in}}%
\pgfpathlineto{\pgfqpoint{1.314504in}{0.751721in}}%
\pgfpathlineto{\pgfqpoint{1.314504in}{0.807234in}}%
\pgfpathlineto{\pgfqpoint{1.272573in}{0.807234in}}%
\pgfpathlineto{\pgfqpoint{1.272573in}{0.751721in}}%
\pgfpathclose%
\pgfusepath{stroke,fill}%
\end{pgfscope}%
\begin{pgfscope}%
\pgfpathrectangle{\pgfqpoint{0.150000in}{0.150000in}}{\pgfqpoint{1.700000in}{1.700000in}}%
\pgfusepath{clip}%
\pgfsetbuttcap%
\pgfsetroundjoin%
\definecolor{currentfill}{rgb}{0.400000,0.600000,0.800000}%
\pgfsetfillcolor{currentfill}%
\pgfsetlinewidth{1.003750pt}%
\definecolor{currentstroke}{rgb}{0.000000,0.266667,0.533333}%
\pgfsetstrokecolor{currentstroke}%
\pgfsetdash{}{0pt}%
\pgfpathmoveto{\pgfqpoint{1.540056in}{0.520228in}}%
\pgfpathlineto{\pgfqpoint{1.579254in}{0.520228in}}%
\pgfpathlineto{\pgfqpoint{1.579254in}{0.564103in}}%
\pgfpathlineto{\pgfqpoint{1.540056in}{0.564103in}}%
\pgfpathlineto{\pgfqpoint{1.540056in}{0.520228in}}%
\pgfpathclose%
\pgfusepath{stroke,fill}%
\end{pgfscope}%
\begin{pgfscope}%
\pgfpathrectangle{\pgfqpoint{0.150000in}{0.150000in}}{\pgfqpoint{1.700000in}{1.700000in}}%
\pgfusepath{clip}%
\pgfsetbuttcap%
\pgfsetroundjoin%
\definecolor{currentfill}{rgb}{0.400000,0.600000,0.800000}%
\pgfsetfillcolor{currentfill}%
\pgfsetlinewidth{1.003750pt}%
\definecolor{currentstroke}{rgb}{0.000000,0.266667,0.533333}%
\pgfsetstrokecolor{currentstroke}%
\pgfsetdash{}{0pt}%
\pgfpathmoveto{\pgfqpoint{1.451486in}{0.440455in}}%
\pgfpathlineto{\pgfqpoint{1.501483in}{0.440455in}}%
\pgfpathlineto{\pgfqpoint{1.501483in}{0.472907in}}%
\pgfpathlineto{\pgfqpoint{1.451486in}{0.472907in}}%
\pgfpathlineto{\pgfqpoint{1.451486in}{0.440455in}}%
\pgfpathclose%
\pgfusepath{stroke,fill}%
\end{pgfscope}%
\begin{pgfscope}%
\pgfpathrectangle{\pgfqpoint{0.150000in}{0.150000in}}{\pgfqpoint{1.700000in}{1.700000in}}%
\pgfusepath{clip}%
\pgfsetbuttcap%
\pgfsetroundjoin%
\definecolor{currentfill}{rgb}{0.400000,0.600000,0.800000}%
\pgfsetfillcolor{currentfill}%
\pgfsetlinewidth{1.003750pt}%
\definecolor{currentstroke}{rgb}{0.000000,0.266667,0.533333}%
\pgfsetstrokecolor{currentstroke}%
\pgfsetdash{}{0pt}%
\pgfpathmoveto{\pgfqpoint{1.368833in}{0.392004in}}%
\pgfpathlineto{\pgfqpoint{1.410580in}{0.392004in}}%
\pgfpathlineto{\pgfqpoint{1.410580in}{0.412098in}}%
\pgfpathlineto{\pgfqpoint{1.368833in}{0.412098in}}%
\pgfpathlineto{\pgfqpoint{1.368833in}{0.392004in}}%
\pgfpathclose%
\pgfusepath{stroke,fill}%
\end{pgfscope}%
\begin{pgfscope}%
\pgfpathrectangle{\pgfqpoint{0.150000in}{0.150000in}}{\pgfqpoint{1.700000in}{1.700000in}}%
\pgfusepath{clip}%
\pgfsetbuttcap%
\pgfsetroundjoin%
\definecolor{currentfill}{rgb}{0.400000,0.600000,0.800000}%
\pgfsetfillcolor{currentfill}%
\pgfsetlinewidth{1.003750pt}%
\definecolor{currentstroke}{rgb}{0.000000,0.266667,0.533333}%
\pgfsetstrokecolor{currentstroke}%
\pgfsetdash{}{0pt}%
\pgfpathmoveto{\pgfqpoint{1.169126in}{0.662303in}}%
\pgfpathlineto{\pgfqpoint{1.215677in}{0.662303in}}%
\pgfpathlineto{\pgfqpoint{1.215677in}{0.706301in}}%
\pgfpathlineto{\pgfqpoint{1.169126in}{0.706301in}}%
\pgfpathlineto{\pgfqpoint{1.169126in}{0.662303in}}%
\pgfpathclose%
\pgfusepath{stroke,fill}%
\end{pgfscope}%
\begin{pgfscope}%
\pgfpathrectangle{\pgfqpoint{0.150000in}{0.150000in}}{\pgfqpoint{1.700000in}{1.700000in}}%
\pgfusepath{clip}%
\pgfsetbuttcap%
\pgfsetroundjoin%
\definecolor{currentfill}{rgb}{0.400000,0.600000,0.800000}%
\pgfsetfillcolor{currentfill}%
\pgfsetlinewidth{1.003750pt}%
\definecolor{currentstroke}{rgb}{0.000000,0.266667,0.533333}%
\pgfsetstrokecolor{currentstroke}%
\pgfsetdash{}{0pt}%
\pgfpathmoveto{\pgfqpoint{1.084487in}{0.618515in}}%
\pgfpathlineto{\pgfqpoint{1.122574in}{0.618515in}}%
\pgfpathlineto{\pgfqpoint{1.122574in}{0.636748in}}%
\pgfpathlineto{\pgfqpoint{1.084487in}{0.636748in}}%
\pgfpathlineto{\pgfqpoint{1.084487in}{0.618515in}}%
\pgfpathclose%
\pgfusepath{stroke,fill}%
\end{pgfscope}%
\begin{pgfscope}%
\pgfpathrectangle{\pgfqpoint{0.150000in}{0.150000in}}{\pgfqpoint{1.700000in}{1.700000in}}%
\pgfusepath{clip}%
\pgfsetbuttcap%
\pgfsetroundjoin%
\definecolor{currentfill}{rgb}{0.400000,0.600000,0.800000}%
\pgfsetfillcolor{currentfill}%
\pgfsetlinewidth{1.003750pt}%
\definecolor{currentstroke}{rgb}{0.000000,0.266667,0.533333}%
\pgfsetstrokecolor{currentstroke}%
\pgfsetdash{}{0pt}%
\pgfpathmoveto{\pgfqpoint{0.999848in}{0.601106in}}%
\pgfpathlineto{\pgfqpoint{1.037935in}{0.601106in}}%
\pgfpathlineto{\pgfqpoint{1.037935in}{0.608314in}}%
\pgfpathlineto{\pgfqpoint{0.999848in}{0.608314in}}%
\pgfpathlineto{\pgfqpoint{0.999848in}{0.601106in}}%
\pgfpathclose%
\pgfusepath{stroke,fill}%
\end{pgfscope}%
\begin{pgfscope}%
\pgfpathrectangle{\pgfqpoint{0.150000in}{0.150000in}}{\pgfqpoint{1.700000in}{1.700000in}}%
\pgfusepath{clip}%
\pgfsetbuttcap%
\pgfsetroundjoin%
\definecolor{currentfill}{rgb}{0.400000,0.600000,0.800000}%
\pgfsetfillcolor{currentfill}%
\pgfsetlinewidth{1.003750pt}%
\definecolor{currentstroke}{rgb}{0.000000,0.266667,0.533333}%
\pgfsetstrokecolor{currentstroke}%
\pgfsetdash{}{0pt}%
\pgfpathmoveto{\pgfqpoint{1.215677in}{0.326900in}}%
\pgfpathlineto{\pgfqpoint{1.272573in}{0.326900in}}%
\pgfpathlineto{\pgfqpoint{1.272573in}{0.340341in}}%
\pgfpathlineto{\pgfqpoint{1.215677in}{0.340341in}}%
\pgfpathlineto{\pgfqpoint{1.215677in}{0.326900in}}%
\pgfpathclose%
\pgfusepath{stroke,fill}%
\end{pgfscope}%
\begin{pgfscope}%
\pgfpathrectangle{\pgfqpoint{0.150000in}{0.150000in}}{\pgfqpoint{1.700000in}{1.700000in}}%
\pgfusepath{clip}%
\pgfsetbuttcap%
\pgfsetroundjoin%
\definecolor{currentfill}{rgb}{0.400000,0.600000,0.800000}%
\pgfsetfillcolor{currentfill}%
\pgfsetlinewidth{1.003750pt}%
\definecolor{currentstroke}{rgb}{0.000000,0.266667,0.533333}%
\pgfsetstrokecolor{currentstroke}%
\pgfsetdash{}{0pt}%
\pgfpathmoveto{\pgfqpoint{1.122574in}{0.311140in}}%
\pgfpathlineto{\pgfqpoint{1.169126in}{0.311140in}}%
\pgfpathlineto{\pgfqpoint{1.169126in}{0.316888in}}%
\pgfpathlineto{\pgfqpoint{1.122574in}{0.316888in}}%
\pgfpathlineto{\pgfqpoint{1.122574in}{0.311140in}}%
\pgfpathclose%
\pgfusepath{stroke,fill}%
\end{pgfscope}%
\begin{pgfscope}%
\pgfpathrectangle{\pgfqpoint{0.150000in}{0.150000in}}{\pgfqpoint{1.700000in}{1.700000in}}%
\pgfusepath{clip}%
\pgfsetbuttcap%
\pgfsetroundjoin%
\definecolor{currentfill}{rgb}{0.400000,0.600000,0.800000}%
\pgfsetfillcolor{currentfill}%
\pgfsetlinewidth{1.003750pt}%
\definecolor{currentstroke}{rgb}{0.000000,0.266667,0.533333}%
\pgfsetstrokecolor{currentstroke}%
\pgfsetdash{}{0pt}%
\pgfpathmoveto{\pgfqpoint{1.037935in}{0.305978in}}%
\pgfpathlineto{\pgfqpoint{1.084487in}{0.305978in}}%
\pgfpathlineto{\pgfqpoint{1.084487in}{0.307015in}}%
\pgfpathlineto{\pgfqpoint{1.037935in}{0.307015in}}%
\pgfpathlineto{\pgfqpoint{1.037935in}{0.305978in}}%
\pgfpathclose%
\pgfusepath{stroke,fill}%
\end{pgfscope}%
\begin{pgfscope}%
\pgfpathrectangle{\pgfqpoint{0.150000in}{0.150000in}}{\pgfqpoint{1.700000in}{1.700000in}}%
\pgfusepath{clip}%
\pgfsetbuttcap%
\pgfsetroundjoin%
\definecolor{currentfill}{rgb}{0.400000,0.600000,0.800000}%
\pgfsetfillcolor{currentfill}%
\pgfsetlinewidth{1.003750pt}%
\definecolor{currentstroke}{rgb}{0.000000,0.266667,0.533333}%
\pgfsetstrokecolor{currentstroke}%
\pgfsetdash{}{0pt}%
\pgfpathmoveto{\pgfqpoint{0.758544in}{1.674359in}}%
\pgfpathlineto{\pgfqpoint{0.835968in}{1.674359in}}%
\pgfpathlineto{\pgfqpoint{0.835968in}{1.690543in}}%
\pgfpathlineto{\pgfqpoint{0.758544in}{1.690543in}}%
\pgfpathlineto{\pgfqpoint{0.758544in}{1.674359in}}%
\pgfpathclose%
\pgfusepath{stroke,fill}%
\end{pgfscope}%
\begin{pgfscope}%
\pgfpathrectangle{\pgfqpoint{0.150000in}{0.150000in}}{\pgfqpoint{1.700000in}{1.700000in}}%
\pgfusepath{clip}%
\pgfsetbuttcap%
\pgfsetroundjoin%
\definecolor{currentfill}{rgb}{0.400000,0.600000,0.800000}%
\pgfsetfillcolor{currentfill}%
\pgfsetlinewidth{1.003750pt}%
\definecolor{currentstroke}{rgb}{0.000000,0.266667,0.533333}%
\pgfsetstrokecolor{currentstroke}%
\pgfsetdash{}{0pt}%
\pgfpathmoveto{\pgfqpoint{0.617773in}{1.616427in}}%
\pgfpathlineto{\pgfqpoint{0.681120in}{1.616427in}}%
\pgfpathlineto{\pgfqpoint{0.681120in}{1.650666in}}%
\pgfpathlineto{\pgfqpoint{0.617773in}{1.650666in}}%
\pgfpathlineto{\pgfqpoint{0.617773in}{1.616427in}}%
\pgfpathclose%
\pgfusepath{stroke,fill}%
\end{pgfscope}%
\begin{pgfscope}%
\pgfpathrectangle{\pgfqpoint{0.150000in}{0.150000in}}{\pgfqpoint{1.700000in}{1.700000in}}%
\pgfusepath{clip}%
\pgfsetbuttcap%
\pgfsetroundjoin%
\definecolor{currentfill}{rgb}{0.400000,0.600000,0.800000}%
\pgfsetfillcolor{currentfill}%
\pgfsetlinewidth{1.003750pt}%
\definecolor{currentstroke}{rgb}{0.000000,0.266667,0.533333}%
\pgfsetstrokecolor{currentstroke}%
\pgfsetdash{}{0pt}%
\pgfpathmoveto{\pgfqpoint{0.862802in}{1.351333in}}%
\pgfpathlineto{\pgfqpoint{0.930598in}{1.351333in}}%
\pgfpathlineto{\pgfqpoint{0.930598in}{1.376473in}}%
\pgfpathlineto{\pgfqpoint{0.862802in}{1.376473in}}%
\pgfpathlineto{\pgfqpoint{0.862802in}{1.351333in}}%
\pgfpathclose%
\pgfusepath{stroke,fill}%
\end{pgfscope}%
\begin{pgfscope}%
\pgfpathrectangle{\pgfqpoint{0.150000in}{0.150000in}}{\pgfqpoint{1.700000in}{1.700000in}}%
\pgfusepath{clip}%
\pgfsetbuttcap%
\pgfsetroundjoin%
\definecolor{currentfill}{rgb}{0.400000,0.600000,0.800000}%
\pgfsetfillcolor{currentfill}%
\pgfsetlinewidth{1.003750pt}%
\definecolor{currentstroke}{rgb}{0.000000,0.266667,0.533333}%
\pgfsetstrokecolor{currentstroke}%
\pgfsetdash{}{0pt}%
\pgfpathmoveto{\pgfqpoint{0.751862in}{1.272765in}}%
\pgfpathlineto{\pgfqpoint{0.807332in}{1.272765in}}%
\pgfpathlineto{\pgfqpoint{0.807332in}{1.314616in}}%
\pgfpathlineto{\pgfqpoint{0.751862in}{1.314616in}}%
\pgfpathlineto{\pgfqpoint{0.751862in}{1.272765in}}%
\pgfpathclose%
\pgfusepath{stroke,fill}%
\end{pgfscope}%
\begin{pgfscope}%
\pgfpathrectangle{\pgfqpoint{0.150000in}{0.150000in}}{\pgfqpoint{1.700000in}{1.700000in}}%
\pgfusepath{clip}%
\pgfsetbuttcap%
\pgfsetroundjoin%
\definecolor{currentfill}{rgb}{0.400000,0.600000,0.800000}%
\pgfsetfillcolor{currentfill}%
\pgfsetlinewidth{1.003750pt}%
\definecolor{currentstroke}{rgb}{0.000000,0.266667,0.533333}%
\pgfsetstrokecolor{currentstroke}%
\pgfsetdash{}{0pt}%
\pgfpathmoveto{\pgfqpoint{0.520296in}{1.540103in}}%
\pgfpathlineto{\pgfqpoint{0.564161in}{1.540103in}}%
\pgfpathlineto{\pgfqpoint{0.564161in}{1.579284in}}%
\pgfpathlineto{\pgfqpoint{0.520296in}{1.579284in}}%
\pgfpathlineto{\pgfqpoint{0.520296in}{1.540103in}}%
\pgfpathclose%
\pgfusepath{stroke,fill}%
\end{pgfscope}%
\begin{pgfscope}%
\pgfpathrectangle{\pgfqpoint{0.150000in}{0.150000in}}{\pgfqpoint{1.700000in}{1.700000in}}%
\pgfusepath{clip}%
\pgfsetbuttcap%
\pgfsetroundjoin%
\definecolor{currentfill}{rgb}{0.400000,0.600000,0.800000}%
\pgfsetfillcolor{currentfill}%
\pgfsetlinewidth{1.003750pt}%
\definecolor{currentstroke}{rgb}{0.000000,0.266667,0.533333}%
\pgfsetstrokecolor{currentstroke}%
\pgfsetdash{}{0pt}%
\pgfpathmoveto{\pgfqpoint{0.440542in}{1.451581in}}%
\pgfpathlineto{\pgfqpoint{0.472988in}{1.451581in}}%
\pgfpathlineto{\pgfqpoint{0.472988in}{1.501548in}}%
\pgfpathlineto{\pgfqpoint{0.440542in}{1.501548in}}%
\pgfpathlineto{\pgfqpoint{0.440542in}{1.451581in}}%
\pgfpathclose%
\pgfusepath{stroke,fill}%
\end{pgfscope}%
\begin{pgfscope}%
\pgfpathrectangle{\pgfqpoint{0.150000in}{0.150000in}}{\pgfqpoint{1.700000in}{1.700000in}}%
\pgfusepath{clip}%
\pgfsetbuttcap%
\pgfsetroundjoin%
\definecolor{currentfill}{rgb}{0.400000,0.600000,0.800000}%
\pgfsetfillcolor{currentfill}%
\pgfsetlinewidth{1.003750pt}%
\definecolor{currentstroke}{rgb}{0.000000,0.266667,0.533333}%
\pgfsetstrokecolor{currentstroke}%
\pgfsetdash{}{0pt}%
\pgfpathmoveto{\pgfqpoint{0.392092in}{1.368973in}}%
\pgfpathlineto{\pgfqpoint{0.412186in}{1.368973in}}%
\pgfpathlineto{\pgfqpoint{0.412186in}{1.410698in}}%
\pgfpathlineto{\pgfqpoint{0.392092in}{1.410698in}}%
\pgfpathlineto{\pgfqpoint{0.392092in}{1.368973in}}%
\pgfpathclose%
\pgfusepath{stroke,fill}%
\end{pgfscope}%
\begin{pgfscope}%
\pgfpathrectangle{\pgfqpoint{0.150000in}{0.150000in}}{\pgfqpoint{1.700000in}{1.700000in}}%
\pgfusepath{clip}%
\pgfsetbuttcap%
\pgfsetroundjoin%
\definecolor{currentfill}{rgb}{0.400000,0.600000,0.800000}%
\pgfsetfillcolor{currentfill}%
\pgfsetlinewidth{1.003750pt}%
\definecolor{currentstroke}{rgb}{0.000000,0.266667,0.533333}%
\pgfsetstrokecolor{currentstroke}%
\pgfsetdash{}{0pt}%
\pgfpathmoveto{\pgfqpoint{0.662442in}{1.169364in}}%
\pgfpathlineto{\pgfqpoint{0.706478in}{1.169364in}}%
\pgfpathlineto{\pgfqpoint{0.706478in}{1.215894in}}%
\pgfpathlineto{\pgfqpoint{0.662442in}{1.215894in}}%
\pgfpathlineto{\pgfqpoint{0.662442in}{1.169364in}}%
\pgfpathclose%
\pgfusepath{stroke,fill}%
\end{pgfscope}%
\begin{pgfscope}%
\pgfpathrectangle{\pgfqpoint{0.150000in}{0.150000in}}{\pgfqpoint{1.700000in}{1.700000in}}%
\pgfusepath{clip}%
\pgfsetbuttcap%
\pgfsetroundjoin%
\definecolor{currentfill}{rgb}{0.400000,0.600000,0.800000}%
\pgfsetfillcolor{currentfill}%
\pgfsetlinewidth{1.003750pt}%
\definecolor{currentstroke}{rgb}{0.000000,0.266667,0.533333}%
\pgfsetstrokecolor{currentstroke}%
\pgfsetdash{}{0pt}%
\pgfpathmoveto{\pgfqpoint{0.618598in}{1.084764in}}%
\pgfpathlineto{\pgfqpoint{0.636859in}{1.084764in}}%
\pgfpathlineto{\pgfqpoint{0.636859in}{1.122834in}}%
\pgfpathlineto{\pgfqpoint{0.618598in}{1.122834in}}%
\pgfpathlineto{\pgfqpoint{0.618598in}{1.084764in}}%
\pgfpathclose%
\pgfusepath{stroke,fill}%
\end{pgfscope}%
\begin{pgfscope}%
\pgfpathrectangle{\pgfqpoint{0.150000in}{0.150000in}}{\pgfqpoint{1.700000in}{1.700000in}}%
\pgfusepath{clip}%
\pgfsetbuttcap%
\pgfsetroundjoin%
\definecolor{currentfill}{rgb}{0.400000,0.600000,0.800000}%
\pgfsetfillcolor{currentfill}%
\pgfsetlinewidth{1.003750pt}%
\definecolor{currentstroke}{rgb}{0.000000,0.266667,0.533333}%
\pgfsetstrokecolor{currentstroke}%
\pgfsetdash{}{0pt}%
\pgfpathmoveto{\pgfqpoint{0.601134in}{1.000164in}}%
\pgfpathlineto{\pgfqpoint{0.608374in}{1.000164in}}%
\pgfpathlineto{\pgfqpoint{0.608374in}{1.038234in}}%
\pgfpathlineto{\pgfqpoint{0.601134in}{1.038234in}}%
\pgfpathlineto{\pgfqpoint{0.601134in}{1.000164in}}%
\pgfpathclose%
\pgfusepath{stroke,fill}%
\end{pgfscope}%
\begin{pgfscope}%
\pgfpathrectangle{\pgfqpoint{0.150000in}{0.150000in}}{\pgfqpoint{1.700000in}{1.700000in}}%
\pgfusepath{clip}%
\pgfsetbuttcap%
\pgfsetroundjoin%
\definecolor{currentfill}{rgb}{0.400000,0.600000,0.800000}%
\pgfsetfillcolor{currentfill}%
\pgfsetlinewidth{1.003750pt}%
\definecolor{currentstroke}{rgb}{0.000000,0.266667,0.533333}%
\pgfsetstrokecolor{currentstroke}%
\pgfsetdash{}{0pt}%
\pgfpathmoveto{\pgfqpoint{0.326960in}{1.215894in}}%
\pgfpathlineto{\pgfqpoint{0.340412in}{1.215894in}}%
\pgfpathlineto{\pgfqpoint{0.340412in}{1.272765in}}%
\pgfpathlineto{\pgfqpoint{0.326960in}{1.272765in}}%
\pgfpathlineto{\pgfqpoint{0.326960in}{1.215894in}}%
\pgfpathclose%
\pgfusepath{stroke,fill}%
\end{pgfscope}%
\begin{pgfscope}%
\pgfpathrectangle{\pgfqpoint{0.150000in}{0.150000in}}{\pgfqpoint{1.700000in}{1.700000in}}%
\pgfusepath{clip}%
\pgfsetbuttcap%
\pgfsetroundjoin%
\definecolor{currentfill}{rgb}{0.400000,0.600000,0.800000}%
\pgfsetfillcolor{currentfill}%
\pgfsetlinewidth{1.003750pt}%
\definecolor{currentstroke}{rgb}{0.000000,0.266667,0.533333}%
\pgfsetstrokecolor{currentstroke}%
\pgfsetdash{}{0pt}%
\pgfpathmoveto{\pgfqpoint{0.311174in}{1.122834in}}%
\pgfpathlineto{\pgfqpoint{0.316935in}{1.122834in}}%
\pgfpathlineto{\pgfqpoint{0.316935in}{1.169364in}}%
\pgfpathlineto{\pgfqpoint{0.311174in}{1.169364in}}%
\pgfpathlineto{\pgfqpoint{0.311174in}{1.122834in}}%
\pgfpathclose%
\pgfusepath{stroke,fill}%
\end{pgfscope}%
\begin{pgfscope}%
\pgfpathrectangle{\pgfqpoint{0.150000in}{0.150000in}}{\pgfqpoint{1.700000in}{1.700000in}}%
\pgfusepath{clip}%
\pgfsetbuttcap%
\pgfsetroundjoin%
\definecolor{currentfill}{rgb}{0.400000,0.600000,0.800000}%
\pgfsetfillcolor{currentfill}%
\pgfsetlinewidth{1.003750pt}%
\definecolor{currentstroke}{rgb}{0.000000,0.266667,0.533333}%
\pgfsetstrokecolor{currentstroke}%
\pgfsetdash{}{0pt}%
\pgfpathmoveto{\pgfqpoint{0.305978in}{1.038234in}}%
\pgfpathlineto{\pgfqpoint{0.307032in}{1.038234in}}%
\pgfpathlineto{\pgfqpoint{0.307032in}{1.084764in}}%
\pgfpathlineto{\pgfqpoint{0.305978in}{1.084764in}}%
\pgfpathlineto{\pgfqpoint{0.305978in}{1.038234in}}%
\pgfpathclose%
\pgfusepath{stroke,fill}%
\end{pgfscope}%
\begin{pgfscope}%
\pgfpathrectangle{\pgfqpoint{0.150000in}{0.150000in}}{\pgfqpoint{1.700000in}{1.700000in}}%
\pgfusepath{clip}%
\pgfsetbuttcap%
\pgfsetroundjoin%
\definecolor{currentfill}{rgb}{0.400000,0.600000,0.800000}%
\pgfsetfillcolor{currentfill}%
\pgfsetlinewidth{1.003750pt}%
\definecolor{currentstroke}{rgb}{0.000000,0.266667,0.533333}%
\pgfsetstrokecolor{currentstroke}%
\pgfsetdash{}{0pt}%
\pgfpathmoveto{\pgfqpoint{0.618724in}{0.876777in}}%
\pgfpathlineto{\pgfqpoint{0.636015in}{0.876777in}}%
\pgfpathlineto{\pgfqpoint{0.636015in}{0.930946in}}%
\pgfpathlineto{\pgfqpoint{0.618724in}{0.930946in}}%
\pgfpathlineto{\pgfqpoint{0.618724in}{0.876777in}}%
\pgfpathclose%
\pgfusepath{stroke,fill}%
\end{pgfscope}%
\begin{pgfscope}%
\pgfpathrectangle{\pgfqpoint{0.150000in}{0.150000in}}{\pgfqpoint{1.700000in}{1.700000in}}%
\pgfusepath{clip}%
\pgfsetbuttcap%
\pgfsetroundjoin%
\definecolor{currentfill}{rgb}{0.400000,0.600000,0.800000}%
\pgfsetfillcolor{currentfill}%
\pgfsetlinewidth{1.003750pt}%
\definecolor{currentstroke}{rgb}{0.000000,0.266667,0.533333}%
\pgfsetstrokecolor{currentstroke}%
\pgfsetdash{}{0pt}%
\pgfpathmoveto{\pgfqpoint{0.659898in}{0.788137in}}%
\pgfpathlineto{\pgfqpoint{0.685374in}{0.788137in}}%
\pgfpathlineto{\pgfqpoint{0.685374in}{0.832457in}}%
\pgfpathlineto{\pgfqpoint{0.659898in}{0.832457in}}%
\pgfpathlineto{\pgfqpoint{0.659898in}{0.788137in}}%
\pgfpathclose%
\pgfusepath{stroke,fill}%
\end{pgfscope}%
\begin{pgfscope}%
\pgfpathrectangle{\pgfqpoint{0.150000in}{0.150000in}}{\pgfqpoint{1.700000in}{1.700000in}}%
\pgfusepath{clip}%
\pgfsetbuttcap%
\pgfsetroundjoin%
\definecolor{currentfill}{rgb}{0.400000,0.600000,0.800000}%
\pgfsetfillcolor{currentfill}%
\pgfsetlinewidth{1.003750pt}%
\definecolor{currentstroke}{rgb}{0.000000,0.266667,0.533333}%
\pgfsetstrokecolor{currentstroke}%
\pgfsetdash{}{0pt}%
\pgfpathmoveto{\pgfqpoint{0.856418in}{0.625915in}}%
\pgfpathlineto{\pgfqpoint{0.930598in}{0.625915in}}%
\pgfpathlineto{\pgfqpoint{0.930598in}{0.655287in}}%
\pgfpathlineto{\pgfqpoint{0.856418in}{0.655287in}}%
\pgfpathlineto{\pgfqpoint{0.856418in}{0.625915in}}%
\pgfpathclose%
\pgfusepath{stroke,fill}%
\end{pgfscope}%
\begin{pgfscope}%
\pgfpathrectangle{\pgfqpoint{0.150000in}{0.150000in}}{\pgfqpoint{1.700000in}{1.700000in}}%
\pgfusepath{clip}%
\pgfsetbuttcap%
\pgfsetroundjoin%
\definecolor{currentfill}{rgb}{0.400000,0.600000,0.800000}%
\pgfsetfillcolor{currentfill}%
\pgfsetlinewidth{1.003750pt}%
\definecolor{currentstroke}{rgb}{0.000000,0.266667,0.533333}%
\pgfsetstrokecolor{currentstroke}%
\pgfsetdash{}{0pt}%
\pgfpathmoveto{\pgfqpoint{0.735032in}{0.699421in}}%
\pgfpathlineto{\pgfqpoint{0.795725in}{0.699421in}}%
\pgfpathlineto{\pgfqpoint{0.795725in}{0.751875in}}%
\pgfpathlineto{\pgfqpoint{0.735032in}{0.751875in}}%
\pgfpathlineto{\pgfqpoint{0.735032in}{0.699421in}}%
\pgfpathclose%
\pgfusepath{stroke,fill}%
\end{pgfscope}%
\begin{pgfscope}%
\pgfpathrectangle{\pgfqpoint{0.150000in}{0.150000in}}{\pgfqpoint{1.700000in}{1.700000in}}%
\pgfusepath{clip}%
\pgfsetbuttcap%
\pgfsetroundjoin%
\definecolor{currentfill}{rgb}{0.400000,0.600000,0.800000}%
\pgfsetfillcolor{currentfill}%
\pgfsetlinewidth{1.003750pt}%
\definecolor{currentstroke}{rgb}{0.000000,0.266667,0.533333}%
\pgfsetstrokecolor{currentstroke}%
\pgfsetdash{}{0pt}%
\pgfpathmoveto{\pgfqpoint{0.309422in}{0.827545in}}%
\pgfpathlineto{\pgfqpoint{0.317497in}{0.827545in}}%
\pgfpathlineto{\pgfqpoint{0.317497in}{0.874076in}}%
\pgfpathlineto{\pgfqpoint{0.309422in}{0.874076in}}%
\pgfpathlineto{\pgfqpoint{0.309422in}{0.827545in}}%
\pgfpathclose%
\pgfusepath{stroke,fill}%
\end{pgfscope}%
\begin{pgfscope}%
\pgfpathrectangle{\pgfqpoint{0.150000in}{0.150000in}}{\pgfqpoint{1.700000in}{1.700000in}}%
\pgfusepath{clip}%
\pgfsetbuttcap%
\pgfsetroundjoin%
\definecolor{currentfill}{rgb}{0.400000,0.600000,0.800000}%
\pgfsetfillcolor{currentfill}%
\pgfsetlinewidth{1.003750pt}%
\definecolor{currentstroke}{rgb}{0.000000,0.266667,0.533333}%
\pgfsetstrokecolor{currentstroke}%
\pgfsetdash{}{0pt}%
\pgfpathmoveto{\pgfqpoint{0.327746in}{0.742945in}}%
\pgfpathlineto{\pgfqpoint{0.341432in}{0.742945in}}%
\pgfpathlineto{\pgfqpoint{0.341432in}{0.781015in}}%
\pgfpathlineto{\pgfqpoint{0.327746in}{0.781015in}}%
\pgfpathlineto{\pgfqpoint{0.327746in}{0.742945in}}%
\pgfpathclose%
\pgfusepath{stroke,fill}%
\end{pgfscope}%
\begin{pgfscope}%
\pgfpathrectangle{\pgfqpoint{0.150000in}{0.150000in}}{\pgfqpoint{1.700000in}{1.700000in}}%
\pgfusepath{clip}%
\pgfsetbuttcap%
\pgfsetroundjoin%
\definecolor{currentfill}{rgb}{0.400000,0.600000,0.800000}%
\pgfsetfillcolor{currentfill}%
\pgfsetlinewidth{1.003750pt}%
\definecolor{currentstroke}{rgb}{0.000000,0.266667,0.533333}%
\pgfsetstrokecolor{currentstroke}%
\pgfsetdash{}{0pt}%
\pgfpathmoveto{\pgfqpoint{0.355338in}{0.658345in}}%
\pgfpathlineto{\pgfqpoint{0.375898in}{0.658345in}}%
\pgfpathlineto{\pgfqpoint{0.375898in}{0.696415in}}%
\pgfpathlineto{\pgfqpoint{0.355338in}{0.696415in}}%
\pgfpathlineto{\pgfqpoint{0.355338in}{0.658345in}}%
\pgfpathclose%
\pgfusepath{stroke,fill}%
\end{pgfscope}%
\begin{pgfscope}%
\pgfpathrectangle{\pgfqpoint{0.150000in}{0.150000in}}{\pgfqpoint{1.700000in}{1.700000in}}%
\pgfusepath{clip}%
\pgfsetbuttcap%
\pgfsetroundjoin%
\definecolor{currentfill}{rgb}{0.400000,0.600000,0.800000}%
\pgfsetfillcolor{currentfill}%
\pgfsetlinewidth{1.003750pt}%
\definecolor{currentstroke}{rgb}{0.000000,0.266667,0.533333}%
\pgfsetstrokecolor{currentstroke}%
\pgfsetdash{}{0pt}%
\pgfpathmoveto{\pgfqpoint{0.849086in}{0.309457in}}%
\pgfpathlineto{\pgfqpoint{0.885766in}{0.309457in}}%
\pgfpathlineto{\pgfqpoint{0.885766in}{0.315444in}}%
\pgfpathlineto{\pgfqpoint{0.849086in}{0.315444in}}%
\pgfpathlineto{\pgfqpoint{0.849086in}{0.309457in}}%
\pgfpathclose%
\pgfusepath{stroke,fill}%
\end{pgfscope}%
\begin{pgfscope}%
\pgfpathrectangle{\pgfqpoint{0.150000in}{0.150000in}}{\pgfqpoint{1.700000in}{1.700000in}}%
\pgfusepath{clip}%
\pgfsetbuttcap%
\pgfsetroundjoin%
\definecolor{currentfill}{rgb}{0.400000,0.600000,0.800000}%
\pgfsetfillcolor{currentfill}%
\pgfsetlinewidth{1.003750pt}%
\definecolor{currentstroke}{rgb}{0.000000,0.266667,0.533333}%
\pgfsetstrokecolor{currentstroke}%
\pgfsetdash{}{0pt}%
\pgfpathmoveto{\pgfqpoint{1.398124in}{1.568475in}}%
\pgfpathlineto{\pgfqpoint{1.549661in}{1.568475in}}%
\pgfpathlineto{\pgfqpoint{1.549661in}{1.637585in}}%
\pgfpathlineto{\pgfqpoint{1.398124in}{1.637585in}}%
\pgfpathlineto{\pgfqpoint{1.398124in}{1.568475in}}%
\pgfpathclose%
\pgfusepath{stroke,fill}%
\end{pgfscope}%
\begin{pgfscope}%
\pgfpathrectangle{\pgfqpoint{0.150000in}{0.150000in}}{\pgfqpoint{1.700000in}{1.700000in}}%
\pgfusepath{clip}%
\pgfsetbuttcap%
\pgfsetroundjoin%
\definecolor{currentfill}{rgb}{0.400000,0.600000,0.800000}%
\pgfsetfillcolor{currentfill}%
\pgfsetlinewidth{1.003750pt}%
\definecolor{currentstroke}{rgb}{0.000000,0.266667,0.533333}%
\pgfsetstrokecolor{currentstroke}%
\pgfsetdash{}{0pt}%
\pgfpathmoveto{\pgfqpoint{1.690091in}{1.073763in}}%
\pgfpathlineto{\pgfqpoint{1.694022in}{1.073763in}}%
\pgfpathlineto{\pgfqpoint{1.694022in}{1.248742in}}%
\pgfpathlineto{\pgfqpoint{1.690091in}{1.248742in}}%
\pgfpathlineto{\pgfqpoint{1.690091in}{1.073763in}}%
\pgfpathclose%
\pgfusepath{stroke,fill}%
\end{pgfscope}%
\begin{pgfscope}%
\pgfpathrectangle{\pgfqpoint{0.150000in}{0.150000in}}{\pgfqpoint{1.700000in}{1.700000in}}%
\pgfusepath{clip}%
\pgfsetbuttcap%
\pgfsetroundjoin%
\definecolor{currentfill}{rgb}{0.400000,0.600000,0.800000}%
\pgfsetfillcolor{currentfill}%
\pgfsetlinewidth{1.003750pt}%
\definecolor{currentstroke}{rgb}{0.000000,0.266667,0.533333}%
\pgfsetstrokecolor{currentstroke}%
\pgfsetdash{}{0pt}%
\pgfpathmoveto{\pgfqpoint{1.314138in}{0.930598in}}%
\pgfpathlineto{\pgfqpoint{1.393846in}{0.930598in}}%
\pgfpathlineto{\pgfqpoint{1.393846in}{1.073763in}}%
\pgfpathlineto{\pgfqpoint{1.314138in}{1.073763in}}%
\pgfpathlineto{\pgfqpoint{1.314138in}{0.930598in}}%
\pgfpathclose%
\pgfusepath{stroke,fill}%
\end{pgfscope}%
\begin{pgfscope}%
\pgfpathrectangle{\pgfqpoint{0.150000in}{0.150000in}}{\pgfqpoint{1.700000in}{1.700000in}}%
\pgfusepath{clip}%
\pgfsetbuttcap%
\pgfsetroundjoin%
\definecolor{currentfill}{rgb}{0.400000,0.600000,0.800000}%
\pgfsetfillcolor{currentfill}%
\pgfsetlinewidth{1.003750pt}%
\definecolor{currentstroke}{rgb}{0.000000,0.266667,0.533333}%
\pgfsetstrokecolor{currentstroke}%
\pgfsetdash{}{0pt}%
\pgfpathmoveto{\pgfqpoint{1.170218in}{1.672824in}}%
\pgfpathlineto{\pgfqpoint{1.274139in}{1.672824in}}%
\pgfpathlineto{\pgfqpoint{1.274139in}{1.688774in}}%
\pgfpathlineto{\pgfqpoint{1.170218in}{1.688774in}}%
\pgfpathlineto{\pgfqpoint{1.170218in}{1.672824in}}%
\pgfpathclose%
\pgfusepath{stroke,fill}%
\end{pgfscope}%
\begin{pgfscope}%
\pgfpathrectangle{\pgfqpoint{0.150000in}{0.150000in}}{\pgfqpoint{1.700000in}{1.700000in}}%
\pgfusepath{clip}%
\pgfsetbuttcap%
\pgfsetroundjoin%
\definecolor{currentfill}{rgb}{0.400000,0.600000,0.800000}%
\pgfsetfillcolor{currentfill}%
\pgfsetlinewidth{1.003750pt}%
\definecolor{currentstroke}{rgb}{0.000000,0.266667,0.533333}%
\pgfsetstrokecolor{currentstroke}%
\pgfsetdash{}{0pt}%
\pgfpathmoveto{\pgfqpoint{1.000165in}{1.694022in}}%
\pgfpathlineto{\pgfqpoint{1.085191in}{1.694022in}}%
\pgfpathlineto{\pgfqpoint{1.085191in}{1.694022in}}%
\pgfpathlineto{\pgfqpoint{1.000165in}{1.694022in}}%
\pgfpathlineto{\pgfqpoint{1.000165in}{1.694022in}}%
\pgfpathclose%
\pgfusepath{stroke,fill}%
\end{pgfscope}%
\begin{pgfscope}%
\pgfpathrectangle{\pgfqpoint{0.150000in}{0.150000in}}{\pgfqpoint{1.700000in}{1.700000in}}%
\pgfusepath{clip}%
\pgfsetbuttcap%
\pgfsetroundjoin%
\definecolor{currentfill}{rgb}{0.400000,0.600000,0.800000}%
\pgfsetfillcolor{currentfill}%
\pgfsetlinewidth{1.003750pt}%
\definecolor{currentstroke}{rgb}{0.000000,0.266667,0.533333}%
\pgfsetstrokecolor{currentstroke}%
\pgfsetdash{}{0pt}%
\pgfpathmoveto{\pgfqpoint{1.085191in}{1.292239in}}%
\pgfpathlineto{\pgfqpoint{1.170218in}{1.292239in}}%
\pgfpathlineto{\pgfqpoint{1.170218in}{1.362742in}}%
\pgfpathlineto{\pgfqpoint{1.085191in}{1.362742in}}%
\pgfpathlineto{\pgfqpoint{1.085191in}{1.292239in}}%
\pgfpathclose%
\pgfusepath{stroke,fill}%
\end{pgfscope}%
\begin{pgfscope}%
\pgfpathrectangle{\pgfqpoint{0.150000in}{0.150000in}}{\pgfqpoint{1.700000in}{1.700000in}}%
\pgfusepath{clip}%
\pgfsetbuttcap%
\pgfsetroundjoin%
\definecolor{currentfill}{rgb}{0.400000,0.600000,0.800000}%
\pgfsetfillcolor{currentfill}%
\pgfsetlinewidth{1.003750pt}%
\definecolor{currentstroke}{rgb}{0.000000,0.266667,0.533333}%
\pgfsetstrokecolor{currentstroke}%
\pgfsetdash{}{0pt}%
\pgfpathmoveto{\pgfqpoint{0.930598in}{1.391533in}}%
\pgfpathlineto{\pgfqpoint{1.000165in}{1.391533in}}%
\pgfpathlineto{\pgfqpoint{1.000165in}{1.394638in}}%
\pgfpathlineto{\pgfqpoint{0.930598in}{1.394638in}}%
\pgfpathlineto{\pgfqpoint{0.930598in}{1.391533in}}%
\pgfpathclose%
\pgfusepath{stroke,fill}%
\end{pgfscope}%
\begin{pgfscope}%
\pgfpathrectangle{\pgfqpoint{0.150000in}{0.150000in}}{\pgfqpoint{1.700000in}{1.700000in}}%
\pgfusepath{clip}%
\pgfsetbuttcap%
\pgfsetroundjoin%
\definecolor{currentfill}{rgb}{0.400000,0.600000,0.800000}%
\pgfsetfillcolor{currentfill}%
\pgfsetlinewidth{1.003750pt}%
\definecolor{currentstroke}{rgb}{0.000000,0.266667,0.533333}%
\pgfsetstrokecolor{currentstroke}%
\pgfsetdash{}{0pt}%
\pgfpathmoveto{\pgfqpoint{1.650656in}{0.617728in}}%
\pgfpathlineto{\pgfqpoint{1.690543in}{0.617728in}}%
\pgfpathlineto{\pgfqpoint{1.690543in}{0.758520in}}%
\pgfpathlineto{\pgfqpoint{1.650656in}{0.758520in}}%
\pgfpathlineto{\pgfqpoint{1.650656in}{0.617728in}}%
\pgfpathclose%
\pgfusepath{stroke,fill}%
\end{pgfscope}%
\begin{pgfscope}%
\pgfpathrectangle{\pgfqpoint{0.150000in}{0.150000in}}{\pgfqpoint{1.700000in}{1.700000in}}%
\pgfusepath{clip}%
\pgfsetbuttcap%
\pgfsetroundjoin%
\definecolor{currentfill}{rgb}{0.400000,0.600000,0.800000}%
\pgfsetfillcolor{currentfill}%
\pgfsetlinewidth{1.003750pt}%
\definecolor{currentstroke}{rgb}{0.000000,0.266667,0.533333}%
\pgfsetstrokecolor{currentstroke}%
\pgfsetdash{}{0pt}%
\pgfpathmoveto{\pgfqpoint{1.272573in}{0.807234in}}%
\pgfpathlineto{\pgfqpoint{1.351279in}{0.807234in}}%
\pgfpathlineto{\pgfqpoint{1.351279in}{0.930598in}}%
\pgfpathlineto{\pgfqpoint{1.272573in}{0.930598in}}%
\pgfpathlineto{\pgfqpoint{1.272573in}{0.807234in}}%
\pgfpathclose%
\pgfusepath{stroke,fill}%
\end{pgfscope}%
\begin{pgfscope}%
\pgfpathrectangle{\pgfqpoint{0.150000in}{0.150000in}}{\pgfqpoint{1.700000in}{1.700000in}}%
\pgfusepath{clip}%
\pgfsetbuttcap%
\pgfsetroundjoin%
\definecolor{currentfill}{rgb}{0.400000,0.600000,0.800000}%
\pgfsetfillcolor{currentfill}%
\pgfsetlinewidth{1.003750pt}%
\definecolor{currentstroke}{rgb}{0.000000,0.266667,0.533333}%
\pgfsetstrokecolor{currentstroke}%
\pgfsetdash{}{0pt}%
\pgfpathmoveto{\pgfqpoint{1.501483in}{0.440455in}}%
\pgfpathlineto{\pgfqpoint{1.579254in}{0.440455in}}%
\pgfpathlineto{\pgfqpoint{1.579254in}{0.520228in}}%
\pgfpathlineto{\pgfqpoint{1.501483in}{0.520228in}}%
\pgfpathlineto{\pgfqpoint{1.501483in}{0.440455in}}%
\pgfpathclose%
\pgfusepath{stroke,fill}%
\end{pgfscope}%
\begin{pgfscope}%
\pgfpathrectangle{\pgfqpoint{0.150000in}{0.150000in}}{\pgfqpoint{1.700000in}{1.700000in}}%
\pgfusepath{clip}%
\pgfsetbuttcap%
\pgfsetroundjoin%
\definecolor{currentfill}{rgb}{0.400000,0.600000,0.800000}%
\pgfsetfillcolor{currentfill}%
\pgfsetlinewidth{1.003750pt}%
\definecolor{currentstroke}{rgb}{0.000000,0.266667,0.533333}%
\pgfsetstrokecolor{currentstroke}%
\pgfsetdash{}{0pt}%
\pgfpathmoveto{\pgfqpoint{1.334676in}{0.361744in}}%
\pgfpathlineto{\pgfqpoint{1.410580in}{0.361744in}}%
\pgfpathlineto{\pgfqpoint{1.410580in}{0.392004in}}%
\pgfpathlineto{\pgfqpoint{1.334676in}{0.392004in}}%
\pgfpathlineto{\pgfqpoint{1.334676in}{0.361744in}}%
\pgfpathclose%
\pgfusepath{stroke,fill}%
\end{pgfscope}%
\begin{pgfscope}%
\pgfpathrectangle{\pgfqpoint{0.150000in}{0.150000in}}{\pgfqpoint{1.700000in}{1.700000in}}%
\pgfusepath{clip}%
\pgfsetbuttcap%
\pgfsetroundjoin%
\definecolor{currentfill}{rgb}{0.400000,0.600000,0.800000}%
\pgfsetfillcolor{currentfill}%
\pgfsetlinewidth{1.003750pt}%
\definecolor{currentstroke}{rgb}{0.000000,0.266667,0.533333}%
\pgfsetstrokecolor{currentstroke}%
\pgfsetdash{}{0pt}%
\pgfpathmoveto{\pgfqpoint{1.084487in}{0.636748in}}%
\pgfpathlineto{\pgfqpoint{1.169126in}{0.636748in}}%
\pgfpathlineto{\pgfqpoint{1.169126in}{0.706301in}}%
\pgfpathlineto{\pgfqpoint{1.084487in}{0.706301in}}%
\pgfpathlineto{\pgfqpoint{1.084487in}{0.636748in}}%
\pgfpathclose%
\pgfusepath{stroke,fill}%
\end{pgfscope}%
\begin{pgfscope}%
\pgfpathrectangle{\pgfqpoint{0.150000in}{0.150000in}}{\pgfqpoint{1.700000in}{1.700000in}}%
\pgfusepath{clip}%
\pgfsetbuttcap%
\pgfsetroundjoin%
\definecolor{currentfill}{rgb}{0.400000,0.600000,0.800000}%
\pgfsetfillcolor{currentfill}%
\pgfsetlinewidth{1.003750pt}%
\definecolor{currentstroke}{rgb}{0.000000,0.266667,0.533333}%
\pgfsetstrokecolor{currentstroke}%
\pgfsetdash{}{0pt}%
\pgfpathmoveto{\pgfqpoint{0.930598in}{0.605362in}}%
\pgfpathlineto{\pgfqpoint{0.999848in}{0.605362in}}%
\pgfpathlineto{\pgfqpoint{0.999848in}{0.608314in}}%
\pgfpathlineto{\pgfqpoint{0.930598in}{0.608314in}}%
\pgfpathlineto{\pgfqpoint{0.930598in}{0.605362in}}%
\pgfpathclose%
\pgfusepath{stroke,fill}%
\end{pgfscope}%
\begin{pgfscope}%
\pgfpathrectangle{\pgfqpoint{0.150000in}{0.150000in}}{\pgfqpoint{1.700000in}{1.700000in}}%
\pgfusepath{clip}%
\pgfsetbuttcap%
\pgfsetroundjoin%
\definecolor{currentfill}{rgb}{0.400000,0.600000,0.800000}%
\pgfsetfillcolor{currentfill}%
\pgfsetlinewidth{1.003750pt}%
\definecolor{currentstroke}{rgb}{0.000000,0.266667,0.533333}%
\pgfsetstrokecolor{currentstroke}%
\pgfsetdash{}{0pt}%
\pgfpathmoveto{\pgfqpoint{1.169126in}{0.311140in}}%
\pgfpathlineto{\pgfqpoint{1.272573in}{0.311140in}}%
\pgfpathlineto{\pgfqpoint{1.272573in}{0.326900in}}%
\pgfpathlineto{\pgfqpoint{1.169126in}{0.326900in}}%
\pgfpathlineto{\pgfqpoint{1.169126in}{0.311140in}}%
\pgfpathclose%
\pgfusepath{stroke,fill}%
\end{pgfscope}%
\begin{pgfscope}%
\pgfpathrectangle{\pgfqpoint{0.150000in}{0.150000in}}{\pgfqpoint{1.700000in}{1.700000in}}%
\pgfusepath{clip}%
\pgfsetbuttcap%
\pgfsetroundjoin%
\definecolor{currentfill}{rgb}{0.400000,0.600000,0.800000}%
\pgfsetfillcolor{currentfill}%
\pgfsetlinewidth{1.003750pt}%
\definecolor{currentstroke}{rgb}{0.000000,0.266667,0.533333}%
\pgfsetstrokecolor{currentstroke}%
\pgfsetdash{}{0pt}%
\pgfpathmoveto{\pgfqpoint{0.930598in}{0.305978in}}%
\pgfpathlineto{\pgfqpoint{0.999848in}{0.305978in}}%
\pgfpathlineto{\pgfqpoint{0.999848in}{0.305978in}}%
\pgfpathlineto{\pgfqpoint{0.930598in}{0.305978in}}%
\pgfpathlineto{\pgfqpoint{0.930598in}{0.305978in}}%
\pgfpathclose%
\pgfusepath{stroke,fill}%
\end{pgfscope}%
\begin{pgfscope}%
\pgfpathrectangle{\pgfqpoint{0.150000in}{0.150000in}}{\pgfqpoint{1.700000in}{1.700000in}}%
\pgfusepath{clip}%
\pgfsetbuttcap%
\pgfsetroundjoin%
\definecolor{currentfill}{rgb}{0.400000,0.600000,0.800000}%
\pgfsetfillcolor{currentfill}%
\pgfsetlinewidth{1.003750pt}%
\definecolor{currentstroke}{rgb}{0.000000,0.266667,0.533333}%
\pgfsetstrokecolor{currentstroke}%
\pgfsetdash{}{0pt}%
\pgfpathmoveto{\pgfqpoint{0.617773in}{1.650666in}}%
\pgfpathlineto{\pgfqpoint{0.758544in}{1.650666in}}%
\pgfpathlineto{\pgfqpoint{0.758544in}{1.690543in}}%
\pgfpathlineto{\pgfqpoint{0.617773in}{1.690543in}}%
\pgfpathlineto{\pgfqpoint{0.617773in}{1.650666in}}%
\pgfpathclose%
\pgfusepath{stroke,fill}%
\end{pgfscope}%
\begin{pgfscope}%
\pgfpathrectangle{\pgfqpoint{0.150000in}{0.150000in}}{\pgfqpoint{1.700000in}{1.700000in}}%
\pgfusepath{clip}%
\pgfsetbuttcap%
\pgfsetroundjoin%
\definecolor{currentfill}{rgb}{0.400000,0.600000,0.800000}%
\pgfsetfillcolor{currentfill}%
\pgfsetlinewidth{1.003750pt}%
\definecolor{currentstroke}{rgb}{0.000000,0.266667,0.533333}%
\pgfsetstrokecolor{currentstroke}%
\pgfsetdash{}{0pt}%
\pgfpathmoveto{\pgfqpoint{0.807332in}{1.272765in}}%
\pgfpathlineto{\pgfqpoint{0.930598in}{1.272765in}}%
\pgfpathlineto{\pgfqpoint{0.930598in}{1.351333in}}%
\pgfpathlineto{\pgfqpoint{0.807332in}{1.351333in}}%
\pgfpathlineto{\pgfqpoint{0.807332in}{1.272765in}}%
\pgfpathclose%
\pgfusepath{stroke,fill}%
\end{pgfscope}%
\begin{pgfscope}%
\pgfpathrectangle{\pgfqpoint{0.150000in}{0.150000in}}{\pgfqpoint{1.700000in}{1.700000in}}%
\pgfusepath{clip}%
\pgfsetbuttcap%
\pgfsetroundjoin%
\definecolor{currentfill}{rgb}{0.400000,0.600000,0.800000}%
\pgfsetfillcolor{currentfill}%
\pgfsetlinewidth{1.003750pt}%
\definecolor{currentstroke}{rgb}{0.000000,0.266667,0.533333}%
\pgfsetstrokecolor{currentstroke}%
\pgfsetdash{}{0pt}%
\pgfpathmoveto{\pgfqpoint{0.440542in}{1.501548in}}%
\pgfpathlineto{\pgfqpoint{0.520296in}{1.501548in}}%
\pgfpathlineto{\pgfqpoint{0.520296in}{1.579284in}}%
\pgfpathlineto{\pgfqpoint{0.440542in}{1.579284in}}%
\pgfpathlineto{\pgfqpoint{0.440542in}{1.501548in}}%
\pgfpathclose%
\pgfusepath{stroke,fill}%
\end{pgfscope}%
\begin{pgfscope}%
\pgfpathrectangle{\pgfqpoint{0.150000in}{0.150000in}}{\pgfqpoint{1.700000in}{1.700000in}}%
\pgfusepath{clip}%
\pgfsetbuttcap%
\pgfsetroundjoin%
\definecolor{currentfill}{rgb}{0.400000,0.600000,0.800000}%
\pgfsetfillcolor{currentfill}%
\pgfsetlinewidth{1.003750pt}%
\definecolor{currentstroke}{rgb}{0.000000,0.266667,0.533333}%
\pgfsetstrokecolor{currentstroke}%
\pgfsetdash{}{0pt}%
\pgfpathmoveto{\pgfqpoint{0.361826in}{1.334835in}}%
\pgfpathlineto{\pgfqpoint{0.392092in}{1.334835in}}%
\pgfpathlineto{\pgfqpoint{0.392092in}{1.410698in}}%
\pgfpathlineto{\pgfqpoint{0.361826in}{1.410698in}}%
\pgfpathlineto{\pgfqpoint{0.361826in}{1.334835in}}%
\pgfpathclose%
\pgfusepath{stroke,fill}%
\end{pgfscope}%
\begin{pgfscope}%
\pgfpathrectangle{\pgfqpoint{0.150000in}{0.150000in}}{\pgfqpoint{1.700000in}{1.700000in}}%
\pgfusepath{clip}%
\pgfsetbuttcap%
\pgfsetroundjoin%
\definecolor{currentfill}{rgb}{0.400000,0.600000,0.800000}%
\pgfsetfillcolor{currentfill}%
\pgfsetlinewidth{1.003750pt}%
\definecolor{currentstroke}{rgb}{0.000000,0.266667,0.533333}%
\pgfsetstrokecolor{currentstroke}%
\pgfsetdash{}{0pt}%
\pgfpathmoveto{\pgfqpoint{0.636859in}{1.084764in}}%
\pgfpathlineto{\pgfqpoint{0.706478in}{1.084764in}}%
\pgfpathlineto{\pgfqpoint{0.706478in}{1.169364in}}%
\pgfpathlineto{\pgfqpoint{0.636859in}{1.169364in}}%
\pgfpathlineto{\pgfqpoint{0.636859in}{1.084764in}}%
\pgfpathclose%
\pgfusepath{stroke,fill}%
\end{pgfscope}%
\begin{pgfscope}%
\pgfpathrectangle{\pgfqpoint{0.150000in}{0.150000in}}{\pgfqpoint{1.700000in}{1.700000in}}%
\pgfusepath{clip}%
\pgfsetbuttcap%
\pgfsetroundjoin%
\definecolor{currentfill}{rgb}{0.400000,0.600000,0.800000}%
\pgfsetfillcolor{currentfill}%
\pgfsetlinewidth{1.003750pt}%
\definecolor{currentstroke}{rgb}{0.000000,0.266667,0.533333}%
\pgfsetstrokecolor{currentstroke}%
\pgfsetdash{}{0pt}%
\pgfpathmoveto{\pgfqpoint{0.605301in}{0.930946in}}%
\pgfpathlineto{\pgfqpoint{0.608374in}{0.930946in}}%
\pgfpathlineto{\pgfqpoint{0.608374in}{1.000164in}}%
\pgfpathlineto{\pgfqpoint{0.605301in}{1.000164in}}%
\pgfpathlineto{\pgfqpoint{0.605301in}{0.930946in}}%
\pgfpathclose%
\pgfusepath{stroke,fill}%
\end{pgfscope}%
\begin{pgfscope}%
\pgfpathrectangle{\pgfqpoint{0.150000in}{0.150000in}}{\pgfqpoint{1.700000in}{1.700000in}}%
\pgfusepath{clip}%
\pgfsetbuttcap%
\pgfsetroundjoin%
\definecolor{currentfill}{rgb}{0.400000,0.600000,0.800000}%
\pgfsetfillcolor{currentfill}%
\pgfsetlinewidth{1.003750pt}%
\definecolor{currentstroke}{rgb}{0.000000,0.266667,0.533333}%
\pgfsetstrokecolor{currentstroke}%
\pgfsetdash{}{0pt}%
\pgfpathmoveto{\pgfqpoint{0.311174in}{1.169364in}}%
\pgfpathlineto{\pgfqpoint{0.326960in}{1.169364in}}%
\pgfpathlineto{\pgfqpoint{0.326960in}{1.272765in}}%
\pgfpathlineto{\pgfqpoint{0.311174in}{1.272765in}}%
\pgfpathlineto{\pgfqpoint{0.311174in}{1.169364in}}%
\pgfpathclose%
\pgfusepath{stroke,fill}%
\end{pgfscope}%
\begin{pgfscope}%
\pgfpathrectangle{\pgfqpoint{0.150000in}{0.150000in}}{\pgfqpoint{1.700000in}{1.700000in}}%
\pgfusepath{clip}%
\pgfsetbuttcap%
\pgfsetroundjoin%
\definecolor{currentfill}{rgb}{0.400000,0.600000,0.800000}%
\pgfsetfillcolor{currentfill}%
\pgfsetlinewidth{1.003750pt}%
\definecolor{currentstroke}{rgb}{0.000000,0.266667,0.533333}%
\pgfsetstrokecolor{currentstroke}%
\pgfsetdash{}{0pt}%
\pgfpathmoveto{\pgfqpoint{0.305978in}{1.000164in}}%
\pgfpathlineto{\pgfqpoint{0.305978in}{1.000164in}}%
\pgfpathlineto{\pgfqpoint{0.305978in}{1.084764in}}%
\pgfpathlineto{\pgfqpoint{0.305978in}{1.084764in}}%
\pgfpathlineto{\pgfqpoint{0.305978in}{1.000164in}}%
\pgfpathclose%
\pgfusepath{stroke,fill}%
\end{pgfscope}%
\begin{pgfscope}%
\pgfpathrectangle{\pgfqpoint{0.150000in}{0.150000in}}{\pgfqpoint{1.700000in}{1.700000in}}%
\pgfusepath{clip}%
\pgfsetbuttcap%
\pgfsetroundjoin%
\definecolor{currentfill}{rgb}{0.400000,0.600000,0.800000}%
\pgfsetfillcolor{currentfill}%
\pgfsetlinewidth{1.003750pt}%
\definecolor{currentstroke}{rgb}{0.000000,0.266667,0.533333}%
\pgfsetstrokecolor{currentstroke}%
\pgfsetdash{}{0pt}%
\pgfpathmoveto{\pgfqpoint{0.636015in}{0.832457in}}%
\pgfpathlineto{\pgfqpoint{0.685374in}{0.832457in}}%
\pgfpathlineto{\pgfqpoint{0.685374in}{0.930946in}}%
\pgfpathlineto{\pgfqpoint{0.636015in}{0.930946in}}%
\pgfpathlineto{\pgfqpoint{0.636015in}{0.832457in}}%
\pgfpathclose%
\pgfusepath{stroke,fill}%
\end{pgfscope}%
\begin{pgfscope}%
\pgfpathrectangle{\pgfqpoint{0.150000in}{0.150000in}}{\pgfqpoint{1.700000in}{1.700000in}}%
\pgfusepath{clip}%
\pgfsetbuttcap%
\pgfsetroundjoin%
\definecolor{currentfill}{rgb}{0.400000,0.600000,0.800000}%
\pgfsetfillcolor{currentfill}%
\pgfsetlinewidth{1.003750pt}%
\definecolor{currentstroke}{rgb}{0.000000,0.266667,0.533333}%
\pgfsetstrokecolor{currentstroke}%
\pgfsetdash{}{0pt}%
\pgfpathmoveto{\pgfqpoint{0.795725in}{0.655287in}}%
\pgfpathlineto{\pgfqpoint{0.930598in}{0.655287in}}%
\pgfpathlineto{\pgfqpoint{0.930598in}{0.751875in}}%
\pgfpathlineto{\pgfqpoint{0.795725in}{0.751875in}}%
\pgfpathlineto{\pgfqpoint{0.795725in}{0.655287in}}%
\pgfpathclose%
\pgfusepath{stroke,fill}%
\end{pgfscope}%
\begin{pgfscope}%
\pgfpathrectangle{\pgfqpoint{0.150000in}{0.150000in}}{\pgfqpoint{1.700000in}{1.700000in}}%
\pgfusepath{clip}%
\pgfsetbuttcap%
\pgfsetroundjoin%
\definecolor{currentfill}{rgb}{0.400000,0.600000,0.800000}%
\pgfsetfillcolor{currentfill}%
\pgfsetlinewidth{1.003750pt}%
\definecolor{currentstroke}{rgb}{0.000000,0.266667,0.533333}%
\pgfsetstrokecolor{currentstroke}%
\pgfsetdash{}{0pt}%
\pgfpathmoveto{\pgfqpoint{0.309422in}{0.742945in}}%
\pgfpathlineto{\pgfqpoint{0.327746in}{0.742945in}}%
\pgfpathlineto{\pgfqpoint{0.327746in}{0.827545in}}%
\pgfpathlineto{\pgfqpoint{0.309422in}{0.827545in}}%
\pgfpathlineto{\pgfqpoint{0.309422in}{0.742945in}}%
\pgfpathclose%
\pgfusepath{stroke,fill}%
\end{pgfscope}%
\begin{pgfscope}%
\pgfpathrectangle{\pgfqpoint{0.150000in}{0.150000in}}{\pgfqpoint{1.700000in}{1.700000in}}%
\pgfusepath{clip}%
\pgfsetbuttcap%
\pgfsetroundjoin%
\definecolor{currentfill}{rgb}{0.400000,0.600000,0.800000}%
\pgfsetfillcolor{currentfill}%
\pgfsetlinewidth{1.003750pt}%
\definecolor{currentstroke}{rgb}{0.000000,0.266667,0.533333}%
\pgfsetstrokecolor{currentstroke}%
\pgfsetdash{}{0pt}%
\pgfpathmoveto{\pgfqpoint{0.355338in}{0.589127in}}%
\pgfpathlineto{\pgfqpoint{0.395899in}{0.589127in}}%
\pgfpathlineto{\pgfqpoint{0.395899in}{0.658345in}}%
\pgfpathlineto{\pgfqpoint{0.355338in}{0.658345in}}%
\pgfpathlineto{\pgfqpoint{0.355338in}{0.589127in}}%
\pgfpathclose%
\pgfusepath{stroke,fill}%
\end{pgfscope}%
\begin{pgfscope}%
\pgfpathrectangle{\pgfqpoint{0.150000in}{0.150000in}}{\pgfqpoint{1.700000in}{1.700000in}}%
\pgfusepath{clip}%
\pgfsetbuttcap%
\pgfsetroundjoin%
\definecolor{currentfill}{rgb}{0.400000,0.600000,0.800000}%
\pgfsetfillcolor{currentfill}%
\pgfsetlinewidth{1.003750pt}%
\definecolor{currentstroke}{rgb}{0.000000,0.266667,0.533333}%
\pgfsetstrokecolor{currentstroke}%
\pgfsetdash{}{0pt}%
\pgfpathmoveto{\pgfqpoint{0.782395in}{0.309457in}}%
\pgfpathlineto{\pgfqpoint{0.849086in}{0.309457in}}%
\pgfpathlineto{\pgfqpoint{0.849086in}{0.322585in}}%
\pgfpathlineto{\pgfqpoint{0.782395in}{0.322585in}}%
\pgfpathlineto{\pgfqpoint{0.782395in}{0.309457in}}%
\pgfpathclose%
\pgfusepath{stroke,fill}%
\end{pgfscope}%
\begin{pgfscope}%
\pgfpathrectangle{\pgfqpoint{0.150000in}{0.150000in}}{\pgfqpoint{1.700000in}{1.700000in}}%
\pgfusepath{clip}%
\pgfsetbuttcap%
\pgfsetroundjoin%
\definecolor{currentfill}{rgb}{0.400000,0.600000,0.800000}%
\pgfsetfillcolor{currentfill}%
\pgfsetlinewidth{1.003750pt}%
\definecolor{currentstroke}{rgb}{0.000000,0.266667,0.533333}%
\pgfsetstrokecolor{currentstroke}%
\pgfsetdash{}{0pt}%
\pgfpathmoveto{\pgfqpoint{0.661138in}{0.340975in}}%
\pgfpathlineto{\pgfqpoint{0.715703in}{0.340975in}}%
\pgfpathlineto{\pgfqpoint{0.715703in}{0.366879in}}%
\pgfpathlineto{\pgfqpoint{0.661138in}{0.366879in}}%
\pgfpathlineto{\pgfqpoint{0.661138in}{0.340975in}}%
\pgfpathclose%
\pgfusepath{stroke,fill}%
\end{pgfscope}%
\begin{pgfscope}%
\pgfpathrectangle{\pgfqpoint{0.150000in}{0.150000in}}{\pgfqpoint{1.700000in}{1.700000in}}%
\pgfusepath{clip}%
\pgfsetbuttcap%
\pgfsetroundjoin%
\definecolor{currentfill}{rgb}{0.400000,0.600000,0.800000}%
\pgfsetfillcolor{currentfill}%
\pgfsetlinewidth{1.003750pt}%
\definecolor{currentstroke}{rgb}{0.000000,0.266667,0.533333}%
\pgfsetstrokecolor{currentstroke}%
\pgfsetdash{}{0pt}%
\pgfpathmoveto{\pgfqpoint{0.539881in}{0.394328in}}%
\pgfpathlineto{\pgfqpoint{0.594446in}{0.394328in}}%
\pgfpathlineto{\pgfqpoint{0.594446in}{0.436801in}}%
\pgfpathlineto{\pgfqpoint{0.539881in}{0.436801in}}%
\pgfpathlineto{\pgfqpoint{0.539881in}{0.394328in}}%
\pgfpathclose%
\pgfusepath{stroke,fill}%
\end{pgfscope}%
\begin{pgfscope}%
\pgfpathrectangle{\pgfqpoint{0.150000in}{0.150000in}}{\pgfqpoint{1.700000in}{1.700000in}}%
\pgfusepath{clip}%
\pgfsetbuttcap%
\pgfsetroundjoin%
\definecolor{currentfill}{rgb}{0.400000,0.600000,0.800000}%
\pgfsetfillcolor{currentfill}%
\pgfsetlinewidth{1.003750pt}%
\definecolor{currentstroke}{rgb}{0.000000,0.266667,0.533333}%
\pgfsetstrokecolor{currentstroke}%
\pgfsetdash{}{0pt}%
\pgfpathmoveto{\pgfqpoint{0.440670in}{0.480426in}}%
\pgfpathlineto{\pgfqpoint{0.489954in}{0.480426in}}%
\pgfpathlineto{\pgfqpoint{0.489954in}{0.529342in}}%
\pgfpathlineto{\pgfqpoint{0.440670in}{0.529342in}}%
\pgfpathlineto{\pgfqpoint{0.440670in}{0.480426in}}%
\pgfpathclose%
\pgfusepath{stroke,fill}%
\end{pgfscope}%
\begin{pgfscope}%
\pgfpathrectangle{\pgfqpoint{0.150000in}{0.150000in}}{\pgfqpoint{1.700000in}{1.700000in}}%
\pgfusepath{clip}%
\pgfsetbuttcap%
\pgfsetroundjoin%
\definecolor{currentfill}{rgb}{0.400000,0.600000,0.800000}%
\pgfsetfillcolor{currentfill}%
\pgfsetlinewidth{1.003750pt}%
\definecolor{currentstroke}{rgb}{0.000000,0.266667,0.533333}%
\pgfsetstrokecolor{currentstroke}%
\pgfsetdash{}{0pt}%
\pgfpathmoveto{\pgfqpoint{1.549661in}{1.423721in}}%
\pgfpathlineto{\pgfqpoint{1.647915in}{1.423721in}}%
\pgfpathlineto{\pgfqpoint{1.647915in}{1.637585in}}%
\pgfpathlineto{\pgfqpoint{1.549661in}{1.637585in}}%
\pgfpathlineto{\pgfqpoint{1.549661in}{1.423721in}}%
\pgfpathclose%
\pgfusepath{stroke,fill}%
\end{pgfscope}%
\begin{pgfscope}%
\pgfpathrectangle{\pgfqpoint{0.150000in}{0.150000in}}{\pgfqpoint{1.700000in}{1.700000in}}%
\pgfusepath{clip}%
\pgfsetbuttcap%
\pgfsetroundjoin%
\definecolor{currentfill}{rgb}{0.400000,0.600000,0.800000}%
\pgfsetfillcolor{currentfill}%
\pgfsetlinewidth{1.003750pt}%
\definecolor{currentstroke}{rgb}{0.000000,0.266667,0.533333}%
\pgfsetstrokecolor{currentstroke}%
\pgfsetdash{}{0pt}%
\pgfpathmoveto{\pgfqpoint{1.085191in}{1.688774in}}%
\pgfpathlineto{\pgfqpoint{1.274139in}{1.688774in}}%
\pgfpathlineto{\pgfqpoint{1.274139in}{1.694022in}}%
\pgfpathlineto{\pgfqpoint{1.085191in}{1.694022in}}%
\pgfpathlineto{\pgfqpoint{1.085191in}{1.688774in}}%
\pgfpathclose%
\pgfusepath{stroke,fill}%
\end{pgfscope}%
\begin{pgfscope}%
\pgfpathrectangle{\pgfqpoint{0.150000in}{0.150000in}}{\pgfqpoint{1.700000in}{1.700000in}}%
\pgfusepath{clip}%
\pgfsetbuttcap%
\pgfsetroundjoin%
\definecolor{currentfill}{rgb}{0.400000,0.600000,0.800000}%
\pgfsetfillcolor{currentfill}%
\pgfsetlinewidth{1.003750pt}%
\definecolor{currentstroke}{rgb}{0.000000,0.266667,0.533333}%
\pgfsetstrokecolor{currentstroke}%
\pgfsetdash{}{0pt}%
\pgfpathmoveto{\pgfqpoint{0.930598in}{1.292239in}}%
\pgfpathlineto{\pgfqpoint{1.085191in}{1.292239in}}%
\pgfpathlineto{\pgfqpoint{1.085191in}{1.391533in}}%
\pgfpathlineto{\pgfqpoint{0.930598in}{1.391533in}}%
\pgfpathlineto{\pgfqpoint{0.930598in}{1.292239in}}%
\pgfpathclose%
\pgfusepath{stroke,fill}%
\end{pgfscope}%
\begin{pgfscope}%
\pgfpathrectangle{\pgfqpoint{0.150000in}{0.150000in}}{\pgfqpoint{1.700000in}{1.700000in}}%
\pgfusepath{clip}%
\pgfsetbuttcap%
\pgfsetroundjoin%
\definecolor{currentfill}{rgb}{0.400000,0.600000,0.800000}%
\pgfsetfillcolor{currentfill}%
\pgfsetlinewidth{1.003750pt}%
\definecolor{currentstroke}{rgb}{0.000000,0.266667,0.533333}%
\pgfsetstrokecolor{currentstroke}%
\pgfsetdash{}{0pt}%
\pgfpathmoveto{\pgfqpoint{1.410580in}{0.361744in}}%
\pgfpathlineto{\pgfqpoint{1.579254in}{0.361744in}}%
\pgfpathlineto{\pgfqpoint{1.579254in}{0.440455in}}%
\pgfpathlineto{\pgfqpoint{1.410580in}{0.440455in}}%
\pgfpathlineto{\pgfqpoint{1.410580in}{0.361744in}}%
\pgfpathclose%
\pgfusepath{stroke,fill}%
\end{pgfscope}%
\begin{pgfscope}%
\pgfpathrectangle{\pgfqpoint{0.150000in}{0.150000in}}{\pgfqpoint{1.700000in}{1.700000in}}%
\pgfusepath{clip}%
\pgfsetbuttcap%
\pgfsetroundjoin%
\definecolor{currentfill}{rgb}{0.400000,0.600000,0.800000}%
\pgfsetfillcolor{currentfill}%
\pgfsetlinewidth{1.003750pt}%
\definecolor{currentstroke}{rgb}{0.000000,0.266667,0.533333}%
\pgfsetstrokecolor{currentstroke}%
\pgfsetdash{}{0pt}%
\pgfpathmoveto{\pgfqpoint{0.930598in}{0.608314in}}%
\pgfpathlineto{\pgfqpoint{1.084487in}{0.608314in}}%
\pgfpathlineto{\pgfqpoint{1.084487in}{0.706301in}}%
\pgfpathlineto{\pgfqpoint{0.930598in}{0.706301in}}%
\pgfpathlineto{\pgfqpoint{0.930598in}{0.608314in}}%
\pgfpathclose%
\pgfusepath{stroke,fill}%
\end{pgfscope}%
\begin{pgfscope}%
\pgfpathrectangle{\pgfqpoint{0.150000in}{0.150000in}}{\pgfqpoint{1.700000in}{1.700000in}}%
\pgfusepath{clip}%
\pgfsetbuttcap%
\pgfsetroundjoin%
\definecolor{currentfill}{rgb}{0.400000,0.600000,0.800000}%
\pgfsetfillcolor{currentfill}%
\pgfsetlinewidth{1.003750pt}%
\definecolor{currentstroke}{rgb}{0.000000,0.266667,0.533333}%
\pgfsetstrokecolor{currentstroke}%
\pgfsetdash{}{0pt}%
\pgfpathmoveto{\pgfqpoint{1.084487in}{0.305978in}}%
\pgfpathlineto{\pgfqpoint{1.272573in}{0.305978in}}%
\pgfpathlineto{\pgfqpoint{1.272573in}{0.311140in}}%
\pgfpathlineto{\pgfqpoint{1.084487in}{0.311140in}}%
\pgfpathlineto{\pgfqpoint{1.084487in}{0.305978in}}%
\pgfpathclose%
\pgfusepath{stroke,fill}%
\end{pgfscope}%
\begin{pgfscope}%
\pgfpathrectangle{\pgfqpoint{0.150000in}{0.150000in}}{\pgfqpoint{1.700000in}{1.700000in}}%
\pgfusepath{clip}%
\pgfsetbuttcap%
\pgfsetroundjoin%
\definecolor{currentfill}{rgb}{0.400000,0.600000,0.800000}%
\pgfsetfillcolor{currentfill}%
\pgfsetlinewidth{1.003750pt}%
\definecolor{currentstroke}{rgb}{0.000000,0.266667,0.533333}%
\pgfsetstrokecolor{currentstroke}%
\pgfsetdash{}{0pt}%
\pgfpathmoveto{\pgfqpoint{0.361826in}{1.410698in}}%
\pgfpathlineto{\pgfqpoint{0.440542in}{1.410698in}}%
\pgfpathlineto{\pgfqpoint{0.440542in}{1.579284in}}%
\pgfpathlineto{\pgfqpoint{0.361826in}{1.579284in}}%
\pgfpathlineto{\pgfqpoint{0.361826in}{1.410698in}}%
\pgfpathclose%
\pgfusepath{stroke,fill}%
\end{pgfscope}%
\begin{pgfscope}%
\pgfpathrectangle{\pgfqpoint{0.150000in}{0.150000in}}{\pgfqpoint{1.700000in}{1.700000in}}%
\pgfusepath{clip}%
\pgfsetbuttcap%
\pgfsetroundjoin%
\definecolor{currentfill}{rgb}{0.400000,0.600000,0.800000}%
\pgfsetfillcolor{currentfill}%
\pgfsetlinewidth{1.003750pt}%
\definecolor{currentstroke}{rgb}{0.000000,0.266667,0.533333}%
\pgfsetstrokecolor{currentstroke}%
\pgfsetdash{}{0pt}%
\pgfpathmoveto{\pgfqpoint{0.608374in}{0.930946in}}%
\pgfpathlineto{\pgfqpoint{0.706478in}{0.930946in}}%
\pgfpathlineto{\pgfqpoint{0.706478in}{1.084764in}}%
\pgfpathlineto{\pgfqpoint{0.608374in}{1.084764in}}%
\pgfpathlineto{\pgfqpoint{0.608374in}{0.930946in}}%
\pgfpathclose%
\pgfusepath{stroke,fill}%
\end{pgfscope}%
\begin{pgfscope}%
\pgfpathrectangle{\pgfqpoint{0.150000in}{0.150000in}}{\pgfqpoint{1.700000in}{1.700000in}}%
\pgfusepath{clip}%
\pgfsetbuttcap%
\pgfsetroundjoin%
\definecolor{currentfill}{rgb}{0.400000,0.600000,0.800000}%
\pgfsetfillcolor{currentfill}%
\pgfsetlinewidth{1.003750pt}%
\definecolor{currentstroke}{rgb}{0.000000,0.266667,0.533333}%
\pgfsetstrokecolor{currentstroke}%
\pgfsetdash{}{0pt}%
\pgfpathmoveto{\pgfqpoint{0.305978in}{1.084764in}}%
\pgfpathlineto{\pgfqpoint{0.311174in}{1.084764in}}%
\pgfpathlineto{\pgfqpoint{0.311174in}{1.272765in}}%
\pgfpathlineto{\pgfqpoint{0.305978in}{1.272765in}}%
\pgfpathlineto{\pgfqpoint{0.305978in}{1.084764in}}%
\pgfpathclose%
\pgfusepath{stroke,fill}%
\end{pgfscope}%
\begin{pgfscope}%
\pgfpathrectangle{\pgfqpoint{0.150000in}{0.150000in}}{\pgfqpoint{1.700000in}{1.700000in}}%
\pgfusepath{clip}%
\pgfsetbuttcap%
\pgfsetroundjoin%
\definecolor{currentfill}{rgb}{0.400000,0.600000,0.800000}%
\pgfsetfillcolor{currentfill}%
\pgfsetlinewidth{1.003750pt}%
\definecolor{currentstroke}{rgb}{0.000000,0.266667,0.533333}%
\pgfsetstrokecolor{currentstroke}%
\pgfsetdash{}{0pt}%
\pgfpathmoveto{\pgfqpoint{0.685374in}{0.751875in}}%
\pgfpathlineto{\pgfqpoint{0.930598in}{0.751875in}}%
\pgfpathlineto{\pgfqpoint{0.930598in}{0.930946in}}%
\pgfpathlineto{\pgfqpoint{0.685374in}{0.930946in}}%
\pgfpathlineto{\pgfqpoint{0.685374in}{0.751875in}}%
\pgfpathclose%
\pgfusepath{stroke,fill}%
\end{pgfscope}%
\begin{pgfscope}%
\pgfpathrectangle{\pgfqpoint{0.150000in}{0.150000in}}{\pgfqpoint{1.700000in}{1.700000in}}%
\pgfusepath{clip}%
\pgfsetbuttcap%
\pgfsetroundjoin%
\definecolor{currentfill}{rgb}{0.400000,0.600000,0.800000}%
\pgfsetfillcolor{currentfill}%
\pgfsetlinewidth{1.003750pt}%
\definecolor{currentstroke}{rgb}{0.000000,0.266667,0.533333}%
\pgfsetstrokecolor{currentstroke}%
\pgfsetdash{}{0pt}%
\pgfpathmoveto{\pgfqpoint{0.309422in}{0.589127in}}%
\pgfpathlineto{\pgfqpoint{0.355338in}{0.589127in}}%
\pgfpathlineto{\pgfqpoint{0.355338in}{0.742945in}}%
\pgfpathlineto{\pgfqpoint{0.309422in}{0.742945in}}%
\pgfpathlineto{\pgfqpoint{0.309422in}{0.589127in}}%
\pgfpathclose%
\pgfusepath{stroke,fill}%
\end{pgfscope}%
\begin{pgfscope}%
\pgfpathrectangle{\pgfqpoint{0.150000in}{0.150000in}}{\pgfqpoint{1.700000in}{1.700000in}}%
\pgfusepath{clip}%
\pgfsetbuttcap%
\pgfsetroundjoin%
\definecolor{currentfill}{rgb}{0.400000,0.600000,0.800000}%
\pgfsetfillcolor{currentfill}%
\pgfsetlinewidth{1.003750pt}%
\definecolor{currentstroke}{rgb}{0.000000,0.266667,0.533333}%
\pgfsetstrokecolor{currentstroke}%
\pgfsetdash{}{0pt}%
\pgfpathmoveto{\pgfqpoint{0.661138in}{0.309457in}}%
\pgfpathlineto{\pgfqpoint{0.782395in}{0.309457in}}%
\pgfpathlineto{\pgfqpoint{0.782395in}{0.340975in}}%
\pgfpathlineto{\pgfqpoint{0.661138in}{0.340975in}}%
\pgfpathlineto{\pgfqpoint{0.661138in}{0.309457in}}%
\pgfpathclose%
\pgfusepath{stroke,fill}%
\end{pgfscope}%
\begin{pgfscope}%
\pgfpathrectangle{\pgfqpoint{0.150000in}{0.150000in}}{\pgfqpoint{1.700000in}{1.700000in}}%
\pgfusepath{clip}%
\pgfsetbuttcap%
\pgfsetroundjoin%
\definecolor{currentfill}{rgb}{0.400000,0.600000,0.800000}%
\pgfsetfillcolor{currentfill}%
\pgfsetlinewidth{1.003750pt}%
\definecolor{currentstroke}{rgb}{0.000000,0.266667,0.533333}%
\pgfsetstrokecolor{currentstroke}%
\pgfsetdash{}{0pt}%
\pgfpathmoveto{\pgfqpoint{0.440670in}{0.394328in}}%
\pgfpathlineto{\pgfqpoint{0.539881in}{0.394328in}}%
\pgfpathlineto{\pgfqpoint{0.539881in}{0.480426in}}%
\pgfpathlineto{\pgfqpoint{0.440670in}{0.480426in}}%
\pgfpathlineto{\pgfqpoint{0.440670in}{0.394328in}}%
\pgfpathclose%
\pgfusepath{stroke,fill}%
\end{pgfscope}%
\begin{pgfscope}%
\pgfpathrectangle{\pgfqpoint{0.150000in}{0.150000in}}{\pgfqpoint{1.700000in}{1.700000in}}%
\pgfusepath{clip}%
\pgfsetbuttcap%
\pgfsetroundjoin%
\definecolor{currentfill}{rgb}{0.400000,0.600000,0.800000}%
\pgfsetfillcolor{currentfill}%
\pgfsetlinewidth{1.003750pt}%
\definecolor{currentstroke}{rgb}{0.000000,0.266667,0.533333}%
\pgfsetstrokecolor{currentstroke}%
\pgfsetdash{}{0pt}%
\pgfpathmoveto{\pgfqpoint{1.647915in}{1.248742in}}%
\pgfpathlineto{\pgfqpoint{1.694022in}{1.248742in}}%
\pgfpathlineto{\pgfqpoint{1.694022in}{1.637585in}}%
\pgfpathlineto{\pgfqpoint{1.647915in}{1.637585in}}%
\pgfpathlineto{\pgfqpoint{1.647915in}{1.248742in}}%
\pgfpathclose%
\pgfusepath{stroke,fill}%
\end{pgfscope}%
\begin{pgfscope}%
\pgfpathrectangle{\pgfqpoint{0.150000in}{0.150000in}}{\pgfqpoint{1.700000in}{1.700000in}}%
\pgfusepath{clip}%
\pgfsetbuttcap%
\pgfsetroundjoin%
\definecolor{currentfill}{rgb}{0.400000,0.600000,0.800000}%
\pgfsetfillcolor{currentfill}%
\pgfsetlinewidth{1.003750pt}%
\definecolor{currentstroke}{rgb}{0.000000,0.266667,0.533333}%
\pgfsetstrokecolor{currentstroke}%
\pgfsetdash{}{0pt}%
\pgfpathmoveto{\pgfqpoint{1.274139in}{0.930598in}}%
\pgfpathlineto{\pgfqpoint{1.314138in}{0.930598in}}%
\pgfpathlineto{\pgfqpoint{1.314138in}{1.248742in}}%
\pgfpathlineto{\pgfqpoint{1.274139in}{1.248742in}}%
\pgfpathlineto{\pgfqpoint{1.274139in}{0.930598in}}%
\pgfpathclose%
\pgfusepath{stroke,fill}%
\end{pgfscope}%
\begin{pgfscope}%
\pgfpathrectangle{\pgfqpoint{0.150000in}{0.150000in}}{\pgfqpoint{1.700000in}{1.700000in}}%
\pgfusepath{clip}%
\pgfsetbuttcap%
\pgfsetroundjoin%
\definecolor{currentfill}{rgb}{0.400000,0.600000,0.800000}%
\pgfsetfillcolor{currentfill}%
\pgfsetlinewidth{1.003750pt}%
\definecolor{currentstroke}{rgb}{0.000000,0.266667,0.533333}%
\pgfsetstrokecolor{currentstroke}%
\pgfsetdash{}{0pt}%
\pgfpathmoveto{\pgfqpoint{1.579254in}{0.361744in}}%
\pgfpathlineto{\pgfqpoint{1.690543in}{0.361744in}}%
\pgfpathlineto{\pgfqpoint{1.690543in}{0.617728in}}%
\pgfpathlineto{\pgfqpoint{1.579254in}{0.617728in}}%
\pgfpathlineto{\pgfqpoint{1.579254in}{0.361744in}}%
\pgfpathclose%
\pgfusepath{stroke,fill}%
\end{pgfscope}%
\begin{pgfscope}%
\pgfpathrectangle{\pgfqpoint{0.150000in}{0.150000in}}{\pgfqpoint{1.700000in}{1.700000in}}%
\pgfusepath{clip}%
\pgfsetbuttcap%
\pgfsetroundjoin%
\definecolor{currentfill}{rgb}{0.400000,0.600000,0.800000}%
\pgfsetfillcolor{currentfill}%
\pgfsetlinewidth{1.003750pt}%
\definecolor{currentstroke}{rgb}{0.000000,0.266667,0.533333}%
\pgfsetstrokecolor{currentstroke}%
\pgfsetdash{}{0pt}%
\pgfpathmoveto{\pgfqpoint{0.361826in}{1.579284in}}%
\pgfpathlineto{\pgfqpoint{0.617773in}{1.579284in}}%
\pgfpathlineto{\pgfqpoint{0.617773in}{1.690543in}}%
\pgfpathlineto{\pgfqpoint{0.361826in}{1.690543in}}%
\pgfpathlineto{\pgfqpoint{0.361826in}{1.579284in}}%
\pgfpathclose%
\pgfusepath{stroke,fill}%
\end{pgfscope}%
\begin{pgfscope}%
\pgfpathrectangle{\pgfqpoint{0.150000in}{0.150000in}}{\pgfqpoint{1.700000in}{1.700000in}}%
\pgfusepath{clip}%
\pgfsetbuttcap%
\pgfsetroundjoin%
\definecolor{currentfill}{rgb}{0.400000,0.600000,0.800000}%
\pgfsetfillcolor{currentfill}%
\pgfsetlinewidth{1.003750pt}%
\definecolor{currentstroke}{rgb}{0.000000,0.266667,0.533333}%
\pgfsetstrokecolor{currentstroke}%
\pgfsetdash{}{0pt}%
\pgfpathmoveto{\pgfqpoint{0.440670in}{0.309457in}}%
\pgfpathlineto{\pgfqpoint{0.661138in}{0.309457in}}%
\pgfpathlineto{\pgfqpoint{0.661138in}{0.394328in}}%
\pgfpathlineto{\pgfqpoint{0.440670in}{0.394328in}}%
\pgfpathlineto{\pgfqpoint{0.440670in}{0.309457in}}%
\pgfpathclose%
\pgfusepath{stroke,fill}%
\end{pgfscope}%
\begin{pgfscope}%
\pgfpathrectangle{\pgfqpoint{0.150000in}{0.150000in}}{\pgfqpoint{1.700000in}{1.700000in}}%
\pgfusepath{clip}%
\pgfsetbuttcap%
\pgfsetroundjoin%
\definecolor{currentfill}{rgb}{0.400000,0.600000,0.800000}%
\pgfsetfillcolor{currentfill}%
\pgfsetlinewidth{1.003750pt}%
\definecolor{currentstroke}{rgb}{0.000000,0.266667,0.533333}%
\pgfsetstrokecolor{currentstroke}%
\pgfsetdash{}{0pt}%
\pgfpathmoveto{\pgfqpoint{1.274139in}{1.637585in}}%
\pgfpathlineto{\pgfqpoint{1.694022in}{1.637585in}}%
\pgfpathlineto{\pgfqpoint{1.694022in}{1.694022in}}%
\pgfpathlineto{\pgfqpoint{1.274139in}{1.694022in}}%
\pgfpathlineto{\pgfqpoint{1.274139in}{1.637585in}}%
\pgfpathclose%
\pgfusepath{stroke,fill}%
\end{pgfscope}%
\begin{pgfscope}%
\pgfpathrectangle{\pgfqpoint{0.150000in}{0.150000in}}{\pgfqpoint{1.700000in}{1.700000in}}%
\pgfusepath{clip}%
\pgfsetbuttcap%
\pgfsetroundjoin%
\definecolor{currentfill}{rgb}{0.400000,0.600000,0.800000}%
\pgfsetfillcolor{currentfill}%
\pgfsetlinewidth{1.003750pt}%
\definecolor{currentstroke}{rgb}{0.000000,0.266667,0.533333}%
\pgfsetstrokecolor{currentstroke}%
\pgfsetdash{}{0pt}%
\pgfpathmoveto{\pgfqpoint{0.930598in}{0.930598in}}%
\pgfpathlineto{\pgfqpoint{1.274139in}{0.930598in}}%
\pgfpathlineto{\pgfqpoint{1.274139in}{1.292239in}}%
\pgfpathlineto{\pgfqpoint{0.930598in}{1.292239in}}%
\pgfpathlineto{\pgfqpoint{0.930598in}{0.930598in}}%
\pgfpathclose%
\pgfusepath{stroke,fill}%
\end{pgfscope}%
\begin{pgfscope}%
\pgfpathrectangle{\pgfqpoint{0.150000in}{0.150000in}}{\pgfqpoint{1.700000in}{1.700000in}}%
\pgfusepath{clip}%
\pgfsetbuttcap%
\pgfsetroundjoin%
\definecolor{currentfill}{rgb}{0.400000,0.600000,0.800000}%
\pgfsetfillcolor{currentfill}%
\pgfsetlinewidth{1.003750pt}%
\definecolor{currentstroke}{rgb}{0.000000,0.266667,0.533333}%
\pgfsetstrokecolor{currentstroke}%
\pgfsetdash{}{0pt}%
\pgfpathmoveto{\pgfqpoint{1.272573in}{0.305978in}}%
\pgfpathlineto{\pgfqpoint{1.690543in}{0.305978in}}%
\pgfpathlineto{\pgfqpoint{1.690543in}{0.361744in}}%
\pgfpathlineto{\pgfqpoint{1.272573in}{0.361744in}}%
\pgfpathlineto{\pgfqpoint{1.272573in}{0.305978in}}%
\pgfpathclose%
\pgfusepath{stroke,fill}%
\end{pgfscope}%
\begin{pgfscope}%
\pgfpathrectangle{\pgfqpoint{0.150000in}{0.150000in}}{\pgfqpoint{1.700000in}{1.700000in}}%
\pgfusepath{clip}%
\pgfsetbuttcap%
\pgfsetroundjoin%
\definecolor{currentfill}{rgb}{0.400000,0.600000,0.800000}%
\pgfsetfillcolor{currentfill}%
\pgfsetlinewidth{1.003750pt}%
\definecolor{currentstroke}{rgb}{0.000000,0.266667,0.533333}%
\pgfsetstrokecolor{currentstroke}%
\pgfsetdash{}{0pt}%
\pgfpathmoveto{\pgfqpoint{0.930598in}{0.706301in}}%
\pgfpathlineto{\pgfqpoint{1.272573in}{0.706301in}}%
\pgfpathlineto{\pgfqpoint{1.272573in}{0.930598in}}%
\pgfpathlineto{\pgfqpoint{0.930598in}{0.930598in}}%
\pgfpathlineto{\pgfqpoint{0.930598in}{0.706301in}}%
\pgfpathclose%
\pgfusepath{stroke,fill}%
\end{pgfscope}%
\begin{pgfscope}%
\pgfpathrectangle{\pgfqpoint{0.150000in}{0.150000in}}{\pgfqpoint{1.700000in}{1.700000in}}%
\pgfusepath{clip}%
\pgfsetbuttcap%
\pgfsetroundjoin%
\definecolor{currentfill}{rgb}{0.400000,0.600000,0.800000}%
\pgfsetfillcolor{currentfill}%
\pgfsetlinewidth{1.003750pt}%
\definecolor{currentstroke}{rgb}{0.000000,0.266667,0.533333}%
\pgfsetstrokecolor{currentstroke}%
\pgfsetdash{}{0pt}%
\pgfpathmoveto{\pgfqpoint{0.305978in}{1.272765in}}%
\pgfpathlineto{\pgfqpoint{0.361826in}{1.272765in}}%
\pgfpathlineto{\pgfqpoint{0.361826in}{1.690543in}}%
\pgfpathlineto{\pgfqpoint{0.305978in}{1.690543in}}%
\pgfpathlineto{\pgfqpoint{0.305978in}{1.272765in}}%
\pgfpathclose%
\pgfusepath{stroke,fill}%
\end{pgfscope}%
\begin{pgfscope}%
\pgfpathrectangle{\pgfqpoint{0.150000in}{0.150000in}}{\pgfqpoint{1.700000in}{1.700000in}}%
\pgfusepath{clip}%
\pgfsetbuttcap%
\pgfsetroundjoin%
\definecolor{currentfill}{rgb}{0.400000,0.600000,0.800000}%
\pgfsetfillcolor{currentfill}%
\pgfsetlinewidth{1.003750pt}%
\definecolor{currentstroke}{rgb}{0.000000,0.266667,0.533333}%
\pgfsetstrokecolor{currentstroke}%
\pgfsetdash{}{0pt}%
\pgfpathmoveto{\pgfqpoint{0.706478in}{0.930946in}}%
\pgfpathlineto{\pgfqpoint{0.930598in}{0.930946in}}%
\pgfpathlineto{\pgfqpoint{0.930598in}{1.272765in}}%
\pgfpathlineto{\pgfqpoint{0.706478in}{1.272765in}}%
\pgfpathlineto{\pgfqpoint{0.706478in}{0.930946in}}%
\pgfpathclose%
\pgfusepath{stroke,fill}%
\end{pgfscope}%
\begin{pgfscope}%
\pgfpathrectangle{\pgfqpoint{0.150000in}{0.150000in}}{\pgfqpoint{1.700000in}{1.700000in}}%
\pgfusepath{clip}%
\pgfsetbuttcap%
\pgfsetroundjoin%
\definecolor{currentfill}{rgb}{0.400000,0.600000,0.800000}%
\pgfsetfillcolor{currentfill}%
\pgfsetlinewidth{1.003750pt}%
\definecolor{currentstroke}{rgb}{0.000000,0.266667,0.533333}%
\pgfsetstrokecolor{currentstroke}%
\pgfsetdash{}{0pt}%
\pgfpathmoveto{\pgfqpoint{0.309422in}{0.309457in}}%
\pgfpathlineto{\pgfqpoint{0.440670in}{0.309457in}}%
\pgfpathlineto{\pgfqpoint{0.440670in}{0.589127in}}%
\pgfpathlineto{\pgfqpoint{0.309422in}{0.589127in}}%
\pgfpathlineto{\pgfqpoint{0.309422in}{0.309457in}}%
\pgfpathclose%
\pgfusepath{stroke,fill}%
\end{pgfscope}%
\begin{pgfscope}%
\pgfpathrectangle{\pgfqpoint{0.150000in}{0.150000in}}{\pgfqpoint{1.700000in}{1.700000in}}%
\pgfusepath{clip}%
\pgfsetbuttcap%
\pgfsetroundjoin%
\definecolor{currentfill}{rgb}{0.400000,0.600000,0.800000}%
\pgfsetfillcolor{currentfill}%
\pgfsetlinewidth{1.003750pt}%
\definecolor{currentstroke}{rgb}{0.000000,0.266667,0.533333}%
\pgfsetstrokecolor{currentstroke}%
\pgfsetdash{}{0pt}%
\pgfpathmoveto{\pgfqpoint{1.690543in}{0.305978in}}%
\pgfpathlineto{\pgfqpoint{1.694022in}{0.305978in}}%
\pgfpathlineto{\pgfqpoint{1.694022in}{0.930598in}}%
\pgfpathlineto{\pgfqpoint{1.690543in}{0.930598in}}%
\pgfpathlineto{\pgfqpoint{1.690543in}{0.305978in}}%
\pgfpathclose%
\pgfusepath{stroke,fill}%
\end{pgfscope}%
\begin{pgfscope}%
\pgfpathrectangle{\pgfqpoint{0.150000in}{0.150000in}}{\pgfqpoint{1.700000in}{1.700000in}}%
\pgfusepath{clip}%
\pgfsetbuttcap%
\pgfsetroundjoin%
\definecolor{currentfill}{rgb}{0.400000,0.600000,0.800000}%
\pgfsetfillcolor{currentfill}%
\pgfsetlinewidth{1.003750pt}%
\definecolor{currentstroke}{rgb}{0.000000,0.266667,0.533333}%
\pgfsetstrokecolor{currentstroke}%
\pgfsetdash{}{0pt}%
\pgfpathmoveto{\pgfqpoint{0.305978in}{0.309457in}}%
\pgfpathlineto{\pgfqpoint{0.309422in}{0.309457in}}%
\pgfpathlineto{\pgfqpoint{0.309422in}{0.930946in}}%
\pgfpathlineto{\pgfqpoint{0.305978in}{0.930946in}}%
\pgfpathlineto{\pgfqpoint{0.305978in}{0.309457in}}%
\pgfpathclose%
\pgfusepath{stroke,fill}%
\end{pgfscope}%
\begin{pgfscope}%
\pgfpathrectangle{\pgfqpoint{0.150000in}{0.150000in}}{\pgfqpoint{1.700000in}{1.700000in}}%
\pgfusepath{clip}%
\pgfsetbuttcap%
\pgfsetroundjoin%
\definecolor{currentfill}{rgb}{0.400000,0.600000,0.800000}%
\pgfsetfillcolor{currentfill}%
\pgfsetlinewidth{1.003750pt}%
\definecolor{currentstroke}{rgb}{0.000000,0.266667,0.533333}%
\pgfsetstrokecolor{currentstroke}%
\pgfsetdash{}{0pt}%
\pgfpathmoveto{\pgfqpoint{0.305978in}{1.690543in}}%
\pgfpathlineto{\pgfqpoint{0.930598in}{1.690543in}}%
\pgfpathlineto{\pgfqpoint{0.930598in}{1.694022in}}%
\pgfpathlineto{\pgfqpoint{0.305978in}{1.694022in}}%
\pgfpathlineto{\pgfqpoint{0.305978in}{1.690543in}}%
\pgfpathclose%
\pgfusepath{stroke,fill}%
\end{pgfscope}%
\begin{pgfscope}%
\pgfpathrectangle{\pgfqpoint{0.150000in}{0.150000in}}{\pgfqpoint{1.700000in}{1.700000in}}%
\pgfusepath{clip}%
\pgfsetbuttcap%
\pgfsetroundjoin%
\definecolor{currentfill}{rgb}{0.400000,0.600000,0.800000}%
\pgfsetfillcolor{currentfill}%
\pgfsetlinewidth{1.003750pt}%
\definecolor{currentstroke}{rgb}{0.000000,0.266667,0.533333}%
\pgfsetstrokecolor{currentstroke}%
\pgfsetdash{}{0pt}%
\pgfpathmoveto{\pgfqpoint{0.305978in}{0.305978in}}%
\pgfpathlineto{\pgfqpoint{0.930598in}{0.305978in}}%
\pgfpathlineto{\pgfqpoint{0.930598in}{0.309457in}}%
\pgfpathlineto{\pgfqpoint{0.305978in}{0.309457in}}%
\pgfpathlineto{\pgfqpoint{0.305978in}{0.305978in}}%
\pgfpathclose%
\pgfusepath{stroke,fill}%
\end{pgfscope}%
\begin{pgfscope}%
\pgfpathrectangle{\pgfqpoint{0.150000in}{0.150000in}}{\pgfqpoint{1.700000in}{1.700000in}}%
\pgfusepath{clip}%
\pgfsetbuttcap%
\pgfsetroundjoin%
\definecolor{currentfill}{rgb}{0.400000,0.600000,0.800000}%
\pgfsetfillcolor{currentfill}%
\pgfsetlinewidth{1.003750pt}%
\definecolor{currentstroke}{rgb}{0.000000,0.266667,0.533333}%
\pgfsetstrokecolor{currentstroke}%
\pgfsetdash{}{0pt}%
\pgfpathmoveto{\pgfqpoint{0.305978in}{1.694022in}}%
\pgfpathlineto{\pgfqpoint{1.694022in}{1.694022in}}%
\pgfpathlineto{\pgfqpoint{1.694022in}{1.850000in}}%
\pgfpathlineto{\pgfqpoint{0.305978in}{1.850000in}}%
\pgfpathlineto{\pgfqpoint{0.305978in}{1.694022in}}%
\pgfpathclose%
\pgfusepath{stroke,fill}%
\end{pgfscope}%
\begin{pgfscope}%
\pgfpathrectangle{\pgfqpoint{0.150000in}{0.150000in}}{\pgfqpoint{1.700000in}{1.700000in}}%
\pgfusepath{clip}%
\pgfsetbuttcap%
\pgfsetroundjoin%
\definecolor{currentfill}{rgb}{0.400000,0.600000,0.800000}%
\pgfsetfillcolor{currentfill}%
\pgfsetlinewidth{1.003750pt}%
\definecolor{currentstroke}{rgb}{0.000000,0.266667,0.533333}%
\pgfsetstrokecolor{currentstroke}%
\pgfsetdash{}{0pt}%
\pgfpathmoveto{\pgfqpoint{0.305978in}{0.150000in}}%
\pgfpathlineto{\pgfqpoint{1.694022in}{0.150000in}}%
\pgfpathlineto{\pgfqpoint{1.694022in}{0.305978in}}%
\pgfpathlineto{\pgfqpoint{0.305978in}{0.305978in}}%
\pgfpathlineto{\pgfqpoint{0.305978in}{0.150000in}}%
\pgfpathclose%
\pgfusepath{stroke,fill}%
\end{pgfscope}%
\begin{pgfscope}%
\pgfpathrectangle{\pgfqpoint{0.150000in}{0.150000in}}{\pgfqpoint{1.700000in}{1.700000in}}%
\pgfusepath{clip}%
\pgfsetbuttcap%
\pgfsetroundjoin%
\definecolor{currentfill}{rgb}{0.400000,0.600000,0.800000}%
\pgfsetfillcolor{currentfill}%
\pgfsetlinewidth{1.003750pt}%
\definecolor{currentstroke}{rgb}{0.000000,0.266667,0.533333}%
\pgfsetstrokecolor{currentstroke}%
\pgfsetdash{}{0pt}%
\pgfpathmoveto{\pgfqpoint{1.694022in}{0.150000in}}%
\pgfpathlineto{\pgfqpoint{1.850000in}{0.150000in}}%
\pgfpathlineto{\pgfqpoint{1.850000in}{1.850000in}}%
\pgfpathlineto{\pgfqpoint{1.694022in}{1.850000in}}%
\pgfpathlineto{\pgfqpoint{1.694022in}{0.150000in}}%
\pgfpathclose%
\pgfusepath{stroke,fill}%
\end{pgfscope}%
\begin{pgfscope}%
\pgfpathrectangle{\pgfqpoint{0.150000in}{0.150000in}}{\pgfqpoint{1.700000in}{1.700000in}}%
\pgfusepath{clip}%
\pgfsetbuttcap%
\pgfsetroundjoin%
\definecolor{currentfill}{rgb}{0.400000,0.600000,0.800000}%
\pgfsetfillcolor{currentfill}%
\pgfsetlinewidth{1.003750pt}%
\definecolor{currentstroke}{rgb}{0.000000,0.266667,0.533333}%
\pgfsetstrokecolor{currentstroke}%
\pgfsetdash{}{0pt}%
\pgfpathmoveto{\pgfqpoint{0.150000in}{0.150000in}}%
\pgfpathlineto{\pgfqpoint{0.305978in}{0.150000in}}%
\pgfpathlineto{\pgfqpoint{0.305978in}{1.850000in}}%
\pgfpathlineto{\pgfqpoint{0.150000in}{1.850000in}}%
\pgfpathlineto{\pgfqpoint{0.150000in}{0.150000in}}%
\pgfpathclose%
\pgfusepath{stroke,fill}%
\end{pgfscope}%
\begin{pgfscope}%
\pgfpathrectangle{\pgfqpoint{0.150000in}{0.150000in}}{\pgfqpoint{1.700000in}{1.700000in}}%
\pgfusepath{clip}%
\pgfsetbuttcap%
\pgfsetroundjoin%
\definecolor{currentfill}{rgb}{0.933333,0.800000,0.400000}%
\pgfsetfillcolor{currentfill}%
\pgfsetlinewidth{1.003750pt}%
\definecolor{currentstroke}{rgb}{0.600000,0.466667,0.000000}%
\pgfsetstrokecolor{currentstroke}%
\pgfsetdash{}{0pt}%
\pgfpathmoveto{\pgfqpoint{1.466315in}{1.464356in}}%
\pgfpathlineto{\pgfqpoint{1.515791in}{1.464356in}}%
\pgfpathlineto{\pgfqpoint{1.515791in}{1.514020in}}%
\pgfpathlineto{\pgfqpoint{1.466315in}{1.514020in}}%
\pgfpathlineto{\pgfqpoint{1.466315in}{1.464356in}}%
\pgfpathclose%
\pgfusepath{stroke,fill}%
\end{pgfscope}%
\begin{pgfscope}%
\pgfpathrectangle{\pgfqpoint{0.150000in}{0.150000in}}{\pgfqpoint{1.700000in}{1.700000in}}%
\pgfusepath{clip}%
\pgfsetbuttcap%
\pgfsetroundjoin%
\definecolor{currentfill}{rgb}{0.933333,0.800000,0.400000}%
\pgfsetfillcolor{currentfill}%
\pgfsetlinewidth{1.003750pt}%
\definecolor{currentstroke}{rgb}{0.600000,0.466667,0.000000}%
\pgfsetstrokecolor{currentstroke}%
\pgfsetdash{}{0pt}%
\pgfpathmoveto{\pgfqpoint{1.515791in}{1.423721in}}%
\pgfpathlineto{\pgfqpoint{1.549661in}{1.423721in}}%
\pgfpathlineto{\pgfqpoint{1.549661in}{1.464356in}}%
\pgfpathlineto{\pgfqpoint{1.515791in}{1.464356in}}%
\pgfpathlineto{\pgfqpoint{1.515791in}{1.423721in}}%
\pgfpathclose%
\pgfusepath{stroke,fill}%
\end{pgfscope}%
\begin{pgfscope}%
\pgfpathrectangle{\pgfqpoint{0.150000in}{0.150000in}}{\pgfqpoint{1.700000in}{1.700000in}}%
\pgfusepath{clip}%
\pgfsetbuttcap%
\pgfsetroundjoin%
\definecolor{currentfill}{rgb}{0.933333,0.800000,0.400000}%
\pgfsetfillcolor{currentfill}%
\pgfsetlinewidth{1.003750pt}%
\definecolor{currentstroke}{rgb}{0.600000,0.466667,0.000000}%
\pgfsetstrokecolor{currentstroke}%
\pgfsetdash{}{0pt}%
\pgfpathmoveto{\pgfqpoint{1.549661in}{1.370790in}}%
\pgfpathlineto{\pgfqpoint{1.586670in}{1.370790in}}%
\pgfpathlineto{\pgfqpoint{1.586670in}{1.423721in}}%
\pgfpathlineto{\pgfqpoint{1.549661in}{1.423721in}}%
\pgfpathlineto{\pgfqpoint{1.549661in}{1.370790in}}%
\pgfpathclose%
\pgfusepath{stroke,fill}%
\end{pgfscope}%
\begin{pgfscope}%
\pgfpathrectangle{\pgfqpoint{0.150000in}{0.150000in}}{\pgfqpoint{1.700000in}{1.700000in}}%
\pgfusepath{clip}%
\pgfsetbuttcap%
\pgfsetroundjoin%
\definecolor{currentfill}{rgb}{0.933333,0.800000,0.400000}%
\pgfsetfillcolor{currentfill}%
\pgfsetlinewidth{1.003750pt}%
\definecolor{currentstroke}{rgb}{0.600000,0.466667,0.000000}%
\pgfsetstrokecolor{currentstroke}%
\pgfsetdash{}{0pt}%
\pgfpathmoveto{\pgfqpoint{1.586670in}{1.327483in}}%
\pgfpathlineto{\pgfqpoint{1.611900in}{1.327483in}}%
\pgfpathlineto{\pgfqpoint{1.611900in}{1.370790in}}%
\pgfpathlineto{\pgfqpoint{1.586670in}{1.370790in}}%
\pgfpathlineto{\pgfqpoint{1.586670in}{1.327483in}}%
\pgfpathclose%
\pgfusepath{stroke,fill}%
\end{pgfscope}%
\begin{pgfscope}%
\pgfpathrectangle{\pgfqpoint{0.150000in}{0.150000in}}{\pgfqpoint{1.700000in}{1.700000in}}%
\pgfusepath{clip}%
\pgfsetbuttcap%
\pgfsetroundjoin%
\definecolor{currentfill}{rgb}{0.933333,0.800000,0.400000}%
\pgfsetfillcolor{currentfill}%
\pgfsetlinewidth{1.003750pt}%
\definecolor{currentstroke}{rgb}{0.600000,0.466667,0.000000}%
\pgfsetstrokecolor{currentstroke}%
\pgfsetdash{}{0pt}%
\pgfpathmoveto{\pgfqpoint{1.611900in}{1.284175in}}%
\pgfpathlineto{\pgfqpoint{1.633175in}{1.284175in}}%
\pgfpathlineto{\pgfqpoint{1.633175in}{1.327483in}}%
\pgfpathlineto{\pgfqpoint{1.611900in}{1.327483in}}%
\pgfpathlineto{\pgfqpoint{1.611900in}{1.284175in}}%
\pgfpathclose%
\pgfusepath{stroke,fill}%
\end{pgfscope}%
\begin{pgfscope}%
\pgfpathrectangle{\pgfqpoint{0.150000in}{0.150000in}}{\pgfqpoint{1.700000in}{1.700000in}}%
\pgfusepath{clip}%
\pgfsetbuttcap%
\pgfsetroundjoin%
\definecolor{currentfill}{rgb}{0.933333,0.800000,0.400000}%
\pgfsetfillcolor{currentfill}%
\pgfsetlinewidth{1.003750pt}%
\definecolor{currentstroke}{rgb}{0.600000,0.466667,0.000000}%
\pgfsetstrokecolor{currentstroke}%
\pgfsetdash{}{0pt}%
\pgfpathmoveto{\pgfqpoint{1.633175in}{1.248742in}}%
\pgfpathlineto{\pgfqpoint{1.647915in}{1.248742in}}%
\pgfpathlineto{\pgfqpoint{1.647915in}{1.284175in}}%
\pgfpathlineto{\pgfqpoint{1.633175in}{1.284175in}}%
\pgfpathlineto{\pgfqpoint{1.633175in}{1.248742in}}%
\pgfpathclose%
\pgfusepath{stroke,fill}%
\end{pgfscope}%
\begin{pgfscope}%
\pgfpathrectangle{\pgfqpoint{0.150000in}{0.150000in}}{\pgfqpoint{1.700000in}{1.700000in}}%
\pgfusepath{clip}%
\pgfsetbuttcap%
\pgfsetroundjoin%
\definecolor{currentfill}{rgb}{0.933333,0.800000,0.400000}%
\pgfsetfillcolor{currentfill}%
\pgfsetlinewidth{1.003750pt}%
\definecolor{currentstroke}{rgb}{0.600000,0.466667,0.000000}%
\pgfsetstrokecolor{currentstroke}%
\pgfsetdash{}{0pt}%
\pgfpathmoveto{\pgfqpoint{1.647915in}{1.195811in}}%
\pgfpathlineto{\pgfqpoint{1.665826in}{1.195811in}}%
\pgfpathlineto{\pgfqpoint{1.665826in}{1.248742in}}%
\pgfpathlineto{\pgfqpoint{1.647915in}{1.248742in}}%
\pgfpathlineto{\pgfqpoint{1.647915in}{1.195811in}}%
\pgfpathclose%
\pgfusepath{stroke,fill}%
\end{pgfscope}%
\begin{pgfscope}%
\pgfpathrectangle{\pgfqpoint{0.150000in}{0.150000in}}{\pgfqpoint{1.700000in}{1.700000in}}%
\pgfusepath{clip}%
\pgfsetbuttcap%
\pgfsetroundjoin%
\definecolor{currentfill}{rgb}{0.933333,0.800000,0.400000}%
\pgfsetfillcolor{currentfill}%
\pgfsetlinewidth{1.003750pt}%
\definecolor{currentstroke}{rgb}{0.600000,0.466667,0.000000}%
\pgfsetstrokecolor{currentstroke}%
\pgfsetdash{}{0pt}%
\pgfpathmoveto{\pgfqpoint{1.665826in}{1.152503in}}%
\pgfpathlineto{\pgfqpoint{1.677059in}{1.152503in}}%
\pgfpathlineto{\pgfqpoint{1.677059in}{1.195811in}}%
\pgfpathlineto{\pgfqpoint{1.665826in}{1.195811in}}%
\pgfpathlineto{\pgfqpoint{1.665826in}{1.152503in}}%
\pgfpathclose%
\pgfusepath{stroke,fill}%
\end{pgfscope}%
\begin{pgfscope}%
\pgfpathrectangle{\pgfqpoint{0.150000in}{0.150000in}}{\pgfqpoint{1.700000in}{1.700000in}}%
\pgfusepath{clip}%
\pgfsetbuttcap%
\pgfsetroundjoin%
\definecolor{currentfill}{rgb}{0.933333,0.800000,0.400000}%
\pgfsetfillcolor{currentfill}%
\pgfsetlinewidth{1.003750pt}%
\definecolor{currentstroke}{rgb}{0.600000,0.466667,0.000000}%
\pgfsetstrokecolor{currentstroke}%
\pgfsetdash{}{0pt}%
\pgfpathmoveto{\pgfqpoint{1.677059in}{1.109196in}}%
\pgfpathlineto{\pgfqpoint{1.685378in}{1.109196in}}%
\pgfpathlineto{\pgfqpoint{1.685378in}{1.152503in}}%
\pgfpathlineto{\pgfqpoint{1.677059in}{1.152503in}}%
\pgfpathlineto{\pgfqpoint{1.677059in}{1.109196in}}%
\pgfpathclose%
\pgfusepath{stroke,fill}%
\end{pgfscope}%
\begin{pgfscope}%
\pgfpathrectangle{\pgfqpoint{0.150000in}{0.150000in}}{\pgfqpoint{1.700000in}{1.700000in}}%
\pgfusepath{clip}%
\pgfsetbuttcap%
\pgfsetroundjoin%
\definecolor{currentfill}{rgb}{0.933333,0.800000,0.400000}%
\pgfsetfillcolor{currentfill}%
\pgfsetlinewidth{1.003750pt}%
\definecolor{currentstroke}{rgb}{0.600000,0.466667,0.000000}%
\pgfsetstrokecolor{currentstroke}%
\pgfsetdash{}{0pt}%
\pgfpathmoveto{\pgfqpoint{1.685378in}{1.073763in}}%
\pgfpathlineto{\pgfqpoint{1.690091in}{1.073763in}}%
\pgfpathlineto{\pgfqpoint{1.690091in}{1.109196in}}%
\pgfpathlineto{\pgfqpoint{1.685378in}{1.109196in}}%
\pgfpathlineto{\pgfqpoint{1.685378in}{1.073763in}}%
\pgfpathclose%
\pgfusepath{stroke,fill}%
\end{pgfscope}%
\begin{pgfscope}%
\pgfpathrectangle{\pgfqpoint{0.150000in}{0.150000in}}{\pgfqpoint{1.700000in}{1.700000in}}%
\pgfusepath{clip}%
\pgfsetbuttcap%
\pgfsetroundjoin%
\definecolor{currentfill}{rgb}{0.933333,0.800000,0.400000}%
\pgfsetfillcolor{currentfill}%
\pgfsetlinewidth{1.003750pt}%
\definecolor{currentstroke}{rgb}{0.600000,0.466667,0.000000}%
\pgfsetstrokecolor{currentstroke}%
\pgfsetdash{}{0pt}%
\pgfpathmoveto{\pgfqpoint{1.690091in}{1.030455in}}%
\pgfpathlineto{\pgfqpoint{1.693354in}{1.030455in}}%
\pgfpathlineto{\pgfqpoint{1.693354in}{1.073763in}}%
\pgfpathlineto{\pgfqpoint{1.690091in}{1.073763in}}%
\pgfpathlineto{\pgfqpoint{1.690091in}{1.030455in}}%
\pgfpathclose%
\pgfusepath{stroke,fill}%
\end{pgfscope}%
\begin{pgfscope}%
\pgfpathrectangle{\pgfqpoint{0.150000in}{0.150000in}}{\pgfqpoint{1.700000in}{1.700000in}}%
\pgfusepath{clip}%
\pgfsetbuttcap%
\pgfsetroundjoin%
\definecolor{currentfill}{rgb}{0.933333,0.800000,0.400000}%
\pgfsetfillcolor{currentfill}%
\pgfsetlinewidth{1.003750pt}%
\definecolor{currentstroke}{rgb}{0.600000,0.466667,0.000000}%
\pgfsetstrokecolor{currentstroke}%
\pgfsetdash{}{0pt}%
\pgfpathmoveto{\pgfqpoint{1.693354in}{0.995022in}}%
\pgfpathlineto{\pgfqpoint{1.694022in}{0.995022in}}%
\pgfpathlineto{\pgfqpoint{1.694022in}{1.030455in}}%
\pgfpathlineto{\pgfqpoint{1.693354in}{1.030455in}}%
\pgfpathlineto{\pgfqpoint{1.693354in}{0.995022in}}%
\pgfpathclose%
\pgfusepath{stroke,fill}%
\end{pgfscope}%
\begin{pgfscope}%
\pgfpathrectangle{\pgfqpoint{0.150000in}{0.150000in}}{\pgfqpoint{1.700000in}{1.700000in}}%
\pgfusepath{clip}%
\pgfsetbuttcap%
\pgfsetroundjoin%
\definecolor{currentfill}{rgb}{0.933333,0.800000,0.400000}%
\pgfsetfillcolor{currentfill}%
\pgfsetlinewidth{1.003750pt}%
\definecolor{currentstroke}{rgb}{0.600000,0.466667,0.000000}%
\pgfsetstrokecolor{currentstroke}%
\pgfsetdash{}{0pt}%
\pgfpathmoveto{\pgfqpoint{1.314138in}{1.195811in}}%
\pgfpathlineto{\pgfqpoint{1.349591in}{1.195811in}}%
\pgfpathlineto{\pgfqpoint{1.349591in}{1.248742in}}%
\pgfpathlineto{\pgfqpoint{1.314138in}{1.248742in}}%
\pgfpathlineto{\pgfqpoint{1.314138in}{1.195811in}}%
\pgfpathclose%
\pgfusepath{stroke,fill}%
\end{pgfscope}%
\begin{pgfscope}%
\pgfpathrectangle{\pgfqpoint{0.150000in}{0.150000in}}{\pgfqpoint{1.700000in}{1.700000in}}%
\pgfusepath{clip}%
\pgfsetbuttcap%
\pgfsetroundjoin%
\definecolor{currentfill}{rgb}{0.933333,0.800000,0.400000}%
\pgfsetfillcolor{currentfill}%
\pgfsetlinewidth{1.003750pt}%
\definecolor{currentstroke}{rgb}{0.600000,0.466667,0.000000}%
\pgfsetstrokecolor{currentstroke}%
\pgfsetdash{}{0pt}%
\pgfpathmoveto{\pgfqpoint{1.349591in}{1.152503in}}%
\pgfpathlineto{\pgfqpoint{1.370538in}{1.152503in}}%
\pgfpathlineto{\pgfqpoint{1.370538in}{1.195811in}}%
\pgfpathlineto{\pgfqpoint{1.349591in}{1.195811in}}%
\pgfpathlineto{\pgfqpoint{1.349591in}{1.152503in}}%
\pgfpathclose%
\pgfusepath{stroke,fill}%
\end{pgfscope}%
\begin{pgfscope}%
\pgfpathrectangle{\pgfqpoint{0.150000in}{0.150000in}}{\pgfqpoint{1.700000in}{1.700000in}}%
\pgfusepath{clip}%
\pgfsetbuttcap%
\pgfsetroundjoin%
\definecolor{currentfill}{rgb}{0.933333,0.800000,0.400000}%
\pgfsetfillcolor{currentfill}%
\pgfsetlinewidth{1.003750pt}%
\definecolor{currentstroke}{rgb}{0.600000,0.466667,0.000000}%
\pgfsetstrokecolor{currentstroke}%
\pgfsetdash{}{0pt}%
\pgfpathmoveto{\pgfqpoint{1.370538in}{1.109196in}}%
\pgfpathlineto{\pgfqpoint{1.385528in}{1.109196in}}%
\pgfpathlineto{\pgfqpoint{1.385528in}{1.152503in}}%
\pgfpathlineto{\pgfqpoint{1.370538in}{1.152503in}}%
\pgfpathlineto{\pgfqpoint{1.370538in}{1.109196in}}%
\pgfpathclose%
\pgfusepath{stroke,fill}%
\end{pgfscope}%
\begin{pgfscope}%
\pgfpathrectangle{\pgfqpoint{0.150000in}{0.150000in}}{\pgfqpoint{1.700000in}{1.700000in}}%
\pgfusepath{clip}%
\pgfsetbuttcap%
\pgfsetroundjoin%
\definecolor{currentfill}{rgb}{0.933333,0.800000,0.400000}%
\pgfsetfillcolor{currentfill}%
\pgfsetlinewidth{1.003750pt}%
\definecolor{currentstroke}{rgb}{0.600000,0.466667,0.000000}%
\pgfsetstrokecolor{currentstroke}%
\pgfsetdash{}{0pt}%
\pgfpathmoveto{\pgfqpoint{1.385528in}{1.073763in}}%
\pgfpathlineto{\pgfqpoint{1.393846in}{1.073763in}}%
\pgfpathlineto{\pgfqpoint{1.393846in}{1.109196in}}%
\pgfpathlineto{\pgfqpoint{1.385528in}{1.109196in}}%
\pgfpathlineto{\pgfqpoint{1.385528in}{1.073763in}}%
\pgfpathclose%
\pgfusepath{stroke,fill}%
\end{pgfscope}%
\begin{pgfscope}%
\pgfpathrectangle{\pgfqpoint{0.150000in}{0.150000in}}{\pgfqpoint{1.700000in}{1.700000in}}%
\pgfusepath{clip}%
\pgfsetbuttcap%
\pgfsetroundjoin%
\definecolor{currentfill}{rgb}{0.933333,0.800000,0.400000}%
\pgfsetfillcolor{currentfill}%
\pgfsetlinewidth{1.003750pt}%
\definecolor{currentstroke}{rgb}{0.600000,0.466667,0.000000}%
\pgfsetstrokecolor{currentstroke}%
\pgfsetdash{}{0pt}%
\pgfpathmoveto{\pgfqpoint{1.393846in}{1.030455in}}%
\pgfpathlineto{\pgfqpoint{1.399535in}{1.030455in}}%
\pgfpathlineto{\pgfqpoint{1.399535in}{1.073763in}}%
\pgfpathlineto{\pgfqpoint{1.393846in}{1.073763in}}%
\pgfpathlineto{\pgfqpoint{1.393846in}{1.030455in}}%
\pgfpathclose%
\pgfusepath{stroke,fill}%
\end{pgfscope}%
\begin{pgfscope}%
\pgfpathrectangle{\pgfqpoint{0.150000in}{0.150000in}}{\pgfqpoint{1.700000in}{1.700000in}}%
\pgfusepath{clip}%
\pgfsetbuttcap%
\pgfsetroundjoin%
\definecolor{currentfill}{rgb}{0.933333,0.800000,0.400000}%
\pgfsetfillcolor{currentfill}%
\pgfsetlinewidth{1.003750pt}%
\definecolor{currentstroke}{rgb}{0.600000,0.466667,0.000000}%
\pgfsetstrokecolor{currentstroke}%
\pgfsetdash{}{0pt}%
\pgfpathmoveto{\pgfqpoint{1.399535in}{0.995022in}}%
\pgfpathlineto{\pgfqpoint{1.400694in}{0.995022in}}%
\pgfpathlineto{\pgfqpoint{1.400694in}{1.030455in}}%
\pgfpathlineto{\pgfqpoint{1.399535in}{1.030455in}}%
\pgfpathlineto{\pgfqpoint{1.399535in}{0.995022in}}%
\pgfpathclose%
\pgfusepath{stroke,fill}%
\end{pgfscope}%
\begin{pgfscope}%
\pgfpathrectangle{\pgfqpoint{0.150000in}{0.150000in}}{\pgfqpoint{1.700000in}{1.700000in}}%
\pgfusepath{clip}%
\pgfsetbuttcap%
\pgfsetroundjoin%
\definecolor{currentfill}{rgb}{0.933333,0.800000,0.400000}%
\pgfsetfillcolor{currentfill}%
\pgfsetlinewidth{1.003750pt}%
\definecolor{currentstroke}{rgb}{0.600000,0.466667,0.000000}%
\pgfsetstrokecolor{currentstroke}%
\pgfsetdash{}{0pt}%
\pgfpathmoveto{\pgfqpoint{1.683312in}{0.878544in}}%
\pgfpathlineto{\pgfqpoint{1.690543in}{0.878544in}}%
\pgfpathlineto{\pgfqpoint{1.690543in}{0.930598in}}%
\pgfpathlineto{\pgfqpoint{1.683312in}{0.930598in}}%
\pgfpathlineto{\pgfqpoint{1.683312in}{0.878544in}}%
\pgfpathclose%
\pgfusepath{stroke,fill}%
\end{pgfscope}%
\begin{pgfscope}%
\pgfpathrectangle{\pgfqpoint{0.150000in}{0.150000in}}{\pgfqpoint{1.700000in}{1.700000in}}%
\pgfusepath{clip}%
\pgfsetbuttcap%
\pgfsetroundjoin%
\definecolor{currentfill}{rgb}{0.933333,0.800000,0.400000}%
\pgfsetfillcolor{currentfill}%
\pgfsetlinewidth{1.003750pt}%
\definecolor{currentstroke}{rgb}{0.600000,0.466667,0.000000}%
\pgfsetstrokecolor{currentstroke}%
\pgfsetdash{}{0pt}%
\pgfpathmoveto{\pgfqpoint{1.674356in}{0.835955in}}%
\pgfpathlineto{\pgfqpoint{1.683312in}{0.835955in}}%
\pgfpathlineto{\pgfqpoint{1.683312in}{0.878544in}}%
\pgfpathlineto{\pgfqpoint{1.674356in}{0.878544in}}%
\pgfpathlineto{\pgfqpoint{1.674356in}{0.835955in}}%
\pgfpathclose%
\pgfusepath{stroke,fill}%
\end{pgfscope}%
\begin{pgfscope}%
\pgfpathrectangle{\pgfqpoint{0.150000in}{0.150000in}}{\pgfqpoint{1.700000in}{1.700000in}}%
\pgfusepath{clip}%
\pgfsetbuttcap%
\pgfsetroundjoin%
\definecolor{currentfill}{rgb}{0.933333,0.800000,0.400000}%
\pgfsetfillcolor{currentfill}%
\pgfsetlinewidth{1.003750pt}%
\definecolor{currentstroke}{rgb}{0.600000,0.466667,0.000000}%
\pgfsetstrokecolor{currentstroke}%
\pgfsetdash{}{0pt}%
\pgfpathmoveto{\pgfqpoint{1.662547in}{0.793365in}}%
\pgfpathlineto{\pgfqpoint{1.674356in}{0.793365in}}%
\pgfpathlineto{\pgfqpoint{1.674356in}{0.835955in}}%
\pgfpathlineto{\pgfqpoint{1.662547in}{0.835955in}}%
\pgfpathlineto{\pgfqpoint{1.662547in}{0.793365in}}%
\pgfpathclose%
\pgfusepath{stroke,fill}%
\end{pgfscope}%
\begin{pgfscope}%
\pgfpathrectangle{\pgfqpoint{0.150000in}{0.150000in}}{\pgfqpoint{1.700000in}{1.700000in}}%
\pgfusepath{clip}%
\pgfsetbuttcap%
\pgfsetroundjoin%
\definecolor{currentfill}{rgb}{0.933333,0.800000,0.400000}%
\pgfsetfillcolor{currentfill}%
\pgfsetlinewidth{1.003750pt}%
\definecolor{currentstroke}{rgb}{0.600000,0.466667,0.000000}%
\pgfsetstrokecolor{currentstroke}%
\pgfsetdash{}{0pt}%
\pgfpathmoveto{\pgfqpoint{1.650656in}{0.758520in}}%
\pgfpathlineto{\pgfqpoint{1.662547in}{0.758520in}}%
\pgfpathlineto{\pgfqpoint{1.662547in}{0.793365in}}%
\pgfpathlineto{\pgfqpoint{1.650656in}{0.793365in}}%
\pgfpathlineto{\pgfqpoint{1.650656in}{0.758520in}}%
\pgfpathclose%
\pgfusepath{stroke,fill}%
\end{pgfscope}%
\begin{pgfscope}%
\pgfpathrectangle{\pgfqpoint{0.150000in}{0.150000in}}{\pgfqpoint{1.700000in}{1.700000in}}%
\pgfusepath{clip}%
\pgfsetbuttcap%
\pgfsetroundjoin%
\definecolor{currentfill}{rgb}{0.933333,0.800000,0.400000}%
\pgfsetfillcolor{currentfill}%
\pgfsetlinewidth{1.003750pt}%
\definecolor{currentstroke}{rgb}{0.600000,0.466667,0.000000}%
\pgfsetstrokecolor{currentstroke}%
\pgfsetdash{}{0pt}%
\pgfpathmoveto{\pgfqpoint{1.633223in}{0.715930in}}%
\pgfpathlineto{\pgfqpoint{1.650656in}{0.715930in}}%
\pgfpathlineto{\pgfqpoint{1.650656in}{0.758520in}}%
\pgfpathlineto{\pgfqpoint{1.633223in}{0.758520in}}%
\pgfpathlineto{\pgfqpoint{1.633223in}{0.715930in}}%
\pgfpathclose%
\pgfusepath{stroke,fill}%
\end{pgfscope}%
\begin{pgfscope}%
\pgfpathrectangle{\pgfqpoint{0.150000in}{0.150000in}}{\pgfqpoint{1.700000in}{1.700000in}}%
\pgfusepath{clip}%
\pgfsetbuttcap%
\pgfsetroundjoin%
\definecolor{currentfill}{rgb}{0.933333,0.800000,0.400000}%
\pgfsetfillcolor{currentfill}%
\pgfsetlinewidth{1.003750pt}%
\definecolor{currentstroke}{rgb}{0.600000,0.466667,0.000000}%
\pgfsetstrokecolor{currentstroke}%
\pgfsetdash{}{0pt}%
\pgfpathmoveto{\pgfqpoint{1.616409in}{0.681084in}}%
\pgfpathlineto{\pgfqpoint{1.633223in}{0.681084in}}%
\pgfpathlineto{\pgfqpoint{1.633223in}{0.715930in}}%
\pgfpathlineto{\pgfqpoint{1.616409in}{0.715930in}}%
\pgfpathlineto{\pgfqpoint{1.616409in}{0.681084in}}%
\pgfpathclose%
\pgfusepath{stroke,fill}%
\end{pgfscope}%
\begin{pgfscope}%
\pgfpathrectangle{\pgfqpoint{0.150000in}{0.150000in}}{\pgfqpoint{1.700000in}{1.700000in}}%
\pgfusepath{clip}%
\pgfsetbuttcap%
\pgfsetroundjoin%
\definecolor{currentfill}{rgb}{0.933333,0.800000,0.400000}%
\pgfsetfillcolor{currentfill}%
\pgfsetlinewidth{1.003750pt}%
\definecolor{currentstroke}{rgb}{0.600000,0.466667,0.000000}%
\pgfsetstrokecolor{currentstroke}%
\pgfsetdash{}{0pt}%
\pgfpathmoveto{\pgfqpoint{0.878552in}{1.683313in}}%
\pgfpathlineto{\pgfqpoint{0.930598in}{1.683313in}}%
\pgfpathlineto{\pgfqpoint{0.930598in}{1.690543in}}%
\pgfpathlineto{\pgfqpoint{0.878552in}{1.690543in}}%
\pgfpathlineto{\pgfqpoint{0.878552in}{1.683313in}}%
\pgfpathclose%
\pgfusepath{stroke,fill}%
\end{pgfscope}%
\begin{pgfscope}%
\pgfpathrectangle{\pgfqpoint{0.150000in}{0.150000in}}{\pgfqpoint{1.700000in}{1.700000in}}%
\pgfusepath{clip}%
\pgfsetbuttcap%
\pgfsetroundjoin%
\definecolor{currentfill}{rgb}{0.933333,0.800000,0.400000}%
\pgfsetfillcolor{currentfill}%
\pgfsetlinewidth{1.003750pt}%
\definecolor{currentstroke}{rgb}{0.600000,0.466667,0.000000}%
\pgfsetstrokecolor{currentstroke}%
\pgfsetdash{}{0pt}%
\pgfpathmoveto{\pgfqpoint{0.835968in}{1.674359in}}%
\pgfpathlineto{\pgfqpoint{0.878552in}{1.674359in}}%
\pgfpathlineto{\pgfqpoint{0.878552in}{1.683313in}}%
\pgfpathlineto{\pgfqpoint{0.835968in}{1.683313in}}%
\pgfpathlineto{\pgfqpoint{0.835968in}{1.674359in}}%
\pgfpathclose%
\pgfusepath{stroke,fill}%
\end{pgfscope}%
\begin{pgfscope}%
\pgfpathrectangle{\pgfqpoint{0.150000in}{0.150000in}}{\pgfqpoint{1.700000in}{1.700000in}}%
\pgfusepath{clip}%
\pgfsetbuttcap%
\pgfsetroundjoin%
\definecolor{currentfill}{rgb}{0.933333,0.800000,0.400000}%
\pgfsetfillcolor{currentfill}%
\pgfsetlinewidth{1.003750pt}%
\definecolor{currentstroke}{rgb}{0.600000,0.466667,0.000000}%
\pgfsetstrokecolor{currentstroke}%
\pgfsetdash{}{0pt}%
\pgfpathmoveto{\pgfqpoint{0.793385in}{1.662553in}}%
\pgfpathlineto{\pgfqpoint{0.835968in}{1.662553in}}%
\pgfpathlineto{\pgfqpoint{0.835968in}{1.674359in}}%
\pgfpathlineto{\pgfqpoint{0.793385in}{1.674359in}}%
\pgfpathlineto{\pgfqpoint{0.793385in}{1.662553in}}%
\pgfpathclose%
\pgfusepath{stroke,fill}%
\end{pgfscope}%
\begin{pgfscope}%
\pgfpathrectangle{\pgfqpoint{0.150000in}{0.150000in}}{\pgfqpoint{1.700000in}{1.700000in}}%
\pgfusepath{clip}%
\pgfsetbuttcap%
\pgfsetroundjoin%
\definecolor{currentfill}{rgb}{0.933333,0.800000,0.400000}%
\pgfsetfillcolor{currentfill}%
\pgfsetlinewidth{1.003750pt}%
\definecolor{currentstroke}{rgb}{0.600000,0.466667,0.000000}%
\pgfsetstrokecolor{currentstroke}%
\pgfsetdash{}{0pt}%
\pgfpathmoveto{\pgfqpoint{0.758544in}{1.650666in}}%
\pgfpathlineto{\pgfqpoint{0.793385in}{1.650666in}}%
\pgfpathlineto{\pgfqpoint{0.793385in}{1.662553in}}%
\pgfpathlineto{\pgfqpoint{0.758544in}{1.662553in}}%
\pgfpathlineto{\pgfqpoint{0.758544in}{1.650666in}}%
\pgfpathclose%
\pgfusepath{stroke,fill}%
\end{pgfscope}%
\begin{pgfscope}%
\pgfpathrectangle{\pgfqpoint{0.150000in}{0.150000in}}{\pgfqpoint{1.700000in}{1.700000in}}%
\pgfusepath{clip}%
\pgfsetbuttcap%
\pgfsetroundjoin%
\definecolor{currentfill}{rgb}{0.933333,0.800000,0.400000}%
\pgfsetfillcolor{currentfill}%
\pgfsetlinewidth{1.003750pt}%
\definecolor{currentstroke}{rgb}{0.600000,0.466667,0.000000}%
\pgfsetstrokecolor{currentstroke}%
\pgfsetdash{}{0pt}%
\pgfpathmoveto{\pgfqpoint{0.715961in}{1.633237in}}%
\pgfpathlineto{\pgfqpoint{0.758544in}{1.633237in}}%
\pgfpathlineto{\pgfqpoint{0.758544in}{1.650666in}}%
\pgfpathlineto{\pgfqpoint{0.715961in}{1.650666in}}%
\pgfpathlineto{\pgfqpoint{0.715961in}{1.633237in}}%
\pgfpathclose%
\pgfusepath{stroke,fill}%
\end{pgfscope}%
\begin{pgfscope}%
\pgfpathrectangle{\pgfqpoint{0.150000in}{0.150000in}}{\pgfqpoint{1.700000in}{1.700000in}}%
\pgfusepath{clip}%
\pgfsetbuttcap%
\pgfsetroundjoin%
\definecolor{currentfill}{rgb}{0.933333,0.800000,0.400000}%
\pgfsetfillcolor{currentfill}%
\pgfsetlinewidth{1.003750pt}%
\definecolor{currentstroke}{rgb}{0.600000,0.466667,0.000000}%
\pgfsetstrokecolor{currentstroke}%
\pgfsetdash{}{0pt}%
\pgfpathmoveto{\pgfqpoint{0.681120in}{1.616427in}}%
\pgfpathlineto{\pgfqpoint{0.715961in}{1.616427in}}%
\pgfpathlineto{\pgfqpoint{0.715961in}{1.633237in}}%
\pgfpathlineto{\pgfqpoint{0.681120in}{1.633237in}}%
\pgfpathlineto{\pgfqpoint{0.681120in}{1.616427in}}%
\pgfpathclose%
\pgfusepath{stroke,fill}%
\end{pgfscope}%
\begin{pgfscope}%
\pgfpathrectangle{\pgfqpoint{0.150000in}{0.150000in}}{\pgfqpoint{1.700000in}{1.700000in}}%
\pgfusepath{clip}%
\pgfsetbuttcap%
\pgfsetroundjoin%
\definecolor{currentfill}{rgb}{0.933333,0.800000,0.400000}%
\pgfsetfillcolor{currentfill}%
\pgfsetlinewidth{1.003750pt}%
\definecolor{currentstroke}{rgb}{0.600000,0.466667,0.000000}%
\pgfsetstrokecolor{currentstroke}%
\pgfsetdash{}{0pt}%
\pgfpathmoveto{\pgfqpoint{0.889799in}{0.605362in}}%
\pgfpathlineto{\pgfqpoint{0.930598in}{0.605362in}}%
\pgfpathlineto{\pgfqpoint{0.930598in}{0.614758in}}%
\pgfpathlineto{\pgfqpoint{0.889799in}{0.614758in}}%
\pgfpathlineto{\pgfqpoint{0.889799in}{0.605362in}}%
\pgfpathclose%
\pgfusepath{stroke,fill}%
\end{pgfscope}%
\begin{pgfscope}%
\pgfpathrectangle{\pgfqpoint{0.150000in}{0.150000in}}{\pgfqpoint{1.700000in}{1.700000in}}%
\pgfusepath{clip}%
\pgfsetbuttcap%
\pgfsetroundjoin%
\definecolor{currentfill}{rgb}{0.933333,0.800000,0.400000}%
\pgfsetfillcolor{currentfill}%
\pgfsetlinewidth{1.003750pt}%
\definecolor{currentstroke}{rgb}{0.600000,0.466667,0.000000}%
\pgfsetstrokecolor{currentstroke}%
\pgfsetdash{}{0pt}%
\pgfpathmoveto{\pgfqpoint{0.856418in}{0.614758in}}%
\pgfpathlineto{\pgfqpoint{0.889799in}{0.614758in}}%
\pgfpathlineto{\pgfqpoint{0.889799in}{0.625915in}}%
\pgfpathlineto{\pgfqpoint{0.856418in}{0.625915in}}%
\pgfpathlineto{\pgfqpoint{0.856418in}{0.614758in}}%
\pgfpathclose%
\pgfusepath{stroke,fill}%
\end{pgfscope}%
\begin{pgfscope}%
\pgfpathrectangle{\pgfqpoint{0.150000in}{0.150000in}}{\pgfqpoint{1.700000in}{1.700000in}}%
\pgfusepath{clip}%
\pgfsetbuttcap%
\pgfsetroundjoin%
\definecolor{currentfill}{rgb}{0.933333,0.800000,0.400000}%
\pgfsetfillcolor{currentfill}%
\pgfsetlinewidth{1.003750pt}%
\definecolor{currentstroke}{rgb}{0.600000,0.466667,0.000000}%
\pgfsetstrokecolor{currentstroke}%
\pgfsetdash{}{0pt}%
\pgfpathmoveto{\pgfqpoint{1.398124in}{1.514020in}}%
\pgfpathlineto{\pgfqpoint{1.466315in}{1.514020in}}%
\pgfpathlineto{\pgfqpoint{1.466315in}{1.568475in}}%
\pgfpathlineto{\pgfqpoint{1.398124in}{1.568475in}}%
\pgfpathlineto{\pgfqpoint{1.398124in}{1.514020in}}%
\pgfpathclose%
\pgfusepath{stroke,fill}%
\end{pgfscope}%
\begin{pgfscope}%
\pgfpathrectangle{\pgfqpoint{0.150000in}{0.150000in}}{\pgfqpoint{1.700000in}{1.700000in}}%
\pgfusepath{clip}%
\pgfsetbuttcap%
\pgfsetroundjoin%
\definecolor{currentfill}{rgb}{0.933333,0.800000,0.400000}%
\pgfsetfillcolor{currentfill}%
\pgfsetlinewidth{1.003750pt}%
\definecolor{currentstroke}{rgb}{0.600000,0.466667,0.000000}%
\pgfsetstrokecolor{currentstroke}%
\pgfsetdash{}{0pt}%
\pgfpathmoveto{\pgfqpoint{1.329932in}{1.568475in}}%
\pgfpathlineto{\pgfqpoint{1.398124in}{1.568475in}}%
\pgfpathlineto{\pgfqpoint{1.398124in}{1.610583in}}%
\pgfpathlineto{\pgfqpoint{1.329932in}{1.610583in}}%
\pgfpathlineto{\pgfqpoint{1.329932in}{1.568475in}}%
\pgfpathclose%
\pgfusepath{stroke,fill}%
\end{pgfscope}%
\begin{pgfscope}%
\pgfpathrectangle{\pgfqpoint{0.150000in}{0.150000in}}{\pgfqpoint{1.700000in}{1.700000in}}%
\pgfusepath{clip}%
\pgfsetbuttcap%
\pgfsetroundjoin%
\definecolor{currentfill}{rgb}{0.933333,0.800000,0.400000}%
\pgfsetfillcolor{currentfill}%
\pgfsetlinewidth{1.003750pt}%
\definecolor{currentstroke}{rgb}{0.600000,0.466667,0.000000}%
\pgfsetstrokecolor{currentstroke}%
\pgfsetdash{}{0pt}%
\pgfpathmoveto{\pgfqpoint{1.274139in}{1.610583in}}%
\pgfpathlineto{\pgfqpoint{1.329932in}{1.610583in}}%
\pgfpathlineto{\pgfqpoint{1.329932in}{1.637585in}}%
\pgfpathlineto{\pgfqpoint{1.274139in}{1.637585in}}%
\pgfpathlineto{\pgfqpoint{1.274139in}{1.610583in}}%
\pgfpathclose%
\pgfusepath{stroke,fill}%
\end{pgfscope}%
\begin{pgfscope}%
\pgfpathrectangle{\pgfqpoint{0.150000in}{0.150000in}}{\pgfqpoint{1.700000in}{1.700000in}}%
\pgfusepath{clip}%
\pgfsetbuttcap%
\pgfsetroundjoin%
\definecolor{currentfill}{rgb}{0.933333,0.800000,0.400000}%
\pgfsetfillcolor{currentfill}%
\pgfsetlinewidth{1.003750pt}%
\definecolor{currentstroke}{rgb}{0.600000,0.466667,0.000000}%
\pgfsetstrokecolor{currentstroke}%
\pgfsetdash{}{0pt}%
\pgfpathmoveto{\pgfqpoint{1.690543in}{0.930598in}}%
\pgfpathlineto{\pgfqpoint{1.694004in}{0.930598in}}%
\pgfpathlineto{\pgfqpoint{1.694004in}{0.995022in}}%
\pgfpathlineto{\pgfqpoint{1.690543in}{0.995022in}}%
\pgfpathlineto{\pgfqpoint{1.690543in}{0.930598in}}%
\pgfpathclose%
\pgfusepath{stroke,fill}%
\end{pgfscope}%
\begin{pgfscope}%
\pgfpathrectangle{\pgfqpoint{0.150000in}{0.150000in}}{\pgfqpoint{1.700000in}{1.700000in}}%
\pgfusepath{clip}%
\pgfsetbuttcap%
\pgfsetroundjoin%
\definecolor{currentfill}{rgb}{0.933333,0.800000,0.400000}%
\pgfsetfillcolor{currentfill}%
\pgfsetlinewidth{1.003750pt}%
\definecolor{currentstroke}{rgb}{0.600000,0.466667,0.000000}%
\pgfsetstrokecolor{currentstroke}%
\pgfsetdash{}{0pt}%
\pgfpathmoveto{\pgfqpoint{1.394638in}{0.930598in}}%
\pgfpathlineto{\pgfqpoint{1.400663in}{0.930598in}}%
\pgfpathlineto{\pgfqpoint{1.400663in}{0.995022in}}%
\pgfpathlineto{\pgfqpoint{1.394638in}{0.995022in}}%
\pgfpathlineto{\pgfqpoint{1.394638in}{0.930598in}}%
\pgfpathclose%
\pgfusepath{stroke,fill}%
\end{pgfscope}%
\begin{pgfscope}%
\pgfpathrectangle{\pgfqpoint{0.150000in}{0.150000in}}{\pgfqpoint{1.700000in}{1.700000in}}%
\pgfusepath{clip}%
\pgfsetbuttcap%
\pgfsetroundjoin%
\definecolor{currentfill}{rgb}{0.933333,0.800000,0.400000}%
\pgfsetfillcolor{currentfill}%
\pgfsetlinewidth{1.003750pt}%
\definecolor{currentstroke}{rgb}{0.600000,0.466667,0.000000}%
\pgfsetstrokecolor{currentstroke}%
\pgfsetdash{}{0pt}%
\pgfpathmoveto{\pgfqpoint{1.216982in}{1.637585in}}%
\pgfpathlineto{\pgfqpoint{1.274139in}{1.637585in}}%
\pgfpathlineto{\pgfqpoint{1.274139in}{1.659231in}}%
\pgfpathlineto{\pgfqpoint{1.216982in}{1.659231in}}%
\pgfpathlineto{\pgfqpoint{1.216982in}{1.637585in}}%
\pgfpathclose%
\pgfusepath{stroke,fill}%
\end{pgfscope}%
\begin{pgfscope}%
\pgfpathrectangle{\pgfqpoint{0.150000in}{0.150000in}}{\pgfqpoint{1.700000in}{1.700000in}}%
\pgfusepath{clip}%
\pgfsetbuttcap%
\pgfsetroundjoin%
\definecolor{currentfill}{rgb}{0.933333,0.800000,0.400000}%
\pgfsetfillcolor{currentfill}%
\pgfsetlinewidth{1.003750pt}%
\definecolor{currentstroke}{rgb}{0.600000,0.466667,0.000000}%
\pgfsetstrokecolor{currentstroke}%
\pgfsetdash{}{0pt}%
\pgfpathmoveto{\pgfqpoint{1.170218in}{1.659231in}}%
\pgfpathlineto{\pgfqpoint{1.216982in}{1.659231in}}%
\pgfpathlineto{\pgfqpoint{1.216982in}{1.672824in}}%
\pgfpathlineto{\pgfqpoint{1.170218in}{1.672824in}}%
\pgfpathlineto{\pgfqpoint{1.170218in}{1.659231in}}%
\pgfpathclose%
\pgfusepath{stroke,fill}%
\end{pgfscope}%
\begin{pgfscope}%
\pgfpathrectangle{\pgfqpoint{0.150000in}{0.150000in}}{\pgfqpoint{1.700000in}{1.700000in}}%
\pgfusepath{clip}%
\pgfsetbuttcap%
\pgfsetroundjoin%
\definecolor{currentfill}{rgb}{0.933333,0.800000,0.400000}%
\pgfsetfillcolor{currentfill}%
\pgfsetlinewidth{1.003750pt}%
\definecolor{currentstroke}{rgb}{0.600000,0.466667,0.000000}%
\pgfsetstrokecolor{currentstroke}%
\pgfsetdash{}{0pt}%
\pgfpathmoveto{\pgfqpoint{1.123453in}{1.672824in}}%
\pgfpathlineto{\pgfqpoint{1.170218in}{1.672824in}}%
\pgfpathlineto{\pgfqpoint{1.170218in}{1.682954in}}%
\pgfpathlineto{\pgfqpoint{1.123453in}{1.682954in}}%
\pgfpathlineto{\pgfqpoint{1.123453in}{1.672824in}}%
\pgfpathclose%
\pgfusepath{stroke,fill}%
\end{pgfscope}%
\begin{pgfscope}%
\pgfpathrectangle{\pgfqpoint{0.150000in}{0.150000in}}{\pgfqpoint{1.700000in}{1.700000in}}%
\pgfusepath{clip}%
\pgfsetbuttcap%
\pgfsetroundjoin%
\definecolor{currentfill}{rgb}{0.933333,0.800000,0.400000}%
\pgfsetfillcolor{currentfill}%
\pgfsetlinewidth{1.003750pt}%
\definecolor{currentstroke}{rgb}{0.600000,0.466667,0.000000}%
\pgfsetstrokecolor{currentstroke}%
\pgfsetdash{}{0pt}%
\pgfpathmoveto{\pgfqpoint{1.085191in}{1.682954in}}%
\pgfpathlineto{\pgfqpoint{1.123453in}{1.682954in}}%
\pgfpathlineto{\pgfqpoint{1.123453in}{1.688774in}}%
\pgfpathlineto{\pgfqpoint{1.085191in}{1.688774in}}%
\pgfpathlineto{\pgfqpoint{1.085191in}{1.682954in}}%
\pgfpathclose%
\pgfusepath{stroke,fill}%
\end{pgfscope}%
\begin{pgfscope}%
\pgfpathrectangle{\pgfqpoint{0.150000in}{0.150000in}}{\pgfqpoint{1.700000in}{1.700000in}}%
\pgfusepath{clip}%
\pgfsetbuttcap%
\pgfsetroundjoin%
\definecolor{currentfill}{rgb}{0.933333,0.800000,0.400000}%
\pgfsetfillcolor{currentfill}%
\pgfsetlinewidth{1.003750pt}%
\definecolor{currentstroke}{rgb}{0.600000,0.466667,0.000000}%
\pgfsetstrokecolor{currentstroke}%
\pgfsetdash{}{0pt}%
\pgfpathmoveto{\pgfqpoint{1.038427in}{1.688774in}}%
\pgfpathlineto{\pgfqpoint{1.085191in}{1.688774in}}%
\pgfpathlineto{\pgfqpoint{1.085191in}{1.692957in}}%
\pgfpathlineto{\pgfqpoint{1.038427in}{1.692957in}}%
\pgfpathlineto{\pgfqpoint{1.038427in}{1.688774in}}%
\pgfpathclose%
\pgfusepath{stroke,fill}%
\end{pgfscope}%
\begin{pgfscope}%
\pgfpathrectangle{\pgfqpoint{0.150000in}{0.150000in}}{\pgfqpoint{1.700000in}{1.700000in}}%
\pgfusepath{clip}%
\pgfsetbuttcap%
\pgfsetroundjoin%
\definecolor{currentfill}{rgb}{0.933333,0.800000,0.400000}%
\pgfsetfillcolor{currentfill}%
\pgfsetlinewidth{1.003750pt}%
\definecolor{currentstroke}{rgb}{0.600000,0.466667,0.000000}%
\pgfsetstrokecolor{currentstroke}%
\pgfsetdash{}{0pt}%
\pgfpathmoveto{\pgfqpoint{1.000165in}{1.692957in}}%
\pgfpathlineto{\pgfqpoint{1.038427in}{1.692957in}}%
\pgfpathlineto{\pgfqpoint{1.038427in}{1.694022in}}%
\pgfpathlineto{\pgfqpoint{1.000165in}{1.694022in}}%
\pgfpathlineto{\pgfqpoint{1.000165in}{1.692957in}}%
\pgfpathclose%
\pgfusepath{stroke,fill}%
\end{pgfscope}%
\begin{pgfscope}%
\pgfpathrectangle{\pgfqpoint{0.150000in}{0.150000in}}{\pgfqpoint{1.700000in}{1.700000in}}%
\pgfusepath{clip}%
\pgfsetbuttcap%
\pgfsetroundjoin%
\definecolor{currentfill}{rgb}{0.933333,0.800000,0.400000}%
\pgfsetfillcolor{currentfill}%
\pgfsetlinewidth{1.003750pt}%
\definecolor{currentstroke}{rgb}{0.600000,0.466667,0.000000}%
\pgfsetstrokecolor{currentstroke}%
\pgfsetdash{}{0pt}%
\pgfpathmoveto{\pgfqpoint{1.216982in}{1.292239in}}%
\pgfpathlineto{\pgfqpoint{1.274139in}{1.292239in}}%
\pgfpathlineto{\pgfqpoint{1.274139in}{1.336859in}}%
\pgfpathlineto{\pgfqpoint{1.216982in}{1.336859in}}%
\pgfpathlineto{\pgfqpoint{1.216982in}{1.292239in}}%
\pgfpathclose%
\pgfusepath{stroke,fill}%
\end{pgfscope}%
\begin{pgfscope}%
\pgfpathrectangle{\pgfqpoint{0.150000in}{0.150000in}}{\pgfqpoint{1.700000in}{1.700000in}}%
\pgfusepath{clip}%
\pgfsetbuttcap%
\pgfsetroundjoin%
\definecolor{currentfill}{rgb}{0.933333,0.800000,0.400000}%
\pgfsetfillcolor{currentfill}%
\pgfsetlinewidth{1.003750pt}%
\definecolor{currentstroke}{rgb}{0.600000,0.466667,0.000000}%
\pgfsetstrokecolor{currentstroke}%
\pgfsetdash{}{0pt}%
\pgfpathmoveto{\pgfqpoint{1.170218in}{1.336859in}}%
\pgfpathlineto{\pgfqpoint{1.216982in}{1.336859in}}%
\pgfpathlineto{\pgfqpoint{1.216982in}{1.362742in}}%
\pgfpathlineto{\pgfqpoint{1.170218in}{1.362742in}}%
\pgfpathlineto{\pgfqpoint{1.170218in}{1.336859in}}%
\pgfpathclose%
\pgfusepath{stroke,fill}%
\end{pgfscope}%
\begin{pgfscope}%
\pgfpathrectangle{\pgfqpoint{0.150000in}{0.150000in}}{\pgfqpoint{1.700000in}{1.700000in}}%
\pgfusepath{clip}%
\pgfsetbuttcap%
\pgfsetroundjoin%
\definecolor{currentfill}{rgb}{0.933333,0.800000,0.400000}%
\pgfsetfillcolor{currentfill}%
\pgfsetlinewidth{1.003750pt}%
\definecolor{currentstroke}{rgb}{0.600000,0.466667,0.000000}%
\pgfsetstrokecolor{currentstroke}%
\pgfsetdash{}{0pt}%
\pgfpathmoveto{\pgfqpoint{1.123453in}{1.362742in}}%
\pgfpathlineto{\pgfqpoint{1.170218in}{1.362742in}}%
\pgfpathlineto{\pgfqpoint{1.170218in}{1.381202in}}%
\pgfpathlineto{\pgfqpoint{1.123453in}{1.381202in}}%
\pgfpathlineto{\pgfqpoint{1.123453in}{1.362742in}}%
\pgfpathclose%
\pgfusepath{stroke,fill}%
\end{pgfscope}%
\begin{pgfscope}%
\pgfpathrectangle{\pgfqpoint{0.150000in}{0.150000in}}{\pgfqpoint{1.700000in}{1.700000in}}%
\pgfusepath{clip}%
\pgfsetbuttcap%
\pgfsetroundjoin%
\definecolor{currentfill}{rgb}{0.933333,0.800000,0.400000}%
\pgfsetfillcolor{currentfill}%
\pgfsetlinewidth{1.003750pt}%
\definecolor{currentstroke}{rgb}{0.600000,0.466667,0.000000}%
\pgfsetstrokecolor{currentstroke}%
\pgfsetdash{}{0pt}%
\pgfpathmoveto{\pgfqpoint{1.085191in}{1.381202in}}%
\pgfpathlineto{\pgfqpoint{1.123453in}{1.381202in}}%
\pgfpathlineto{\pgfqpoint{1.123453in}{1.391533in}}%
\pgfpathlineto{\pgfqpoint{1.085191in}{1.391533in}}%
\pgfpathlineto{\pgfqpoint{1.085191in}{1.381202in}}%
\pgfpathclose%
\pgfusepath{stroke,fill}%
\end{pgfscope}%
\begin{pgfscope}%
\pgfpathrectangle{\pgfqpoint{0.150000in}{0.150000in}}{\pgfqpoint{1.700000in}{1.700000in}}%
\pgfusepath{clip}%
\pgfsetbuttcap%
\pgfsetroundjoin%
\definecolor{currentfill}{rgb}{0.933333,0.800000,0.400000}%
\pgfsetfillcolor{currentfill}%
\pgfsetlinewidth{1.003750pt}%
\definecolor{currentstroke}{rgb}{0.600000,0.466667,0.000000}%
\pgfsetstrokecolor{currentstroke}%
\pgfsetdash{}{0pt}%
\pgfpathmoveto{\pgfqpoint{1.038427in}{1.391533in}}%
\pgfpathlineto{\pgfqpoint{1.085191in}{1.391533in}}%
\pgfpathlineto{\pgfqpoint{1.085191in}{1.398847in}}%
\pgfpathlineto{\pgfqpoint{1.038427in}{1.398847in}}%
\pgfpathlineto{\pgfqpoint{1.038427in}{1.391533in}}%
\pgfpathclose%
\pgfusepath{stroke,fill}%
\end{pgfscope}%
\begin{pgfscope}%
\pgfpathrectangle{\pgfqpoint{0.150000in}{0.150000in}}{\pgfqpoint{1.700000in}{1.700000in}}%
\pgfusepath{clip}%
\pgfsetbuttcap%
\pgfsetroundjoin%
\definecolor{currentfill}{rgb}{0.933333,0.800000,0.400000}%
\pgfsetfillcolor{currentfill}%
\pgfsetlinewidth{1.003750pt}%
\definecolor{currentstroke}{rgb}{0.600000,0.466667,0.000000}%
\pgfsetstrokecolor{currentstroke}%
\pgfsetdash{}{0pt}%
\pgfpathmoveto{\pgfqpoint{1.000165in}{1.398847in}}%
\pgfpathlineto{\pgfqpoint{1.038427in}{1.398847in}}%
\pgfpathlineto{\pgfqpoint{1.038427in}{1.400694in}}%
\pgfpathlineto{\pgfqpoint{1.000165in}{1.400694in}}%
\pgfpathlineto{\pgfqpoint{1.000165in}{1.398847in}}%
\pgfpathclose%
\pgfusepath{stroke,fill}%
\end{pgfscope}%
\begin{pgfscope}%
\pgfpathrectangle{\pgfqpoint{0.150000in}{0.150000in}}{\pgfqpoint{1.700000in}{1.700000in}}%
\pgfusepath{clip}%
\pgfsetbuttcap%
\pgfsetroundjoin%
\definecolor{currentfill}{rgb}{0.933333,0.800000,0.400000}%
\pgfsetfillcolor{currentfill}%
\pgfsetlinewidth{1.003750pt}%
\definecolor{currentstroke}{rgb}{0.600000,0.466667,0.000000}%
\pgfsetstrokecolor{currentstroke}%
\pgfsetdash{}{0pt}%
\pgfpathmoveto{\pgfqpoint{1.579254in}{0.617728in}}%
\pgfpathlineto{\pgfqpoint{1.616409in}{0.617728in}}%
\pgfpathlineto{\pgfqpoint{1.616409in}{0.681084in}}%
\pgfpathlineto{\pgfqpoint{1.579254in}{0.681084in}}%
\pgfpathlineto{\pgfqpoint{1.579254in}{0.617728in}}%
\pgfpathclose%
\pgfusepath{stroke,fill}%
\end{pgfscope}%
\begin{pgfscope}%
\pgfpathrectangle{\pgfqpoint{0.150000in}{0.150000in}}{\pgfqpoint{1.700000in}{1.700000in}}%
\pgfusepath{clip}%
\pgfsetbuttcap%
\pgfsetroundjoin%
\definecolor{currentfill}{rgb}{0.933333,0.800000,0.400000}%
\pgfsetfillcolor{currentfill}%
\pgfsetlinewidth{1.003750pt}%
\definecolor{currentstroke}{rgb}{0.600000,0.466667,0.000000}%
\pgfsetstrokecolor{currentstroke}%
\pgfsetdash{}{0pt}%
\pgfpathmoveto{\pgfqpoint{1.376454in}{0.862748in}}%
\pgfpathlineto{\pgfqpoint{1.394638in}{0.862748in}}%
\pgfpathlineto{\pgfqpoint{1.394638in}{0.930598in}}%
\pgfpathlineto{\pgfqpoint{1.376454in}{0.930598in}}%
\pgfpathlineto{\pgfqpoint{1.376454in}{0.862748in}}%
\pgfpathclose%
\pgfusepath{stroke,fill}%
\end{pgfscope}%
\begin{pgfscope}%
\pgfpathrectangle{\pgfqpoint{0.150000in}{0.150000in}}{\pgfqpoint{1.700000in}{1.700000in}}%
\pgfusepath{clip}%
\pgfsetbuttcap%
\pgfsetroundjoin%
\definecolor{currentfill}{rgb}{0.933333,0.800000,0.400000}%
\pgfsetfillcolor{currentfill}%
\pgfsetlinewidth{1.003750pt}%
\definecolor{currentstroke}{rgb}{0.600000,0.466667,0.000000}%
\pgfsetstrokecolor{currentstroke}%
\pgfsetdash{}{0pt}%
\pgfpathmoveto{\pgfqpoint{1.351279in}{0.807234in}}%
\pgfpathlineto{\pgfqpoint{1.376454in}{0.807234in}}%
\pgfpathlineto{\pgfqpoint{1.376454in}{0.862748in}}%
\pgfpathlineto{\pgfqpoint{1.351279in}{0.862748in}}%
\pgfpathlineto{\pgfqpoint{1.351279in}{0.807234in}}%
\pgfpathclose%
\pgfusepath{stroke,fill}%
\end{pgfscope}%
\begin{pgfscope}%
\pgfpathrectangle{\pgfqpoint{0.150000in}{0.150000in}}{\pgfqpoint{1.700000in}{1.700000in}}%
\pgfusepath{clip}%
\pgfsetbuttcap%
\pgfsetroundjoin%
\definecolor{currentfill}{rgb}{0.933333,0.800000,0.400000}%
\pgfsetfillcolor{currentfill}%
\pgfsetlinewidth{1.003750pt}%
\definecolor{currentstroke}{rgb}{0.600000,0.466667,0.000000}%
\pgfsetstrokecolor{currentstroke}%
\pgfsetdash{}{0pt}%
\pgfpathmoveto{\pgfqpoint{1.314504in}{0.751721in}}%
\pgfpathlineto{\pgfqpoint{1.351279in}{0.751721in}}%
\pgfpathlineto{\pgfqpoint{1.351279in}{0.807234in}}%
\pgfpathlineto{\pgfqpoint{1.314504in}{0.807234in}}%
\pgfpathlineto{\pgfqpoint{1.314504in}{0.751721in}}%
\pgfpathclose%
\pgfusepath{stroke,fill}%
\end{pgfscope}%
\begin{pgfscope}%
\pgfpathrectangle{\pgfqpoint{0.150000in}{0.150000in}}{\pgfqpoint{1.700000in}{1.700000in}}%
\pgfusepath{clip}%
\pgfsetbuttcap%
\pgfsetroundjoin%
\definecolor{currentfill}{rgb}{0.933333,0.800000,0.400000}%
\pgfsetfillcolor{currentfill}%
\pgfsetlinewidth{1.003750pt}%
\definecolor{currentstroke}{rgb}{0.600000,0.466667,0.000000}%
\pgfsetstrokecolor{currentstroke}%
\pgfsetdash{}{0pt}%
\pgfpathmoveto{\pgfqpoint{1.272573in}{0.706301in}}%
\pgfpathlineto{\pgfqpoint{1.314504in}{0.706301in}}%
\pgfpathlineto{\pgfqpoint{1.314504in}{0.751721in}}%
\pgfpathlineto{\pgfqpoint{1.272573in}{0.751721in}}%
\pgfpathlineto{\pgfqpoint{1.272573in}{0.706301in}}%
\pgfpathclose%
\pgfusepath{stroke,fill}%
\end{pgfscope}%
\begin{pgfscope}%
\pgfpathrectangle{\pgfqpoint{0.150000in}{0.150000in}}{\pgfqpoint{1.700000in}{1.700000in}}%
\pgfusepath{clip}%
\pgfsetbuttcap%
\pgfsetroundjoin%
\definecolor{currentfill}{rgb}{0.933333,0.800000,0.400000}%
\pgfsetfillcolor{currentfill}%
\pgfsetlinewidth{1.003750pt}%
\definecolor{currentstroke}{rgb}{0.600000,0.466667,0.000000}%
\pgfsetstrokecolor{currentstroke}%
\pgfsetdash{}{0pt}%
\pgfpathmoveto{\pgfqpoint{1.540056in}{0.564103in}}%
\pgfpathlineto{\pgfqpoint{1.579254in}{0.564103in}}%
\pgfpathlineto{\pgfqpoint{1.579254in}{0.617728in}}%
\pgfpathlineto{\pgfqpoint{1.540056in}{0.617728in}}%
\pgfpathlineto{\pgfqpoint{1.540056in}{0.564103in}}%
\pgfpathclose%
\pgfusepath{stroke,fill}%
\end{pgfscope}%
\begin{pgfscope}%
\pgfpathrectangle{\pgfqpoint{0.150000in}{0.150000in}}{\pgfqpoint{1.700000in}{1.700000in}}%
\pgfusepath{clip}%
\pgfsetbuttcap%
\pgfsetroundjoin%
\definecolor{currentfill}{rgb}{0.933333,0.800000,0.400000}%
\pgfsetfillcolor{currentfill}%
\pgfsetlinewidth{1.003750pt}%
\definecolor{currentstroke}{rgb}{0.600000,0.466667,0.000000}%
\pgfsetstrokecolor{currentstroke}%
\pgfsetdash{}{0pt}%
\pgfpathmoveto{\pgfqpoint{1.501483in}{0.520228in}}%
\pgfpathlineto{\pgfqpoint{1.540056in}{0.520228in}}%
\pgfpathlineto{\pgfqpoint{1.540056in}{0.564103in}}%
\pgfpathlineto{\pgfqpoint{1.501483in}{0.564103in}}%
\pgfpathlineto{\pgfqpoint{1.501483in}{0.520228in}}%
\pgfpathclose%
\pgfusepath{stroke,fill}%
\end{pgfscope}%
\begin{pgfscope}%
\pgfpathrectangle{\pgfqpoint{0.150000in}{0.150000in}}{\pgfqpoint{1.700000in}{1.700000in}}%
\pgfusepath{clip}%
\pgfsetbuttcap%
\pgfsetroundjoin%
\definecolor{currentfill}{rgb}{0.933333,0.800000,0.400000}%
\pgfsetfillcolor{currentfill}%
\pgfsetlinewidth{1.003750pt}%
\definecolor{currentstroke}{rgb}{0.600000,0.466667,0.000000}%
\pgfsetstrokecolor{currentstroke}%
\pgfsetdash{}{0pt}%
\pgfpathmoveto{\pgfqpoint{1.451486in}{0.472907in}}%
\pgfpathlineto{\pgfqpoint{1.501483in}{0.472907in}}%
\pgfpathlineto{\pgfqpoint{1.501483in}{0.520228in}}%
\pgfpathlineto{\pgfqpoint{1.451486in}{0.520228in}}%
\pgfpathlineto{\pgfqpoint{1.451486in}{0.472907in}}%
\pgfpathclose%
\pgfusepath{stroke,fill}%
\end{pgfscope}%
\begin{pgfscope}%
\pgfpathrectangle{\pgfqpoint{0.150000in}{0.150000in}}{\pgfqpoint{1.700000in}{1.700000in}}%
\pgfusepath{clip}%
\pgfsetbuttcap%
\pgfsetroundjoin%
\definecolor{currentfill}{rgb}{0.933333,0.800000,0.400000}%
\pgfsetfillcolor{currentfill}%
\pgfsetlinewidth{1.003750pt}%
\definecolor{currentstroke}{rgb}{0.600000,0.466667,0.000000}%
\pgfsetstrokecolor{currentstroke}%
\pgfsetdash{}{0pt}%
\pgfpathmoveto{\pgfqpoint{1.410580in}{0.440455in}}%
\pgfpathlineto{\pgfqpoint{1.451486in}{0.440455in}}%
\pgfpathlineto{\pgfqpoint{1.451486in}{0.472907in}}%
\pgfpathlineto{\pgfqpoint{1.410580in}{0.472907in}}%
\pgfpathlineto{\pgfqpoint{1.410580in}{0.440455in}}%
\pgfpathclose%
\pgfusepath{stroke,fill}%
\end{pgfscope}%
\begin{pgfscope}%
\pgfpathrectangle{\pgfqpoint{0.150000in}{0.150000in}}{\pgfqpoint{1.700000in}{1.700000in}}%
\pgfusepath{clip}%
\pgfsetbuttcap%
\pgfsetroundjoin%
\definecolor{currentfill}{rgb}{0.933333,0.800000,0.400000}%
\pgfsetfillcolor{currentfill}%
\pgfsetlinewidth{1.003750pt}%
\definecolor{currentstroke}{rgb}{0.600000,0.466667,0.000000}%
\pgfsetstrokecolor{currentstroke}%
\pgfsetdash{}{0pt}%
\pgfpathmoveto{\pgfqpoint{1.368833in}{0.412098in}}%
\pgfpathlineto{\pgfqpoint{1.410580in}{0.412098in}}%
\pgfpathlineto{\pgfqpoint{1.410580in}{0.440455in}}%
\pgfpathlineto{\pgfqpoint{1.368833in}{0.440455in}}%
\pgfpathlineto{\pgfqpoint{1.368833in}{0.412098in}}%
\pgfpathclose%
\pgfusepath{stroke,fill}%
\end{pgfscope}%
\begin{pgfscope}%
\pgfpathrectangle{\pgfqpoint{0.150000in}{0.150000in}}{\pgfqpoint{1.700000in}{1.700000in}}%
\pgfusepath{clip}%
\pgfsetbuttcap%
\pgfsetroundjoin%
\definecolor{currentfill}{rgb}{0.933333,0.800000,0.400000}%
\pgfsetfillcolor{currentfill}%
\pgfsetlinewidth{1.003750pt}%
\definecolor{currentstroke}{rgb}{0.600000,0.466667,0.000000}%
\pgfsetstrokecolor{currentstroke}%
\pgfsetdash{}{0pt}%
\pgfpathmoveto{\pgfqpoint{1.334676in}{0.392004in}}%
\pgfpathlineto{\pgfqpoint{1.368833in}{0.392004in}}%
\pgfpathlineto{\pgfqpoint{1.368833in}{0.412098in}}%
\pgfpathlineto{\pgfqpoint{1.334676in}{0.412098in}}%
\pgfpathlineto{\pgfqpoint{1.334676in}{0.392004in}}%
\pgfpathclose%
\pgfusepath{stroke,fill}%
\end{pgfscope}%
\begin{pgfscope}%
\pgfpathrectangle{\pgfqpoint{0.150000in}{0.150000in}}{\pgfqpoint{1.700000in}{1.700000in}}%
\pgfusepath{clip}%
\pgfsetbuttcap%
\pgfsetroundjoin%
\definecolor{currentfill}{rgb}{0.933333,0.800000,0.400000}%
\pgfsetfillcolor{currentfill}%
\pgfsetlinewidth{1.003750pt}%
\definecolor{currentstroke}{rgb}{0.600000,0.466667,0.000000}%
\pgfsetstrokecolor{currentstroke}%
\pgfsetdash{}{0pt}%
\pgfpathmoveto{\pgfqpoint{1.215677in}{0.662303in}}%
\pgfpathlineto{\pgfqpoint{1.272573in}{0.662303in}}%
\pgfpathlineto{\pgfqpoint{1.272573in}{0.706301in}}%
\pgfpathlineto{\pgfqpoint{1.215677in}{0.706301in}}%
\pgfpathlineto{\pgfqpoint{1.215677in}{0.662303in}}%
\pgfpathclose%
\pgfusepath{stroke,fill}%
\end{pgfscope}%
\begin{pgfscope}%
\pgfpathrectangle{\pgfqpoint{0.150000in}{0.150000in}}{\pgfqpoint{1.700000in}{1.700000in}}%
\pgfusepath{clip}%
\pgfsetbuttcap%
\pgfsetroundjoin%
\definecolor{currentfill}{rgb}{0.933333,0.800000,0.400000}%
\pgfsetfillcolor{currentfill}%
\pgfsetlinewidth{1.003750pt}%
\definecolor{currentstroke}{rgb}{0.600000,0.466667,0.000000}%
\pgfsetstrokecolor{currentstroke}%
\pgfsetdash{}{0pt}%
\pgfpathmoveto{\pgfqpoint{1.169126in}{0.636748in}}%
\pgfpathlineto{\pgfqpoint{1.215677in}{0.636748in}}%
\pgfpathlineto{\pgfqpoint{1.215677in}{0.662303in}}%
\pgfpathlineto{\pgfqpoint{1.169126in}{0.662303in}}%
\pgfpathlineto{\pgfqpoint{1.169126in}{0.636748in}}%
\pgfpathclose%
\pgfusepath{stroke,fill}%
\end{pgfscope}%
\begin{pgfscope}%
\pgfpathrectangle{\pgfqpoint{0.150000in}{0.150000in}}{\pgfqpoint{1.700000in}{1.700000in}}%
\pgfusepath{clip}%
\pgfsetbuttcap%
\pgfsetroundjoin%
\definecolor{currentfill}{rgb}{0.933333,0.800000,0.400000}%
\pgfsetfillcolor{currentfill}%
\pgfsetlinewidth{1.003750pt}%
\definecolor{currentstroke}{rgb}{0.600000,0.466667,0.000000}%
\pgfsetstrokecolor{currentstroke}%
\pgfsetdash{}{0pt}%
\pgfpathmoveto{\pgfqpoint{1.122574in}{0.618515in}}%
\pgfpathlineto{\pgfqpoint{1.169126in}{0.618515in}}%
\pgfpathlineto{\pgfqpoint{1.169126in}{0.636748in}}%
\pgfpathlineto{\pgfqpoint{1.122574in}{0.636748in}}%
\pgfpathlineto{\pgfqpoint{1.122574in}{0.618515in}}%
\pgfpathclose%
\pgfusepath{stroke,fill}%
\end{pgfscope}%
\begin{pgfscope}%
\pgfpathrectangle{\pgfqpoint{0.150000in}{0.150000in}}{\pgfqpoint{1.700000in}{1.700000in}}%
\pgfusepath{clip}%
\pgfsetbuttcap%
\pgfsetroundjoin%
\definecolor{currentfill}{rgb}{0.933333,0.800000,0.400000}%
\pgfsetfillcolor{currentfill}%
\pgfsetlinewidth{1.003750pt}%
\definecolor{currentstroke}{rgb}{0.600000,0.466667,0.000000}%
\pgfsetstrokecolor{currentstroke}%
\pgfsetdash{}{0pt}%
\pgfpathmoveto{\pgfqpoint{1.084487in}{0.608314in}}%
\pgfpathlineto{\pgfqpoint{1.122574in}{0.608314in}}%
\pgfpathlineto{\pgfqpoint{1.122574in}{0.618515in}}%
\pgfpathlineto{\pgfqpoint{1.084487in}{0.618515in}}%
\pgfpathlineto{\pgfqpoint{1.084487in}{0.608314in}}%
\pgfpathclose%
\pgfusepath{stroke,fill}%
\end{pgfscope}%
\begin{pgfscope}%
\pgfpathrectangle{\pgfqpoint{0.150000in}{0.150000in}}{\pgfqpoint{1.700000in}{1.700000in}}%
\pgfusepath{clip}%
\pgfsetbuttcap%
\pgfsetroundjoin%
\definecolor{currentfill}{rgb}{0.933333,0.800000,0.400000}%
\pgfsetfillcolor{currentfill}%
\pgfsetlinewidth{1.003750pt}%
\definecolor{currentstroke}{rgb}{0.600000,0.466667,0.000000}%
\pgfsetstrokecolor{currentstroke}%
\pgfsetdash{}{0pt}%
\pgfpathmoveto{\pgfqpoint{1.037935in}{0.601106in}}%
\pgfpathlineto{\pgfqpoint{1.084487in}{0.601106in}}%
\pgfpathlineto{\pgfqpoint{1.084487in}{0.608314in}}%
\pgfpathlineto{\pgfqpoint{1.037935in}{0.608314in}}%
\pgfpathlineto{\pgfqpoint{1.037935in}{0.601106in}}%
\pgfpathclose%
\pgfusepath{stroke,fill}%
\end{pgfscope}%
\begin{pgfscope}%
\pgfpathrectangle{\pgfqpoint{0.150000in}{0.150000in}}{\pgfqpoint{1.700000in}{1.700000in}}%
\pgfusepath{clip}%
\pgfsetbuttcap%
\pgfsetroundjoin%
\definecolor{currentfill}{rgb}{0.933333,0.800000,0.400000}%
\pgfsetfillcolor{currentfill}%
\pgfsetlinewidth{1.003750pt}%
\definecolor{currentstroke}{rgb}{0.600000,0.466667,0.000000}%
\pgfsetstrokecolor{currentstroke}%
\pgfsetdash{}{0pt}%
\pgfpathmoveto{\pgfqpoint{0.999848in}{0.599306in}}%
\pgfpathlineto{\pgfqpoint{1.037935in}{0.599306in}}%
\pgfpathlineto{\pgfqpoint{1.037935in}{0.601106in}}%
\pgfpathlineto{\pgfqpoint{0.999848in}{0.601106in}}%
\pgfpathlineto{\pgfqpoint{0.999848in}{0.599306in}}%
\pgfpathclose%
\pgfusepath{stroke,fill}%
\end{pgfscope}%
\begin{pgfscope}%
\pgfpathrectangle{\pgfqpoint{0.150000in}{0.150000in}}{\pgfqpoint{1.700000in}{1.700000in}}%
\pgfusepath{clip}%
\pgfsetbuttcap%
\pgfsetroundjoin%
\definecolor{currentfill}{rgb}{0.933333,0.800000,0.400000}%
\pgfsetfillcolor{currentfill}%
\pgfsetlinewidth{1.003750pt}%
\definecolor{currentstroke}{rgb}{0.600000,0.466667,0.000000}%
\pgfsetstrokecolor{currentstroke}%
\pgfsetdash{}{0pt}%
\pgfpathmoveto{\pgfqpoint{1.215677in}{0.340341in}}%
\pgfpathlineto{\pgfqpoint{1.272573in}{0.340341in}}%
\pgfpathlineto{\pgfqpoint{1.272573in}{0.361744in}}%
\pgfpathlineto{\pgfqpoint{1.215677in}{0.361744in}}%
\pgfpathlineto{\pgfqpoint{1.215677in}{0.340341in}}%
\pgfpathclose%
\pgfusepath{stroke,fill}%
\end{pgfscope}%
\begin{pgfscope}%
\pgfpathrectangle{\pgfqpoint{0.150000in}{0.150000in}}{\pgfqpoint{1.700000in}{1.700000in}}%
\pgfusepath{clip}%
\pgfsetbuttcap%
\pgfsetroundjoin%
\definecolor{currentfill}{rgb}{0.933333,0.800000,0.400000}%
\pgfsetfillcolor{currentfill}%
\pgfsetlinewidth{1.003750pt}%
\definecolor{currentstroke}{rgb}{0.600000,0.466667,0.000000}%
\pgfsetstrokecolor{currentstroke}%
\pgfsetdash{}{0pt}%
\pgfpathmoveto{\pgfqpoint{1.169126in}{0.326900in}}%
\pgfpathlineto{\pgfqpoint{1.215677in}{0.326900in}}%
\pgfpathlineto{\pgfqpoint{1.215677in}{0.340341in}}%
\pgfpathlineto{\pgfqpoint{1.169126in}{0.340341in}}%
\pgfpathlineto{\pgfqpoint{1.169126in}{0.326900in}}%
\pgfpathclose%
\pgfusepath{stroke,fill}%
\end{pgfscope}%
\begin{pgfscope}%
\pgfpathrectangle{\pgfqpoint{0.150000in}{0.150000in}}{\pgfqpoint{1.700000in}{1.700000in}}%
\pgfusepath{clip}%
\pgfsetbuttcap%
\pgfsetroundjoin%
\definecolor{currentfill}{rgb}{0.933333,0.800000,0.400000}%
\pgfsetfillcolor{currentfill}%
\pgfsetlinewidth{1.003750pt}%
\definecolor{currentstroke}{rgb}{0.600000,0.466667,0.000000}%
\pgfsetstrokecolor{currentstroke}%
\pgfsetdash{}{0pt}%
\pgfpathmoveto{\pgfqpoint{1.122574in}{0.316888in}}%
\pgfpathlineto{\pgfqpoint{1.169126in}{0.316888in}}%
\pgfpathlineto{\pgfqpoint{1.169126in}{0.326900in}}%
\pgfpathlineto{\pgfqpoint{1.122574in}{0.326900in}}%
\pgfpathlineto{\pgfqpoint{1.122574in}{0.316888in}}%
\pgfpathclose%
\pgfusepath{stroke,fill}%
\end{pgfscope}%
\begin{pgfscope}%
\pgfpathrectangle{\pgfqpoint{0.150000in}{0.150000in}}{\pgfqpoint{1.700000in}{1.700000in}}%
\pgfusepath{clip}%
\pgfsetbuttcap%
\pgfsetroundjoin%
\definecolor{currentfill}{rgb}{0.933333,0.800000,0.400000}%
\pgfsetfillcolor{currentfill}%
\pgfsetlinewidth{1.003750pt}%
\definecolor{currentstroke}{rgb}{0.600000,0.466667,0.000000}%
\pgfsetstrokecolor{currentstroke}%
\pgfsetdash{}{0pt}%
\pgfpathmoveto{\pgfqpoint{1.084487in}{0.311140in}}%
\pgfpathlineto{\pgfqpoint{1.122574in}{0.311140in}}%
\pgfpathlineto{\pgfqpoint{1.122574in}{0.316888in}}%
\pgfpathlineto{\pgfqpoint{1.084487in}{0.316888in}}%
\pgfpathlineto{\pgfqpoint{1.084487in}{0.311140in}}%
\pgfpathclose%
\pgfusepath{stroke,fill}%
\end{pgfscope}%
\begin{pgfscope}%
\pgfpathrectangle{\pgfqpoint{0.150000in}{0.150000in}}{\pgfqpoint{1.700000in}{1.700000in}}%
\pgfusepath{clip}%
\pgfsetbuttcap%
\pgfsetroundjoin%
\definecolor{currentfill}{rgb}{0.933333,0.800000,0.400000}%
\pgfsetfillcolor{currentfill}%
\pgfsetlinewidth{1.003750pt}%
\definecolor{currentstroke}{rgb}{0.600000,0.466667,0.000000}%
\pgfsetstrokecolor{currentstroke}%
\pgfsetdash{}{0pt}%
\pgfpathmoveto{\pgfqpoint{1.037935in}{0.307015in}}%
\pgfpathlineto{\pgfqpoint{1.084487in}{0.307015in}}%
\pgfpathlineto{\pgfqpoint{1.084487in}{0.311140in}}%
\pgfpathlineto{\pgfqpoint{1.037935in}{0.311140in}}%
\pgfpathlineto{\pgfqpoint{1.037935in}{0.307015in}}%
\pgfpathclose%
\pgfusepath{stroke,fill}%
\end{pgfscope}%
\begin{pgfscope}%
\pgfpathrectangle{\pgfqpoint{0.150000in}{0.150000in}}{\pgfqpoint{1.700000in}{1.700000in}}%
\pgfusepath{clip}%
\pgfsetbuttcap%
\pgfsetroundjoin%
\definecolor{currentfill}{rgb}{0.933333,0.800000,0.400000}%
\pgfsetfillcolor{currentfill}%
\pgfsetlinewidth{1.003750pt}%
\definecolor{currentstroke}{rgb}{0.600000,0.466667,0.000000}%
\pgfsetstrokecolor{currentstroke}%
\pgfsetdash{}{0pt}%
\pgfpathmoveto{\pgfqpoint{0.999848in}{0.305978in}}%
\pgfpathlineto{\pgfqpoint{1.037935in}{0.305978in}}%
\pgfpathlineto{\pgfqpoint{1.037935in}{0.307015in}}%
\pgfpathlineto{\pgfqpoint{0.999848in}{0.307015in}}%
\pgfpathlineto{\pgfqpoint{0.999848in}{0.305978in}}%
\pgfpathclose%
\pgfusepath{stroke,fill}%
\end{pgfscope}%
\begin{pgfscope}%
\pgfpathrectangle{\pgfqpoint{0.150000in}{0.150000in}}{\pgfqpoint{1.700000in}{1.700000in}}%
\pgfusepath{clip}%
\pgfsetbuttcap%
\pgfsetroundjoin%
\definecolor{currentfill}{rgb}{0.933333,0.800000,0.400000}%
\pgfsetfillcolor{currentfill}%
\pgfsetlinewidth{1.003750pt}%
\definecolor{currentstroke}{rgb}{0.600000,0.466667,0.000000}%
\pgfsetstrokecolor{currentstroke}%
\pgfsetdash{}{0pt}%
\pgfpathmoveto{\pgfqpoint{0.617773in}{1.579284in}}%
\pgfpathlineto{\pgfqpoint{0.681120in}{1.579284in}}%
\pgfpathlineto{\pgfqpoint{0.681120in}{1.616427in}}%
\pgfpathlineto{\pgfqpoint{0.617773in}{1.616427in}}%
\pgfpathlineto{\pgfqpoint{0.617773in}{1.579284in}}%
\pgfpathclose%
\pgfusepath{stroke,fill}%
\end{pgfscope}%
\begin{pgfscope}%
\pgfpathrectangle{\pgfqpoint{0.150000in}{0.150000in}}{\pgfqpoint{1.700000in}{1.700000in}}%
\pgfusepath{clip}%
\pgfsetbuttcap%
\pgfsetroundjoin%
\definecolor{currentfill}{rgb}{0.933333,0.800000,0.400000}%
\pgfsetfillcolor{currentfill}%
\pgfsetlinewidth{1.003750pt}%
\definecolor{currentstroke}{rgb}{0.600000,0.466667,0.000000}%
\pgfsetstrokecolor{currentstroke}%
\pgfsetdash{}{0pt}%
\pgfpathmoveto{\pgfqpoint{0.862802in}{1.376473in}}%
\pgfpathlineto{\pgfqpoint{0.930598in}{1.376473in}}%
\pgfpathlineto{\pgfqpoint{0.930598in}{1.394638in}}%
\pgfpathlineto{\pgfqpoint{0.862802in}{1.394638in}}%
\pgfpathlineto{\pgfqpoint{0.862802in}{1.376473in}}%
\pgfpathclose%
\pgfusepath{stroke,fill}%
\end{pgfscope}%
\begin{pgfscope}%
\pgfpathrectangle{\pgfqpoint{0.150000in}{0.150000in}}{\pgfqpoint{1.700000in}{1.700000in}}%
\pgfusepath{clip}%
\pgfsetbuttcap%
\pgfsetroundjoin%
\definecolor{currentfill}{rgb}{0.933333,0.800000,0.400000}%
\pgfsetfillcolor{currentfill}%
\pgfsetlinewidth{1.003750pt}%
\definecolor{currentstroke}{rgb}{0.600000,0.466667,0.000000}%
\pgfsetstrokecolor{currentstroke}%
\pgfsetdash{}{0pt}%
\pgfpathmoveto{\pgfqpoint{0.807332in}{1.351333in}}%
\pgfpathlineto{\pgfqpoint{0.862802in}{1.351333in}}%
\pgfpathlineto{\pgfqpoint{0.862802in}{1.376473in}}%
\pgfpathlineto{\pgfqpoint{0.807332in}{1.376473in}}%
\pgfpathlineto{\pgfqpoint{0.807332in}{1.351333in}}%
\pgfpathclose%
\pgfusepath{stroke,fill}%
\end{pgfscope}%
\begin{pgfscope}%
\pgfpathrectangle{\pgfqpoint{0.150000in}{0.150000in}}{\pgfqpoint{1.700000in}{1.700000in}}%
\pgfusepath{clip}%
\pgfsetbuttcap%
\pgfsetroundjoin%
\definecolor{currentfill}{rgb}{0.933333,0.800000,0.400000}%
\pgfsetfillcolor{currentfill}%
\pgfsetlinewidth{1.003750pt}%
\definecolor{currentstroke}{rgb}{0.600000,0.466667,0.000000}%
\pgfsetstrokecolor{currentstroke}%
\pgfsetdash{}{0pt}%
\pgfpathmoveto{\pgfqpoint{0.751862in}{1.314616in}}%
\pgfpathlineto{\pgfqpoint{0.807332in}{1.314616in}}%
\pgfpathlineto{\pgfqpoint{0.807332in}{1.351333in}}%
\pgfpathlineto{\pgfqpoint{0.751862in}{1.351333in}}%
\pgfpathlineto{\pgfqpoint{0.751862in}{1.314616in}}%
\pgfpathclose%
\pgfusepath{stroke,fill}%
\end{pgfscope}%
\begin{pgfscope}%
\pgfpathrectangle{\pgfqpoint{0.150000in}{0.150000in}}{\pgfqpoint{1.700000in}{1.700000in}}%
\pgfusepath{clip}%
\pgfsetbuttcap%
\pgfsetroundjoin%
\definecolor{currentfill}{rgb}{0.933333,0.800000,0.400000}%
\pgfsetfillcolor{currentfill}%
\pgfsetlinewidth{1.003750pt}%
\definecolor{currentstroke}{rgb}{0.600000,0.466667,0.000000}%
\pgfsetstrokecolor{currentstroke}%
\pgfsetdash{}{0pt}%
\pgfpathmoveto{\pgfqpoint{0.706478in}{1.272765in}}%
\pgfpathlineto{\pgfqpoint{0.751862in}{1.272765in}}%
\pgfpathlineto{\pgfqpoint{0.751862in}{1.314616in}}%
\pgfpathlineto{\pgfqpoint{0.706478in}{1.314616in}}%
\pgfpathlineto{\pgfqpoint{0.706478in}{1.272765in}}%
\pgfpathclose%
\pgfusepath{stroke,fill}%
\end{pgfscope}%
\begin{pgfscope}%
\pgfpathrectangle{\pgfqpoint{0.150000in}{0.150000in}}{\pgfqpoint{1.700000in}{1.700000in}}%
\pgfusepath{clip}%
\pgfsetbuttcap%
\pgfsetroundjoin%
\definecolor{currentfill}{rgb}{0.933333,0.800000,0.400000}%
\pgfsetfillcolor{currentfill}%
\pgfsetlinewidth{1.003750pt}%
\definecolor{currentstroke}{rgb}{0.600000,0.466667,0.000000}%
\pgfsetstrokecolor{currentstroke}%
\pgfsetdash{}{0pt}%
\pgfpathmoveto{\pgfqpoint{0.564161in}{1.540103in}}%
\pgfpathlineto{\pgfqpoint{0.617773in}{1.540103in}}%
\pgfpathlineto{\pgfqpoint{0.617773in}{1.579284in}}%
\pgfpathlineto{\pgfqpoint{0.564161in}{1.579284in}}%
\pgfpathlineto{\pgfqpoint{0.564161in}{1.540103in}}%
\pgfpathclose%
\pgfusepath{stroke,fill}%
\end{pgfscope}%
\begin{pgfscope}%
\pgfpathrectangle{\pgfqpoint{0.150000in}{0.150000in}}{\pgfqpoint{1.700000in}{1.700000in}}%
\pgfusepath{clip}%
\pgfsetbuttcap%
\pgfsetroundjoin%
\definecolor{currentfill}{rgb}{0.933333,0.800000,0.400000}%
\pgfsetfillcolor{currentfill}%
\pgfsetlinewidth{1.003750pt}%
\definecolor{currentstroke}{rgb}{0.600000,0.466667,0.000000}%
\pgfsetstrokecolor{currentstroke}%
\pgfsetdash{}{0pt}%
\pgfpathmoveto{\pgfqpoint{0.520296in}{1.501548in}}%
\pgfpathlineto{\pgfqpoint{0.564161in}{1.501548in}}%
\pgfpathlineto{\pgfqpoint{0.564161in}{1.540103in}}%
\pgfpathlineto{\pgfqpoint{0.520296in}{1.540103in}}%
\pgfpathlineto{\pgfqpoint{0.520296in}{1.501548in}}%
\pgfpathclose%
\pgfusepath{stroke,fill}%
\end{pgfscope}%
\begin{pgfscope}%
\pgfpathrectangle{\pgfqpoint{0.150000in}{0.150000in}}{\pgfqpoint{1.700000in}{1.700000in}}%
\pgfusepath{clip}%
\pgfsetbuttcap%
\pgfsetroundjoin%
\definecolor{currentfill}{rgb}{0.933333,0.800000,0.400000}%
\pgfsetfillcolor{currentfill}%
\pgfsetlinewidth{1.003750pt}%
\definecolor{currentstroke}{rgb}{0.600000,0.466667,0.000000}%
\pgfsetstrokecolor{currentstroke}%
\pgfsetdash{}{0pt}%
\pgfpathmoveto{\pgfqpoint{0.472988in}{1.451581in}}%
\pgfpathlineto{\pgfqpoint{0.520296in}{1.451581in}}%
\pgfpathlineto{\pgfqpoint{0.520296in}{1.501548in}}%
\pgfpathlineto{\pgfqpoint{0.472988in}{1.501548in}}%
\pgfpathlineto{\pgfqpoint{0.472988in}{1.451581in}}%
\pgfpathclose%
\pgfusepath{stroke,fill}%
\end{pgfscope}%
\begin{pgfscope}%
\pgfpathrectangle{\pgfqpoint{0.150000in}{0.150000in}}{\pgfqpoint{1.700000in}{1.700000in}}%
\pgfusepath{clip}%
\pgfsetbuttcap%
\pgfsetroundjoin%
\definecolor{currentfill}{rgb}{0.933333,0.800000,0.400000}%
\pgfsetfillcolor{currentfill}%
\pgfsetlinewidth{1.003750pt}%
\definecolor{currentstroke}{rgb}{0.600000,0.466667,0.000000}%
\pgfsetstrokecolor{currentstroke}%
\pgfsetdash{}{0pt}%
\pgfpathmoveto{\pgfqpoint{0.440542in}{1.410698in}}%
\pgfpathlineto{\pgfqpoint{0.472988in}{1.410698in}}%
\pgfpathlineto{\pgfqpoint{0.472988in}{1.451581in}}%
\pgfpathlineto{\pgfqpoint{0.440542in}{1.451581in}}%
\pgfpathlineto{\pgfqpoint{0.440542in}{1.410698in}}%
\pgfpathclose%
\pgfusepath{stroke,fill}%
\end{pgfscope}%
\begin{pgfscope}%
\pgfpathrectangle{\pgfqpoint{0.150000in}{0.150000in}}{\pgfqpoint{1.700000in}{1.700000in}}%
\pgfusepath{clip}%
\pgfsetbuttcap%
\pgfsetroundjoin%
\definecolor{currentfill}{rgb}{0.933333,0.800000,0.400000}%
\pgfsetfillcolor{currentfill}%
\pgfsetlinewidth{1.003750pt}%
\definecolor{currentstroke}{rgb}{0.600000,0.466667,0.000000}%
\pgfsetstrokecolor{currentstroke}%
\pgfsetdash{}{0pt}%
\pgfpathmoveto{\pgfqpoint{0.412186in}{1.368973in}}%
\pgfpathlineto{\pgfqpoint{0.440542in}{1.368973in}}%
\pgfpathlineto{\pgfqpoint{0.440542in}{1.410698in}}%
\pgfpathlineto{\pgfqpoint{0.412186in}{1.410698in}}%
\pgfpathlineto{\pgfqpoint{0.412186in}{1.368973in}}%
\pgfpathclose%
\pgfusepath{stroke,fill}%
\end{pgfscope}%
\begin{pgfscope}%
\pgfpathrectangle{\pgfqpoint{0.150000in}{0.150000in}}{\pgfqpoint{1.700000in}{1.700000in}}%
\pgfusepath{clip}%
\pgfsetbuttcap%
\pgfsetroundjoin%
\definecolor{currentfill}{rgb}{0.933333,0.800000,0.400000}%
\pgfsetfillcolor{currentfill}%
\pgfsetlinewidth{1.003750pt}%
\definecolor{currentstroke}{rgb}{0.600000,0.466667,0.000000}%
\pgfsetstrokecolor{currentstroke}%
\pgfsetdash{}{0pt}%
\pgfpathmoveto{\pgfqpoint{0.392092in}{1.334835in}}%
\pgfpathlineto{\pgfqpoint{0.412186in}{1.334835in}}%
\pgfpathlineto{\pgfqpoint{0.412186in}{1.368973in}}%
\pgfpathlineto{\pgfqpoint{0.392092in}{1.368973in}}%
\pgfpathlineto{\pgfqpoint{0.392092in}{1.334835in}}%
\pgfpathclose%
\pgfusepath{stroke,fill}%
\end{pgfscope}%
\begin{pgfscope}%
\pgfpathrectangle{\pgfqpoint{0.150000in}{0.150000in}}{\pgfqpoint{1.700000in}{1.700000in}}%
\pgfusepath{clip}%
\pgfsetbuttcap%
\pgfsetroundjoin%
\definecolor{currentfill}{rgb}{0.933333,0.800000,0.400000}%
\pgfsetfillcolor{currentfill}%
\pgfsetlinewidth{1.003750pt}%
\definecolor{currentstroke}{rgb}{0.600000,0.466667,0.000000}%
\pgfsetstrokecolor{currentstroke}%
\pgfsetdash{}{0pt}%
\pgfpathmoveto{\pgfqpoint{0.662442in}{1.215894in}}%
\pgfpathlineto{\pgfqpoint{0.706478in}{1.215894in}}%
\pgfpathlineto{\pgfqpoint{0.706478in}{1.272765in}}%
\pgfpathlineto{\pgfqpoint{0.662442in}{1.272765in}}%
\pgfpathlineto{\pgfqpoint{0.662442in}{1.215894in}}%
\pgfpathclose%
\pgfusepath{stroke,fill}%
\end{pgfscope}%
\begin{pgfscope}%
\pgfpathrectangle{\pgfqpoint{0.150000in}{0.150000in}}{\pgfqpoint{1.700000in}{1.700000in}}%
\pgfusepath{clip}%
\pgfsetbuttcap%
\pgfsetroundjoin%
\definecolor{currentfill}{rgb}{0.933333,0.800000,0.400000}%
\pgfsetfillcolor{currentfill}%
\pgfsetlinewidth{1.003750pt}%
\definecolor{currentstroke}{rgb}{0.600000,0.466667,0.000000}%
\pgfsetstrokecolor{currentstroke}%
\pgfsetdash{}{0pt}%
\pgfpathmoveto{\pgfqpoint{0.636859in}{1.169364in}}%
\pgfpathlineto{\pgfqpoint{0.662442in}{1.169364in}}%
\pgfpathlineto{\pgfqpoint{0.662442in}{1.215894in}}%
\pgfpathlineto{\pgfqpoint{0.636859in}{1.215894in}}%
\pgfpathlineto{\pgfqpoint{0.636859in}{1.169364in}}%
\pgfpathclose%
\pgfusepath{stroke,fill}%
\end{pgfscope}%
\begin{pgfscope}%
\pgfpathrectangle{\pgfqpoint{0.150000in}{0.150000in}}{\pgfqpoint{1.700000in}{1.700000in}}%
\pgfusepath{clip}%
\pgfsetbuttcap%
\pgfsetroundjoin%
\definecolor{currentfill}{rgb}{0.933333,0.800000,0.400000}%
\pgfsetfillcolor{currentfill}%
\pgfsetlinewidth{1.003750pt}%
\definecolor{currentstroke}{rgb}{0.600000,0.466667,0.000000}%
\pgfsetstrokecolor{currentstroke}%
\pgfsetdash{}{0pt}%
\pgfpathmoveto{\pgfqpoint{0.618598in}{1.122834in}}%
\pgfpathlineto{\pgfqpoint{0.636859in}{1.122834in}}%
\pgfpathlineto{\pgfqpoint{0.636859in}{1.169364in}}%
\pgfpathlineto{\pgfqpoint{0.618598in}{1.169364in}}%
\pgfpathlineto{\pgfqpoint{0.618598in}{1.122834in}}%
\pgfpathclose%
\pgfusepath{stroke,fill}%
\end{pgfscope}%
\begin{pgfscope}%
\pgfpathrectangle{\pgfqpoint{0.150000in}{0.150000in}}{\pgfqpoint{1.700000in}{1.700000in}}%
\pgfusepath{clip}%
\pgfsetbuttcap%
\pgfsetroundjoin%
\definecolor{currentfill}{rgb}{0.933333,0.800000,0.400000}%
\pgfsetfillcolor{currentfill}%
\pgfsetlinewidth{1.003750pt}%
\definecolor{currentstroke}{rgb}{0.600000,0.466667,0.000000}%
\pgfsetstrokecolor{currentstroke}%
\pgfsetdash{}{0pt}%
\pgfpathmoveto{\pgfqpoint{0.608374in}{1.084764in}}%
\pgfpathlineto{\pgfqpoint{0.618598in}{1.084764in}}%
\pgfpathlineto{\pgfqpoint{0.618598in}{1.122834in}}%
\pgfpathlineto{\pgfqpoint{0.608374in}{1.122834in}}%
\pgfpathlineto{\pgfqpoint{0.608374in}{1.084764in}}%
\pgfpathclose%
\pgfusepath{stroke,fill}%
\end{pgfscope}%
\begin{pgfscope}%
\pgfpathrectangle{\pgfqpoint{0.150000in}{0.150000in}}{\pgfqpoint{1.700000in}{1.700000in}}%
\pgfusepath{clip}%
\pgfsetbuttcap%
\pgfsetroundjoin%
\definecolor{currentfill}{rgb}{0.933333,0.800000,0.400000}%
\pgfsetfillcolor{currentfill}%
\pgfsetlinewidth{1.003750pt}%
\definecolor{currentstroke}{rgb}{0.600000,0.466667,0.000000}%
\pgfsetstrokecolor{currentstroke}%
\pgfsetdash{}{0pt}%
\pgfpathmoveto{\pgfqpoint{0.601134in}{1.038234in}}%
\pgfpathlineto{\pgfqpoint{0.608374in}{1.038234in}}%
\pgfpathlineto{\pgfqpoint{0.608374in}{1.084764in}}%
\pgfpathlineto{\pgfqpoint{0.601134in}{1.084764in}}%
\pgfpathlineto{\pgfqpoint{0.601134in}{1.038234in}}%
\pgfpathclose%
\pgfusepath{stroke,fill}%
\end{pgfscope}%
\begin{pgfscope}%
\pgfpathrectangle{\pgfqpoint{0.150000in}{0.150000in}}{\pgfqpoint{1.700000in}{1.700000in}}%
\pgfusepath{clip}%
\pgfsetbuttcap%
\pgfsetroundjoin%
\definecolor{currentfill}{rgb}{0.933333,0.800000,0.400000}%
\pgfsetfillcolor{currentfill}%
\pgfsetlinewidth{1.003750pt}%
\definecolor{currentstroke}{rgb}{0.600000,0.466667,0.000000}%
\pgfsetstrokecolor{currentstroke}%
\pgfsetdash{}{0pt}%
\pgfpathmoveto{\pgfqpoint{0.599306in}{1.000164in}}%
\pgfpathlineto{\pgfqpoint{0.601134in}{1.000164in}}%
\pgfpathlineto{\pgfqpoint{0.601134in}{1.038234in}}%
\pgfpathlineto{\pgfqpoint{0.599306in}{1.038234in}}%
\pgfpathlineto{\pgfqpoint{0.599306in}{1.000164in}}%
\pgfpathclose%
\pgfusepath{stroke,fill}%
\end{pgfscope}%
\begin{pgfscope}%
\pgfpathrectangle{\pgfqpoint{0.150000in}{0.150000in}}{\pgfqpoint{1.700000in}{1.700000in}}%
\pgfusepath{clip}%
\pgfsetbuttcap%
\pgfsetroundjoin%
\definecolor{currentfill}{rgb}{0.933333,0.800000,0.400000}%
\pgfsetfillcolor{currentfill}%
\pgfsetlinewidth{1.003750pt}%
\definecolor{currentstroke}{rgb}{0.600000,0.466667,0.000000}%
\pgfsetstrokecolor{currentstroke}%
\pgfsetdash{}{0pt}%
\pgfpathmoveto{\pgfqpoint{0.340412in}{1.215894in}}%
\pgfpathlineto{\pgfqpoint{0.361826in}{1.215894in}}%
\pgfpathlineto{\pgfqpoint{0.361826in}{1.272765in}}%
\pgfpathlineto{\pgfqpoint{0.340412in}{1.272765in}}%
\pgfpathlineto{\pgfqpoint{0.340412in}{1.215894in}}%
\pgfpathclose%
\pgfusepath{stroke,fill}%
\end{pgfscope}%
\begin{pgfscope}%
\pgfpathrectangle{\pgfqpoint{0.150000in}{0.150000in}}{\pgfqpoint{1.700000in}{1.700000in}}%
\pgfusepath{clip}%
\pgfsetbuttcap%
\pgfsetroundjoin%
\definecolor{currentfill}{rgb}{0.933333,0.800000,0.400000}%
\pgfsetfillcolor{currentfill}%
\pgfsetlinewidth{1.003750pt}%
\definecolor{currentstroke}{rgb}{0.600000,0.466667,0.000000}%
\pgfsetstrokecolor{currentstroke}%
\pgfsetdash{}{0pt}%
\pgfpathmoveto{\pgfqpoint{0.326960in}{1.169364in}}%
\pgfpathlineto{\pgfqpoint{0.340412in}{1.169364in}}%
\pgfpathlineto{\pgfqpoint{0.340412in}{1.215894in}}%
\pgfpathlineto{\pgfqpoint{0.326960in}{1.215894in}}%
\pgfpathlineto{\pgfqpoint{0.326960in}{1.169364in}}%
\pgfpathclose%
\pgfusepath{stroke,fill}%
\end{pgfscope}%
\begin{pgfscope}%
\pgfpathrectangle{\pgfqpoint{0.150000in}{0.150000in}}{\pgfqpoint{1.700000in}{1.700000in}}%
\pgfusepath{clip}%
\pgfsetbuttcap%
\pgfsetroundjoin%
\definecolor{currentfill}{rgb}{0.933333,0.800000,0.400000}%
\pgfsetfillcolor{currentfill}%
\pgfsetlinewidth{1.003750pt}%
\definecolor{currentstroke}{rgb}{0.600000,0.466667,0.000000}%
\pgfsetstrokecolor{currentstroke}%
\pgfsetdash{}{0pt}%
\pgfpathmoveto{\pgfqpoint{0.316935in}{1.122834in}}%
\pgfpathlineto{\pgfqpoint{0.326960in}{1.122834in}}%
\pgfpathlineto{\pgfqpoint{0.326960in}{1.169364in}}%
\pgfpathlineto{\pgfqpoint{0.316935in}{1.169364in}}%
\pgfpathlineto{\pgfqpoint{0.316935in}{1.122834in}}%
\pgfpathclose%
\pgfusepath{stroke,fill}%
\end{pgfscope}%
\begin{pgfscope}%
\pgfpathrectangle{\pgfqpoint{0.150000in}{0.150000in}}{\pgfqpoint{1.700000in}{1.700000in}}%
\pgfusepath{clip}%
\pgfsetbuttcap%
\pgfsetroundjoin%
\definecolor{currentfill}{rgb}{0.933333,0.800000,0.400000}%
\pgfsetfillcolor{currentfill}%
\pgfsetlinewidth{1.003750pt}%
\definecolor{currentstroke}{rgb}{0.600000,0.466667,0.000000}%
\pgfsetstrokecolor{currentstroke}%
\pgfsetdash{}{0pt}%
\pgfpathmoveto{\pgfqpoint{0.311174in}{1.084764in}}%
\pgfpathlineto{\pgfqpoint{0.316935in}{1.084764in}}%
\pgfpathlineto{\pgfqpoint{0.316935in}{1.122834in}}%
\pgfpathlineto{\pgfqpoint{0.311174in}{1.122834in}}%
\pgfpathlineto{\pgfqpoint{0.311174in}{1.084764in}}%
\pgfpathclose%
\pgfusepath{stroke,fill}%
\end{pgfscope}%
\begin{pgfscope}%
\pgfpathrectangle{\pgfqpoint{0.150000in}{0.150000in}}{\pgfqpoint{1.700000in}{1.700000in}}%
\pgfusepath{clip}%
\pgfsetbuttcap%
\pgfsetroundjoin%
\definecolor{currentfill}{rgb}{0.933333,0.800000,0.400000}%
\pgfsetfillcolor{currentfill}%
\pgfsetlinewidth{1.003750pt}%
\definecolor{currentstroke}{rgb}{0.600000,0.466667,0.000000}%
\pgfsetstrokecolor{currentstroke}%
\pgfsetdash{}{0pt}%
\pgfpathmoveto{\pgfqpoint{0.307032in}{1.038234in}}%
\pgfpathlineto{\pgfqpoint{0.311174in}{1.038234in}}%
\pgfpathlineto{\pgfqpoint{0.311174in}{1.084764in}}%
\pgfpathlineto{\pgfqpoint{0.307032in}{1.084764in}}%
\pgfpathlineto{\pgfqpoint{0.307032in}{1.038234in}}%
\pgfpathclose%
\pgfusepath{stroke,fill}%
\end{pgfscope}%
\begin{pgfscope}%
\pgfpathrectangle{\pgfqpoint{0.150000in}{0.150000in}}{\pgfqpoint{1.700000in}{1.700000in}}%
\pgfusepath{clip}%
\pgfsetbuttcap%
\pgfsetroundjoin%
\definecolor{currentfill}{rgb}{0.933333,0.800000,0.400000}%
\pgfsetfillcolor{currentfill}%
\pgfsetlinewidth{1.003750pt}%
\definecolor{currentstroke}{rgb}{0.600000,0.466667,0.000000}%
\pgfsetstrokecolor{currentstroke}%
\pgfsetdash{}{0pt}%
\pgfpathmoveto{\pgfqpoint{0.305978in}{1.000164in}}%
\pgfpathlineto{\pgfqpoint{0.307032in}{1.000164in}}%
\pgfpathlineto{\pgfqpoint{0.307032in}{1.038234in}}%
\pgfpathlineto{\pgfqpoint{0.305978in}{1.038234in}}%
\pgfpathlineto{\pgfqpoint{0.305978in}{1.000164in}}%
\pgfpathclose%
\pgfusepath{stroke,fill}%
\end{pgfscope}%
\begin{pgfscope}%
\pgfpathrectangle{\pgfqpoint{0.150000in}{0.150000in}}{\pgfqpoint{1.700000in}{1.700000in}}%
\pgfusepath{clip}%
\pgfsetbuttcap%
\pgfsetroundjoin%
\definecolor{currentfill}{rgb}{0.933333,0.800000,0.400000}%
\pgfsetfillcolor{currentfill}%
\pgfsetlinewidth{1.003750pt}%
\definecolor{currentstroke}{rgb}{0.600000,0.466667,0.000000}%
\pgfsetstrokecolor{currentstroke}%
\pgfsetdash{}{0pt}%
\pgfpathmoveto{\pgfqpoint{0.605301in}{0.876777in}}%
\pgfpathlineto{\pgfqpoint{0.618724in}{0.876777in}}%
\pgfpathlineto{\pgfqpoint{0.618724in}{0.930946in}}%
\pgfpathlineto{\pgfqpoint{0.605301in}{0.930946in}}%
\pgfpathlineto{\pgfqpoint{0.605301in}{0.876777in}}%
\pgfpathclose%
\pgfusepath{stroke,fill}%
\end{pgfscope}%
\begin{pgfscope}%
\pgfpathrectangle{\pgfqpoint{0.150000in}{0.150000in}}{\pgfqpoint{1.700000in}{1.700000in}}%
\pgfusepath{clip}%
\pgfsetbuttcap%
\pgfsetroundjoin%
\definecolor{currentfill}{rgb}{0.933333,0.800000,0.400000}%
\pgfsetfillcolor{currentfill}%
\pgfsetlinewidth{1.003750pt}%
\definecolor{currentstroke}{rgb}{0.600000,0.466667,0.000000}%
\pgfsetstrokecolor{currentstroke}%
\pgfsetdash{}{0pt}%
\pgfpathmoveto{\pgfqpoint{0.618724in}{0.832457in}}%
\pgfpathlineto{\pgfqpoint{0.636015in}{0.832457in}}%
\pgfpathlineto{\pgfqpoint{0.636015in}{0.876777in}}%
\pgfpathlineto{\pgfqpoint{0.618724in}{0.876777in}}%
\pgfpathlineto{\pgfqpoint{0.618724in}{0.832457in}}%
\pgfpathclose%
\pgfusepath{stroke,fill}%
\end{pgfscope}%
\begin{pgfscope}%
\pgfpathrectangle{\pgfqpoint{0.150000in}{0.150000in}}{\pgfqpoint{1.700000in}{1.700000in}}%
\pgfusepath{clip}%
\pgfsetbuttcap%
\pgfsetroundjoin%
\definecolor{currentfill}{rgb}{0.933333,0.800000,0.400000}%
\pgfsetfillcolor{currentfill}%
\pgfsetlinewidth{1.003750pt}%
\definecolor{currentstroke}{rgb}{0.600000,0.466667,0.000000}%
\pgfsetstrokecolor{currentstroke}%
\pgfsetdash{}{0pt}%
\pgfpathmoveto{\pgfqpoint{0.636015in}{0.788137in}}%
\pgfpathlineto{\pgfqpoint{0.659898in}{0.788137in}}%
\pgfpathlineto{\pgfqpoint{0.659898in}{0.832457in}}%
\pgfpathlineto{\pgfqpoint{0.636015in}{0.832457in}}%
\pgfpathlineto{\pgfqpoint{0.636015in}{0.788137in}}%
\pgfpathclose%
\pgfusepath{stroke,fill}%
\end{pgfscope}%
\begin{pgfscope}%
\pgfpathrectangle{\pgfqpoint{0.150000in}{0.150000in}}{\pgfqpoint{1.700000in}{1.700000in}}%
\pgfusepath{clip}%
\pgfsetbuttcap%
\pgfsetroundjoin%
\definecolor{currentfill}{rgb}{0.933333,0.800000,0.400000}%
\pgfsetfillcolor{currentfill}%
\pgfsetlinewidth{1.003750pt}%
\definecolor{currentstroke}{rgb}{0.600000,0.466667,0.000000}%
\pgfsetstrokecolor{currentstroke}%
\pgfsetdash{}{0pt}%
\pgfpathmoveto{\pgfqpoint{0.659898in}{0.751875in}}%
\pgfpathlineto{\pgfqpoint{0.685374in}{0.751875in}}%
\pgfpathlineto{\pgfqpoint{0.685374in}{0.788137in}}%
\pgfpathlineto{\pgfqpoint{0.659898in}{0.788137in}}%
\pgfpathlineto{\pgfqpoint{0.659898in}{0.751875in}}%
\pgfpathclose%
\pgfusepath{stroke,fill}%
\end{pgfscope}%
\begin{pgfscope}%
\pgfpathrectangle{\pgfqpoint{0.150000in}{0.150000in}}{\pgfqpoint{1.700000in}{1.700000in}}%
\pgfusepath{clip}%
\pgfsetbuttcap%
\pgfsetroundjoin%
\definecolor{currentfill}{rgb}{0.933333,0.800000,0.400000}%
\pgfsetfillcolor{currentfill}%
\pgfsetlinewidth{1.003750pt}%
\definecolor{currentstroke}{rgb}{0.600000,0.466667,0.000000}%
\pgfsetstrokecolor{currentstroke}%
\pgfsetdash{}{0pt}%
\pgfpathmoveto{\pgfqpoint{0.795725in}{0.625915in}}%
\pgfpathlineto{\pgfqpoint{0.856418in}{0.625915in}}%
\pgfpathlineto{\pgfqpoint{0.856418in}{0.655287in}}%
\pgfpathlineto{\pgfqpoint{0.795725in}{0.655287in}}%
\pgfpathlineto{\pgfqpoint{0.795725in}{0.625915in}}%
\pgfpathclose%
\pgfusepath{stroke,fill}%
\end{pgfscope}%
\begin{pgfscope}%
\pgfpathrectangle{\pgfqpoint{0.150000in}{0.150000in}}{\pgfqpoint{1.700000in}{1.700000in}}%
\pgfusepath{clip}%
\pgfsetbuttcap%
\pgfsetroundjoin%
\definecolor{currentfill}{rgb}{0.933333,0.800000,0.400000}%
\pgfsetfillcolor{currentfill}%
\pgfsetlinewidth{1.003750pt}%
\definecolor{currentstroke}{rgb}{0.600000,0.466667,0.000000}%
\pgfsetstrokecolor{currentstroke}%
\pgfsetdash{}{0pt}%
\pgfpathmoveto{\pgfqpoint{0.735032in}{0.655287in}}%
\pgfpathlineto{\pgfqpoint{0.795725in}{0.655287in}}%
\pgfpathlineto{\pgfqpoint{0.795725in}{0.699421in}}%
\pgfpathlineto{\pgfqpoint{0.735032in}{0.699421in}}%
\pgfpathlineto{\pgfqpoint{0.735032in}{0.655287in}}%
\pgfpathclose%
\pgfusepath{stroke,fill}%
\end{pgfscope}%
\begin{pgfscope}%
\pgfpathrectangle{\pgfqpoint{0.150000in}{0.150000in}}{\pgfqpoint{1.700000in}{1.700000in}}%
\pgfusepath{clip}%
\pgfsetbuttcap%
\pgfsetroundjoin%
\definecolor{currentfill}{rgb}{0.933333,0.800000,0.400000}%
\pgfsetfillcolor{currentfill}%
\pgfsetlinewidth{1.003750pt}%
\definecolor{currentstroke}{rgb}{0.600000,0.466667,0.000000}%
\pgfsetstrokecolor{currentstroke}%
\pgfsetdash{}{0pt}%
\pgfpathmoveto{\pgfqpoint{0.685374in}{0.699421in}}%
\pgfpathlineto{\pgfqpoint{0.735032in}{0.699421in}}%
\pgfpathlineto{\pgfqpoint{0.735032in}{0.751875in}}%
\pgfpathlineto{\pgfqpoint{0.685374in}{0.751875in}}%
\pgfpathlineto{\pgfqpoint{0.685374in}{0.699421in}}%
\pgfpathclose%
\pgfusepath{stroke,fill}%
\end{pgfscope}%
\begin{pgfscope}%
\pgfpathrectangle{\pgfqpoint{0.150000in}{0.150000in}}{\pgfqpoint{1.700000in}{1.700000in}}%
\pgfusepath{clip}%
\pgfsetbuttcap%
\pgfsetroundjoin%
\definecolor{currentfill}{rgb}{0.933333,0.800000,0.400000}%
\pgfsetfillcolor{currentfill}%
\pgfsetlinewidth{1.003750pt}%
\definecolor{currentstroke}{rgb}{0.600000,0.466667,0.000000}%
\pgfsetstrokecolor{currentstroke}%
\pgfsetdash{}{0pt}%
\pgfpathmoveto{\pgfqpoint{0.309422in}{0.874076in}}%
\pgfpathlineto{\pgfqpoint{0.317497in}{0.874076in}}%
\pgfpathlineto{\pgfqpoint{0.317497in}{0.930946in}}%
\pgfpathlineto{\pgfqpoint{0.309422in}{0.930946in}}%
\pgfpathlineto{\pgfqpoint{0.309422in}{0.874076in}}%
\pgfpathclose%
\pgfusepath{stroke,fill}%
\end{pgfscope}%
\begin{pgfscope}%
\pgfpathrectangle{\pgfqpoint{0.150000in}{0.150000in}}{\pgfqpoint{1.700000in}{1.700000in}}%
\pgfusepath{clip}%
\pgfsetbuttcap%
\pgfsetroundjoin%
\definecolor{currentfill}{rgb}{0.933333,0.800000,0.400000}%
\pgfsetfillcolor{currentfill}%
\pgfsetlinewidth{1.003750pt}%
\definecolor{currentstroke}{rgb}{0.600000,0.466667,0.000000}%
\pgfsetstrokecolor{currentstroke}%
\pgfsetdash{}{0pt}%
\pgfpathmoveto{\pgfqpoint{0.317497in}{0.827545in}}%
\pgfpathlineto{\pgfqpoint{0.327746in}{0.827545in}}%
\pgfpathlineto{\pgfqpoint{0.327746in}{0.874076in}}%
\pgfpathlineto{\pgfqpoint{0.317497in}{0.874076in}}%
\pgfpathlineto{\pgfqpoint{0.317497in}{0.827545in}}%
\pgfpathclose%
\pgfusepath{stroke,fill}%
\end{pgfscope}%
\begin{pgfscope}%
\pgfpathrectangle{\pgfqpoint{0.150000in}{0.150000in}}{\pgfqpoint{1.700000in}{1.700000in}}%
\pgfusepath{clip}%
\pgfsetbuttcap%
\pgfsetroundjoin%
\definecolor{currentfill}{rgb}{0.933333,0.800000,0.400000}%
\pgfsetfillcolor{currentfill}%
\pgfsetlinewidth{1.003750pt}%
\definecolor{currentstroke}{rgb}{0.600000,0.466667,0.000000}%
\pgfsetstrokecolor{currentstroke}%
\pgfsetdash{}{0pt}%
\pgfpathmoveto{\pgfqpoint{0.327746in}{0.781015in}}%
\pgfpathlineto{\pgfqpoint{0.341432in}{0.781015in}}%
\pgfpathlineto{\pgfqpoint{0.341432in}{0.827545in}}%
\pgfpathlineto{\pgfqpoint{0.327746in}{0.827545in}}%
\pgfpathlineto{\pgfqpoint{0.327746in}{0.781015in}}%
\pgfpathclose%
\pgfusepath{stroke,fill}%
\end{pgfscope}%
\begin{pgfscope}%
\pgfpathrectangle{\pgfqpoint{0.150000in}{0.150000in}}{\pgfqpoint{1.700000in}{1.700000in}}%
\pgfusepath{clip}%
\pgfsetbuttcap%
\pgfsetroundjoin%
\definecolor{currentfill}{rgb}{0.933333,0.800000,0.400000}%
\pgfsetfillcolor{currentfill}%
\pgfsetlinewidth{1.003750pt}%
\definecolor{currentstroke}{rgb}{0.600000,0.466667,0.000000}%
\pgfsetstrokecolor{currentstroke}%
\pgfsetdash{}{0pt}%
\pgfpathmoveto{\pgfqpoint{0.341432in}{0.742945in}}%
\pgfpathlineto{\pgfqpoint{0.355338in}{0.742945in}}%
\pgfpathlineto{\pgfqpoint{0.355338in}{0.781015in}}%
\pgfpathlineto{\pgfqpoint{0.341432in}{0.781015in}}%
\pgfpathlineto{\pgfqpoint{0.341432in}{0.742945in}}%
\pgfpathclose%
\pgfusepath{stroke,fill}%
\end{pgfscope}%
\begin{pgfscope}%
\pgfpathrectangle{\pgfqpoint{0.150000in}{0.150000in}}{\pgfqpoint{1.700000in}{1.700000in}}%
\pgfusepath{clip}%
\pgfsetbuttcap%
\pgfsetroundjoin%
\definecolor{currentfill}{rgb}{0.933333,0.800000,0.400000}%
\pgfsetfillcolor{currentfill}%
\pgfsetlinewidth{1.003750pt}%
\definecolor{currentstroke}{rgb}{0.600000,0.466667,0.000000}%
\pgfsetstrokecolor{currentstroke}%
\pgfsetdash{}{0pt}%
\pgfpathmoveto{\pgfqpoint{0.355338in}{0.696415in}}%
\pgfpathlineto{\pgfqpoint{0.375898in}{0.696415in}}%
\pgfpathlineto{\pgfqpoint{0.375898in}{0.742945in}}%
\pgfpathlineto{\pgfqpoint{0.355338in}{0.742945in}}%
\pgfpathlineto{\pgfqpoint{0.355338in}{0.696415in}}%
\pgfpathclose%
\pgfusepath{stroke,fill}%
\end{pgfscope}%
\begin{pgfscope}%
\pgfpathrectangle{\pgfqpoint{0.150000in}{0.150000in}}{\pgfqpoint{1.700000in}{1.700000in}}%
\pgfusepath{clip}%
\pgfsetbuttcap%
\pgfsetroundjoin%
\definecolor{currentfill}{rgb}{0.933333,0.800000,0.400000}%
\pgfsetfillcolor{currentfill}%
\pgfsetlinewidth{1.003750pt}%
\definecolor{currentstroke}{rgb}{0.600000,0.466667,0.000000}%
\pgfsetstrokecolor{currentstroke}%
\pgfsetdash{}{0pt}%
\pgfpathmoveto{\pgfqpoint{0.375898in}{0.658345in}}%
\pgfpathlineto{\pgfqpoint{0.395899in}{0.658345in}}%
\pgfpathlineto{\pgfqpoint{0.395899in}{0.696415in}}%
\pgfpathlineto{\pgfqpoint{0.375898in}{0.696415in}}%
\pgfpathlineto{\pgfqpoint{0.375898in}{0.658345in}}%
\pgfpathclose%
\pgfusepath{stroke,fill}%
\end{pgfscope}%
\begin{pgfscope}%
\pgfpathrectangle{\pgfqpoint{0.150000in}{0.150000in}}{\pgfqpoint{1.700000in}{1.700000in}}%
\pgfusepath{clip}%
\pgfsetbuttcap%
\pgfsetroundjoin%
\definecolor{currentfill}{rgb}{0.933333,0.800000,0.400000}%
\pgfsetfillcolor{currentfill}%
\pgfsetlinewidth{1.003750pt}%
\definecolor{currentstroke}{rgb}{0.600000,0.466667,0.000000}%
\pgfsetstrokecolor{currentstroke}%
\pgfsetdash{}{0pt}%
\pgfpathmoveto{\pgfqpoint{0.885766in}{0.309457in}}%
\pgfpathlineto{\pgfqpoint{0.930598in}{0.309457in}}%
\pgfpathlineto{\pgfqpoint{0.930598in}{0.315444in}}%
\pgfpathlineto{\pgfqpoint{0.885766in}{0.315444in}}%
\pgfpathlineto{\pgfqpoint{0.885766in}{0.309457in}}%
\pgfpathclose%
\pgfusepath{stroke,fill}%
\end{pgfscope}%
\begin{pgfscope}%
\pgfpathrectangle{\pgfqpoint{0.150000in}{0.150000in}}{\pgfqpoint{1.700000in}{1.700000in}}%
\pgfusepath{clip}%
\pgfsetbuttcap%
\pgfsetroundjoin%
\definecolor{currentfill}{rgb}{0.933333,0.800000,0.400000}%
\pgfsetfillcolor{currentfill}%
\pgfsetlinewidth{1.003750pt}%
\definecolor{currentstroke}{rgb}{0.600000,0.466667,0.000000}%
\pgfsetstrokecolor{currentstroke}%
\pgfsetdash{}{0pt}%
\pgfpathmoveto{\pgfqpoint{0.849086in}{0.315444in}}%
\pgfpathlineto{\pgfqpoint{0.885766in}{0.315444in}}%
\pgfpathlineto{\pgfqpoint{0.885766in}{0.322585in}}%
\pgfpathlineto{\pgfqpoint{0.849086in}{0.322585in}}%
\pgfpathlineto{\pgfqpoint{0.849086in}{0.315444in}}%
\pgfpathclose%
\pgfusepath{stroke,fill}%
\end{pgfscope}%
\begin{pgfscope}%
\pgfpathrectangle{\pgfqpoint{0.150000in}{0.150000in}}{\pgfqpoint{1.700000in}{1.700000in}}%
\pgfusepath{clip}%
\pgfsetbuttcap%
\pgfsetroundjoin%
\definecolor{currentfill}{rgb}{0.933333,0.800000,0.400000}%
\pgfsetfillcolor{currentfill}%
\pgfsetlinewidth{1.003750pt}%
\definecolor{currentstroke}{rgb}{0.600000,0.466667,0.000000}%
\pgfsetstrokecolor{currentstroke}%
\pgfsetdash{}{0pt}%
\pgfpathmoveto{\pgfqpoint{1.274139in}{1.248742in}}%
\pgfpathlineto{\pgfqpoint{1.314138in}{1.248742in}}%
\pgfpathlineto{\pgfqpoint{1.314138in}{1.292239in}}%
\pgfpathlineto{\pgfqpoint{1.274139in}{1.292239in}}%
\pgfpathlineto{\pgfqpoint{1.274139in}{1.248742in}}%
\pgfpathclose%
\pgfusepath{stroke,fill}%
\end{pgfscope}%
\begin{pgfscope}%
\pgfpathrectangle{\pgfqpoint{0.150000in}{0.150000in}}{\pgfqpoint{1.700000in}{1.700000in}}%
\pgfusepath{clip}%
\pgfsetbuttcap%
\pgfsetroundjoin%
\definecolor{currentfill}{rgb}{0.933333,0.800000,0.400000}%
\pgfsetfillcolor{currentfill}%
\pgfsetlinewidth{1.003750pt}%
\definecolor{currentstroke}{rgb}{0.600000,0.466667,0.000000}%
\pgfsetstrokecolor{currentstroke}%
\pgfsetdash{}{0pt}%
\pgfpathmoveto{\pgfqpoint{0.930598in}{1.690543in}}%
\pgfpathlineto{\pgfqpoint{1.000165in}{1.690543in}}%
\pgfpathlineto{\pgfqpoint{1.000165in}{1.694022in}}%
\pgfpathlineto{\pgfqpoint{0.930598in}{1.694022in}}%
\pgfpathlineto{\pgfqpoint{0.930598in}{1.690543in}}%
\pgfpathclose%
\pgfusepath{stroke,fill}%
\end{pgfscope}%
\begin{pgfscope}%
\pgfpathrectangle{\pgfqpoint{0.150000in}{0.150000in}}{\pgfqpoint{1.700000in}{1.700000in}}%
\pgfusepath{clip}%
\pgfsetbuttcap%
\pgfsetroundjoin%
\definecolor{currentfill}{rgb}{0.933333,0.800000,0.400000}%
\pgfsetfillcolor{currentfill}%
\pgfsetlinewidth{1.003750pt}%
\definecolor{currentstroke}{rgb}{0.600000,0.466667,0.000000}%
\pgfsetstrokecolor{currentstroke}%
\pgfsetdash{}{0pt}%
\pgfpathmoveto{\pgfqpoint{0.930598in}{1.394638in}}%
\pgfpathlineto{\pgfqpoint{1.000165in}{1.394638in}}%
\pgfpathlineto{\pgfqpoint{1.000165in}{1.400694in}}%
\pgfpathlineto{\pgfqpoint{0.930598in}{1.400694in}}%
\pgfpathlineto{\pgfqpoint{0.930598in}{1.394638in}}%
\pgfpathclose%
\pgfusepath{stroke,fill}%
\end{pgfscope}%
\begin{pgfscope}%
\pgfpathrectangle{\pgfqpoint{0.150000in}{0.150000in}}{\pgfqpoint{1.700000in}{1.700000in}}%
\pgfusepath{clip}%
\pgfsetbuttcap%
\pgfsetroundjoin%
\definecolor{currentfill}{rgb}{0.933333,0.800000,0.400000}%
\pgfsetfillcolor{currentfill}%
\pgfsetlinewidth{1.003750pt}%
\definecolor{currentstroke}{rgb}{0.600000,0.466667,0.000000}%
\pgfsetstrokecolor{currentstroke}%
\pgfsetdash{}{0pt}%
\pgfpathmoveto{\pgfqpoint{1.272573in}{0.361744in}}%
\pgfpathlineto{\pgfqpoint{1.334676in}{0.361744in}}%
\pgfpathlineto{\pgfqpoint{1.334676in}{0.392004in}}%
\pgfpathlineto{\pgfqpoint{1.272573in}{0.392004in}}%
\pgfpathlineto{\pgfqpoint{1.272573in}{0.361744in}}%
\pgfpathclose%
\pgfusepath{stroke,fill}%
\end{pgfscope}%
\begin{pgfscope}%
\pgfpathrectangle{\pgfqpoint{0.150000in}{0.150000in}}{\pgfqpoint{1.700000in}{1.700000in}}%
\pgfusepath{clip}%
\pgfsetbuttcap%
\pgfsetroundjoin%
\definecolor{currentfill}{rgb}{0.933333,0.800000,0.400000}%
\pgfsetfillcolor{currentfill}%
\pgfsetlinewidth{1.003750pt}%
\definecolor{currentstroke}{rgb}{0.600000,0.466667,0.000000}%
\pgfsetstrokecolor{currentstroke}%
\pgfsetdash{}{0pt}%
\pgfpathmoveto{\pgfqpoint{0.930598in}{0.599306in}}%
\pgfpathlineto{\pgfqpoint{0.999848in}{0.599306in}}%
\pgfpathlineto{\pgfqpoint{0.999848in}{0.605362in}}%
\pgfpathlineto{\pgfqpoint{0.930598in}{0.605362in}}%
\pgfpathlineto{\pgfqpoint{0.930598in}{0.599306in}}%
\pgfpathclose%
\pgfusepath{stroke,fill}%
\end{pgfscope}%
\begin{pgfscope}%
\pgfpathrectangle{\pgfqpoint{0.150000in}{0.150000in}}{\pgfqpoint{1.700000in}{1.700000in}}%
\pgfusepath{clip}%
\pgfsetbuttcap%
\pgfsetroundjoin%
\definecolor{currentfill}{rgb}{0.933333,0.800000,0.400000}%
\pgfsetfillcolor{currentfill}%
\pgfsetlinewidth{1.003750pt}%
\definecolor{currentstroke}{rgb}{0.600000,0.466667,0.000000}%
\pgfsetstrokecolor{currentstroke}%
\pgfsetdash{}{0pt}%
\pgfpathmoveto{\pgfqpoint{0.930598in}{0.305978in}}%
\pgfpathlineto{\pgfqpoint{0.999848in}{0.305978in}}%
\pgfpathlineto{\pgfqpoint{0.999848in}{0.309457in}}%
\pgfpathlineto{\pgfqpoint{0.930598in}{0.309457in}}%
\pgfpathlineto{\pgfqpoint{0.930598in}{0.305978in}}%
\pgfpathclose%
\pgfusepath{stroke,fill}%
\end{pgfscope}%
\begin{pgfscope}%
\pgfpathrectangle{\pgfqpoint{0.150000in}{0.150000in}}{\pgfqpoint{1.700000in}{1.700000in}}%
\pgfusepath{clip}%
\pgfsetbuttcap%
\pgfsetroundjoin%
\definecolor{currentfill}{rgb}{0.933333,0.800000,0.400000}%
\pgfsetfillcolor{currentfill}%
\pgfsetlinewidth{1.003750pt}%
\definecolor{currentstroke}{rgb}{0.600000,0.466667,0.000000}%
\pgfsetstrokecolor{currentstroke}%
\pgfsetdash{}{0pt}%
\pgfpathmoveto{\pgfqpoint{0.361826in}{1.272765in}}%
\pgfpathlineto{\pgfqpoint{0.392092in}{1.272765in}}%
\pgfpathlineto{\pgfqpoint{0.392092in}{1.334835in}}%
\pgfpathlineto{\pgfqpoint{0.361826in}{1.334835in}}%
\pgfpathlineto{\pgfqpoint{0.361826in}{1.272765in}}%
\pgfpathclose%
\pgfusepath{stroke,fill}%
\end{pgfscope}%
\begin{pgfscope}%
\pgfpathrectangle{\pgfqpoint{0.150000in}{0.150000in}}{\pgfqpoint{1.700000in}{1.700000in}}%
\pgfusepath{clip}%
\pgfsetbuttcap%
\pgfsetroundjoin%
\definecolor{currentfill}{rgb}{0.933333,0.800000,0.400000}%
\pgfsetfillcolor{currentfill}%
\pgfsetlinewidth{1.003750pt}%
\definecolor{currentstroke}{rgb}{0.600000,0.466667,0.000000}%
\pgfsetstrokecolor{currentstroke}%
\pgfsetdash{}{0pt}%
\pgfpathmoveto{\pgfqpoint{0.599306in}{0.930946in}}%
\pgfpathlineto{\pgfqpoint{0.605301in}{0.930946in}}%
\pgfpathlineto{\pgfqpoint{0.605301in}{1.000164in}}%
\pgfpathlineto{\pgfqpoint{0.599306in}{1.000164in}}%
\pgfpathlineto{\pgfqpoint{0.599306in}{0.930946in}}%
\pgfpathclose%
\pgfusepath{stroke,fill}%
\end{pgfscope}%
\begin{pgfscope}%
\pgfpathrectangle{\pgfqpoint{0.150000in}{0.150000in}}{\pgfqpoint{1.700000in}{1.700000in}}%
\pgfusepath{clip}%
\pgfsetbuttcap%
\pgfsetroundjoin%
\definecolor{currentfill}{rgb}{0.933333,0.800000,0.400000}%
\pgfsetfillcolor{currentfill}%
\pgfsetlinewidth{1.003750pt}%
\definecolor{currentstroke}{rgb}{0.600000,0.466667,0.000000}%
\pgfsetstrokecolor{currentstroke}%
\pgfsetdash{}{0pt}%
\pgfpathmoveto{\pgfqpoint{0.305978in}{0.930946in}}%
\pgfpathlineto{\pgfqpoint{0.309422in}{0.930946in}}%
\pgfpathlineto{\pgfqpoint{0.309422in}{1.000164in}}%
\pgfpathlineto{\pgfqpoint{0.305978in}{1.000164in}}%
\pgfpathlineto{\pgfqpoint{0.305978in}{0.930946in}}%
\pgfpathclose%
\pgfusepath{stroke,fill}%
\end{pgfscope}%
\begin{pgfscope}%
\pgfpathrectangle{\pgfqpoint{0.150000in}{0.150000in}}{\pgfqpoint{1.700000in}{1.700000in}}%
\pgfusepath{clip}%
\pgfsetbuttcap%
\pgfsetroundjoin%
\definecolor{currentfill}{rgb}{0.933333,0.800000,0.400000}%
\pgfsetfillcolor{currentfill}%
\pgfsetlinewidth{1.003750pt}%
\definecolor{currentstroke}{rgb}{0.600000,0.466667,0.000000}%
\pgfsetstrokecolor{currentstroke}%
\pgfsetdash{}{0pt}%
\pgfpathmoveto{\pgfqpoint{0.395899in}{0.589127in}}%
\pgfpathlineto{\pgfqpoint{0.440670in}{0.589127in}}%
\pgfpathlineto{\pgfqpoint{0.440670in}{0.658345in}}%
\pgfpathlineto{\pgfqpoint{0.395899in}{0.658345in}}%
\pgfpathlineto{\pgfqpoint{0.395899in}{0.589127in}}%
\pgfpathclose%
\pgfusepath{stroke,fill}%
\end{pgfscope}%
\begin{pgfscope}%
\pgfpathrectangle{\pgfqpoint{0.150000in}{0.150000in}}{\pgfqpoint{1.700000in}{1.700000in}}%
\pgfusepath{clip}%
\pgfsetbuttcap%
\pgfsetroundjoin%
\definecolor{currentfill}{rgb}{0.933333,0.800000,0.400000}%
\pgfsetfillcolor{currentfill}%
\pgfsetlinewidth{1.003750pt}%
\definecolor{currentstroke}{rgb}{0.600000,0.466667,0.000000}%
\pgfsetstrokecolor{currentstroke}%
\pgfsetdash{}{0pt}%
\pgfpathmoveto{\pgfqpoint{0.782395in}{0.322585in}}%
\pgfpathlineto{\pgfqpoint{0.849086in}{0.322585in}}%
\pgfpathlineto{\pgfqpoint{0.849086in}{0.340975in}}%
\pgfpathlineto{\pgfqpoint{0.782395in}{0.340975in}}%
\pgfpathlineto{\pgfqpoint{0.782395in}{0.322585in}}%
\pgfpathclose%
\pgfusepath{stroke,fill}%
\end{pgfscope}%
\begin{pgfscope}%
\pgfpathrectangle{\pgfqpoint{0.150000in}{0.150000in}}{\pgfqpoint{1.700000in}{1.700000in}}%
\pgfusepath{clip}%
\pgfsetbuttcap%
\pgfsetroundjoin%
\definecolor{currentfill}{rgb}{0.933333,0.800000,0.400000}%
\pgfsetfillcolor{currentfill}%
\pgfsetlinewidth{1.003750pt}%
\definecolor{currentstroke}{rgb}{0.600000,0.466667,0.000000}%
\pgfsetstrokecolor{currentstroke}%
\pgfsetdash{}{0pt}%
\pgfpathmoveto{\pgfqpoint{0.715703in}{0.340975in}}%
\pgfpathlineto{\pgfqpoint{0.782395in}{0.340975in}}%
\pgfpathlineto{\pgfqpoint{0.782395in}{0.366879in}}%
\pgfpathlineto{\pgfqpoint{0.715703in}{0.366879in}}%
\pgfpathlineto{\pgfqpoint{0.715703in}{0.340975in}}%
\pgfpathclose%
\pgfusepath{stroke,fill}%
\end{pgfscope}%
\begin{pgfscope}%
\pgfpathrectangle{\pgfqpoint{0.150000in}{0.150000in}}{\pgfqpoint{1.700000in}{1.700000in}}%
\pgfusepath{clip}%
\pgfsetbuttcap%
\pgfsetroundjoin%
\definecolor{currentfill}{rgb}{0.933333,0.800000,0.400000}%
\pgfsetfillcolor{currentfill}%
\pgfsetlinewidth{1.003750pt}%
\definecolor{currentstroke}{rgb}{0.600000,0.466667,0.000000}%
\pgfsetstrokecolor{currentstroke}%
\pgfsetdash{}{0pt}%
\pgfpathmoveto{\pgfqpoint{0.661138in}{0.366879in}}%
\pgfpathlineto{\pgfqpoint{0.715703in}{0.366879in}}%
\pgfpathlineto{\pgfqpoint{0.715703in}{0.394328in}}%
\pgfpathlineto{\pgfqpoint{0.661138in}{0.394328in}}%
\pgfpathlineto{\pgfqpoint{0.661138in}{0.366879in}}%
\pgfpathclose%
\pgfusepath{stroke,fill}%
\end{pgfscope}%
\begin{pgfscope}%
\pgfpathrectangle{\pgfqpoint{0.150000in}{0.150000in}}{\pgfqpoint{1.700000in}{1.700000in}}%
\pgfusepath{clip}%
\pgfsetbuttcap%
\pgfsetroundjoin%
\definecolor{currentfill}{rgb}{0.933333,0.800000,0.400000}%
\pgfsetfillcolor{currentfill}%
\pgfsetlinewidth{1.003750pt}%
\definecolor{currentstroke}{rgb}{0.600000,0.466667,0.000000}%
\pgfsetstrokecolor{currentstroke}%
\pgfsetdash{}{0pt}%
\pgfpathmoveto{\pgfqpoint{0.594446in}{0.394328in}}%
\pgfpathlineto{\pgfqpoint{0.661138in}{0.394328in}}%
\pgfpathlineto{\pgfqpoint{0.661138in}{0.436801in}}%
\pgfpathlineto{\pgfqpoint{0.594446in}{0.436801in}}%
\pgfpathlineto{\pgfqpoint{0.594446in}{0.394328in}}%
\pgfpathclose%
\pgfusepath{stroke,fill}%
\end{pgfscope}%
\begin{pgfscope}%
\pgfpathrectangle{\pgfqpoint{0.150000in}{0.150000in}}{\pgfqpoint{1.700000in}{1.700000in}}%
\pgfusepath{clip}%
\pgfsetbuttcap%
\pgfsetroundjoin%
\definecolor{currentfill}{rgb}{0.933333,0.800000,0.400000}%
\pgfsetfillcolor{currentfill}%
\pgfsetlinewidth{1.003750pt}%
\definecolor{currentstroke}{rgb}{0.600000,0.466667,0.000000}%
\pgfsetstrokecolor{currentstroke}%
\pgfsetdash{}{0pt}%
\pgfpathmoveto{\pgfqpoint{0.539881in}{0.436801in}}%
\pgfpathlineto{\pgfqpoint{0.594446in}{0.436801in}}%
\pgfpathlineto{\pgfqpoint{0.594446in}{0.480426in}}%
\pgfpathlineto{\pgfqpoint{0.539881in}{0.480426in}}%
\pgfpathlineto{\pgfqpoint{0.539881in}{0.436801in}}%
\pgfpathclose%
\pgfusepath{stroke,fill}%
\end{pgfscope}%
\begin{pgfscope}%
\pgfpathrectangle{\pgfqpoint{0.150000in}{0.150000in}}{\pgfqpoint{1.700000in}{1.700000in}}%
\pgfusepath{clip}%
\pgfsetbuttcap%
\pgfsetroundjoin%
\definecolor{currentfill}{rgb}{0.933333,0.800000,0.400000}%
\pgfsetfillcolor{currentfill}%
\pgfsetlinewidth{1.003750pt}%
\definecolor{currentstroke}{rgb}{0.600000,0.466667,0.000000}%
\pgfsetstrokecolor{currentstroke}%
\pgfsetdash{}{0pt}%
\pgfpathmoveto{\pgfqpoint{0.440670in}{0.529342in}}%
\pgfpathlineto{\pgfqpoint{0.489954in}{0.529342in}}%
\pgfpathlineto{\pgfqpoint{0.489954in}{0.589127in}}%
\pgfpathlineto{\pgfqpoint{0.440670in}{0.589127in}}%
\pgfpathlineto{\pgfqpoint{0.440670in}{0.529342in}}%
\pgfpathclose%
\pgfusepath{stroke,fill}%
\end{pgfscope}%
\begin{pgfscope}%
\pgfpathrectangle{\pgfqpoint{0.150000in}{0.150000in}}{\pgfqpoint{1.700000in}{1.700000in}}%
\pgfusepath{clip}%
\pgfsetbuttcap%
\pgfsetroundjoin%
\definecolor{currentfill}{rgb}{0.933333,0.800000,0.400000}%
\pgfsetfillcolor{currentfill}%
\pgfsetlinewidth{1.003750pt}%
\definecolor{currentstroke}{rgb}{0.600000,0.466667,0.000000}%
\pgfsetstrokecolor{currentstroke}%
\pgfsetdash{}{0pt}%
\pgfpathmoveto{\pgfqpoint{0.489954in}{0.480426in}}%
\pgfpathlineto{\pgfqpoint{0.539881in}{0.480426in}}%
\pgfpathlineto{\pgfqpoint{0.539881in}{0.529342in}}%
\pgfpathlineto{\pgfqpoint{0.489954in}{0.529342in}}%
\pgfpathlineto{\pgfqpoint{0.489954in}{0.480426in}}%
\pgfpathclose%
\pgfusepath{stroke,fill}%
\end{pgfscope}%
\end{pgfpicture}%
\makeatother%
\endgroup%

				\subcaption{Expected union of three adjacent ring separators}
			\end{subfigure}
			\caption{Union of adjacent boundary-overlapping separators. In red the inner set, in blue the outer set, in yellow the uncertain set.}
			\label{fig:general_case}
		\end{figure}

	% REFERENCES
	\subsection*{References}
		\begin{description}

			\item[{[1]}] L. Jaulin, M. Kieffer, O. Didrit, and E. Walter, Applied Interval Analysis: With Examples in Parameter and State Estimation, Robust Control and Robotics. Springer Science \& Business Media, 2012.
			
			\item[{[2]}] L. Jaulin and B. Desrochers, “Introduction to the algebra of separators with application to path planning,” Engineering Applications of Artificial Intelligence, vol. 33, pp. 141–147, Aug. 2014
			% , doi: 10.1016/j.engappai.2014.04.010.
			
			\item[{[3]}] R. Guyonneau, “Méthodes ensemblistes pour la localisation en robotique mobile,” These de doctorat, Angers, 2013.
			% Accessed: Jun. 06, 2023. [Online]. Available: \url{https://www.theses.fr/2013ANGE0046}
				
	\end{description}

\end{document}
        

%%% Local Variables:
%%% mode: latex
%%% TeX-master: t
%%% End:
