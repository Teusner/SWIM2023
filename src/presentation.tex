\PassOptionsToPackage{dvipsnames}{xcolor}
\documentclass[10pt, xcolor={usenames, dvipsnames}]{beamer}

\usepackage[scale=2]{ccicons}
\usepackage{graphicx}
\usepackage{booktabs}
\usepackage{gensymb}
\usepackage{multimedia}
\usepackage{hyperref}
\usepackage{txfonts}
\usepackage{caption}
\usepackage{subcaption}
\usepackage{siunitx}
\usepackage{colortbl}
\usepackage{arydshln}
\usepackage{pgf}
\usepackage{adjustbox}

\usepackage[style=authoryear,backend=biber]{biblatex}
\renewcommand*{\nameyeardelim}{\addcomma\addspace}
\addbibresource{bib/abstract.bib}

% Beamer configuration
\usetheme[sectionpage=progressbar, numbering=counter, progressbar=frametitle]{metropolis}

\usepackage{xfp}
\usepackage{pgfplots}
\usepackage{pgfplotsthemetol}
\usepackage{tikz}
\usetikzlibrary{automata,positioning,arrows,decorations.pathmorphing,calc,patterns,decorations.markings,decorations.shapes,shapes.geometric,matrix,arrows.meta,decorations.pathreplacing,overlay-beamer-styles,karnaugh,overlay-beamer-styles,fit,backgrounds}
\usepgfplotslibrary{groupplots,units}
\pgfplotsset{width=7cm,compat=1.18}

\tikzset{paint/.style={ draw=#1, fill=#1 },
         decorate with/.style=
{decorate,decoration={shape backgrounds,shape=#1,shape size=1mm,shape sep=.5cm}}}

\newcommand{\mysetminusD}{\hbox{\tikz{\draw[line width=0.6pt,line cap=round] (3pt,0) -- (0,6pt);}}}
\newcommand{\mysetminusT}{\mysetminusD}
\newcommand{\mysetminusS}{\hbox{\tikz{\draw[line width=0.45pt,line cap=round] (2pt,0) -- (0,4pt);}}}
\newcommand{\mysetminusSS}{\hbox{\tikz{\draw[line width=0.4pt,line cap=round] (1.5pt,0) -- (0,3pt);}}}

\newcommand{\mysetminus}{\mathbin{\mathchoice{\mysetminusD}{\mysetminusT}{\mysetminusS}{\mysetminusSS}}}

%command for alg-closure that automatically adapts its 'bar' to the arg based on the args length (including '\')
\newcommand{\ols}[1]{\mskip.5\thinmuskip\overline{\mskip-.5\thinmuskip {#1} \mskip-.5\thinmuskip}\mskip.5\thinmuskip} % overline short
\newcommand{\olsi}[1]{\,\overline{\!{#1}}} % overline short italic
\makeatletter
\newcommand\closure[1]{
  \tctestifnum{\count@stringtoks{#1}>1} %checks if number of chars in arg > 1 (including '\')
  {\ols{#1}} %if arg is longer than just one char, e.g. \mathbb{Q}, \mathbb{F},...
  {\olsi{#1}} %if arg is just one char, e.g. K, L,...
}
% FROM TOKCYCLE:
\long\def\count@stringtoks#1{\tc@earg\count@toks{\string#1}}
\long\def\count@toks#1{\the\numexpr-1\count@@toks#1.\tc@endcnt}
\long\def\count@@toks#1#2\tc@endcnt{+1\tc@ifempty{#2}{\relax}{\count@@toks#2\tc@endcnt}}
\def\tc@ifempty#1{\tc@testxifx{\expandafter\relax\detokenize{#1}\relax}}
\long\def\tc@earg#1#2{\expandafter#1\expandafter{#2}}
\long\def\tctestifnum#1{\tctestifcon{\ifnum#1\relax}}
\long\def\tctestifcon#1{#1\expandafter\tc@exfirst\else\expandafter\tc@exsecond\fi}
\long\def\tc@testxifx{\tc@earg\tctestifx}
\long\def\tctestifx#1{\tctestifcon{\ifx#1}}
\long\def\tc@exfirst#1#2{#1}
\long\def\tc@exsecond#1#2{#2}
\makeatother

% Plot style
\pgfplotsset{
    mplot/.style={
        width=.48\textwidth,
        height=.4\textheight,
        grid=major,
        grid style={dashed,gray!30},
        ylabel style={align=center, font=\bfseries\boldmath},
        xlabel style={align=center, font=\bfseries\boldmath},
        x tick label style={font=\bfseries\boldmath},
        y tick label style={font=\bfseries\boldmath},
        scaled ticks=false,
        label style={font=\footnotesize},
        xticklabel style={font=\footnotesize},
        yticklabel style={font=\footnotesize},
        every axis plot/.append style={solid,thick},
    },
}

% Progressbar
\setbeamercolor{progress bar}{
    fg=TolLightGreen,
    bg=TolLightGreen!50!black!30
}
\makeatletter
    \setlength{\metropolis@titleseparator@linewidth}{2pt}
    \setlength{\metropolis@progressonsectionpage@linewidth}{2pt}
    \setlength{\metropolis@progressinheadfoot@linewidth}{2pt}
\makeatother

% Footer
\setbeamertemplate{frame footer}{Quentin Brateau, ENSTA Bretagne}

% Block fill
\metroset{block=fill}

% Section pages numbering
\makeatletter
\renewcommand{\metropolis@enablesectionpage}{
  \AtBeginSection{
    \ifbeamer@inframe
      \sectionpage
    \else
      \frame[c,plain]{\sectionpage}
    \fi
  }
}
\metropolis@enablesectionpage
\makeatother

% Title
\title{Union of adjacent contractors}
\date{\today}
\author{Quentin Brateau}
\institute{ENSTA Bretagne}

\titlegraphic{
    \centering
    \begin{tabular}{lllll}
        \href{https://www.defense.gouv.fr/aid}{\includegraphics[height=0.6cm]{imgs/logo_aid}} &
        \href{https://www.gdr-robotique.org/}{\includegraphics[height=0.6cm]{imgs/logo_gdr}} &
        \href{https://www.ensta-bretagne.fr/fr/}{\includegraphics[height=0.6cm]{imgs/logo_ensta}} &
        \href{https://labsticc.fr/fr}{\includegraphics[height=0.6cm]{imgs/logo_labsticc}} &
        \href{https://www.ensta-bretagne.fr/robex/}{\includegraphics[height=0.6cm]{imgs/logo_robex}}
    \end{tabular}
}

\addtobeamertemplate{frametitle}{}{%
    \begin{tikzpicture}[remember picture,overlay]
    \node[anchor=north east,yshift=2pt] at (current page.north east) {\includegraphics[height=0.85cm]{imgs/logo_ensta_aid}};
    \end{tikzpicture}
}

\begin{document}

    \maketitle

    \section{Context}

        \begin{frame}{PhD thesis}
            \centering
            \begin{minipage}[c]{0.58\textwidth}
                \begin{block}{Research laboratory}
                    \vspace{0.2cm}
                    \begin{itemize}
                        \item ENSTA Bretagne, UMR 6285, Lab-STICC, IAO, ROBEX
                    \end{itemize}
                \end{block}

                \begin{block}{Supervisiors}
                    \begin{itemize}
                        \item Luc Jaulin
                        \item Fabrice Le Bars
                    \end{itemize}
                \end{block}

                \begin{block}{Funding}
                    \begin{itemize}
                        \item AID funding: Jean-Daniel Masson
                    \end{itemize}
                \end{block}
            \end{minipage}
            \hfill
            \begin{minipage}[c]{0.4\textwidth}
                \includegraphics[height=0.7\textheight, trim={24cm 0 16cm 0}, clip]{imgs/ensta.jpg}
            \end{minipage}
        \end{frame}

        \begin{frame}{Goal}
            \begin{minipage}[c]{0.55\textwidth}
                \begin{block}{AUV}
                    \vspace{0.25cm}
                    \begin{itemize}
                        \item Control of torpedo-like AUV \\ 
                        \item Riptide's micro-uuv
                    \end{itemize}
                \end{block}
                \begin{block}{Environment}
                    \begin{itemize}
                        \item Constrained environment \\ 
                        \item Pool, harbor, ...
                    \end{itemize}
                \end{block}
                \begin{block}{Goals}
                    \begin{itemize}
                        \item Reactivity \\
                        \item Manoeuvrability
                    \end{itemize}
                \end{block}
            \end{minipage}
            \hfill
            \begin{minipage}[c]{0.4\textwidth}
                \begin{figure}[htb]
                    \includegraphics[width=\textwidth]{imgs/harbour.png}

                    \vspace{.1cm}

                    \includegraphics[width=\textwidth]{imgs/Riptide.jpeg}
                    \caption{Harbor and Riptide in the ENSTA Bretagne pool}
                \end{figure}
            \end{minipage}
        \end{frame}

        \begin{frame}{Robot sensors}
            \begin{minipage}[t]{.55\textwidth}
                \begin{exampleblock}{Riptide's sensors}
                    \vspace{0.25cm}
                    \begin{itemize}[<+->]
                        \item Proprioceptive
                        \begin{itemize}
                            \item IMU
                        \end{itemize}
                        \item Exteroceptive
                        \begin{itemize}
                            \item Pressure sensor \\ 
                            \item Echosounder
                        \end{itemize}
                    \end{itemize}
                \end{exampleblock}
                % \begin{alertblock}<+->{Echosounder measurements}
                %     \begin{itemize}[<+->]
                %         \item Sound propagation in pool causes multiple reflections
                %         \item Use information provided by the silence
                %     \end{itemize}
                % \end{alertblock}
            \end{minipage}
            \hfill
            \begin{minipage}[t]{.42\textwidth}
                \begin{figure}[htb]
                    \includegraphics[width=\textwidth]{imgs/echosounder.jpg}
                    \caption{Riptide's echosounder located below the head}
                \end{figure}
            \end{minipage}
        \end{frame}

        \begin{frame}{Localization}
            \begin{minipage}[t]{.45\textwidth}
                \vspace{-4mm}
                \centering
                \begin{figure}
                    \begin{overprint}
                        \onslide<1>\centerline{\input{imgs/robot_unknown.pgf}}
                        \onslide<2>\centerline{\input{imgs/robot_pool.pgf}}
                        \onslide<3->\centerline{\input{imgs/robot_echosounder.pgf}}
                    \end{overprint}
                    \caption{\only<1>{Unknown localization}\only<2>{Robot is in the pool}\only<3->{Distance measurement}}
               \end{figure}
            \end{minipage}%
            \hfill
            \begin{minipage}[t]{.55\textwidth}
                \centering
                \onslide<4->{
                    \begin{figure}
                        \resizebox*{.75\textwidth}{!}{
                            \input{imgs/robot_zoom.pgf}
                        }
                        \caption{Uncertain area}
                    \end{figure}
                }
                \vspace{-5mm}
                \begin{alertblock}<5->{Problem statement}
                    Why is there this uncertain area?
                \end{alertblock}
            \end{minipage}
        \end{frame}

    \section{Geometric contractors}

    \begin{frame}{Geometric contractors}
        \begin{minipage}[c]{.45\textwidth}
            \begin{block}<+->{Geometric contractors}
                \vspace{2.5mm}
                \begin{itemize}[<+->]
                    \item Contractors based on geometrical constraints
                    \item Usefull in localization
                \end{itemize}
            \end{block}
            \begin{exampleblock}<+->{Example}
                \begin{itemize}[<+->]
                    \item CtcCross
                    \item CtcVisible
                \end{itemize}
            \end{exampleblock}
            \begin{block}<+->{Polygon extension}
                \begin{itemize}[<+->]
                    \item Contractors on segments
                    \item Extension on polygons
                \end{itemize}
            \end{block}
        \end{minipage}%
        \hfill
        \begin{minipage}[c]{.5\textwidth}
            \begin{figure}
                \begin{overprint}
                    \onslide<5>\centerline{\input{imgs/sepcross_segment.pgf}}
                    \onslide<6->\centerline{\input{imgs/sepvisible_segment.pgf}}
                \end{overprint}
                \vspace{-8mm}
                \begin{overprint}
                    \onslide<5>\caption{SepCross}
                    \onslide<6->\caption{SepVisible}
                \end{overprint}
            \end{figure}
        \end{minipage}
    \end{frame}

    \begin{frame}[fragile]{Union problem}
        \begin{minipage}[t]{.5\textwidth}
            \begin{figure}
                \centering
                \adjustbox{max width=\textwidth}{
                    \begin{tikzpicture}
                        [   
                            ArrowLine/.style n args={4}{
                                line width=.75pt, #1,
                                postaction={
                                    decorate, decoration={
                                        markings, mark=at position #4 with {
                                            \arrow[line width=.75pt, #1] {Latex};
                                            \node[#3] {#2};
                                        }
                                    }
                                }
                            }
                        ]

                        % Specify the bounding box
                        \useasboundingbox (-.5,-.5) rectangle (3.6,3.6);

                        % Define nodes
                        \coordinate (n0) at (0,0);
                        \coordinate (n1) at (0,1.8);
                        \coordinate (n2) at (1,1);
                        \coordinate (n3) at (2.2,0.8);

                        % Set overshoot length
                        \def\prevovershoot{0.5}
                        \def\nextovershoot{1.5}

                        % Iterate over nodes
                        \foreach \i in {1,2,3} {
                            % Do something with each node
                            \pgfmathanglebetweenpoints{\pgfpointanchor{n0}{center}}{\pgfpointanchor{n\i}{center}}
                            \let\angle\pgfmathresult
                            \coordinate (prev0\i) at ($(n0) + (\angle+180:\prevovershoot)$);
                            \coordinate (next0\i) at ($(n\i) + (\angle:\nextovershoot)$);
                        }
                            
                        % Iterate over nodes
                        \foreach \i in {1,3} {
                            % Do something with each node 
                            \pgfmathanglebetweenpoints{\pgfpointanchor{n2}{center}}{\pgfpointanchor{n\i}{center}}
                            \let\angle\pgfmathresult
                            \coordinate (prev2\i) at ($(n2) + (\angle+180:\prevovershoot)$);
                            \coordinate (next2\i) at ($(n\i) + (\angle:\prevovershoot)$);
                        }

                        % Draw obstacles
                        \draw<+->[line width=1.5pt] (n1) -- node[midway, above] {$e_1$} (n2) -- node[midway, above] {$e_2$} (n3);
                        
                        % First visibility set
                        \draw<+-5>[densely dotted, gray, line width=.75pt] (prev01) -- (next01);
                        \draw<.-5>[densely dotted, gray, line width=.75pt] (prev21) -- (next21);
                        \draw<.-5>[densely dotted, gray, line width=.75pt] (prev02) -- (next02);

                        \draw<+->[ArrowLine={line width=.75pt,RoyalBlue}{$a$}{above left}{.62}] (next01) -- (n1);
                        \draw<+->[ArrowLine={line width=.75pt,RoyalBlue}{$b$}{below left}{.62}] (n1) -- (n2);
                        \draw<+->[ArrowLine={line width=.75pt,RoyalBlue}{$c$}{xshift=-5mm}{.45}] (n2) -- (next02);

                        \filldraw<+->[RoyalBlue,opacity=.3] (n1) -- (n2) -- (next02) -- (next02) to[out=180,in=15] (next01) -- cycle;
                        \node<.->[RoyalBlue,xshift=3mm,yshift=-3mm] at (next01) {$A_1$};
                        
                        % Second visibility set
                        \draw<+-10>[densely dotted, gray, line width=.75pt] (prev02) -- (next02);
                        \draw<.-10>[densely dotted, gray, line width=.75pt] (prev23) -- (next23);
                        \draw<.-10>[densely dotted, gray, line width=.75pt] (prev03) -- (next03);

                        \draw<+->[ArrowLine={line width=.75pt,RoyalBlue}{$d$}{above left}{.45}] (next02) -- (n2);
                        \draw<+->[ArrowLine={line width=.75pt,RoyalBlue}{$e$}{below left}{.62}] (n2) -- (n3);
                        \draw<+->[ArrowLine={line width=.75pt,RoyalBlue}{$f$}{above left}{.62}] (n3) -- (next03);

                        % Increment the overlay counter manually
                        \filldraw<+->[RoyalBlue,opacity=.3] (n2) -- (n3) -- (next03) to[out=190,in=15] (next02) -- cycle;
                        \node<.->[RoyalBlue,xshift=1mm,yshift=-3mm] at (next02) {$A_2$};

                        % Draw nodes
                        \fill[RubineRed,draw=black] (n0) circle (2pt) node[xshift=3mm] {$p$};
                        \fill[white,draw=black,line width=1pt] (n1) circle (1.5pt);
                        \fill[white,draw=black,line width=1pt] (n2) circle (1.5pt);
                        \fill[white,draw=black,line width=1pt] (n3) circle (1.5pt);
                    \end{tikzpicture}
                }
                \caption{Sepvisible on a polygon}
            \end{figure}
        \end{minipage}%
        \hfill
        \begin{minipage}[t]{.5\textwidth}
            \begin{overprint}
                \only<10-20>{
                    \begin{block}<+->{Masked area}
                        \vspace{2.5mm}
                        $$A = A_1 \cup A_2$$
                    \end{block}
                    \begin{block}<+->{Boundary aproach}
                        \vspace{2.5mm}
                        \begin{itemize}[<+->]
                            \item $\partial A_1 = a + b + c$
                            \item $\partial A_2 = d + e + f$
                            \item
                            \begin{overprint}
                                \only<.>{$\partial A \neq \partial A_1 + \partial A_2$}
                                \only<+>{$\partial A = a + b + c + d + e + f$}
                                \only<+>{$\partial A = a + b + {\color{RubineRed}c} + {\color{RubineRed}d} + e + f$}
                                \only<+>{$\partial A = a + b + {\color{RubineRed}c} + {\color{RubineRed}(-c)} + e + f$}
                                \only<+->{$\partial A = a + b + e + f$}
                            \end{overprint}
                        \end{itemize}
                        \vspace{-5mm}
                    \end{block}
                }
                \only<21->{
                    \begin{figure}
                        \resizebox*{.8\textwidth}{!}{
                            \input{imgs/sepvisible_polygon.pgf}
                        }
                        \caption{SepVisible on a polygon}
                    \end{figure}
                    \vspace{-2cm}
                    
                }
            \end{overprint}
        \end{minipage}
        \begin{center}
            \begin{minipage}[c]{.5\textwidth}
                \vspace{-5mm}
                \begin{alertblock}<22->{Solution}
                    \vspace{2.5mm}
                    \begin{itemize}
                        \item<23-> $\mathcal{C}_{A\cup B} \neq \mathcal{C}_{A} \cup \mathcal{C}_{B}$
                        \item<24-> $c + d = \mathbb{R}^2$
                    \end{itemize}
                \end{alertblock}
            \end{minipage}
        \end{center}
    \end{frame}

    \section{Union of contractors}

        \begin{frame}{Frame of the problem}
            \begin{minipage}[c]{.56\textwidth}
                \begin{block}{Set definition}
                    \vspace{2.5mm}
                    \begin{itemize}[<+->]
                        \item $A: \left\{x_1 + 3 \cdot x_2 \in [-\infty; 0]\right\}$
                        \item $B: \left\{(x_1+0.5)^2 + x_2^2 - 2^2\in [-\infty; 0]\right\}$
                        \item $C: \left\{(x_1-0.5)^2 + x_2^2 - 2^2\in [-\infty; 0]\right\}$
                    \end{itemize}
                \end{block}
                \begin{block}<6>{Set expression}
                    $$Z = (A \cap B) \cup (\closure{A} \cap C)$$
                \end{block}
            \end{minipage}%
            \hfill
            \begin{minipage}[c]{.4\textwidth}
                \begin{figure}
                    \begin{overprint}
                        \onslide<1>\centerline{\input{imgs/problem_a.pgf}}
                        \onslide<2>\centerline{\input{imgs/problem_b.pgf}}
                        \onslide<3>\centerline{\input{imgs/problem_c.pgf}}
                        \onslide<4>\centerline{\input{imgs/problem_e1.pgf}}
                        \onslide<5>\centerline{\input{imgs/problem_e2.pgf}}
                        \onslide<6>\centerline{\input{imgs/problem_e.pgf}}
                    \end{overprint}
                    \caption{
                        \only<1>{$A$}\only<2>{$B$}\only<3>{$C$}\only<4>{$A \cap B$}\only<5>{$\closure{A} \cap C$}\only<6>{$Z = (A \cap B) \cup (\closure{A} \cap C)$}
                    }
                \end{figure}
            \end{minipage}
        \end{frame}

        \begin{frame}[fragile]{Karnaugh map}
            \begin{center}
                \begin{minipage}[b]{.6\textwidth}
                    \begin{figure}
                        \centering
                        \begin{tikzpicture}[
                                karnaugh cell size=.85cm,
                                karnaugh, American style,
                                kmlabel top/.style={font=\small,above},
                                kmlabel left/.style={font=\small,left},
                                kmsep line length=0.75\kmunitlength,
                                thick, grp/.style n args={3}{
                                    #1,fill=#1,fill opacity=0.4,rectangle,draw,
                                    minimum width=#2\kmunitlength,
                                    minimum height=#3\kmunitlength,
                                    rounded corners=.05\kmunitlength,
                                },
                            ]
                            \karnaughmaptab{3}{$f(a,b,c)$}{{$a$}{$b$}{$c$}}{0110 0011}{
                                \draw<-2>[grp={RubineRed}{1.9}{.9}] (3.95\kmunitlength,.05) -- (3.95\kmunitlength,.95\kmunitlength) -- (2.95\kmunitlength,.95\kmunitlength) {[rounded corners=false] -- (2.95\kmunitlength,2\kmunitlength) -- (2.05\kmunitlength,2\kmunitlength)} -- (2.05\kmunitlength,1.95\kmunitlength) -- (1.05\kmunitlength,1.95\kmunitlength) -- (1.05\kmunitlength,1.05\kmunitlength) -- (2.05\kmunitlength,1.05\kmunitlength) {[rounded corners=false] -- (2.05\kmunitlength,0) -- (2.95\kmunitlength,0)} -- (2.95\kmunitlength,.05) -- cycle;
                                \draw<2->[Dandelion,thick] (2\kmunitlength,0) -- (2\kmunitlength,\kmunitlength) -- (\kmunitlength,\kmunitlength) -- (\kmunitlength,2\kmunitlength) -- (2\kmunitlength,2\kmunitlength);
                                \draw<2->[Dandelion,thick] (3\kmunitlength,0) -- (4\kmunitlength,0) -- (4\kmunitlength,\kmunitlength) -- (3\kmunitlength,\kmunitlength) -- (3\kmunitlength,2\kmunitlength);
                                \node<3->[alt=<4>{grp={gray}{1.9}{.9}}{grp={RubineRed}{1.9}{.9}}] (innerB) at (3\kmunitlength,0.5\kmunitlength) {};
                                \node<4->[grp={RubineRed}{1.9}{.9}] (innerC) at (2\kmunitlength,1.5\kmunitlength) {};
                                \draw<5-6>[RoyalBlue,line width=.5mm] (2\kmunitlength,0) -- (3\kmunitlength,0);
                                \draw<5-6>[RoyalBlue,line width=.5mm] (2\kmunitlength,\kmunitlength) -- (3\kmunitlength,\kmunitlength);
                                \draw<5-6>[RoyalBlue,line width=.5mm] (2\kmunitlength,2\kmunitlength) -- (3\kmunitlength,2\kmunitlength);
                                \draw<7->[grp={RoyalBlue}{.9}{1.9}] [rounded corners=false] (2.05\kmunitlength,0) -- (2.05\kmunitlength,2) -- (2.95\kmunitlength,2) -- (2.95\kmunitlength,0) -- cycle;
                            }
                        \end{tikzpicture}
                        \caption{Karnaugh map - $Z$}
                    \end{figure}
                \end{minipage}%
                \hfill
                \begin{minipage}[b]{.4\textwidth}
                    \begin{figure}
                        \begin{overprint}
                            \only<1-2>{
                                \resizebox{.7\textwidth}{!}{\input{imgs/problem_paving.pgf}}
                            }
                            \only<3>{
                                \resizebox{.7\textwidth}{!}{\input{imgs/problem_e1.pgf}}
                            }
                            \only<4>{
                                \resizebox{.7\textwidth}{!}{\input{imgs/problem_e2.pgf}}
                            }
                            \only<5-6>{
                                \resizebox{.7\textwidth}{!}{\input{imgs/problem_e.pgf}}
                            }
                            \only<7>{
                                \resizebox{.7\textwidth}{!}{\input{imgs/problem_e3.pgf}}
                            }
                            \only<8>{
                                \resizebox{.7\textwidth}{!}{\input{imgs/problem_paving.pgf}}
                            }
                        \end{overprint}
                        \vspace{-1cm}
                        \caption{
                            \only<1-2,8>{$Z$}\only<3>{$A \cap B$}\only<4>{$\closure{A} \cap C$}\only<5-6>{$(A \cap B) \cup (\closure{A} \cap C)$}\only<7>{$B \cap C$}
                        }
                    \end{figure}
                \end{minipage}
                \begin{minipage}[t]{.65\textwidth}
                    \vspace{-8mm}
                    \begin{block}<3->{Disjunctive Normal Form}
                        \vspace{2.5mm}
                        \centering
                        $f(a, b, c) = \onslide<3->{(a \land b)} \onslide<4->{\lor (\lnot a \land c)}$
                    \end{block}
                    \begin{block}<6->{Boundary Preserving Form}
                        \vspace{2.5mm}
                        \centering
                        $f(a, b, c) = (a \land b) \lor (\lnot a \land c) \onslide<7->{\lor (b \land c)}$
                    \end{block}
                \end{minipage}
            \end{center}
        \end{frame}

        \begin{frame}[fragile]{Adjacent Sets}
            \begin{minipage}[c]{.6\textwidth}
                \begin{figure}
                    \centering
                    \begin{tikzpicture}[
                            karnaugh cell size=.85cm,
                            karnaugh, American style,
                            kmlabel top/.style={font=\small,above},
                            kmlabel left/.style={font=\small,left},
                            kmsep line length=0.75\kmunitlength,
                            thick, grp/.style n args={3}{
                                #1,fill=#1,fill opacity=0.4,rectangle,draw,
                                minimum width=#2\kmunitlength,
                                minimum height=#3\kmunitlength,
                                rounded corners=.05\kmunitlength,
                            },
                        ]
                        \karnaughmaptab{2}{$f(a,b)$}{{$a$}{$b$}}{01 11}{
                            \draw<1>[grp={RubineRed}{1.9}{.9}] (1.95\kmunitlength,.05) -- (1.95\kmunitlength,1.95\kmunitlength) -- (1.05\kmunitlength,1.95\kmunitlength) -- (1.05\kmunitlength,.95\kmunitlength) -- (.05\kmunitlength,.95\kmunitlength) -- (.05\kmunitlength,.05\kmunitlength) -- cycle;
                            \node<2->[alt=<3>{grp={gray}{1.9}{.9}}{grp={RubineRed}{1.9}{.9}}] (innerA) at (\kmunitlength,.5\kmunitlength) {};
                            \node<3->[grp={RubineRed}{.9}{1.9}] (innerB) at (1.5\kmunitlength,\kmunitlength) {};
                        }
                    \end{tikzpicture}
                    \caption{Karnaugh map - $A \cup B$}
                \end{figure}
            \end{minipage}%
            \hfill
            \begin{minipage}[c]{.4\textwidth}
                \begin{figure}
                    \begin{overprint}
                        \only<1>{
                            \resizebox{.9\textwidth}{!}{\input{imgs/circle_expected.pgf}}
                        }
                        \only<2>{
                            \resizebox{.9\textwidth}{!}{\input{imgs/circle_1.pgf}}
                        }
                        \only<3>{
                            \resizebox{.9\textwidth}{!}{\input{imgs/circle_2.pgf}}
                        }
                        \only<4>{
                            \resizebox{.9\textwidth}{!}{\input{imgs/circle_discontinuity.pgf}}
                        }
                    \end{overprint}
                    \vspace{-1cm}
                    \caption{\only<1>{Expected $A \cup B$}\only<2>{$A$}\only<3>{$B$}\only<4>{$A \cup B$}}
                \end{figure}
            \end{minipage}
        \end{frame}

        \begin{frame}[fragile]{Boundary approach}
            \begin{minipage}[t]{.5\textwidth}
                \begin{figure}
                    \centering
                    \begin{tikzpicture}[
                            karnaugh cell size=.85cm,
                            karnaugh, American style,
                            kmlabel top/.style={font=\small,above},
                            kmlabel left/.style={font=\small,left},
                            kmsep line length=0.75\kmunitlength,
                            thick, grp/.style n args={3}{
                                #1,fill=#1,fill opacity=0.4,rectangle,draw,
                                minimum width=#2\kmunitlength,
                                minimum height=#3\kmunitlength,
                                rounded corners=.05\kmunitlength,
                            },
                        ]
                        \karnaughmaptab{3}{$f(a,b,c)$}{{$a$}{$b$}{$c$}}{0110 0011}{
                            \draw[grp={RubineRed}{1.9}{.9}] (3.95\kmunitlength,.05) -- (3.95\kmunitlength,.95\kmunitlength) -- (2.95\kmunitlength,.95\kmunitlength) {[rounded corners=false] -- (2.95\kmunitlength,2\kmunitlength) -- (2.05\kmunitlength,2\kmunitlength)} -- (2.05\kmunitlength,1.95\kmunitlength) -- (1.05\kmunitlength,1.95\kmunitlength) -- (1.05\kmunitlength,1.05\kmunitlength) -- (2.05\kmunitlength,1.05\kmunitlength) {[rounded corners=false] -- (2.05\kmunitlength,0) -- (2.95\kmunitlength,0)} -- (2.95\kmunitlength,.05) -- cycle;
                            % dBnA
                            \draw<2->[alt=<2>{RoyalBlue,line width=.5mm}{Dandelion,line width=.3mm}] (2\kmunitlength,0) -- (2\kmunitlength,\kmunitlength);
                            \draw<2->[alt=<2>{RoyalBlue,line width=.5mm}{Dandelion,line width=.3mm}] (4\kmunitlength,0) -- (4\kmunitlength,\kmunitlength);
                            % dCnA
                            \draw<3->[alt=<3>{RoyalBlue,line width=.5mm}{Dandelion,line width=.3mm}] (\kmunitlength,\kmunitlength) -- (\kmunitlength,2\kmunitlength);
                            \draw<3->[alt=<3>{RoyalBlue,line width=.5mm}{Dandelion,line width=.3mm}] (3\kmunitlength,\kmunitlength) -- (3\kmunitlength,2\kmunitlength);
                            % dAnBn~C
                            \draw<4->[alt=<4>{RoyalBlue,line width=.5mm}{Dandelion,line width=.3mm}] (3\kmunitlength,0) -- (4\kmunitlength,0);
                            \draw<4->[alt=<4>{RoyalBlue,line width=.5mm}{Dandelion,line width=.3mm}] (3\kmunitlength,\kmunitlength) -- (4\kmunitlength,\kmunitlength);
                            % dAn~BnC
                            \draw<5->[alt=<5>{RoyalBlue,line width=.5mm}{Dandelion,line width=.3mm}] (\kmunitlength,\kmunitlength) -- (2\kmunitlength,\kmunitlength);
                            \draw<5->[alt=<5>{RoyalBlue,line width=.5mm}{Dandelion,line width=.3mm}] (\kmunitlength,2\kmunitlength) -- (2\kmunitlength,2\kmunitlength);
                        }
                    \end{tikzpicture}
                    \caption{Karnaugh map of the expression}
                \end{figure}
                \vspace{2cm}
            \end{minipage}%
            \hfill
            \begin{minipage}[t]{.5\textwidth}
                \begin{figure}
                    \centering
                    \begin{overprint}
                        \only<2>{\input{imgs/problem_dBA.pgf}}
                        \only<3>{\input{imgs/problem_dC_A.pgf}}
                        \only<4>{\input{imgs/problem_dAB_C.pgf}}
                        \only<5>{\input{imgs/problem_dA_BC.pgf}}
                        \only<6>{\input{imgs/problem_dZ.pgf}}
                        \only<7>{\input{imgs/problem_paving.pgf}}
                    \end{overprint}
                    \vspace{-1.2cm}
                    \only<2->{
                        \caption{\only<2>{$\partial B \cap A$}\only<3>{$\partial C \cap \closure{A}$}\only<4>{$\partial A \cap B \cap \closure{C}$}\only<5>{$\partial A \cap \closure{B} \cap C$}\only<6>{$\partial Z$}\only<7>{$Z$}}
                    }
                \end{figure}
            \end{minipage}
        \end{frame}

        \begin{frame}{Outlook}
            \centering
            \begin{minipage}[c]{.65\textwidth}
                \begin{block}<+->{Problem statement}
                    \vspace{2.5mm}
                    \begin{itemize}[<+->]
                        \item Union contractors problem
                        \item Adjacent contractors
                        \item Fake boundary
                    \end{itemize}
                \end{block}
                \begin{block}<+->{Geometric contractors}
                    \begin{itemize}[<+->]
                        \item Define $\mathcal{C}_{A \cup B}$
                        \item Add constraint $A \cup \closure{A} = \mathbb{R}^2$
                    \end{itemize}
                \end{block}
                \begin{block}<+->{General union}
                    \begin{itemize}[<+->]
                        \item Boundary Preserving Form
                        \item Boundary approach + Predicate
                    \end{itemize}
                \end{block}
            \end{minipage}
        \end{frame}

        \appendix

        \begin{frame}[standout]
            Questions?
        \end{frame}

        \maketitle

    %%%%%%%%%%%%%%%%%%%%%%%%%%%%%%%%%%%%%%%%%%%%%%%%%%%%%%%%%%%%%%%%%%%%

    % \begin{frame}{Set operators}
    %     \centering
    %     \begin{figure}
    %         % Definition of circles
    %         \def\firstcircle{(-0.75,0) circle (1)}
    %         \def\secondcircle{(0.75,0) circle (1)}

    %         \colorlet{circle edge}{TolDarkBlue}
    %         \colorlet{circle area}{TolLightBlue}

    %         \tikzset{filled/.style={fill=circle area, draw=circle edge, thick},
    %             outline/.style={draw=circle edge, thick}}

    %         \centering 

    %         \onslide<+->\begin{subfigure}[b]{0.5\textwidth}
    %             \centering
    %             \begin{tikzpicture}
    %                 \draw[filled] \firstcircle node {$A$}
    %                             \secondcircle node {$B$};
    %                 \node[anchor=south] at (current bounding box.north) {$A \cup B$};
    %             \end{tikzpicture}
    %             \subcaption{Set union}
    %         \end{subfigure}%
    %         \hfill
    %         \onslide<+->\begin{subfigure}[b]{0.5\textwidth}
    %             \centering
    %             \begin{tikzpicture}
    %                 \begin{scope}
    %                     \clip \firstcircle;
    %                     \fill[filled] \secondcircle;
    %                 \end{scope}
    %                 \draw[outline] \firstcircle node {$A$};
    %                 \draw[outline] \secondcircle node {$B$};
    %                 \node[anchor=south] at (current bounding box.north) {$A \cap B$};
    %             \end{tikzpicture}
    %             \subcaption{Set intersection}
    %         \end{subfigure}%

    %         \onslide<+->\begin{subfigure}[b]{0.5\textwidth}
    %             \centering
    %             \begin{tikzpicture}
    %                 \begin{scope}
    %                     \clip \secondcircle;
    %                     \draw[filled, even odd rule] \secondcircle node {$B$} \firstcircle;
    %                 \end{scope}
    %                 \draw[outline] \secondcircle \firstcircle node {$A$};
    %                 \node[anchor=south] at (current bounding box.north) {$\bar{A}$};
    %             \end{tikzpicture}
    %             \subcaption{Set negation}
    %         \end{subfigure}%
    %         \hfill
    %         \onslide<+->\begin{subfigure}[b]{0.5\textwidth}
    %             \centering
    %             \begin{tikzpicture}
    %                 \begin{scope}
    %                     \clip \firstcircle;
    %                     \draw[filled, even odd rule] \firstcircle node {$A$} \secondcircle;
    %                 \end{scope}
    %                 \draw[outline] \firstcircle
    %                             \secondcircle node {$B$};
    %                 \node[anchor=south] at (current bounding box.north) {$A \mysetminus B$};
    %             \end{tikzpicture}
    %             \subcaption{Set deprivation}
    %         \end{subfigure}
    %     \end{figure}
    % \end{frame}

    % \begin{frame}{Interval example}

    %     \centering
    %     \begin{minipage}[t]{\textwidth}
    %         \begin{figure}
    %             \begin{tikzpicture}
    %                 \draw[stealth-stealth, thick] (-5,0) node[below]{$-\infty$} -- (5,0) node[below]{$\infty$};
    %                 \draw foreach \X in {-4,-3,...,4} {(\X,.1) -- (\X,-.1) node[below=0.2em]{$\X$}};
    %                 \onslide<2-5>{\draw[very thick,RoyalBlue,alt=<4>{{Bracket[width=1.2em]}-{Bracket[width=1.2em,RubineRed]}}{{Bracket[width=1.2em]}-{Bracket[width=1.2em]}}] (-2,0) -- (1,0) node[midway,yshift=2.5mm]{$A$};}
    %                 \onslide<3-5>{\draw[very thick,Dandelion,alt=<4>{{Bracket[width=1.2em,RubineRed]}-{Bracket[width=1.2em]}}{{Bracket[width=1.2em]}-{Bracket[width=1.2em]}}] (1,0) -- (3,0) node[midway,yshift=2.5mm]{$B$};}
    %                 \onslide<5,10>{\draw[decorate,decoration=brace] (-2,0.5) -- (3,0.5) node[midway,above=0.1em]{$A \cup B$};}
    %                 \onslide<6>{\draw[very thick,RoyalPurple,{Bracket[width=1.2em]}-{Bracket[width=1.2em]}] (-2,0) -- (3,0) node[midway,yshift=2.5mm]{$C$};}

    %                 % Boundary
    %                 \onslide<7-10>{
    %                     \filldraw[RoyalBlue] (-2,0) circle[radius=2.5pt];
    %                     \node[RoyalBlue] at (-.5,.25) {$A$};
    %                     \filldraw[RoyalBlue] (1,0) circle[radius=2.5pt];
    %                 }
    %                 \onslide<8-10>{
    %                     \filldraw[Dandelion] (1,2.5pt) arc (90:-90:2.5pt) -- cycle;
    %                     \node[Dandelion] at (2,.25) {$B$};
    %                     \filldraw[Dandelion] (3,0) circle[radius=2.5pt];
    %                 }
    %                 \onslide<9>{
    %                     \filldraw[RubineRed] (1,0) circle[radius=2.5pt];
    %                 }
    %                 \onslide<11>{
    %                     \filldraw[RoyalPurple] (-2,0) circle[radius=2.5pt];
    %                     \node[RoyalPurple] at (.5,.25) {$C$};
    %                     \filldraw[RoyalPurple] (3,0) circle[radius=2.5pt];
    %                 }
    %             \end{tikzpicture}
    %             \caption{Interval example representation}
    %         \end{figure}
    %     \end{minipage}

    %     \begin{minipage}[t]{0.48\textwidth}
    %         \begin{exampleblock}<2->{Intervals}
    %             \vspace{2.5mm}
    %             \begin{itemize}
    %                 \item<2-> $A = [-2, {\color<4>{RubineRed}1}]$
    %                 \item<3-> $B = [{\color<4>{RubineRed}1}, 3]$
    %             \end{itemize}
    %         \end{exampleblock}
    %         \begin{exampleblock}<5->{Union}
    %             \begin{itemize}
    %                 \item<5-> $C = A \cup B \onslide<6->{= [-2, 3]}$
    %             \end{itemize}
    %         \end{exampleblock}
    %     \end{minipage}%
    %     \hfill
    %     \begin{minipage}[t]{0.48\textwidth}
    %         \begin{exampleblock}<7->{Boundary}
    %             \begin{itemize}
    %                 \vspace{2.5mm}
    %                 \item<7-> $\partial A = \{-2, {\color<9>{RubineRed}1}\}$
    %                 \item<8-> $\partial B = \{{\color<9>{RubineRed}1}, 3\}$
    %             \end{itemize}
    %         \end{exampleblock}
    %         \begin{exampleblock}<10->{Union Boundary}
    %             \begin{itemize}
    %                 \item<10-> $\partial C = \partial (A \cup B) \onslide<11>{= \{-2, 3\}}$
    %             \end{itemize}
    %         \end{exampleblock}
    %     \end{minipage}%
    % \end{frame}

    % \begin{frame}{Property}
    %     \centering
    %     \begin{minipage}{.75\textwidth}
    %         \begin{alertblock}{Boundary property}
    %             \vspace{2.5mm}
    %             \begin{itemize}
    %                 \item Set union doesn't preserve the boundary,
    %                 \item Set union transforms boundary points into interior points.
    %             \end{itemize}
    %         \end{alertblock}

    %         \begin{alertblock}{Boundary of union}
    %             \begin{equation*}
    %                 \partial (A \cup B) \subseteq \partial A \cup \partial B
    %             \end{equation*}
    %         \end{alertblock}
    %     \end{minipage}
    % \end{frame}

    
\end{document}